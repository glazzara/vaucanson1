% help.tex
% $Id$

\documentclass{article}

% packages.tex
%
% $Id$

\usepackage{makeidx}
\usepackage{verbatim}
\usepackage{listings}
\usepackage{url}
\usepackage{ulem}
\usepackage{amsmath}
\usepackage{amsfonts}
\usepackage{palatino}
\usepackage{a4wide}
\usepackage{xspace}
\usepackage{listings}

% macros.tex
%
% $Id$

\newenvironment{code}[1][]%
{\begin{center}
     \rule{1cm}{0.02cm} \textit{#1} \rule{1cm}{0.02cm} 
  \end{center}
  \verbatim }
{\endverbatim \begin{center} \rule{2cm}{0.02cm} \end{center}}


% definitions.tex
%
%

\newcommand\MyDefAlphabet{
\textbf{alphabet:} collection of symbol
}

\begin{document}

\title{Your first program in Vaucanson}

\maketitle
\tableofcontents

\section{Introduction}

This document is designed to help beginners writing their first
program in Vaucanson. For this purpose, the library provides some c++
headers that contain shortcuts for the basic usage of the library.

\section{How to build an automaton}

The following listing is a valid Vaucanson program:

\begin{code}
1  #include <vaucanson/vaucanson_boolean_automaton.hh>
2  using namespace vcsn::boolean_automaton;
3  int main()
4  {
5    alphabet_t alphabet;
6    alphabet.insert('a');
7    alphabet.insert('b');
8    automaton_t a = new_automaton(alphabet);
9    hstate_t p = a.add_state();
10   hstate_t q = a.add_state();
11   a.set_initial(p);
12   a.set_final(q);
13   a.add_letter_edge(p, q, 'a');
14   a.add_letter_edge(q, p, 'b');
15   tools::dot_dump(std::cout, a, "automaton");
16 }
\end{code}

\begin{list}{$\triangleright$}{}
\item \textbf{line 1 and 2}: To program standard boolean automata (the
so-called acceptors), the user can include this shortcuts' header. The
\verb!using namespace! command makes all shorcuts directly
available. Otherwise, the user has to prefix every types and functions
by \verb!vcsn::boolean_automaton::!.
\item \textbf{line 5}: The first thing to do is to declare the
alphabet we are working with. The type \verb!alphabet_t! is predefined
into the header. We obtain an object instance called \verb!alphabet!.
\item \textbf{line 6 and 7}: \verb!alphabet! is an object instance
that provides services. For example, we can insert 'a' and 'b' into
the alphabet (other services can be consulted in the file 
\verb!vaucanson/algebra/concept/alphabets_base.hh!).
\item \textbf{line 8}: The function \verb!new_automaton! defined in
the header takes an alphabet and returns a fresh empty automaton. Here,
we store this automaton into the variable \verb!a!.
\item \textbf{line 9 and 10}: As an object instance, \verb!a! provides
services like the ability to create a new state. This state is
characterized by a handler (concretely, a little integer). In Vaucanson,
every handler for state has the \verb!hstate_t! type. 
\item \textbf{line 11 and 12}: \verb!a! can also change the status of
its state. For example, the \verb!set_initial! method mark a state as
initial. 
\item \textbf{line 13 and 14}: We can define a transition between two
states labelled by a letter using the method \verb!add_letter_edge!.
The other methods that the user can expect from an automaton are located in
the file: \verb!vaucanson/automata/concept/automata_base.hh!.
\item \textbf{line 15}: Vaucanson provides different way to interact
with the user. For example, we can use the \verb!DOT! format to
display the automaton 'a' with \verb!dotty!.
% FIXME: add reference to the ATT site.
\end{list}

\subsection{How to compile a stand-alone program}

In a shell, if your file is called \verb!automaton.cc! and
if vaucanson is installed on your system,
type the following command:

\begin{verbatim}
% g++ automaton.cc -o automaton
\end{verbatim}

Note: if your Vaucanson is not installed or if it is not installed
into a standard location, add \verb!-I the_vaucanson_directory/include!
to your command.

To execute the program and to display the resulting automaton:

\begin{verbatim}
% ./automaton | dotty -
\end{verbatim}

\section{How to use standard algorithms}

The second step is to test the algorithms of Vaucanson. For this
purpose, the user can include also shortcut header. For example, in
the following code, the program build an automaton and compute its
associated deterministic automaton.

\begin{code}
1  #include <vaucanson/vaucanson_boolean_automaton.hh>
2  #include <vaucanson/standard_algorithms.hh>
3  using namespace vcsn::boolean_automaton;
4  int main()
5  {
6    alphabet_t alphabet;
7    alphabet.insert('a');
8    alphabet.insert('b');
9    automaton_t a = new_automaton(alphabet);
10   hstate_t p = a.add_state();
11   hstate_t q = a.add_state();
12   hstate_t r = a.add_state();
13   a.set_initial(p);
14   a.set_final(r);
15   a.add_letter_edge(q, q, 'a');
16   a.add_letter_edge(p, q, 'a');
17   a.add_letter_edge(q, p, 'b');
18   a.add_letter_edge(q, r, 'b');
19   a.add_letter_edge(r, q, 'a');
20   a.add_letter_edge(r, r, 'b');
21   automaton_t a_det = determinize(a);
22   tools::dot_dump(std::cout, a_det, "det_automaton");
23 }
\end{code}

\begin{list}{$\triangleright$}{}
\item \textbf{line 2}: This header is including several headers present
in the \verb!vaucanson/algorithms! directory.
\item \textbf{line 21}: Algorithms are functions not object instance.
\end{list}

\end{document}