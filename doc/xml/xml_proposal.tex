\documentclass[a4paper]{article}
\pagestyle{plain}
\usepackage{amsmath,amsfonts,amscd,amssymb}
\usepackage{alltt}
\usepackage{xspace}
\usepackage{vaucanson-g}
\usepackage{myhyperref}
\usepackage{texi}
\usepackage{natbib}
%%%%%%%%%%%%%%%%%%%%%%%%%%%%%%%%%%%%%%%%%%%

\newcommand{\bigskipneg}{\vspace*{-5ex}} %960728
\newcommand{\medskipneg}{\vspace*{-2ex}} %960728
\newcommand{\madskipneg}{\vspace*{-1.5ex}} %960728
\newcommand{\smallskipneg}{\vspace*{-1ex}} %960728
\newcommand{\miniskipneg}{\vspace*{-0.25ex}} %960728
\newcommand{\skipneg}{\vspace*{-0.5cm}}



\newcommand{\Vauc}{{\sc Vaucanson}\xspace}

%% XML Tag.
\newcommand{\xtag}[1]{\texttt{<#1>}}
%% XML Attribute.
\newcommand{\xattr}[1]{\texttt{#1}}


\begin{document}

\title{XML proposal for automata description}

\author{The \Vauc group}

\maketitle{}

\begin{abstract}
This paper presents an XML description format for the automaton
representation. We introduce the proposal through some examples that
enlight characteristic features of the format, with a progressive
complexity. Finally, we briefly focus on implementation concerns.
\end{abstract}

%%%%%%%%%%%%%%%%%%%%%%%%%%%%%%%%%%%%%
\section*{Introduction}

Conceiving a universal automaton exchange format aims at providing the
community with a communication tool for the connection
of the various programs that deal with automata and transducers.  This
system would enable someone to load an automaton in any automata
handling program from a XML file, or to store an existing one from
this program into an XML file.

%%%%%%%%%%%%%%%%%%%%%%%%%%%%%%%%%%%%%
\section{Overview}

As it will be described further, we use an XSD file \cite{vaucanson.www.xsd}
for the description format of this XML proposition, since it fits our needs.
This report first presents an XML representation of a classical Boolean
automaton. Then this example will be fleshed out, so as to deal with transducers
and more general automata with multiplicity.

The automaton description is structured in two parts.  The
\xtag{labelType} tag provides some automaton type definitions. This can be a
Boolean automaton, or a weighted one with the ability to specify the
weight type.  It can also have some alphabet specifications, etc. The
\xtag{content} tag provides the definition of the automaton
``structure''.

The visual representation of automata involves a very large amount of
information. This is why two different types of information are
distinguished in this proposition, described in the following
two tags.  The \xtag{geometry} data correspond to the embedding
of the automaton in a plane. They represent the way the automaton is
placed in it. The tag consequently contains information such as the
coordinates of the states, or the directions and types of the
transitions. The \xtag{drawing} data contain the definition of
attributes that characterize the graphical aspects of the automaton's
elements. Therefore, this tag contains information like the color of
the states or the style of the transitions.

The proposed policy expects these properties to be checked by the
program, and that it is not complicated nor more costly than to test
whether an announced property is indeed fulfilled.

%%%%%%%%%%%%%%%%%%%%%%%%%%%%%%%%%%%%%
\section{Simple examples with default types}
\subsection{A Boolean automaton}

As a first example, the automaton of \autoref{B1} is represented in
\autoref{B1xml}. This Boolean automaton recognizes the set of words
over the alphabet $\{a,b\}$ that contain at least one $b$.

\begin{figure}[ht]
\begin{center}
\VCDraw[1.5]{%
\begin{VCPicture}{(-3,-1.2)(3,.7)}
\State{(-1,0)}{A}
\State{(1,0)}{B}
\Initial{A}\Final{B}
\EdgeL{A}{B}{b}
\LoopN{A}{a}\LoopN{B}{a}
\LoopS{A}{b}\LoopS{B}{b}
\end{VCPicture}}
\end{center}
\vspace*{-.8cm}
\caption{The automaton~$B_1$}\label{B1}
\medskipneg
\end{figure}

\begin{figure}[ht]
  \small
  \begin{center}
\begin{alltt}
<automaton>
  <content>
    <states>
       <state name="s0"/>
       <state name="s1"/>
    </states>
    <transitions>
       <transition \xattr{src}="s0" \xattr{dst}="s0" label="a"/>
       <transition \xattr{src}="s0" \xattr{dst}="s0" label="b"/>
       <transition \xattr{src}="s0" \xattr{dst}="s1" label="b"/>
       <transition \xattr{src}="s1" \xattr{dst}="s1" label="a"/>
       <transition \xattr{src}="s1" \xattr{dst}="s1" label="b"/>
       <initial state="s0"/>
       <final state="s1"/>
    </transitions>
  </content>
</automaton>
\end{alltt}

\caption{The XML description of the automaton $B_1$}
\label{B1xml}
  \end{center}
\end{figure}

\subsection{A Boolean transducer}

The XML proposition can also be used to represent transducers. The example of
\autoref{bindiv3} gives the quotient by 3 of a binary number. It is
represented in \autoref{bindiv3xml}.


\begin{figure}[ht]
  \begin{center}
\VCDraw[1.2]{%
\begin{VCPicture}{(-2,-2)(5,2)}
% states
\State{(0,0)}{A} \State{(3,0)}{B} \State{(6,0)}{C}
\Initial[w]{A}
\Final[s]{A}
% transitions
\LoopN{A}{\IOL{0}{0}}
\LoopN{C}{\IOL{1}{1}}
\ArcL{A}{B}{\IOL{1}{0}}
\ArcL{B}{A}{\IOL{1}{1}}
\ArcL{B}{C}{\IOL{0}{0}}
\ArcL{C}{B}{\IOL{0}{1}}

\end{VCPicture}}
\caption{Transducer T giving the quotient by 3 of a binary number}
\label{bindiv3}
  \end{center}
\end{figure}

\small
\begin{figure}[ht]
  \begin{center}
\begin{alltt}
<transducer>
  <content>
    <states>
       <state name="s0"/>
       <state name="s1"/>
       <state name="s2"/>
    </states>
    <transitions>
       <transition \xattr{src}="s0" \xattr{dst}="s0" in="0" out="0"/>
       <transition \xattr{src}="s0" \xattr{dst}="s1" in="1" out="0"/>
       <transition \xattr{src}="s1" \xattr{dst}="s0" in="1" out="1"/>
       <transition \xattr{src}="s1" \xattr{dst}="s2" in="0" out="0"/>
       <transition \xattr{src}="s2" \xattr{dst}="s2" in="1" out="1"/>
       <transition \xattr{src}="s2" \xattr{dst}="s1" in="0" out="1"/>
       <initial state="s0"/>
       <final state="s0"/>
    </transitions>
  </content>
</transducer>
\end{alltt}

\caption{The XML description of T}
\label{bindiv3xml}
  \end{center}
\end{figure}

\newpage
\subsection{Naive description of the \xtag{content} tag}

The \xtag{content} tag aims at describing the structure of the
automaton. It has two children, both mandatory and expected
in a specific order. These tags enable definitions of states
and transitions.

The first tag, \xtag{states}, introduces the declaration of
the set of the automaton states. A state has three attributes: a
\xattr{name} (which is mandatory and has to be unique), a \xattr{label}
and a \xattr{number}. The latter can be used to put an ordering on states,
or to add special integer data to the state.

The second tag, \xtag{transitions}, introduces the declaration
of the set of transitions. The initial and final
transitions are represented as children of
\xtag{transitions}. Effectively, an initial state $s$ can be seen as a transition
which destination is $s$. This transition can have a \xattr{label} or a
\xattr{number} in some cases, so it seems to be logical to have the list of
initial states in the \xtag{transitions} tag.  Similarly, the final
states can be found at the same place in the XML description.

It is mandatory for a \xtag{transition} to have two attributes:
\xattr{src} and \xattr{dst}, representing source and destination state
names of the
transition. In the case of an \xtag{initial} or \xtag{final} transition,
the only mandatory attribute is \xattr{state}, referring to the initial
or final state the transition belongs to.

There is no limitation of the format for the content of attributes, as
it is a non-restricted string. For example, a user can store a
rational expression in the \xattr{label}, \xattr{in} or \xattr{out}
attributes of a transition. The use of these attributes depends on the
structure of automaton one defines. When omitting them, the XSD
grammar proposes the empty string as the default value.

At this point, most of automata can be easily described. These
examples use only a part of the XML description that we present, but
allow the reader to understand the basis of this format and to easily
deal with a great amount of automata and transducers.

%%%%%%%%%%%%%%%%%%%%%%%%%%%%%%%%%%%%%
\section{Description of the format}

This part describes in details some tags of the XML format.  Firstly,
the tags \xtag{automaton} and \xtag{transducer} will be introduced.
Then, the two main tags, namely \xtag{labelType} and \xtag{content}, will be
described.  Eventually, the main \xtag{session} tag that holds all the
other ones will be presented.

\subsection{The \xtag{automaton} and \xtag{transducer} tags}

As one can see in the previous examples, \autoref{B1xml} and
\autoref{bindiv3xml}, these tags specify the type(s) of the object(s)
contained in the XML session. The content of an object is then
specific, and it is linked to the type of tag that is used.  In any
case, an attribute ``name'', present in these tags, allows to bring an
explicit name to an automaton and to store it in a XML file.

\subsection{The \xtag{labelType} tag}

In many cases, automata are graphs whose transitions are labeled by
symbols called letters, taken in a set called alphabet. In full
generality, this label can be a polynomial, or even a rational series,
over a monoid with coefficients taken in a semiring. The \xtag{labelType}
tag allows to refer to this semiring and this monoid.

In the automaton described in \autoref{B1xml}, no specific information
is given on the type of the automaton. The proposal comes with a set
of predefined types, in order to limit amount of needed declarations
for widely used structures. When the document starts with the
\xtag{automaton} tag and when the \xtag{labelType} tag is omitted, the
default automaton type is a Boolean automaton, on the standard
alphabet (all the letters of the alphabet, including capitalized ones,
and digits).  Concerning the \xtag{transducer} tag without any
\xtag{labelType} tag, the default transducer is Boolean with two monoids
built on the same standard alphabet.  These default types will be
described further in the XML format.


\subsubsection{The \xtag{monoid} tag}

There are cases for which the default alphabet proposed to build the
monoid doesn't fit. For this reason, this tag enables the user to
determine the basic symbols set that she wants to use as an alphabet
in the labels of the transitions.

For instance, in the current state, the automaton of \autoref{B1} is
defined with the default alphabet. It could be better to set an
alphabet that only contains the letters \textit{a} and \textit{b}, so
as to prevent the user from possible errors subsequent to the first
definition. So, the resticted alphabet would be defined as shown in
\autoref{alpha1}.

\begin{figure}[ht]
  \small
  \begin{center}
\begin{alltt}
<labelType>
  <monoid>
     <generator value="a"/>
     <generator value="b"/>
  </monoid>
</labelType>
\end{alltt}

\caption{Setting $\{a, b\}$ alphabet}
\label{alpha1}
  \end{center}
\end{figure}

Similarly, one can also set a restriction on the alphabet like in
\autoref{restriction1}.

\begin{figure}[ht]
  \small
  \begin{center}
\begin{alltt}
<labelType>
  <monoid generators="digits" type="free">
    <generator value="0"/>
    <generator value="1"/>
  </monoid>
</labelType>
\end{alltt}

\caption{Example of a restriction}
\label{restriction1}
  \end{center}
\end{figure}


The user can set the string denoting the empty word with the attribute
\xattr{identitySymbol} of the monoid. This way, the empty word
symbol can always be different from any character of the used
alphabet.

To create a transducer based on a free monoid product, it is necessary
to declare the two monoids in order to determine the two needed
alphabets. A suitable example can be found in the
\autoref{transducertype}. It represents the type of a default
transducer.

The XSD description of this tag is a little complicated. Some elements
allow the proposition to be extensive and to fully describe any monoid
type. This is why the first elements of the \xtag{monoid} tag are a
choice. It can be one or more monoid types to allow some complex
definitions (\autoref{transducertype} for an example), or it can
be one or more generator types to describe the letters composing the
alphabet of the monoid. But both types cannot exist in the same monoid
type description. Concerning transducers, only two monoids can be
defined under the main \xtag{monoid} tag so as to remain a free monoid
product.

The monoid attributes are:
\begin{itemize}
\item \xattr{type}\\
  This is used to set the type of the monoid. Choices are \code{unit},
  \code{free} or \code{product}.
\item \xattr{generators}\\
  This attribute sets a global restriction to the alphabet. The
  current possibilities are letters, digits, pair or weighted.
\item \xattr{identitySymbol}\\
  Used to set an empty word symbol.
\end{itemize}

The XSD description of the \xtag{generator} tag is:
\begin{itemize}
\item \xattr{value}\\
  This is used to add one letter in the alphabet. One can put several
  \xtag{generator} tags in a monoid description so as to have bigger
  alphabets.
\item \xattr{range}\\
  This allows to set a fixed range without being obliged to add all
  the symbols one by one. For instance, the range \textsc{implicitAlphabet}
  sets the alphabet on the lowercase and uppercase alphabet characters.
\end{itemize}

\subsubsection{The \xtag{semiring} tag}

The XML proposition enables a full description of the automaton type.
It consequently proposes a way to write weighted automata or
transducers seen as a weighted automaton with its weights in
$Rat(B^*)$.

To describe a weighted automaton, the \xtag{labelType} tag provides a set of
customizable tags to specify the type of multiplicities. The example
of \autoref{B1Zxml} shows how to turn the automaton $B_1$ into a
weighted automaton with weights in ${\mathbb Z}$ -- so it counts the
number of $b$ in a word.

\begin{figure}[ht]
  \small
  \begin{center}
\begin{alltt}
<automaton>
  \xtag{labelType}
     <semiring set="Z"/>
  </labelType>
  <content>
  ...
  </content>
</automaton>
\end{alltt}

\caption{The XML description of the $\mathbb{Z}$-automaton $B_1$}
\label{B1Zxml}
  \end{center}
\end{figure}

The \xtag{semiring} tag can be described with two attributes:
\begin{itemize}
\item \xattr{set}\\
  The set on which the automaton is built. The possible sets are
  ${\mathbb B}$, ${\mathbb R}$, ${\mathbb Z}$, ${\mathbb N}$ and
  $ratSeries$ (which will be discussed later).
\item \xattr{operations}\\
  The type of operations that can be performed on this set.  The
  possibilities are \textit{numerical}, \textit{boolean},
  \textit{minPlus} or \textit{maxPlus}.
\end{itemize}

For instance, describing the tropical semiring $({\mathbb Z}, max, +)$
is achieved with:
\begin{center}
{\small
\verb|<semiring set="Z" operations="maxPlus">|}
\end{center}

The content definition previously defined in \autoref{B1xml} is
still totally compatible with a weighted automaton, and can remain
unchanged.

Two different ways are proposed to set the weight of a transition. One
can directly store the multiplicity in the \xattr{label} attribute, or
use the dedicated \xattr{weight} attribute. These attributes can
indistinctly be used in a \xtag{transition}, an \xtag{initial} or a
\xtag{final} tag. When omitting the \xattr{weight} attribute, the XSD
grammar proposes the identity of the semiring as the default value.

\medskip

The \xtag{semiring} tag proposes some solutions for the transducers,
like the example of \autoref{ratseries1}. It is seen as a weighted automaton
with weight in $Rat(B^*)$. \xtag{monoid} and \xtag{semiring} tags can
recursively be defined, in order to describe a complex type. Only the
\xtag{labelType} is shown so as to remain clear.

\begin{figure}[ht]
  \begin{center}
\begin{alltt}
<transducer>
  <labelType>
    <monoid generators="digits" type="free">
      <generator value="0"/>
      <generator value="1"/>
      <generator value="2"/>
    </monoid>
    <semiring set="ratSeries">
      <monoid generators="digits" type="free">
        <generator value="0"/>
        <generator value="1"/>
      </monoid>
      <semiring operations="numerical" set="B"/>
    </semiring>
  </labelType>
  <content>
    ...
  </content>
<transducer>
\end{alltt}
\caption{Right transducer for binary addition}
\label{ratseries1}
  \end{center}
\end{figure}

The beginning of the XSD description of the \xtag{semiring} is a sequence that
contains two elements, namely the monoid and the semiring. This allows a
recursion in the definition of an automaton structure.

\subsection{The \xtag{content} tag}

For automata and transducers, the \xtag{content} tag has the same
structure.  The following is the description of this tag. Bold
elements are mandatory.  The special tags \xtag{geometry} and
\xtag{drawing} can be at any place of the document. They are ignored
here, but more information can be found in
\autoref{title_vizualisation}.

\medskip

The first tag, \textbf{\xtag{states}}, gives the possibility to fully
describe the states of an automaton and their content. The
\xtag{states} tag is composed of:
\begin{itemize}
\item \xtag{state}\\
  This tag represents one state. There must be as many of these tags
  as there are states in the automaton. It has the following
  elements and attributes:
  \begin{itemize}
  \item \textbf{\xattr{name}}\\
    The name to the state.
  \item \xattr{label}\\
    Optional label for a state.
  \item \xattr{number}\\
    Used if one wants to set an order on the states.~\\
  \end{itemize}
\end{itemize}

With the second tag, \textbf{\xtag{transitions}}, one can describe the
transitions. It is composed of:
\begin{itemize}
\item \xtag{transition}\\
  This tag describes the content of one transition of the automaton.
  It is composed of:
  \begin{itemize}
  \item \xattr{\textbf{src}}\\
    The state that is the source of the transition.
  \item \xattr{\textbf{dst}}\\
    The state that is the destination of the transition.
  \item \xattr{name}
  \item \xattr{label}
  \item \xattr{weight}\\
    Optional name, label and weight for the transition.
  \end{itemize}
\item \xtag{initial}\\
  This tag describes the content of an initial state of the automaton:
  \begin{itemize}
  \item \xattr{\textbf{state}}\\
    The state that is initial.
  \item \xattr{label}
  \item \xattr{weight}\\
    Optional label and weigth for the entering transition.
  \end{itemize}
\item \xtag{final}\\
  This tag describes the content of a final state of the automaton.
  Its strucure is the same as the \xtag{initial} tag's one.
\end{itemize}

Concerning transducers, the \xtag{content} tag follows the same
structure as for automata description, although a noticeable
difference is the extension for transition definitions. Two new
attributes are proposed for transducer description: \xattr{in} and
\xattr{out}, respectively corresponding to the input and the output of
a transition. These two attributes are proposed in addition to the
classical \xattr{label} and \xattr{weight} attributes, that can still be
used for transducer description. Of course, they can be used
indistinctly in \xtag{transition}, \xtag{initial} or \xtag{final}.

\subsection{Dealing with labels}
\subsubsection{About the label attributes}

A label as an attribute of a transition or a state can be useful. It can keep an
XML description of an automata human-readable and more simple to write. But this
system comes with its limitations. Written as a string, a label automatically
comes with a grammar to parse its content. Consequenlty, using attributes should
only be done with very simple labels. If not, two programs using different
grammars would surely be unable to parse a complicated label the same way.
To avoid such errors, a label node has been defined.

\subsubsection{The \xtag{label} tag}

This proposal now possesses a strict definition for labels. The \xtag{transition}
and \xtag{state} tags now have an inner element named \xtag{label} that enables to
define the content of transitions or states. Its structure represents the
grammar of the regular expressions that the label node can contain.

A label element can contain one of the following ones:
\begin{itemize}
\item \xtag{star}\\
  This tag is used to ``starify'' the expression inside it.
\item \xtag{sum}\\
  This tag is used if the label to be defined is the union of two
  expressions. For this reason, it can contain two tags amongst those that
  define the label node.
\item \xtag{product}\\
  This tag has the same properties as the \xtag{sum} one. It is used if the label
  is a product of expressions, \textit{i.e} their concatenation. It is
  represented by ``.'' in the explanation of the use examples.
\item \xtag{word}\\
  First of the three possible terminal elements of the label grammar, this tag
  is used to define a word. It has a mandatory attribute \xattr{value} that must
  be valid elements taken from the automaton alphabet.
\item \xtag{zero}\\
  This tag represents the zero word. The result of an evaluation on such an
  element will always return false.
\item \xtag{identity}\\
  This tag represents the empty word. It is similar to a label with an empty
  string value, or with a value equal to the optional \xattr{identitySymbol}
  attribute of the \xtag{monoid} node.
\end{itemize}

A \xattr{weight} attribute can be added to each of these adds. It enables the
user to set a weight to any element of the expression.

The XSD structure of the label has been designed so as to follow the grammar of
the expected expressions. One can easily notice that this definition allows to
define a complicated expression, recursively using the different tags.

In figure \autoref{labeltags}, some examples of use of the \xtag{label} tag are
shown. It is easy to understand the label values of the three examples, namely
\textit{a}, \textit{(3 a+b)*} and \textit{a+(b.(a.c))}.

\begin{figure}[htp]
  \begin{center}
\begin{alltt}
<label>
  <word value="a"/>
</label>

<label>
  <star>
    <sum>
      <word value="a" weight="3"/>
      <word value="b"/>
    </sum>
  </star>
</label>

<label>
  <sum>
    <word value="a"/>
    <product>
      <word value="b"/>
      <product>
        <word value="a"/>
        <word value="c"/>
      </product>
    </product>
  </sum>
</label>
\end{alltt}

\caption{Examples of use of the \xtag{label} tag}
\label{labeltags}
  \end{center}
\end{figure}

as for the \xattr{label} attribute, the same system has been enabled for the
\xattr{in}and \xattr{out} ones. Indeed, one can define the input and the output
of a transition using the \xtag{in} and \xtag{out} tags.
If the labels are described this way in an XML file, one can now be sure that
the labels will be perfectly understood, whatever the program that deals with.



\subsection{The \xtag{session} tag}

A way to manipulate many automata would be to combine them in a single
document. The proposal offers this feature through the \xtag{session}
tag. An unlimited number of automata or transducers can be combined in
a single XML document. A XML session can also be named.  An
application can be found in \autoref{session1}.

\begin{figure}[htp]
  \small
  \begin{center}
\begin{alltt}
<session name="session1">
  <automaton name="a1">...</automaton>
  <transducer name="t1">...</transducer>
  <transducer name="t2">...</transducer>
</session>
\end{alltt}

\caption{A session with several automata}
\label{session1}
  \end{center}
\end{figure}

\subsection{The cascade of default options}

The \xtag{labelType} tag has two children: the \xtag{monoid} tag and the
\xtag{semiring} tag. None of these tags is mandatory, and both have
different values according to the root tag. \autoref{automatontype}
shows the equivalent XML code if one omits the \xtag{labelType} tag when
declaring an automaton. Similarly, \autoref{transducertype} shows the
default type for transducers.  The \xattr{operations} attribute is
set to \textit{numerical}, which means that usual laws over
$\mathbb{B}$ shall be applied.

\begin{figure}[htp]
  \begin{center}
\begin{alltt}
<labelType>
  <monoid type="free" generators="letters">
     <generator range="implicitAlphabet"/>
  </monoid>
  <semiring set="B" operations="numerical"/>
</labelType>
\end{alltt}

\caption{Default type for an automaton}
\label{automatontype}
  \end{center}
\end{figure}


\begin{figure}[htp]
  \begin{center}
\begin{alltt}
<labelType>
  <monoid type="product">
     <monoid type="free" generators="letters">
       <generator range="implicitAlphabet"/>
     </monoid>
     <monoid type="free" generators="letters">
       <generator range="implicitAlphabet"/>
     </monoid>
  </monoid>
  <semiring set="B" operations="numerical"/>
</labelType>
\end{alltt}

\caption{Default type for a transducer}
\label{transducertype}
  \end{center}
\end{figure}

%%%%%%%%%%%%%%%%%%%%%%%%%%%%%%%%%%%%%
\newpage
\section{The visualization tags}
\label{title_vizualisation}
The visual representation of automata involves a very large amount of
information. On the one hand, the \xtag{geometry} is context sensitive
with data such as state coordinates or transition type. It gives some
information about the way the automata are set in the plane only.  On
the other hand, the \xtag{drawing} tag gives some graphical information
about the way the automata of a session must be drawn.

Let us note that these tags can be used at any level of the document.
In this case, the defined properties are applied to the tag in which
they are defined, and to its children. It is possible to define some
properties in a tag and to locally override them in a child tag. For
example in \autoref{override_properties}, the filling color of the
states is globally set to \textit{black}, and the color of state $s_1$
is locally set to red. As a result, $s_0$ and $s_2$ will be black, and
$s_1$ will be red.

\begin{figure}[htp]
  \small
  \begin{center}
\begin{alltt}
<transducer>
  <drawing stateFillColor="black"/>
  <content>
     <states>
        <state name="s0"/>
        <state name="s1">
            <drawing stateFillColor="red"/>
        </state>
        <state name="s2"/>
      ...
</transducer>
\end{alltt}

    \caption{Example of overriding drawing properties}
    \label{override_properties}
  \end{center}
\end{figure}

\subsection{The \xtag{geometry} tag}

The \xtag{geometry} tag is context sensitive. If it is a child of the
\xtag{state} tag, the only two properties that can be set are the
position, \textit{x} and \textit{y}, of the state. These values can only
be numeric.

If it is a child of \xtag{transition}, \xtag{initial} or \xtag{final},
two attributes can be set. Firstly, the \xattr{transitionType}
attribute, that assigns the type of the transition (\textit{line},
\textit{arcL}, \textit{arcR}, \textit{curve}). Then, the
\xattr{direction} attribute, that can be used to assign the direction
angle of a loop, for instance. This attribute is numeric.

The example of \autoref{geom1} sets a global offset for the document,
and then places the states in the plane.  It also sets the type of the
transition as a left arc.

\begin{figure}[htp]
  \small
  \begin{center}
\begin{alltt}
<automaton>
  <geometry x="-2" y="-2"/>
  <content>
     <states>
        <state name="s0"><geometry x="0" y="0"/>
        </state>
        <state name="s1"><geometry x="3" y="0"/>
        </state>
     <transitions>
        <transition \xattr{src}="s0 \xattr{dst}="s1" label="a">
          <geometry transitionType="arcL">
        </transition>
     </transitions>
</automaton>
\end{alltt}

\caption{Setting geometry properties}
\label{geom1}
  \end{center}
\end{figure}


\subsection{The \xtag{drawing} tag}

The \xtag{drawing} tag contains the definition of attributes that
characterize the actual drawing of the graph.  Most of them are indeed
implicit and provided by drawing programs; the format only provides
the possibility to make them explicit.

Since it's not possible to exhaustively name all needed attributes
users may need, the proposal offers a very limited set of properties. For
example, \xattr{stateFillColor} or \xattr{transitionLabelColor} usage are
shown in \autoref{drawing1}. These attributes use a string representation to
describe their values. Providing more attributes will be one of our future
works.

One of the powerful features of XSD files is the \textit{anyAttribute}
modifier. This modifier allows the user to easily extend the main XSD,
and then use its own attributes and still be compliant with the
grammar. The \xtag{drawing} tag contains a \textit{anyAttribute}
modifier in the proposal, so the grammar is not limited to a specific
set of drawing properties.

\begin{figure}[ht]
  \small
  \begin{center}
\begin{alltt}
<transducer>
  <geometry x="-5" y="0"/>
  <drawing stateFillColor="black" transitionLabelColor="dashed"/>
  <content>
     <states>
        <state name="s0">
            <drawing stateFillColor="red"/>
        </state>
      ...
</transducer>
\end{alltt}

\caption{Setting drawing properties}
\label{drawing1}
  \end{center}
\end{figure}


%%%%%%%%%%%%%%%%%%%%%%%%%%%%%%%%%%%%%
\section{Using an XSD file}

Some specifications of the XML format have led us to describe it using an XSD
(XML Schema Description) Schema.

It is desirable to keep the description of automata simple when
describing widely used structures while giving the possibility to
describe the most complex ones. For XML, this simplification enables
to have default types, in order to omit \xtag{labelType} tag when describing
common Boolean automata or transducers.

The problem then arises when describing an automaton or a transducer,
the default values for the \xtag{labelType} tag must of course be different.
This is not possible with a DTD description.  The use of an XSD
overcomes this difficulty, since it is possible to define different
properties for a same element, according to the embracing context. It
is so possible to locally change the behavior of a tag, and make it
context-sensitive. With this feature, default values for the
\xtag{labelType} tag are achieved, whether it is a child of
\xtag{transducer} or \xtag{automaton}.

For information about the XSD Schema are available on the
\textit{World Wide Web Consortium} website \cite{w3c.www}.

%%%%%%%%%%%%%%%%%%%%%%%%%%%%%%%%%%%%%%%%%
\section{Conclusion}

For the past years we experimented the different proposals in the
\Vauc platform. It begun with a first one in 2004 \cite{vaucanson.04.techrep}
and we presented our progress at CIAA'05 \cite{claveirole.05.ciaa}.
This version 0.3.2 comes as a result of this experiment, with
simplifications where possible. Thus, the \Vauc platform deals
with numerous automata types, and it is important to be able to define
precisely the type of the automaton in addition to its content.

This proposal comes as a combination of two needs, shorten declaration
of widely used structure and make possible definitions of complex
types. We hope to have proposed a description format that fulfills, at
least partially, both needs.

\nocite{lombardy.04.tcs}
\nocite{claveirole.05.ciaa}
\nocite{vaucanson.04.techrep}
\nocite{vaucanson.www.xml}
\nocite{vaucanson.www.xsd}
\nocite{w3c.www}


\bibliographystyle{apalike}
\bibliography{lrde,vaucanson}


\end{document}
