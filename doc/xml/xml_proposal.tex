\documentclass[a4paper]{article}
\pagestyle{plain}
\usepackage{amsmath,amsfonts,amscd,amssymb}
\usepackage{alltt}
\usepackage{xspace}
\usepackage{vaucanson-g}
%%%%%%%%%%%%%%%%%%%%%%%%%%%%%%%%%%%%%%%%%%%



%%%%%%%%%%%%%%%%%%%%%%%%%%%%%%%%%%%%%%%%
\newcommand{\bigskipneg}{\vspace*{-5ex}} %960728
\newcommand{\medskipneg}{\vspace*{-2ex}} %960728
\newcommand{\madskipneg}{\vspace*{-1.5ex}} %960728
\newcommand{\smallskipneg}{\vspace*{-1ex}} %960728
\newcommand{\miniskipneg}{\vspace*{-0.25ex}} %960728
\newcommand{\skipneg}{\vspace*{-0.5cm}}



\newcommand{\Vauc}{{\sc Vaucanson}\xspace}

\newcommand{\typetag}{\verb|<type>|\xpsace}
\newcommand{\contenttag}{\verb|<content>|\xpsace}
\newcommand{\statestag}{\verb|<states>|\xpsace}
\newcommand{\transitionstag}{\verb|<transitions>|\xpsace}


\begin{document}

\title{XML proposal for automata description}

\author{The \Vauc group}

% \institute{http://vaucanson.lrde.epita.fr}
%%%

\maketitle{}

\begin{abstract}
This paper present an XML description format for automata
representation. We introduce the proposal through some examples that
enlight characteristic features of the format, with a progressive
complexity. Finally, we focus briefly on implementation concerns.
\end{abstract}


%%
\section{Introduction}

The aim of conceiving a universal automata exchange format is to
provide the community with a communication tool that could be used
for the connexion of the various softwares that deal with
automata and transducers.

The idea of establishing an XML format for automata has been
discussed at the CIAA conferences for several years.
At CIAA'04, a special session was organized on the subject and two
proposals were presented, one by our group (see~\cite{VXML04}).
We come again at CIAA'05 with a new proposal, which is an evolution of
the former but has undergone profound modifications based on our
experience in using this format as an input-output format
for the \Vauc platform.

The most important difference with our previous proposal is the change
from a DTD (Document Type Definition) to an XSD Schema for the
description of the format. 
We explain later the reason for this change.


%%
\section{Overview}

The description of automata is structured in two parts.  The
\verb|<type>| tag provides automaton type definition, like Boolean
automaton, or weighted ones with the ability to specify weight type,
alphabet specification, etc. The \verb|<content>| tag provides the
definition of the automaton ``structure''.\\

The visual representation of automata involves a very large amount of
informations.  The \verb|<geometry>| data corresponds to the embedding
of the automaton in a plane (with informations such as state
coordinates or edge type for a transition).  The \verb|<drawing>| data
contains the definition of attributes that characterize the actual
drawing of the graph (such as label position or state color for
instance). \\

\section{Description of the format}
\subsubsection{A first example}


%% The \contenttag aims to describe the structure of the automaton. An
%% automaton is a set of states, where some can be initials or finals,
%% and transitions; so the \contenttag has two children: \statestag and
%% \transitions.

As a first example, the automaton of Figure \ref{B1} is represented in
Figure \ref{B1xml}. This automaton recognizes the set of
words over the alphabet $\{a,b\}$ that contains at least one $b$.\\

\begin{figure}[ht]
\begin{center}
\VCDraw[1.5]{%
\begin{VCPicture}{(-3,-1.2)(3,.7)}
\State{(-1,0)}{A}
\State{(1,0)}{B}
\Initial{A}\Final{B}
\EdgeL{A}{B}{b}
\LoopN{A}{a}\LoopN{B}{a}
\LoopS{A}{b}\LoopS{B}{b}
\end{VCPicture}}
\end{center}
\vspace*{-.8cm}
\caption{The automaton~$B_1$}\label{B1}
\medskipneg
\end{figure}

{\small

\begin{figure}[h]
  \begin{center}
\begin{alltt}
<automaton>
  <content>
    <states>
       <state name="s0"/>
       <state name="s1"/>
    </states>
    <transitions>
       <transition src="s0" dst="s0" label="a"/>
       <transition src="s0" dst="s0" label="b"/>
       <transition src="s0" dst="s1" label="b"/>
       <transition src="s0" dst="s1" label="a"/>
       <transition src="s0" dst="s1" label="b"/>
       <initial state="s0"/>
       <final state="s1"/>
    </transitions>
    </finals>
  </content>
</automaton>
\end{alltt} 

\caption{The XML description of the automaton $B_1$}
\label{B1xml}
  \end{center}
\end{figure}
}

\subsection{The content tag}
The \verb|<content>| tag aims to describe the structure of the
automaton. It has two children, mandatory and supposed to appear in a
specific order. These two tags allow definitions of states and
transitions. \\

The first tag is \verb|<states>|, representing start declaration of
the set of states of the automaton. A state has three attributes: a
\verb|name| (which is mandatory and has to be unique), a \verb|label|
and a \verb|number|. The latter can be used to put an ordering on states,
or to add special data to the state.

The second tag is \verb|<transitions>|, representing start declaration
of the set of transitions. Let us note that the initial and final
\textit{transitions} are represented as children of
\verb|<transitions>|. It is mandatory for a \verb|<transition>| to
have two attributes: \verb|src| and \verb|dst|, representing source
and destination of the transition. In the case of an \verb|<initial>|
or \verb|<final>| transition, the only mandatory attribute is
\verb|state|, referring to
the initial or final state the transition belongs.\\

Let us note that there is no limitation of the format for the content
of attributes, as it is a non-restricted string. For example, a user
can store a rational expression in the label.  Let us note also that
when omiting the \verb|label| attribute the XSD grammar propose the
identity of the monoid ({\it i.e.} the empty word) as the default value.

\subsection{The type tag}
In the automaton described in Figure \ref{B1xml}, no specific
information is given on the type of the automaton. The proposal comes
with a set of predefined types, in order to limit amount of needed
declarations for widely used structures. When the document starts with
the \verb|<automaton>| tag and when the \verb|<type>| tag is omitted,
the default automaton type is Boolean automaton, on the standard
alphabet (all letters of the alphabet, capitalized or not, and digits).


\subsubsection{Weighted automata}
\label{weightedautomata}

To describe a weighted automaton, the \verb|<type>| tag provides a set
of customizable tags to specify the type of multiplicities. The example
of Figure \ref{B1Zxml} shows how to turn the automaton $B_1$ into a
weighted automaton with weight in ${\mathbb Z}$ -- so it counts the
number of $b$ in a word.

{\small

\begin{figure}[h]
  \begin{center}
\begin{verbatim}
<automaton>
  <type>
     <semiring set="Z"/>
  </type>
  <content>
  ...
  </content>
</automaton>
\end{verbatim}

\caption{The XML description of the $\mathbb{Z}$-automaton $B_1$}
\label{B1Zxml}
  \end{center}
\end{figure}

}

The multiplicity semiring can be described with two attributes. The
\verb|set| indicates the set of weights, while the \verb|operations|
attributes indicates the corresponding operations. The possible sets
are ${\mathbb B}$, ${\mathbb R}$, ${\mathbb Z}$, ${\mathbb N}$ and
$ratSeries$ (which will be discussed later). 

For instance, describing the tropical semiring $({\mathbb Z}, max, +)$
is achieved with: {\small
\verb|<semiring set="Z" operations="tropicalMax">|}\\


All the content definition previously defined in Figure \ref{B1xml} is
still totally compatible with a weighted automaton, and can remain
unchanged. \\

Two different ways are proposed to set the weight of an edge. One can
directly store the multiplicity in the \verb|label| attribute, or use
the dedicated \verb|weight| attribute. These attributes can
indistinctly be used in a \verb|<transition>|, an
\verb|<initial>| or a \verb|<final>| tag. When omiting the
\verb|weight| attribute, the XSD grammar propose the identity of
the semiring as default value.

%
\subsubsection{Transducers}

As already mentioned above, this proposal aims to limit amount of
declarations for widely used structures. Description of transducers is
now achieved through the \verb|<transducer>| tag. The example of
Figure \ref{binadd}, the right transducer for binary addition, is
represented in Figure \ref{binaddxml}. 

\begin{figure}[h]
  \begin{center}
\ShowFrame
\ShowGrid
\VCDraw[1.2]{%
\begin{VCPicture}{(-2,-2)(5,2)}
% states
\State[C]{(0,0)}{A} \State[N]{(3,0)}{B}
\Initial[w]{A}
\Final[se]{A} \FinalR{e}{B}{\IOL{}{1}}
%transitions
\LoopN{A}{\IOL{0}{0}}
\LoopS{A}{\IOL{1}{1}}
\LoopN[.75]{B}{\IOL{2}{1}}
\LoopS[.75]{B}{\IOL{1}{0}}
\ArcL{A}{B}{\IOL{2}{0}}
\ArcL{B}{A}{\IOL{0}{1}}
%
\end{VCPicture}}
\caption{Right transducer for binary addition}
\label{binadd}
  \end{center}
\end{figure}

{\small
\begin{figure}[h]
  \begin{center}
\begin{alltt}
<transducer>
  <content>
    <states>
       <state name="s0" label="C"/>
       <state name="s1" label="N"/>
    </states>
    <transitions>
       <transition src="s0" dst="s0" in="0" out="0"/>
       <transition src="s0" dst="s0" in="1" out="1"/>
       <transition src="s0" dst="s1" in="2" out="0"/>
       <transition src="s1" dst="s0" in="0" out="1"/>
       <transition src="s1" dst="s1" in="1" out="0"/>
       <transition src="s1" dst="s1" in="2" out="1"/>
       <initial state="s0"/>
       <final state="s0"/>
       <final state="s1" out="1"/>
    </transitions>
  </content>
</transducer>
\end{alltt}

\caption{The XML description of the right transducer for binary addition}
\label{binaddxml}
  \end{center}
\end{figure}

}

The \verb|<content>| tag follows the exact same structure as for automata
description. Although a noticeable difference is the extension for
transitions definitions. Two new attributes are proposed for transducer
description: \verb|in| and \verb|out|, respectively corresponding to
the input and the output of a transition. These two attributes are
proposed in addition to the classical \verb|label| and \verb|weight|
attributes, that can still be used for transducer description. They of
course can be used indistinctly in \verb|<transition>|,
\verb|<initial>| or \verb|<final>|. \\


In the example of Figure \ref{binaddxml}, the \verb|<type>| tag is
omitted. The XSD grammar propose also a default type for transducer:
automaton over the free monoid product.
% The two alphabets of free
%monoid are standard alphabets (all letters of the alphabet,
%capitalized or not, and digits).

\subsection{The geometry tag}

The visual representation of automata involves a very large amount of
informations.  The \verb|<geometry>| data corresponds to the embedding
of the automaton in a plane (with informations such as state
coordinates or edge type for a transition). The
format provides the possibility to set these properties at any level of the
document and to locally override them in a child tag.

The example of Figure \ref{geom1} sets a global offset
for the document, and then places a state in the plane.
{\small

\begin{figure}[h]
  \begin{center}
\begin{alltt}
<transducer>
  <geometry x="-2" y="-2"/> 
  <content>
     <states>
        <state name="s0" label="C"><geometry x="0" y="0"/>
        </state>
        <state name="s0" label="N"><geometry x="3" y="0"/>
        </state>
      ...
</transducer>
\end{alltt}

\caption{Setting geometry properties}
\label{geom1}
  \end{center}
\end{figure}

}

The \verb|<geometry>| tag is context sensitive. If it is a child of
the \verb|<state>| tag, the only two properties that can be set is the
position, \verb|x| and \verb|y|, of the state. Note that these values
can only be numeric.

If it is a child of \verb|<transition>|, \verb|<initial>| or
\verb|<final>|, two attributes can be set. First, the \verb|edgeType|
attribute, that assign the type of the edge (\textit{line},
\textit{arcL}, \textit{arcR}, \textit{curve}). Then, the
\verb|direction| attribute, that can be used to assign the direction
angle of a loop, for instance. Note that this attribute is numeric.


\subsection{The drawing tag}
The \verb|<drawing>| data
contains the definition of attributes that characterize the actual
drawing of the graph (such as label position or state color for
instance).
\\Most of them are indeed implicit and provided by drawing programs; the
format only provides the possibility to make them explicit.
As the geometry tag, this tag can be used at any level of the
document and be locally overridden in a child tag.

Since it's not possible to exhaustively
name all needed attributes users may need, the proposal offers a
limited set of properties. For example, \verb|stateFillColor| or
\verb|edgeStyle| usage are shown in Figure \ref{drawing1}. These
attributes use a string representation to describe their values.

One of the powerful features of XSD Schema descriptions is the
\verb|anyAttribute| modifier. This modifier allows the user to easily
extend the main XSD, and then use its own attributes and still be
compliant with the grammar. The \verb|<drawing>| tag contains a
\verb|anyAttribute| modifier in the proposal, so the grammar is not
limited to a specific set of drawing properties.


{\small

\begin{figure}[h]
  \begin{center}
\begin{alltt}
<transducer>
  <geometry x="-5" y="0"/> 
  <drawing stateFillColor="black" edgeStyle="dashed"/>
  <content>
     <states>
        <state name="s0">
            <drawing stateFillColor="red"/>
        </state>
      ...
</transducer>
\end{alltt}

\caption{Setting drawing properties}
\label{drawing1}
  \end{center}
\end{figure}

}



\section{Discussion}

\subsection{The complexity of the {\tt type} tag}
\label{deepertypetag}

In section \ref{weightedautomata}, we briefly introduced how the
\verb|<type>| tag can be used to specify weight types for a weighted
automaton. The aim of the \verb|<type>| tag is to provide a set of
tags that allows full description of the automaton type.


\subsubsection{Alphabet specification}

The user may need to use an alphabet that is not necessarily the
standard letter alphabet. For example, a restriction of the alphabet
to $\{a, b\}$ is proposed in Figure \ref{alpha1}.


{\small

\begin{figure}[h]
  \begin{center}
\begin{alltt}
<type>
  <monoid>
     <generator value="a"/>
     <generator value="b"/>
  </monoid>
</type>
\end{alltt}

\caption{Setting $\{a, b\}$ alphabet}
\label{alpha1}
  \end{center}
\end{figure}

}


\subsubsection{Default types}

The \verb|<type>| tag has two children: the \verb|<monoid>| tag and
the \verb|<semiring>| tag. Note that both of these tags are not
mandatory, and have different values according to the root tag. Figure
\ref{automatontype} shows the equivalent XML code if you omit the
\verb|<type>| tag when declaring an automaton. Similarly,
Figure~\ref{transducertype} shows the default type for transducers.
The \verb|operations| attributes is setted to \verb|"numerical"|,
which means that usual laws over $\mathbb{B}$ shall be applied.

\begin{figure}[h]
  \begin{center}
\begin{alltt}
<type>
  <monoid type="free" generators="letters">
     <generator range="ascii"/>
  </monoid>
  <semiring set="B" operations="numerical"/>
</type>
\end{alltt}

\caption{Default type for an automaton}
\label{automatontype}
  \end{center}
\end{figure}


\begin{figure}[h]
  \begin{center}
\begin{alltt}
<type>
  <monoid type="product">
     <monoid type="free" generators="letters">
       <generator range="ascii"/>
     </monoid>
     <monoid type="free" generators="letters">
       <generator range="ascii"/>
     </monoid>
  </monoid>
  <semiring set="B" operations="numerical"/>
</type>
\end{alltt}

\caption{Default type for a transducer}
\label{transducertype}
  \end{center}
\end{figure}



\subsubsection{Advanced example}

The power of the type tag is enlighted with the example of Figure
\ref{ratseries1}. This example describes the right transducer for
binary addition (Figure~\ref{binadd}) seen as a weighted automaton with
weight in $Rat(B^*)$. \verb|<monoid>| and \verb|<semiring>| tags can
recursively be defined, in order to describe a complex type. For the
sake of space saving, the content part is omitted, but is
\textit{verbatim} the one proposed in Figure \ref{binaddxml}.


\begin{figure}[h]
  \begin{center}
\begin{alltt}
<transducer>
  <type>
    <monoid generators="integers" type="free">
      <generator value="0"/>
      <generator value="1"/>
      <generator value="2"/>
    </monoid>
    <semiring set="ratSeries">
      <monoid generators="integers" type="free">
        <generator value="0"/>
        <generator value="1"/>
      </monoid>
      <semiring operations="numerical" set="B"/>
    </semiring>
  </type>
  <content>
     ...
  </content>
<transducer>
\end{alltt}

\caption{Right transducer for binary addition}
\label{ratseries1}
  \end{center}
\end{figure}


\subsection{From DTD to XSD}
The most important difference with our previous proposal is the change
from a DTD (Document Type Definition) describing the tags for automata
representation to an XSD Schema.

It is desirable to keep the description of automata simple when
describing widely used structures while, giving the possibility to
describe the most complex ones. For XML, this simplification amounts
to have default types, in order to omit \verb|<type>| tag when
describing common Boolean automata or transducers.

The problem then arises when describing an automaton or a transducer,
the default values for the \verb|<type>| tag must of course be
different. This is not possible with a DTD description.  The use of a
XSD overcomes this difficulty, since it is possible to define
different properties for a same element, according to the embracing
context. Is is so possible to locally alter the behavior of a tag, and
make it context-sensitive. With this feature, default values for the
\verb|<type>| tag are achieved, whether it is a child of
\verb|<transducer>| or of \verb|<automaton>|.


\subsection{Convenient}

\subsubsection{Sessions}

A way to manipulate many automata would be to combine them in
a single document. The proposal offers this feature, through the
\verb|<session>| tag. An unlimited number of automata or transducers
can be combined in a single XML document, as shown in Figure~\ref{session1}.



{\small

\begin{figure}[h]
  \begin{center}
\begin{alltt}
<session>
  <automaton name="a1">...</automaton>
  <transducer name="t1">...</transducer>
  <transducer name="t2">...</transducer>
</session>
\end{alltt}

\caption{Session of numerous automata}
\label{session1}
  \end{center}
\end{figure}

}


%% %%
%% \section{Advanced examples}

%% \subsection{Automaton description}

%% \subsection{Transducer description}



%%
%% \section{Implementation concerns}

%% This proposal is the result of a one year experiment of the former
%% proposal we made at CIAA'04. We had to overcome two major issues:
%% describe an XML grammar which is context-sensitive, and provide in
%% \Vauc a corresponding implementation that would let us to easily extend
%% or modify the grammar.






%% \subsubsection{Implementation}
%% To parse the XML document and create the associated tree, the Apache
%% Xerces C++ parser \cite{xerces} is appropriate. Xerces is a validating
%% XML parser, and well handles DTD document validation or XSD
%% validation.

%% The factory method design patterns fulfills our needs of modularity
%% and adaptability of the implementation. This design pattern, described
%% in \cite{gof}, is well adapted to dynamic content such as XML
%% document. We provide an implementation of the format proposal in \Vauc
%% that relies on this pattern, and it proved its flexibility numerous
%% times.



%%
\section{Conclusion}

For the past year we experimented the proposal made at CIAA'04 in the
\Vauc platform. This new proposal comes as a result of this
experiment, with simplifications where it was possible. Thus, the
\Vauc platform deals with numerous automata types, and it is important
to be able to define precisely the type of the automaton in addition
to its content.

This proposal comes as a combination of two needs, shorten declaration
of widely used structure and make possible definitions of complex
types. We hope to have proposed a description format that fulfills, at
least partially, both of these needs.




{\small%
\begin{thebibliography}{1}

\bibitem{gof}
{\sc Gamma E., Helm R., Johnson R., and Vlissides J.},
\newblock {\em Design Patterns: Elements of Reusable Object-Oriented 
Software},
\newblock Addison-Wesley, 1995.

\bibitem{LombReGiSaka04}
{\sc Lombardy S., R\'egis-Gianas Y., and Sakarovitch J.},
\newblock Introducing Vaucanson
\newblock {\em Theoretical Comput. Sci. 328\/} (2004), 77--96.
\newblock Journal version of 
{\em Proc. of CIAA 2003, Lect. Notes in Comp. Sc. 2759}, (2003), 
96--107 
\newblock (with {\sc R. Poss}).

\bibitem{VXML04}
{\sc Claveirole T.},
\newblock Proposal: an XML representation for automata,
\newblock Technical report, LRDE (2004).


\bibitem{xerces}
\verb+http://xml.apache.org/xerces-c/+
\end{thebibliography}}


\end{document}
