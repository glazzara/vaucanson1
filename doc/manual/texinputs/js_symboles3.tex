% -*- coding: mac-roman -*-
%%%%%%%%%%%%%%%%%%%%%%%%%%%%%%%%%%%%%%%%%%%%%%%%%%%%%%%%%%%%%
%                                                           %
%                  js_symboles.tex                         %
%
%       Fichier general de symboles                         %
%                                                           %
%   (doit etre complete par un fichier general de           %
%               commandes et de macros                      %
%   et par fichier particulier pour chaque article)         %
%                                                           %
%                                                           %
%%%%%%%%%%%%%%%%%%%%%%%%%%%%%%%%%%%%%%%%%%%%%%%%%%%%%%%%%%%%%
%
%
%     Symboles mathematiques   abreviations de LaTeX
%
\newcommand{\fa}{\forall}
\newcommand{\ext}{\exists}
\newcommand{\extuni}{\ext \, \mathbf{!} \, } % il existe un unique 990926
\newcommand{\es}{\emptyset}
\newcommand{\ORA}{\overrightarrow}
\newcommand{\OA}{\overrightarrow} % pour compatibilit\'e
\newcommand{\OLA}{\overleftarrow}
\newcommand{\OL}{\overline}
\newcommand{\UL}{\underline}
\newcommand{\ULS}[1]{\underline{\rule[-.4ex]{0pt}{1ex} #1}}% 000426
\newcommand{\WH}{\widehat}
\newcommand{\WT}{\widetilde}
\newcommand{\UB}{\underbrace}
\newcommand{\OB}{\overbrace}
\newcommand{\bk}{\mathrel{\backslash }}
\newcommand{\sm}{\setminus}
\newcommand{\mto}{\mapsto}
\newcommand{\lmto}{\longmapsto}
%%%%%% \section{mep formules}
%       commandes de mise en page des formules
\newcommand{\e}{\text{\quad}}                 % un moins petit espace
\newcommand{\ee}{\text{\qquad}}               % un espace
\newcommand{\eee}{\text{\qquad \qquad}} % et un grand
%%% nouvelle programmation des espaces mode math 021009
\newsavebox{\InterSymbolSpace}
\savebox{\InterSymbolSpace}{\hspace{0.125em}}
\newsavebox{\SideFormulaSpace}
\savebox{\SideFormulaSpace}{\hspace{0.2em}}
\newcommand{\msp}{\usebox{\SideFormulaSpace}} % espace pour faire ressortir
\newcommand{\xmd}{\usebox{\InterSymbolSpace}} % espace entre les symboles
% ponctuation (dependra peut-etre de la langue)
\newcommand{\eqpnt}{\makebox[0pt][l]{\: .}}
\newcommand{\eqpntvrg}{\makebox[0pt][l]{\: ;}}
\newcommand{\eqvrg}{\makebox[0pt][l]{\: ,}}
\newcommand{\EqVrgInt}{\: , \e }
\newcommand{\EqVrgSmInt}{\: , \ }
\newcommand{\EqPntVrgInt}{\: ; \e }
\newcommand{\EqVrg}{\: ,}
\newcommand{\EqPnt}{\: .}
\newcommand{\EgalComp}{\: : \; \;}
\newcommand{\quantvrg}{\, , \;}
\newcommand{\quantsp}{\ee }
\newcommand{\quantsmsp}{\e }
%
\newcommand{\ine}{\, \in \,}
\newcommand{\ege}{\quad = \quad}
%%%%%% \section{texte formules}
%%% 070127 new definition 
\newcommand{\TextInFormula}[1]{\text{\quad #1 \quad }}
%
\newcommand{\et}{\TextInFormula{and}}
\newcommand{\finite}{\TextInFormula{finite}}
\newcommand{\si}{\TextInFormula{if}}
\newcommand{\isin}{\TextInFormula{is in}}
\newcommand{\ou}{\TextInFormula{or}}
\newcommand{\otherwise}{\TextInFormula{otherwise}}
\newcommand{\suchthat}{\TextInFormula{such that}}
\newcommand{\alors}{\TextInFormula{then}}
\newcommand{\where}{\TextInFormula{where}}
\newcommand{\with}{\TextInFormula{with}}
%%%%%%
%%%%%% \section{abr. latines}
% abbreviations de locutions latines
%%% 070127 new definition 
\newcommand{\LatinLocution}[1]{{\itshape #1}\xspace}
% 
\newcommand{\acontrario}{\LatinLocution{a contrario}} %
\newcommand{\Acontrario}{\LatinLocution{A contrario}} %
\newcommand{\adhoc}{\LatinLocution{ad hoc}}
\newcommand{\apriori}{\LatinLocution{a priori}} %
\newcommand{\Apriori}{\LatinLocution{A priori}} %
\newcommand{\afortiori}{\LatinLocution{a fortiori}\xspace}
\newcommand{\artcit}{\LatinLocution{art. cit.}\xspace}
\newcommand{\cf}{\LatinLocution{cf.}}
\newcommand{\Cf}{\LatinLocution{Cf.}}
\newcommand{\defacto}{\textit{de facto}\xspace}
\newcommand{\eg}{\LatinLocution{e.g.}}
\newcommand{\etalii}{\LatinLocution{et~al.}}
\newcommand{\etc}{\LatinLocution{etc.}} % 
\newcommand{\ibid}{\LatinLocution{ibid.}}
\newcommand{\idem}{\LatinLocution{idem}}
\newcommand{\Idem}{\LatinLocution{Idem}}
\newcommand{\ie}{{that is, }}
\newcommand{\NB}{\LatinLocution{N.B.}}
\newcommand{\infra}{\LatinLocution{infra}\xspace}
\newcommand{\ipsofacto}{\LatinLocution{ipso facto}\xspace}
\newcommand{\itin}{\LatinLocution{in}\xspace}
\newcommand{\Itin}{\LatinLocution{In}\xspace}
\newcommand{\modulo}{\LatinLocution{modulo}\xspace}
\newcommand{\mutmat}{\LatinLocution{mutatis mutandis}\xspace} 
\newcommand{\opcit}{\LatinLocution{op. cit.}\xspace}
\newcommand{\redadabs}{\LatinLocution{reductio ad absurdum}\xspace}
\newcommand{\sed}{\LatinLocution{sed}\xspace}
\newcommand{\sqq}{\LatinLocution{et seq.}}
\newcommand{\supra}{\LatinLocution{supra}\xspace}
\newcommand{\via}{via\xspace}
\newcommand{\verbat}{\LatinLocution{verbatim}\xspace}% verbatim 
\newcommand{\vv}{\LatinLocution{vice versa}\xspace} 
%%%%%%%%%%%%%%%%%%%%%%%%%%%%%%%%%%%%
% 990508 implications (param\'etr\'ees)
\newcommand{\jsImpli}[2]{\noindent
   \makebox[5em][c]{$\mathrm{#1} \; \Rightarrow \; \mathrm{#2}$ }}
%
\newcommand{\ieii}{\jsImpli{(i)}{(ii)}}
\newcommand{\iieiii}{\jsImpli{(ii)}{(iii)}}
\newcommand{\iiiei}{\jsImpli{(iii)}{(i)}}
%
%%%%%%%%%%%%%%%%%%%%%%%%%%
% quelques chiffres en "oldstyle"
% \def\zold{\oldstyle{0}}
% \def\uold{\oldstyle{1}}
% \def\dold{\oldstyle{2}}
% en attendant mieux
\def\zold{\mathbf{0}}
\def\uold{\mathbf{1}}
\def\dold{\mathbf{2}}
\def\told{\mathbf{3}}
\def\qold{\mathbf{4}}
%%%%%%%%%%%%%%%%%%%%%%%%%%
% parametrage des fontes pour certains symboles 
\newcommand{\mathjsu}[1]{\mathsf{#1}}
%%%%%%%%%%%%%%%%%%%%%%%%%%
%%%%%% \section{capitales grasses}
% Lettres capitales grasses avec corps evide
%    (pour les semi-anneaux)
%  nouvelle version uniquement \mathbb 000510
%%% 070127 new definition 
\newcommand{\Ambb}{\mathbb{A}}
\newcommand{\Bmbb}{\mathbb{B}}
\newcommand{\Cmbb}{\mathbb{C}} 
\newcommand{\Dmbb}{\mathbb{D}}
\newcommand{\Embb}{\mathbb{E}}
\newcommand{\Fmbb}{\mathbb{F}}
\newcommand{\Gmbb}{\mathbb{G}}
\newcommand{\Hmbb}{\mathbb{H}}
\newcommand{\Imbb}{\mathbb{I}}
\newcommand{\Jmbb}{\mathbb{J}}
\newcommand{\Kmbb}{\mathbb{K}}
\newcommand{\Lmbb}{\mathbb{L}}
\newcommand{\Mmbb}{\mathbb{M}}
\newcommand{\Nmbb}{\mathbb{N}}
\newcommand{\Ombb}{\mathbb{O}}
\newcommand{\Pmbb}{\mathbb{P}}
\newcommand{\Qmbb}{\mathbb{Q}}
\newcommand{\Rmbb}{\mathbb{R}}
\newcommand{\Smbb}{\mathbb{S}}
\newcommand{\Tmbb}{\mathbb{T}}
\newcommand{\Umbb}{\mathbb{U}} 
\newcommand{\Vmbb}{\mathbb{V}}
\newcommand{\Wmbb}{\mathbb{W}}
\newcommand{\Xmbb}{\mathbb{X}}
\newcommand{\Ymbb}{\mathbb{Y}}
\newcommand{\Zmbb}{\mathbb{Z}}
%
\newcommand{\UNmbb}{{\mathchoice
{\hbox{$\textstyle\rm 1\kern-0.2em I$}}%
{\hbox{$\textstyle\rm 1\kern-0.2em I$}}%
{\hbox{$\scriptstyle\rm 1\kern-0.15em I$}}%
{\hbox{$\scriptscriptstyle\rm 1\kern-0.1em I$}}%
}}
%%%%%%%%%%%%%%%%%%%%%%%%%%
%%%%%% \section{cap. cal}
\newcommand{\Ac}{\mathcal{A}}
\newcommand{\Bc}{\mathcal{B}}
\newcommand{\Cc}{\mathcal{C}}
\newcommand{\Dc}{\mathcal{D}}
\newcommand{\Ec}{\mathcal{E}}
\newcommand{\Fc}{\mathcal{F}}
\newcommand{\Gc}{\mathcal{G}}
\newcommand{\Hc}{\mathcal{H}}
\newcommand{\Ic}{\mathcal{I}}
\newcommand{\Jc}{\mathcal{J}}
\newcommand{\Kc}{\mathcal{K}}
\newcommand{\Lc}{\mathcal{L}}
\newcommand{\Mc}{\mathcal{M}}
\newcommand{\Nc}{\mathcal{N}}
\newcommand{\Oc}{\mathcal{O}}
\newcommand{\Pc}{\mathcal{P}}
\newcommand{\Qc}{\mathcal{Q}}
\newcommand{\Rc}{\mathcal{R}}
\newcommand{\Sc}{\mathcal{S}}
\newcommand{\Tc}{\mathcal{T}}
\newcommand{\Uc}{\mathcal{U}}
\newcommand{\Vc}{\mathcal{V}}
\newcommand{\Wc}{\mathcal{W}}
\newcommand{\Xc}{\mathcal{X}}
\newcommand{\Yc}{\mathcal{Y}}
\newcommand{\Zc}{\mathcal{Z}}
%
%%%%%%%%%%%%%%%%%%%%%%%%%%
% Lettres grasses
%%%%%% \section{boldface}
% lettres "boldface" pour les maths
\newcommand{\ambf}{\mathbf{a}}
\newcommand{\bmbf}{\mathbf{b}}
\newcommand{\cmbf}{\mathbf{c}}
\newcommand{\dmbf}{\mathbf{d}}
\newcommand{\embf}{\mathbf{e}}
\newcommand{\fmbf}{\mathbf{f}}
\newcommand{\gmbf}{\mathbf{g}}
\newcommand{\hmbf}{\mathbf{h}}
\newcommand{\imbf}{\mathbf{i}}
\newcommand{\jmbf}{\mathbf{j}}
\newcommand{\kmbf}{\mathbf{k}}
\newcommand{\lmbf}{\mathbf{l}}
\newcommand{\mmbf}{\mathbf{m}}
\newcommand{\nmbf}{\mathbf{n}}
\newcommand{\ombf}{\mathbf{o}}
\newcommand{\pmbf}{\mathbf{p}}
\newcommand{\qmbf}{\mathbf{q}}
\newcommand{\rmbf}{\mathbf{r}}
\newcommand{\smbf}{\mathbf{s}}
\newcommand{\tmbf}{\mathbf{t}}
\newcommand{\umbf}{\mathbf{u}}
\newcommand{\vmbf}{\mathbf{v}}
\newcommand{\wmbf}{\mathbf{w}}
\newcommand{\xmbf}{\mathbf{x}}
\newcommand{\ymbf}{\mathbf{y}}
\newcommand{\zmbf}{\mathbf{z}}
%
\newcommand{\Ambf}{\mathbf{A}}
\newcommand{\Bmbf}{\mathbf{B}}
\newcommand{\Cmbf}{\mathbf{C}}
\newcommand{\Dmbf}{\mathbf{D}}
\newcommand{\Embf}{\mathbf{E}}
\newcommand{\Fmbf}{\mathbf{F}}
\newcommand{\Gmbf}{\mathbf{G}}
\newcommand{\Hmbf}{\mathbf{H}}
\newcommand{\Imbf}{\mathbf{I}}
\newcommand{\Jmbf}{\mathbf{J}}
\newcommand{\Kmbf}{\mathbf{K}}
\newcommand{\Lmbf}{\mathbf{L}}
\newcommand{\Mmbf}{\mathbf{M}}
\newcommand{\Nmbf}{\mathbf{N}}
\newcommand{\Ombf}{\mathbf{O}}
\newcommand{\Pmbf}{\mathbf{P}}
\newcommand{\Qmbf}{\mathbf{Q}}
\newcommand{\Rmbf}{\mathbf{R}}
\newcommand{\Smbf}{\mathbf{S}}
\newcommand{\Tmbf}{\mathbf{T}}
\newcommand{\Umbf}{\mathbf{U}}
\newcommand{\Vmbf}{\mathbf{V}}
\newcommand{\Wmbf}{\mathbf{W}}
\newcommand{\Xmbf}{\mathbf{X}}
\newcommand{\Ymbf}{\mathbf{Y}}
\newcommand{\Zmbf}{\mathbf{Z}}
%%%%%% \section{boldsymbol}
% lettres "boldsymbol" pour les maths
\newcommand{\absy}{\boldsymbol{a}}
\newcommand{\bbsy}{\boldsymbol{b}}
\newcommand{\cbsy}{\boldsymbol{c}}
\newcommand{\dbsy}{\boldsymbol{d}}
\newcommand{\ebsy}{\boldsymbol{e}}
\newcommand{\fbsy}{\boldsymbol{f}}
\newcommand{\gbsy}{\boldsymbol{g}}
\newcommand{\hbsy}{\boldsymbol{h}}
\newcommand{\ibsy}{\boldsymbol{i}}
\newcommand{\jbsy}{\boldsymbol{j}}
\newcommand{\kbsy}{\boldsymbol{k}}
\newcommand{\lbsy}{\boldsymbol{l}}
\newcommand{\mbsy}{\boldsymbol{m}}
\newcommand{\nbsy}{\boldsymbol{n}}
\newcommand{\obsy}{\boldsymbol{o}}
\newcommand{\pbsy}{\boldsymbol{p}}
\newcommand{\qbsy}{\boldsymbol{q}}
\newcommand{\rbsy}{\boldsymbol{r}}
\newcommand{\sbsy}{\boldsymbol{s}}
\newcommand{\tbsy}{\boldsymbol{t}}
\newcommand{\ubsy}{\boldsymbol{u}}
\newcommand{\vbsy}{\boldsymbol{v}}
\newcommand{\wbsy}{\boldsymbol{w}}
\newcommand{\xbsy}{\boldsymbol{x}}
\newcommand{\ybsy}{\boldsymbol{y}}
\newcommand{\zbsy}{\boldsymbol{z}}
%
\newcommand{\Absy}{\boldsymbol{A}}
\newcommand{\Bbsy}{\boldsymbol{B}}
\newcommand{\Cbsy}{\boldsymbol{C}}
\newcommand{\Dbsy}{\boldsymbol{D}}
\newcommand{\Ebsy}{\boldsymbol{E}}
\newcommand{\Fbsy}{\boldsymbol{F}}
\newcommand{\Gbsy}{\boldsymbol{G}}
\newcommand{\Hbsy}{\boldsymbol{H}}
\newcommand{\Ibsy}{\boldsymbol{I}}
\newcommand{\Jbsy}{\boldsymbol{J}}
\newcommand{\Kbsy}{\boldsymbol{K}}
\newcommand{\Lbsy}{\boldsymbol{L}}
\newcommand{\Mbsy}{\boldsymbol{M}}
\newcommand{\Nbsy}{\boldsymbol{N}}
\newcommand{\Obsy}{\boldsymbol{O}}
\newcommand{\Pbsy}{\boldsymbol{P}}
\newcommand{\Qbsy}{\boldsymbol{Q}}
\newcommand{\Rbsy}{\boldsymbol{R}}
\newcommand{\Sbsy}{\boldsymbol{S}}
\newcommand{\Tbsy}{\boldsymbol{T}}
\newcommand{\Ubsy}{\boldsymbol{U}}
\newcommand{\Vbsy}{\boldsymbol{V}}
\newcommand{\Wbsy}{\boldsymbol{W}}
\newcommand{\Xbsy}{\boldsymbol{X}}
\newcommand{\Ybsy}{\boldsymbol{Y}}
\newcommand{\Zbsy}{\boldsymbol{Z}}
%%%%%%%%%%%%%%%%%%%%%%%%%%
%%%%%% \section{sans serif}
% lettres "sans serif" pour les maths
\newcommand{\amsf}{\mathsf{a}}
\newcommand{\bmsf}{\mathsf{b}}
\newcommand{\cmsf}{\mathsf{c}}
\newcommand{\dmsf}{\mathsf{d}}
\newcommand{\emsf}{\mathsf{e}}
\newcommand{\fmsf}{\mathsf{f}}
\newcommand{\gmsf}{\mathsf{g}}
\newcommand{\hmsf}{\mathsf{h}}
\newcommand{\imsf}{\mathsf{i}}
\newcommand{\jmsf}{\mathsf{j}}
\newcommand{\kmsf}{\mathsf{k}}
\newcommand{\lmsf}{\mathsf{l}}
\newcommand{\mmsf}{\mathsf{m}}
\newcommand{\nmsf}{\mathsf{n}}
\newcommand{\omsf}{\mathsf{o}}
\newcommand{\pmsf}{\mathsf{p}}
\newcommand{\qmsf}{\mathsf{q}}
\newcommand{\rmsf}{\mathsf{r}}
\newcommand{\smsf}{\mathsf{s}}
\newcommand{\tmsf}{\mathsf{t}}
\newcommand{\umsf}{\mathsf{u}}
\newcommand{\vmsf}{\mathsf{v}}
\newcommand{\wmsf}{\mathsf{w}}
\newcommand{\xmsf}{\mathsf{x}}
\newcommand{\ymsf}{\mathsf{y}}
\newcommand{\zmsf}{\mathsf{z}}
%
\newcommand{\Amsf}{\mathsf{A}}
\newcommand{\Bmsf}{\mathsf{B}}
\newcommand{\Cmsf}{\mathsf{C}}
\newcommand{\Dmsf}{\mathsf{D}}
\newcommand{\Emsf}{\mathsf{E}}
\newcommand{\Fmsf}{\mathsf{F}}
\newcommand{\Gmsf}{\mathsf{G}}
\newcommand{\Hmsf}{\mathsf{H}}
\newcommand{\Imsf}{\mathsf{I}}
\newcommand{\Jmsf}{\mathsf{J}}
\newcommand{\Kmsf}{\mathsf{K}}
\newcommand{\Lmsf}{\mathsf{L}}
\newcommand{\Mmsf}{\mathsf{M}}
\newcommand{\Nmsf}{\mathsf{N}}
\newcommand{\Omsf}{\mathsf{O}}
\newcommand{\Pmsf}{\mathsf{P}}
\newcommand{\Qmsf}{\mathsf{Q}}
\newcommand{\Rmsf}{\mathsf{R}}
\newcommand{\Smsf}{\mathsf{S}}
\newcommand{\Tmsf}{\mathsf{T}}
\newcommand{\Umsf}{\mathsf{U}}
\newcommand{\Vmsf}{\mathsf{V}}
\newcommand{\Wmsf}{\mathsf{W}}
\newcommand{\Xmsf}{\mathsf{X}}
\newcommand{\Ymsf}{\mathsf{Y}}
\newcommand{\Zmsf}{\mathsf{Z}}
%%%%%%%%%%%%%%%%%%%%%%%%%%
%%%%%% \section{lettres surlignees}
% lettres "barr\'ees"
% toutes les barres sont � la m�me hauteur, 
% ind�pendamment de la lettre
\newcommand{\ETAbar}[1]{\overline{\rule{0pt}{1.5ex}\smash{#1}}}
\newcommand{\abar}{\ETAbar{a}}
\newcommand{\bbar}{\ETAbar{b}}
\newcommand{\cbar}{\ETAbar{c}}
\newcommand{\dbar}{\ETAbar{d}}
\newcommand{\ebar}{\ETAbar{e}}
\newcommand{\fbar}{\ETAbar{f}}
\newcommand{\gbar}{\ETAbar{g}}
\newcommand{\hbarjs}{\ETAbar{h}}
\newcommand{\ibar}{\ETAbar{i}}
\newcommand{\jbar}{\ETAbar{j}}
\newcommand{\kbar}{\ETAbar{k}}
\newcommand{\lbar}{\ETAbar{l}}
\newcommand{\mbar}{\ETAbar{m}}
\newcommand{\nbar}{\ETAbar{n}}
\newcommand{\obar}{\ETAbar{o}}
\newcommand{\pbar}{\ETAbar{p}}
\newcommand{\qbar}{\ETAbar{q}}
\newcommand{\rbar}{\ETAbar{r}}
\newcommand{\sbar}{\ETAbar{s}}
\newcommand{\tbar}{\ETAbar{t}}
\newcommand{\ubar}{\ETAbar{u}}
\newcommand{\vbar}{\ETAbar{v}}
\newcommand{\wbar}{\ETAbar{w}}
\newcommand{\xbar}{\ETAbar{x}}
\newcommand{\ybar}{\ETAbar{y}}
\newcommand{\zbar}{\ETAbar{z}}
\newcommand{\unbar}{\ETAbar{1}}
%
\newcommand{\Abar}{\ETAbar{A}}
\newcommand{\Bbar}{\ETAbar{B}}
\newcommand{\Cbar}{\ETAbar{C}}
\newcommand{\Dbar}{\ETAbar{D}}
\newcommand{\Ebar}{\ETAbar{E}}
\newcommand{\Fbar}{\ETAbar{F}}
\newcommand{\Gbar}{\ETAbar{G}}
\newcommand{\Hbar}{\ETAbar{H}}
\newcommand{\Ibar}{\ETAbar{I}}
\newcommand{\Jbar}{\ETAbar{J}}
\newcommand{\Kbar}{\ETAbar{K}}
\newcommand{\Lbar}{\ETAbar{L}}
\newcommand{\Mbar}{\ETAbar{M}}
\newcommand{\Nbar}{\ETAbar{N}}
\newcommand{\Obar}{\ETAbar{O}}
\newcommand{\Pbar}{\ETAbar{P}}
\newcommand{\Qbar}{\ETAbar{Q}}
\newcommand{\Rbar}{\ETAbar{R}}
\newcommand{\Sbar}{\ETAbar{S}}
\newcommand{\Tbar}{\ETAbar{T}}
\newcommand{\Ubar}{\ETAbar{U}}
\newcommand{\Vbar}{\ETAbar{V}}
\newcommand{\Wbar}{\ETAbar{W}}
\newcommand{\Xbar}{\ETAbar{X}}
\newcommand{\Ybar}{\ETAbar{Y}}
\newcommand{\Zbar}{\ETAbar{Z}}
%%% 070328  de js-macros3
% entiers sign\'es 021110 c'est le boxon y a deux sortes de
% lettres barr\'ees cf js_symboles.tex  \abar et \bara
\newcommand{\jsBar}[1]{\bar{#1}}
\newcommand{\baru}{\jsBar{1}}
\newcommand{\bard}{\jsBar{2}}
\newcommand{\bart}{\jsBar{3}}
\newcommand{\barq}{\jsBar{4}}
\newcommand{\barc}{\jsBar{5}}
\newcommand{\bars}{\jsBar{6}}
\newcommand{\barl}{\jsBar{l}}
\newcommand{\barh}{\jsBar{h}}
% % lettres "tild\'ees"
\newcommand{\Atil}{\widetilde{A}}
\newcommand{\Btil}{\widetilde{B}}
\newcommand{\Ctil}{\widetilde{C}}
\newcommand{\Dtil}{\widetilde{D}}
\newcommand{\Etil}{\widetilde{E}}
\newcommand{\Ftil}{\widetilde{F}}
\newcommand{\Gtil}{\widetilde{G}}
\newcommand{\Htil}{\widetilde{H}}
\newcommand{\Itil}{\widetilde{I}}
\newcommand{\Jtil}{\widetilde{J}}
\newcommand{\Ktil}{\widetilde{K}}
\newcommand{\Ltil}{\widetilde{L}}
\newcommand{\Mtil}{\widetilde{M}}
\newcommand{\Ntil}{\widetilde{N}}
\newcommand{\Otil}{\widetilde{O}}
\newcommand{\Ptil}{\widetilde{P}}
\newcommand{\Qtil}{\widetilde{Q}}
\newcommand{\Rtil}{\widetilde{R}}
\newcommand{\Stil}{\widetilde{S}}
\newcommand{\Ttil}{\widetilde{T}}
\newcommand{\Util}{\widetilde{U}}
\newcommand{\Vtil}{\widetilde{V}}
\newcommand{\Wtil}{\widetilde{W}}
\newcommand{\Xtil}{\widetilde{X}}
\newcommand{\Ytil}{\widetilde{Y}}
\newcommand{\Ztil}{\widetilde{Z}}
%%%%%%%%%%%%%%%%%%%%%%%%%%%
%%%%%%%%%%%%%%%%%%%%%%%%%%
%%%%%% \section{zeros et uns}
% gros chiffres (pour les matrices en blocs
% 010128
\newcommand{\bigz}{\mbox{\large{$0$}}}
\newcommand{\Bigz}{\mbox{\Large{$0$}}}
\newcommand{\BIGz}{\mbox{\LARGE{$0$}}}
\newcommand{\bz}{\BIGz}% compatibilit\'e
%%%%%%%%%%%%%%%%%%%%%%%%%%
%%%%%% \section{symboles r�duits}
%%% symboles r�duits, plus jolis dans le texte
%%%%%%\section{adresse internet}   030821
\newcommand{\tildett}{$\scriptstyle{\pmb{\sim}}$}
%%% dollard \dol 010128
\def\dol{{\mathchoice
{\hbox{{\small $\textstyle \$ $}}}%
{\hbox{{\small $\textstyle \$ $}}}%
{\hbox{$\scriptstyle \$ $}}%
{\hbox{$\scriptscriptstyle \$ $}}%
}}
%%% infini 020223 (NB utilise \scalebox, donc pstricks!}
\newcommand{\etainfty}{\scalebox{0.85}{\scalebox{0.85}{+}\infty}}
\newcommand{\etamininfty}{\scalebox{0.85}{\scalebox{0.85}{-}\infty}}
%%%%%%%%%%%%%%%%%%%%%%%%%%
% macros utilis�es dans les figures
\newcommand{\jesBar}[1]{\text{\textbf{-}}{#1}}
\newcommand{\sunBe}{\scalebox{0.8}{\unBe}}
%%%%%%%%%%%%%%%%%%%%%%%%%%%
%%%%%% \section{h\'ebreu}
% 020116
\newcommand{\etabeth}{\scalebox{0.85}{\beth}}
%%%%%%%%%%%%%%%%%%%%%%%%%%
%%%%%% \section{diagrammes}
% environnement pour les diagrammes (utilise pstricks)
%
\newcommand{\diagechu}{0.8}     %% echelle 1
% syle de diagramme pour les slides
\newlength{\ArrowDiagSize}
\setlength{\ArrowDiagSize}{6pt}
\newlength{\ArrowDiagWidth}
\setlength{\ArrowDiagWidth}{2pt}
\newpsstyle{SLDiagStyle}%
   {colsep=6ex,rowsep=5ex,nodesep=1ex,npos=.45,%
    arrows=->,linewidth=\ArrowDiagWidth,arrowsize=\ArrowDiagSize,%
        linestyle=solid,linecolor=\ArrowDiagColor}
\newcommand{\SLDiagStyle}{\psset{style=SLDiagStyle}}
%
\newenvironment{SLDiag}%
   {\psset{style=SLDiagStyle}\begin{psmatrix}}%
   {\end{psmatrix}}%
\newcommand{\CDSL}{\begin{SLDiag}}
\newcommand{\CDSLF}{\end{SLDiag}}
%
\newenvironment{DiagraBig}%
{\psmatrix[colsep=7ex,rowsep=6ex,arrows=->,nodesep=1ex,npos=.45]}%
{\endpsmatrix}
\newcommand{\CDB}{\begin{DiagraBig}}
\newcommand{\CDBF}{\end{DiagraBig}}
% la meme chose en plus petit
\newenvironment{DiagraSmall}%
{\psmatrix[colsep=3ex,rowsep=3ex,arrows=->,nodesep=1ex,npos=.45]}%
{\endpsmatrix}
\newcommand{\CDS}{\begin{DiagraSmall}}
\newcommand{\CDSF}{\end{DiagraSmall}}
% exemple d'utilisation des diagrammes
% a reprendre par copier coller
% \CD
% [name=A] F & & [name=B] E \\[0ex]
% [name=C] R & & [name=D] Q
% \ncline{A}{B}^{\varphi }
% \ncline{A}{C}<{\iota }
% \ncline{B}{D}>{\iota }
% \ncline{C}{D}_{\varphi }
% \CDF
\newcommand{\uddots}{\scalebox{1 -1}{\ddots}}
%%%%%% diagramme sp\'ecial pour le groupe libre (II.6) %%%%%
\newcommand{\DiaGroLibU}{\etaCD
[name=A] \Atileto & &  [name=B] F(A) \\[0ex]
        & [name=C] \Atileto &
\ncline{A}{B}^{\gamma _{A}}
\ncline{A}{C}<{\rho _{A}}
\psset{offset=.5ex}
\ncline{C}{B}
\naput[npos=.3]{\gamma _{A}}
\ncline{B}{C}
\naput[npos=.3]{\iota }
\etaCDF}
%%%%%% \section{matrices et vecteurs}
%    Matrices et vecteurs
%
%  commandes pour rapprocher les colonnes et ecarter les
%  lignes des matrices. Retablissent les valeurs habituelles
%  a la sortie de chaque macro de matrices
%
%%% mod 010130  utilisation de \pmatrix, etc. (AmsTeX)
%  Matrice 1 x 1
\newcommand{\matriceuu}[1]%
    {\begin{pmatrix} #1 \end{pmatrix}}
%  Matrice 2 x 2  modif 010128
\newcommand{\matricedd}[4]%
    {\begin{pmatrix} #1 & #2 \\ #3 & #4 \end{pmatrix}}
%  Vecteur-colonne de dimension 2 modif 010128
\newcommand{\vecteurd}[2]%
    {\begin{pmatrix} #1 \\ #2 \end{pmatrix}}
%  Vecteur-ligne de dimension 2 modif 010128
\newcommand{\ligned}[2]%
    {\begin{pmatrix} #1 & #2 \end{pmatrix}}
%  Matrice 3 x 3
\newcommand{\matricett}[9]%
    {\begin{pmatrix}  #1 & #2 & #3 \\
                      #4 & #5 & #6 \\
                      #7 & #8 & #9 \end{pmatrix}}
%  Vecteur-colonne de dimension 3
\newcommand{\vecteurt}[3]%
    {\begin{pmatrix} #1 \\ #2 \\ #3 \end{pmatrix}}
%  Vecteur-ligne de dimension 3
\newcommand{\lignet}[3]%
    {\begin{pmatrix} #1 & #2 & #3 \end{pmatrix}}
    %%%%%%%%%%%%%% 030211
    \newcommand{\matricein}
        {\addtolength{\arraycolsep}{-0.5\arraycolsep}%
          \renewcommand{\arraystretch}{2}}
    \newcommand{\matriceout}
        {\addtolength{\arraycolsep}{\arraycolsep}%
          \renewcommand{\arraystretch}{1}}
    \newcommand{\MatBlocIn}
        {\addtolength{\arraycolsep}{-0.5\arraycolsep}%
          \renewcommand{\arraystretch}{1.5}}
    \newcommand{\MatBlocOut}
        {\addtolength{\arraycolsep}{\arraycolsep}%
          \renewcommand{\arraystretch}{1}}
% construction de matrices-blocs
\newlength{\jsWidthCol}
\setlength{\jsWidthCol}{0pt}
\newcommand{\Ent}[1]{\makebox[\jsWidthCol]{$#1$}}
\newlength{\blocinterligne}
\setlength{\blocinterligne}{1.4ex}
\newcommand{\bitl}{\rule[-\blocinterligne]{0 mm}{\blocinterligne}}
\newlength{\blocinterligned}
\setlength{\blocinterligned}{2ex}
\newcommand{\bitld}{\rule[-\blocinterligned]{0 mm}{\blocinterligned}}
%
\newcommand{\blocmattt}[9]{%
{\scriptstyle \ematricett{#1}{#2}{#3}{#4}{#5}{#6}{#7}{#8}{#9}}}
\newcommand{\blocligt}[3]{%
{\scriptstyle \elignet{#1}{#2}{#3}}}
\newcommand{\blocvect}[3]{%
{\scriptstyle \evecteurt{#1}{#2}{#3}}}
% %  r\'eduction des grosse matrices
% \newcommand{\redmatu}[1]{\scalebox{0.83}{#1}}
% \newcommand{\redmatd}[1]{\scalebox{0.66}{#1}}
%
%
% cadres pour visualiser la decomposition des matrices en blocs
%
\newlength{\temparraycolsep}
\newlength{\longueurbloc}
\newlength{\hauteurbloc}
\newlength{\centragebloc}
\setlength{\longueurbloc}{9ex}
\setlength{\hauteurbloc}{7ex}
\setlength{\centragebloc}{-3ex}
% nouvelle longueurs 020927
\newlength{\longueurblc}
\newlength{\hauteurblc}
\newlength{\centrageblc}
\setlength{\longueurblc}{6.5ex}
\setlength{\hauteurblc}{5ex}
\setlength{\centrageblc}{-2ex}
%
\newcommand{\blocligne}[1]%
    {\framebox[\longueurbloc]{$#1$}}
\newcommand{\blocmatrice}[1]%
    {\framebox[\longueurbloc]{\rule[\centragebloc]{0mm}{\hauteurbloc}$#1$}}
\newcommand{\blocvecteur}[1]%
    {\framebox{\rule[\centragebloc]{0mm}{\hauteurbloc}$#1$}}
%
\newcommand{\blcligne}[1]%
    {\framebox[\longueurblc]{$#1$}}
\newcommand{\blcmatrice}[1]%
    {\framebox[\longueurblc]{\rule[\centrageblc]{0mm}{\hauteurblc}$#1$}}
\newcommand{\blcvecteur}[1]%
    {\framebox{\rule[\centrageblc]{0mm}{\hauteurblc}$#1$}}
%
%  Matrice 2 x 2  avec blocs visualis\'es 020927
\newcommand{\matriceddblvs}[4]%%
   {\setlength{\temparraycolsep}{\arraycolsep}%
    \setlength{\arraycolsep}{1.3pt}%
        \renewcommand{\arraystretch}{1.2}%
        \left (%
    \begin{array}{cc}%
                #1  & \blcligne{#2} \\
            \blcvecteur{#3} & \blcmatrice{#4}
        \end{array}%
        \right )%
    \setlength{\arraycolsep}{\temparraycolsep}%
        \renewcommand{\arraystretch}{1.0}%
   }%
%  Vecteur-colonne de dimension 2
\newcommand{\vecteurdblvs}[2]%
   {\setlength{\temparraycolsep}{\arraycolsep}%
    \setlength{\arraycolsep}{1.5pt}%
        \renewcommand{\arraystretch}{1.2}%
        \left (%
    \begin{array}{c}%
                #1  \\
            \blcvecteur{#2}
        \end{array}%
        \right )%
    \setlength{\arraycolsep}{\temparraycolsep}%
        \renewcommand{\arraystretch}{1.0}%
   }%
%     {\begin{pmatrix} #1 \\ \blcvecteur{#2} \end{pmatrix}}
%  Vecteur-ligne de dimension 2
\newcommand{\lignedblvs}[2]%
   {\setlength{\temparraycolsep}{\arraycolsep}%
    \setlength{\arraycolsep}{1.5pt}%
        \renewcommand{\arraystretch}{1.2}%
        \left (%
    \begin{array}{cc}%
                #1  & \blcligne{#2}
        \end{array}%
        \right )%
    \setlength{\arraycolsep}{\temparraycolsep}%
        \renewcommand{\arraystretch}{1.0}%
   }%
%     {\begin{pmatrix} #1 & \blcligne{#2} \end{pmatrix}}
%
%  Matrice 3 x 3  avec blocs visualis\'es 020928
\newcommand{\matricettblvs}[9]%%
   {\setlength{\temparraycolsep}{\arraycolsep}%
    \setlength{\arraycolsep}{1.5pt}%
        \renewcommand{\arraystretch}{1.2}%
        \left (%
    \begin{array}{ccc}%
                #1  & \blcligne{#2} & #3\\
            \blcvecteur{#4} & \blcmatrice{#5} & \blcvecteur{#6}\\
                #7  & \blcligne{#8} & #9\\
        \end{array}%
        \right )%
    \setlength{\arraycolsep}{\temparraycolsep}%
        \renewcommand{\arraystretch}{1.0}%
   }%
%  Vecteur-colonne de dimension 3
\newcommand{\vecteurtblvs}[3]%
   {\setlength{\temparraycolsep}{\arraycolsep}%
    \setlength{\arraycolsep}{1.5pt}%
        \renewcommand{\arraystretch}{1.2}%
        \left (%
    \begin{array}{c}%
                #1  \\
            \blcvecteur{#2}\\
                #3
        \end{array}%
        \right )%
    \setlength{\arraycolsep}{\temparraycolsep}%
        \renewcommand{\arraystretch}{1.0}%
   }%
%  Vecteur-ligne de dimension 3
\newcommand{\lignetblvs}[3]%
   {\setlength{\temparraycolsep}{\arraycolsep}%
    \setlength{\arraycolsep}{1.5pt}%
        \renewcommand{\arraystretch}{1.2}%
        \left (%
    \begin{array}{ccc}%
                #1  & \blcligne{#2} & #3
        \end{array}%
        \right )%
    \setlength{\arraycolsep}{\temparraycolsep}%
        \renewcommand{\arraystretch}{1.0}%
   }%
%
%  Matrice 3 x 3  avec blocs visualis\'es 020928
%  autre sorte de blocs
\newcommand{\matricettblblvs}[9]%%
   {\setlength{\temparraycolsep}{\arraycolsep}%
    \setlength{\arraycolsep}{1.5pt}%
        \renewcommand{\arraystretch}{1.2}%
        \left (%
    \begin{array}{ccc}%
                #1  & \blcligne{#2} & \blcligne{#3}\\
            \blcvecteur{#4} & \blcmatrice{#5} & \blcmatrice{#6}\\
                \blcvecteur{#7}  & \blcmatrice{#8} & \blcmatrice{#9}\\
        \end{array}%
        \right )%
    \setlength{\arraycolsep}{\temparraycolsep}%
        \renewcommand{\arraystretch}{1.0}%
   }%
%  Vecteur-colonne de dimension 3
\newcommand{\vecteurtblblvs}[3]%
   {\setlength{\temparraycolsep}{\arraycolsep}%
    \setlength{\arraycolsep}{1.5pt}%
        \renewcommand{\arraystretch}{1.2}%
        \left (%
    \begin{array}{c}%
                #1  \\
            \blcvecteur{#2}\\
                \blcvecteur{#3}
        \end{array}%
        \right )%
    \setlength{\arraycolsep}{\temparraycolsep}%
        \renewcommand{\arraystretch}{1.0}%
   }%
%     {\begin{pmatrix} #1 \\ \blcvecteur{#2} \end{pmatrix}}
%  Vecteur-ligne de dimension 3
\newcommand{\lignetblblvs}[3]%
   {\setlength{\temparraycolsep}{\arraycolsep}%
    \setlength{\arraycolsep}{1.5pt}%
        \renewcommand{\arraystretch}{1.2}%
        \left (%
    \begin{array}{ccc}%
                #1  & \blcligne{#2} & \blcligne{#3}
        \end{array}%
        \right )%
    \setlength{\arraycolsep}{\temparraycolsep}%
        \renewcommand{\arraystretch}{1.0}%
   }%
%     {\begin{pmatrix} #1 & \blcligne{#2} \end{pmatrix}}
%
%%%%%%%%%%%%%%%%%%%%%%%%%%
\endinput
%%%%%%%%
