
\chapter{Automata repository and factory}
\label{chp:aut-rep-fac}

The \vcsnv distribution contains a folder
\command{data/automata/} where a number of automata and of \vcsn
programs which build automata are ready for use by the \tafkit
commands.

\section{$\B$-automata}

\subsection{Repository}

The following automata files are stored
\file{data/automata/char-b/} directory (and accessible by the command
\command{vcsn-char-b cat}).


\subsubsection{\file{a1.xml}  for $\Ac_1$}

\begin{figure}[ht]
    \centering
% \begin{center}
\VCDraw{%
\begin{VCPicture}{(-1.4,-1.4cm)(7.4,1.4cm)}
% etats[p][r][q]
\State{(0,0)}{A}\State{(3,0)}{B}\State{(6,0)}{C}
%
\Initial{A}\Final{C}
% transitions
\EdgeL{A}{B}{a}\EdgeL{B}{C}{b}
%
\LoopN{A}{a}\LoopS[.22]{A}{b}
\LoopN{C}{a}\LoopS[.22]{C}{b}
%
\end{VCPicture}}
% \end{center}
    \caption{The Boolean automaton $\Ac_1$ over
    $\{a,b\}^{*}$ (\cf \protect\figur{a1}). }
\label{fig:a1-app}
\end{figure}

\subsubsection{\file{b1.xml}  for $\Bc_1$}

\begin{figure}[ht]
    \centering
% \begin{center}
\VCDraw{%
\begin{VCPicture}{(-1.4,-1.4cm)(5.4,1.4cm)}
% etats[p][r][q]\State{(3,0)}{B}
\State{(0,0)}{A}\State{(4,0)}{C}
%
\Initial{A}\Final{C}
% transitions \EdgeL{A}{B}{a}
\EdgeL{A}{C}{b}
%
\LoopN{A}{a}\LoopS[.22]{A}{b}
\LoopN{C}{a}\LoopS[.22]{C}{b}
%
\end{VCPicture}}
% \end{center}
    \caption{The Boolean automaton $\Bc_1$ over
    $\{a,b\}^{*}$. }
\label{fig:b1-app}
\end{figure}
\pagebreak
\subsubsection{\file{evena.xml}}

\begin{figure}[ht]
    \centering
\VCDraw{%
\begin{VCPicture}{(-1.4,-1.4cm)(5.4,1.4cm)}
\State{(0,0)}{A}\State{(4,0)}{C}

\Initial[n]{A}\Final[s]{A}
\ArcL{A}{C}{a}\ArcL{C}{A}{a}
%
\LoopW{A}{b}
\LoopE{C}{b}
%
\end{VCPicture}}
    \caption{The Boolean automaton \file{evena.xml} over
    $\{a,b\}^{*}$. }
\label{fig:evena-app}
\end{figure}

\subsubsection{\file{oddb.xml}}

\begin{figure}[ht]
    \centering
\VCDraw{%
\begin{VCPicture}{(-1.4,-1.4cm)(5.4,1.4cm)}
\State{(0,0)}{A}\State{(4,0)}{C}

\Initial[n]{A}\Final[s]{C}
\ArcL{A}{C}{b}\ArcL{C}{A}{b}
%
\LoopW{A}{a}
\LoopE{C}{a}
%
\end{VCPicture}}
    \caption{The Boolean automaton \file{oddb.xml}  over
    $\{a,b\}^{*}$. }
\label{fig:oddb-app}
\end{figure}

\subsection{Factory}

The following programs are in the
\file{data/automata/char-b/} directory.

\subsubsection{Program \file{divkbaseb}}


\begin{SwClCmd}
\begin{shell}
$ \kbd{divkbaseb 4 3 > div4base3.xml}
$
\end{shell}%
\end{SwClCmd}%
\begin{SwClTxt}
    Generates an automaton over the digit alphabet $\{0,\ldots,b-1\}$
    that recognises the writing in base~\Prm{b} of the numbers
    divisible by the integer~\Prm{k}.
\end{SwClTxt}%

\Comt
The `divisor' \file{div3base2.xml} (\figur{div-3b2}) is already in the
repository.

\begin{figure}[ht]
    \centering
% \begin{center}
\VCDraw{%
\begin{VCPicture}{(-1.5,-1.4)(7.5,1.4)}
% etats[\zold][\uold][\dold]
\MediumState
\State{(0,0)}{A}
\State{(3,0)}{B}
\State{(6,0)}{C}
%
\Initial[n]{A}\Final[s]{A}
% transitions
\ArcL{A}{B}{1}\ArcL{B}{A}{1}
\ArcL{B}{C}{0}\ArcL{C}{B}{0}
%
\LoopW{A}{0}\LoopE{C}{1}
%
\end{VCPicture}
}
% \end{center}
    \caption{The `divisor' \file{div3base2.xml} over
    $\{0,1\}^{*}$. }
\label{fig:div-3b2}
\end{figure}

\subsubsection{Program \file{double\_ring}}
\SetTwClPrm{\TwClThree}%


\begin{SwClCmd}
\begin{shell}
$ \kbd{double\_ring 6 1 3 4 5 > double-6-1-3-4-5.xml}
$
\end{shell}%
\end{SwClCmd}%
\begin{SwClTxt}
    Generates an~\Prm{n} state automaton over the alphabet $\{a,b\}$
    that consists in a double ring of transitions: a counter
    clockwise ring of transitions labelled by~\Prm{a} and a clockwise
    ring of transitions labelled by~\Prm{b}.
\end{SwClTxt}%

\Spec
The states are labelled from~\Prm{0} to~\Prm{n-1}.
State~\Prm{0} is initial.
The number of states~\Prm{n} is the first parameter and the next
parameters give the list of final states.
\figur{H6-app} shows the automaton built by the above command.

\Comt
The double-ring automata are closely related to the star-height
problem.
Sch\"utzenberger used them to give the first example of automata over
a 2 letter alphabet that have arbitrary large loop complexity and
McNaughton to give the simplest example of minimal automata which do
not achieve the minimal loop complexity for the language the recognize.
This was then reinterpreted in terms of \emph{universal automata}
(\cf \cite[Sec.~II.8]{Saka03}).

The automaton \file{double-3-1.xml} (\figur{H6-app}) is already in the
repository.

\begin{figure}[ht]
    \centering
% \begin{center}
\VCDraw[1.4]{%
\begin{VCPicture}{(-3,-2.5)(3,2.5)}
\MediumState
% etats
\State[0]{(-2,0)}{A}
\State[1]{(-1,\SQRThree)}{B}
\State[2]{(1,\SQRThree)}{C}
\State[3]{(2,0)}{D}
\State[4]{(1,-\SQRThree)}{E}
\State[5]{(-1,-\SQRThree)}{F}
%
\Initial{A}
\Final[n]{B}
% \Final[n]{C}
\Final{D}\Final[s]{E}\Final[s]{F}
% transitions
\ArcL{A}{B}{a}
\ArcL{B}{C}{a}
\ArcL{C}{D}{a}
\ArcL{D}{E}{a}
\ArcL{E}{F}{a}
\ArcL{F}{A}{a}
%
\ArcL{A}{F}{b}
\ArcL{F}{E}{b}
\ArcL{E}{D}{b}
\ArcL{D}{C}{b}
\ArcL{C}{B}{b}
\ArcL{B}{A}{b}
%
\end{VCPicture}
}
\eee
\VCDraw[1.2]{%
\begin{VCPicture}{(-3,-2.5)(3,2.5)}
\MediumState
% etats
\State[0]{(-2,0)}{A}
% \State[1]{(-1,\SQRThree)}{B}
\State[1]{(1,\SQRThree)}{C}
% \State[3]{(2,0)}{D}
\State[2]{(1,-\SQRThree)}{E}
% \State[5]{(-1,-\SQRThree)}{F}
%
\Initial{A}
% \Final[n]{B}
\Final[e]{C}
% \Final{D}\Final[s]{E}\Final[s]{F}
% transitions
\ArcL{A}{C}{a}
% \ArcL{B}{C}{a}
\ArcL{C}{E}{a}
% \ArcL{D}{E}{a}
\ArcL{E}{A}{a}
% \ArcL{F}{A}{a}
%
\EdgeL{A}{E}{b}
% \ArcL{F}{E}{b}
\EdgeL{E}{C}{b}
% \ArcL{D}{C}{b}
\EdgeL{C}{A}{b}
% \ArcL{B}{A}{b}
%
\end{VCPicture}
}
% \end{center}
\caption{The `double rings' $\Hc_{6}$ and \file{double-3-1.xml}}
\label{fig:H6-app}
\end{figure}

\SetTwClPrm{\TwClOne}%


\subsubsection{Program \file{ladybird}}

\begin{SwClCmd}
\begin{shell}
$ \kbd{ladybird 6  > ladybird-6.xml}
$
\end{shell}%
\end{SwClCmd}%
\begin{SwClTxt}
    Generates an~\Prm{n} state automaton over the alphabet $\{a,b,c\}$
    whose equivalent minimal deterministic automaton has~$2^{n}$
    states.
\end{SwClTxt}%

\Spec
The
states are labelled from~\Prm{0} to~\Prm{n-1}.
State~\Prm{0} is initial and final.
The number of states~\Prm{n} is the first parameter and the next
parameters give the list of final states.
\figur{H6-app} shows the automaton built by the above command.

\Comt
The determinisation of \file{ladybird-n} has~$2^{n}$ states and is
minimal as it is co-deterministic.

\file{ladybird-n} is used in the benchmarking of \vcsn.

The automaton \file{ladybird-6.xml} (\figur{ldb-6}) is already in the
repository.

\begin{figure}[ht]
    \centering
% \begin{center}
\VCDraw[1.4]{%
\begin{VCPicture}{(-3,-3)(3,3)}
\MediumState
% etats
\State[0]{(-2,0)}{A}
\State[1]{(-1,\SQRThree)}{B}
\State[2]{(1,\SQRThree)}{C}
\State[3]{(2,0)}{D}
\State[4]{(1,-\SQRThree)}{E}
\State[5]{(-1,-\SQRThree)}{F}
%
\Initial[nw]{A}\Final[sw]{A}
% transitions
\ArcL{A}{B}{a}
\ArcL{B}{C}{a}
\ArcL{C}{D}{a}
\ArcL{D}{E}{a}
\ArcL{E}{F}{a}
\ArcL{F}{A}{a}
%
\EdgeR{F}{A}{c}
\EdgeR{E}{A}{c}
\EdgeR{D}{A}{c}
\EdgeR{C}{A}{c}
\EdgeL{B}{A}{c}
%
\LoopS{F}{b,c}
\LoopS[.75]{E}{b,c}
\LoopE{D}{b,c}
\LoopN[.75]{C}{b,c}
\LoopN{B}{b,c}
\end{VCPicture}
}
% \end{center}
\caption{The `ladybird' $\Lc_{6}$}
\label{fig:ldb-6}
\end{figure}



\section{$\Z$-automata}

\subsection{Repository}

The following automata files are stored
\file{data/automata/char-z/} directory (and accessible by the command
\command{vcsn-char-z cat}).

\subsubsection{\file{b1.xml}}

The chacteristic automaton of the automaton~$\Bc_1$ (\cf
\figur{b1-app}).
The different outcomes of functions such as
\command{power n b1.xml \bslash quotient -} on the automaton
\file{b1.xml} illustrate well the influence of the weights.

\subsubsection{\file{c1.xml} for~$\Cc_1$ }

\Comt
The series realised by~$\Cc_1$ associates every word~$w$
of~$\{a,b\}^{*}$ with its value in the binary notation, when~$a$ is
interpreted as~$0$ and~$b$ as~$1$.

\begin{figure}[ht]
    \centering
% \begin{center}
\VCDraw{%
\begin{VCPicture}{(-1.4,-1.4cm)(5.4,1.4cm)}
% etats[p][r][q]\State{(3,0)}{B}
\State{(0,0)}{A}\State{(4,0)}{C}
%
\Initial{A}\Final{C}
% transitions \EdgeL{A}{B}{a}
\EdgeL{A}{C}{b}
%
\LoopN{A}{a}\LoopS[.22]{A}{b}
\LoopN{C}{2\xmd a}\LoopS[.22]{C}{2\xmd b}
%
\end{VCPicture}}
% \end{center}
    \caption{The $\Z$-automaton $\Cc_1$ over
    $\{a,b\}^{*}$. }
\label{fig:c1-app}
\end{figure}

\subsubsection{\file{d1.xml}}

\Comt
The series realised by this automaton associates every word~$w$
of~$\{a,b\}^{*}$ with its number of `a' minus its number of `b'.

\begin{figure}[ht]
    \centering
% \begin{center}
\VCDraw{%
\begin{VCPicture}{(-1.4,-1.4cm)(5.4,1.4cm)}
% etats[p][r][q]\State{(3,0)}{B}
\State{(0,0)}{A}\State{(4,0)}{C}
%
\Initial{A}\Final{C}
% transitions \EdgeL{A}{B}{a}
\ArcL{A}{C}{a}
\ArcR{A}{C}{-b}
%
\LoopN{A}{a}\LoopS[.22]{A}{b}
\LoopN{C}{a}\LoopS[.22]{C}{b}
%
\end{VCPicture}}
% \end{center}
    \caption{The $\Z$-automaton \file{d1.xml} over
    $\{a,b\}^{*}$. }
\label{fig:d1-app}
\end{figure}



%\subsection{Factory}

%The following program is in the
%\file{data/automata/char-z/} directory.


%\subsubsection{Program \file{rem\_divkbaseb}}
%\SetTwClPrm{\TwClThree}%


%\begin{SwClCmd}
%\begin{shell}
%$ \kbd{rem\_divkbaseb 4 3 > rem-div4base3.xml}
%$
%\end{shell}%
%\end{SwClCmd}%
%\begin{SwClTxt}
    %Generates an automaton over the digit alphabet $\{0,\ldots,b-1\}$
    %that computes the remainder of the division by the
    %integer~\Prm{k} of the numbers written in base~\Prm{b} (\cf
    %\figur{rem-app}).
%\end{SwClTxt}%

%\Comt
%The `divisor' \file{rem-div3base10.xml} is already in the
%repository.

%\begin{figure}[ht]
    %\centering
%% \begin{center}
%\VCDraw{%
%\begin{VCPicture}{(-2,-5)(8,5)}
%% etats
%\MediumState
%\State[0]{(0,0)}{A}\State[3]{(6,0)}{C}
%\State[1]{(3,3)}{B}\State[2]{(3,-3)}{D}
%\Initial[n]{A}
%\FinalL{e}{B}{\{1\}\xmd e}%
%\FinalL{s}{C}{\{3\}\xmd e}%
%\FinalR{e}{D}{\{2\}\xmd e}%
%% transitions
%\ArcL{A}{B}{1}\ArcL{B}{A}{1}\ArcL{B}{C}{0}\ArcL{C}{B}{0}
%\ArcL{A}{D}{2}\ArcL{D}{A}{2}\ArcL{D}{C}{1}\ArcL{C}{D}{1}
%\LoopW{A}{0}\LoopE{C}{2}
%\LoopN{B}{2}\LoopS{D}{0}
%%
%\end{VCPicture}
%}
%% \end{center}
    %\caption{The `divisor' \file{rem-div4base3.xml} over
    %$\{0,1,2\}^{*}$. }
%\label{fig:rem-app}
%\end{figure}

%\SetTwClPrm{\TwClOne}%




\section{$\Zmin$-automata}
\subsection{Repository}

The following automata files are stored
\file{data/automata/char-zmin/} directory (and accessible by the command
\command{vcsn-char-zmin cat}).

\subsubsection{\file{minab.xml}}

\Comt
The series realised by this automaton associates every word~$w$
of~$\{a,b\}^{*}$ with $min$(number of `$a$', number of `$b$').

\begin{figure}[ht]
\centering
\VCDraw{%[.9]
\begin{VCPicture}{(0,-1.4)(6,1.4)}
% \SmallState
%
% \VCPut{(0,6)}{%
\State{(0,0)}{X} \State{(6,0)}{Y}
% }
\Initial[n]{X}\Final[s]{X}\Initial[n]{Y}\Final[s]{Y}
%
\LoopW{X}{\{1\}\xmd a}
\LoopE{X}{\{0\}\xmd b}
\LoopW{Y}{\{0\}\xmd a}
\LoopE{Y}{\{1\}\xmd b}
%%%
\end{VCPicture}
}%
\caption{The $\Zmin$-automaton \file{minab.xml} over $\{a,b\}^*$}
\label{fig:minab-app}
\end{figure}

\subsubsection{\file{minblocka.xml}}

\Comt
The series realised by this automaton associates every word~$w$
of~$\{a,b\}^{*}$ with the length of the smallest block of `$a$' between two `$b$'s.

\begin{figure}[ht]
\centering
\VCDraw{%[.9]
\begin{VCPicture}{(0,-1.4)(9,2)}
% \SmallState
%
% \VCPut{(0,6)}{%
\State{(0,0)}{X} \State{(4,0)}{Y}\State{(8,0)}{Z}
% }
\Initial[w]{X}\Final[e]{Z}
%
\LoopN{X}{\{0\}\xmd a}
\LoopS{X}{\{0\}\xmd b}
\LoopN[0.5]{Y}{\{1\}\xmd a}
\LoopN[0.75]{Z}{\{0\}\xmd a}
\LoopS[0.75]{Z}{\{0\}\xmd b}

\EdgeL{X}{Y}{\{0\}\xmd b}
\EdgeL{Y}{Z}{\{0\}\xmd b}
%%%
\end{VCPicture}
}%
\caption{The $\Zmin$-automaton \file{sag.xml} over $\{a,b\}^*$}
\label{fig:minblocka-app}
\end{figure}

\subsubsection{\file{slowgrow.xml}}

\Comt
The smallest word associated with the value $n$ is $abaab\cdots a^{(n-1)} b a^n b$.

\begin{figure}[ht]
\centering
\VCDraw{%[.9]
\begin{VCPicture}{(0,-1.4)(9,1.8)}
% \SmallState
%
% \VCPut{(0,6)}{%
\State{(0,0)}{X} \State{(4,0)}{Y}\State{(8,0)}{Z}
% }
\Initial[w]{X}\Final[e]{Z}
%
\LoopN{X}{\{0\}\xmd a}
\LoopS{X}{\{0\}\xmd b}
\LoopN[0.5]{Y}{\{1\}\xmd a}
\LoopN[0.75]{Z}{\{0\}\xmd a}


\EdgeL{X}{Y}{\{0\}\xmd b}
\ArcL[0.5]{Y}{Z}{\{0\}\xmd b}
\ArcL[0.5]{Z}{Y}{\{1\}\xmd a}
%%%
\end{VCPicture}
}%
\caption{The $\Zmin$-automaton \file{sag.xml} over $\{a,b\}^*$}
\label{fig:slowgrow-app}
\end{figure}

\section{$\Zmax$-automata}
\subsection{Repository}

The following automata files are stored
\file{data/automata/char-zmax/} directory (and accessible by the command
\command{vcsn-char-zmax cat}).

\subsubsection{\file{maxab}}
\Comt
The series realised by this automaton associates every word~$w$
of~$\{a,b\}^{*}$ with $max$(number of `$a$', number of `$b$') (\cf \figur{minab-app}).


\subsubsection{\file{maxblocka.xml}}

\Comt
The series realised by this automaton associates every word~$w$
of~$\{a,b\}^{*}$ with the length of the greatest block of `$a$'.

\begin{figure}[ht]
\centering
\VCDraw{%[.9]
\begin{VCPicture}{(0,-1.4)(9,2)}
% \SmallState
%
% \VCPut{(0,6)}{%
\State{(0,0)}{X} \VCPut[22.5]{(4,0)}{\State{(0,0)}{Y}}\State{(8,0)}{Z}
% }
\Initial[w]{X}\Final[e]{Z}
\Initial[sw]{Y}\Final[s]{Y}
%
\LoopN{X}{\{0\}\xmd a}
\LoopS{X}{\{0\}\xmd b}
\LoopN[0.5]{Y}{\{1\}\xmd a}
\LoopN[0.75]{Z}{\{0\}\xmd a}
\LoopS[0.75]{Z}{\{0\}\xmd b}

\EdgeL{X}{Y}{\{0\}\xmd b}
\EdgeL{Y}{Z}{\{0\}\xmd b}
%%%
\end{VCPicture}
}%
\caption{The $\Zmin$-automaton \file{maxblocka.xml} over $\{a,b\}^*$}
\label{fig:maxblocka-app}
\end{figure}

\section{$\B$-\fmpts}

\subsection{Repository}

The following automata files are stored
\file{data/automata/char-fmp-b/} directory (and accessible by the command
\command{vcsn-char-fmp-b cat}).

\subsubsection{\file{t1.xml} for~$\Tc_{1}$}

\begin{figure}[ht]
\centering
\VCDraw{%[.9]
\begin{VCPicture}{(-3,-1.4)(3,1.4)}
% \SmallState
%
% \VCPut{(0,6)}{%
\State{(0,0)}{X}\State{(3,0)}{Y}\State{(-3,0)}{Z}%
% }
\Initial[n]{X}\Final[s]{X}\Final[s]{Z}
%
\ArcR{X}{Z}{\IOL{a}{1}}\ArcR{X}{Y}{\IOL{1}{x}}
\ArcR{Y}{X}{\IOL{b}{1}}\ArcR{Z}{X}{\IOL{1}{y}}
%%%
\end{VCPicture}
}%
\caption{The \fmpt $\Tc_{1}$ over $\{a,b\}^{*}\x\{x,y\}^{*}$
(\cf \protect\sbsct{fmp-com-E}).}
\label{fig:t1-app}
\end{figure}

\subsubsection{\file{u1.xml} for~$\Uc_{1}$}


\begin{figure}[ht]
\centering
\VCDraw{%[.9]
\begin{VCPicture}{(-3,-1.4)(3,1.4)}
% \SmallState
%
% \VCPut{(0,6)}{%
\State{(0,0)}{X}\State{(3,0)}{Y}\State{(-3,0)}{Z}%
% }
\Initial[n]{X}\Final[s]{X}\Initial[n]{Z}
%
\ArcR{Z}{X}{\IOL{x}{1}}\ArcR{X}{Y}{\IOL{y}{1}}
\ArcR{Y}{X}{\IOL{1}{u}}\ArcR{X}{Z}{\IOL{1}{v}}
%%%
\end{VCPicture}
}%
\caption{The \fmpt $\Uc_{1}$ over $\{x,y\}^{*}\x\{u,v\}^{*}$
(\cf \protect\sbsct{fmp-com-E}).}
\label{fig:u1-app}
\end{figure}


\subsection{Factory}

The following program is in the
\file{data/automata/char-fmp-b/} directory.


\subsubsection{Program \file{quotkbaseb}}
% \SetTwClPrm{\TwClThree}%


\begin{SwClCmd}
\begin{shell}
$ \kbd{quotkbaseb 3 2 > quot3base2.xml}
$
\end{shell}%
\end{SwClCmd}%
\begin{SwClTxt}
    Generates an \fmpt over the digit alphabets $\{0,\ldots,b-1\}$
    that computes the \emph{integer quotient} of the division by the
    integer~\Prm{k} (first parameter) of the numbers written in
    base~\Prm{b} (second parameter).
\end{SwClTxt}%

\Comt
The `divisor' \file{quot3base2.xml} (\cf
    \figur{quo-app}) is already in the
repository.

\begin{figure}[ht]
    \centering
% \begin{center}
\VCDraw{%
\begin{VCPicture}{(-1.5,-1.4)(7.5,1.4)}
% etats[\zold][\uold][\dold]
\MediumState
\State{(0,0)}{A}
\State{(3,0)}{B}
\State{(6,0)}{C}
%
\Initial[n]{A}\Final[s]{A}
% transitions
\ArcL{A}{B}{\IOL{1}{0}}\ArcL{B}{A}{\IOL{1}{1}}
\ArcL{B}{C}{\IOL{0}{0}}\ArcL{C}{B}{\IOL{0}{1}}
%
\LoopW{A}{\IOL{0}{0}}\LoopE{C}{\IOL{1}{1}}
%
\end{VCPicture}
}
% \end{center}
    \caption{The `divisor' \file{quot3base2.xml} over
    $\{0,1,2\}^{*}$. }
\label{fig:quo-app}
\end{figure}

\subsubsection{Program \file{ORR}}
% \SetTwClPrm{\TwClThree}%


\begin{SwClCmd}
\begin{shell}
$ \kbd{ORR abb baa fibred}
$
\end{shell}%
\end{SwClCmd}%
\begin{SwClTxt}
    Generates two \fmpts over the alphabets guessed from the two patterns (first two parameters).
    The first transducer is left-sequential; it realises the replacement of the
    first pattern by the second in a left to right reading of the input. The
    second is right-sequential; it realises the replacement of the first pattern by the second
    in a right to left reading of the input.

    The third parameter of the program, completed by \code{\_left} and \code{\_right}, gives the
    name of the two transducers.

\end{SwClTxt}%

\Comt
The \fmpts \file{fibred\_left.xml} and \file{fibred\_right.xml} are already in the repository.



\SetTwClPrm{\TwClOne}%

\endinput
