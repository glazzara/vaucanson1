\chapter{Algorithm specification, \protect\\
\ee description and discussion}
\label{app:alg-spe}

\section{General automata and rational expressions functions }
\label{sec:aut-fct-spe-A}

\subsection{Graph functions}
\label{sec:gra-fct-A}

\addtocounter{subsubsection}{-1}
\subsubsection{\Fct{reverse}}

This is a hidden (and ancillary) graph function, not accessible to the user through 
\tafkit (because it would be somewhat confusing with \Fct{transpose}). 
It builds the transpose of the graph including the initial and final 
function that can be seen as labels of arcs from subliminal to real 
states, but leaves the labels untouched. 

\subsubsection{\Fct{accessible}, \Fct{coaccessible}, \Fct{trim}}

Graph traversal. Implemented by breadth-first search.

\longonly{%
Even this simple algorithm is to be described with more details:
whether it is in-place or not, \etc.
This discussion is not relevant for \vcsnv indeed, but for \vcsn~2.0.
}%

% \begin{enumerate}
%     
% \addtocounter{enumi}{-1}
% \item \Fct{transpose-g}
% 
% 
% \item \Fct{accessible}, \Fct{coaccessible}, \Fct{trim}
% 
% 
% The algorithm should access \emph{entries} rather than 
% \emph{transitions}.
% 
% \Fctp{coaccessible} = \Fctq{transpose-g}{\Fctq{accessible}{\Fctp{transpose-g}}}
% 
% \Fctp{trim} = \Fctq{coaccessible}{\Fctp{accessible}}
% 
% Complexity should be  proportional to the size \emph{of the result}.
% This implies that tags `number of states' and `number of transitions' 
% but attached to automata, and updated by every function that modify 
% these parameters. 
% 
% This can be considered too heavy, in which case these functions will have 
% a complexity proportional to the size \emph{of the input}.
% 
% \item \Fct{is-accessible}, \Fct{is-coaccessible}, \Fct{is-trim}
% 
% Boolean answer. Silent in the non-verbose mode.
% 
% Raise the question of having flag tags that could be used in trusted 
% components.
% 
% \item \Fct{is-empty}, \Fct{is-useless}
% 
% cf. la note : sur la nature du vide. Introuvable maintenant sur le 
% trac.
% 
% \end{enumerate}


\subsection{Transformations of automata}

\subsubsection{\Fct{proper}}
\label{ssc:aut-pro-A}%


From a theoretical point of view, the algorithm \Fct{proper} cannot 
be described, nor  
understood, before addressing the problem of the star in a semiring 
of series. 

\thnu
If~$M$ is graded, then~$\KM$, equipped with the Cauchy product, is a 
semiring as well.\footnote{%
   If~$M$ is not graded, this may not be the case 
   anymore, but is out of the scope of \vcsn for the time being, and 
   for certain while.}


If~$\T$ is a semiring, and~$t$ is in~$\T$, by definition
\begin{equation}
    t^{*}=\sum_{n\in \N} t^{n}
    \notag
%     \label{}
\end{equation}
and as infinite sums are not always defined, $t^{*}$ is not always 
defined.
Hence a semiring should be equipped with two supplementary methods 
(supplementary to the defining operations of the semiring)
\Fctp{is-starable} and \Fctp{star}, with obvious meaning and result.

If~$s$ is a series in~$\KM$, we denote by~$\TermCst{s}$ its \emph{constant 
term}, that is, the coefficient of~$\unM$.
Thus, a series~$s$ is {proper} if its constant term is nul:
$\TermCst{s}=\zeK$.
And the \emph{proper part} of an arbitrary series~$s$ is the proper 
series~$\spr$ such that 
$s = \TermCst{s}\xmd\unM + \spr$.
Under a natural, and not restrictive, hypothesis on~$\K$ (\cf 
\cite{Saka03,Saka09}), the following property holds.

\begin{property}
    A series~$s$ in~$\KM$ is starable if, and only if, $\TermCst{s}$ 
    is starable and it holds:
    $ s^{*}= (\TermCst{s})^{*}\left(\spr\xmd(\TermCst{s})^{*}\right)^{*}$
\end{property}

As a conclusion to this paragraph, we can say that star is not always 
defined in~$\K$, and thus in~$\KM$. 

\thnd
Let~$\Ac$ be an automaton over~$M$ with multiplicityin~$\K$.
We say that the \emph{behaviour of}~$\Ac$, $\CompAuto{\Ac}$, is 
defined if, and only if, for every pair of states~$p$ and~$q$ 
in~$\Ac$, the family of labels of computations from~$p$ to~$q$ is 
\emph{summable}.

Let~$\Ac_{0}$ be the automaton obtained 
from~$\Ac$ by retaining the transitions labelled by~$\unM$ only.
We then have:
\begin{property}
    The behaviour of~$\Ac$ is defined if, and only if, 
    the behaviour of~$\Ac_{0}$ is.
\end{property}


\thnt
Let
$\msp\Ac=\aut{I,E,T}\msp$
and
$\msp\Ac_{0}=\aut{I,E_{0},T}\msp$
be their respective matrix description.
We write~$\PartProp{E}$ for the proper matrix such that
$\msp E= \PartProp{E}+E_{0}\msp$.

\begin{property}
    If the behaviour of~$\Ac$ is defined, we have:
    \begin{equation}
        \CompAuto{\Ac} = 
        I\cdot 
        \left({E_{0}}^{*}\cdot\PartProp{E}\right)^{*}\cdot{E_{0}}^{*}
        \cdot T
        \eqpnt
        \notag
%         \label{}
    \end{equation}
\end{property}

It is important to note that it is not true that~$\CompAuto{\Ac}$ is 
necessarily defined when~${E_{0}}^{*}$, and thus
$\msp I\cdot 
        \left({E_{0}}^{*}\cdot\PartProp{E}\right)^{*}\cdot{E_{0}}^{*}
        \cdot T
\msp$
are defined (\cf \cite{Saka03,Saka09} for more details and example).

The algorithm, whose implementation depends indeed on~$\K$, has the 
double goal of deciding if the behaviour of~$\Ac_{0}$ is defined and 
of computing~${E_{0}}^{*}\cdot\PartProp{E}$ and~${E_{0}}^{*}\cdot T$.
It will be described in \cite{LombSakaXX}.


\subsubsection{\Fct{standardize}}
\label{ssc:aut-sta-A}%

An automaton is
said to be \emph{standard} if it has a \emph{unique initial state} which is the
destination of no transition and whose \emph{initial multiplicity} is equal to
the \emph{unit} (of the multiplicity semiring).

Not only every automaton is equivalent to a standard one, but a simple
procedure, called `standardization', transforms every 
automaton~$\Ac$ in an equivalent standard one.
The difficulty in specifying standardization comes from the fact that 
on the one hand side a standard automaton is not necessarily proper 
nor accessible and on the other the initial function of a state may a 
priori be any polynomial. 


The procedure goes as follows.

\thi Add a new state~$s$ , make it initial, with initial multiplicity equal to
the {unit} of the multiplicity semiring.

\thii For every initial state~$i$ of~$\Ac$ , with initial function~$I(i)$ , 
add a transition from~$s$ to~$i$ with label~$I(i)$, and set~$I(i)$ 
to~$\zeK$
(the zero of the multiplicity semiring) --- which is equivalent to 
say that~$i$ is \emph{not initial anymore}.

\thiii Suppress  by a \emph{backward closure}
every \emph{spontaneous transition} that has been created in~(ii).

By \emph{convention}, we consider that a transition from~$s$ to~$i$ 
is spontaneous if~$I(i)$ is \emph{scalar}, that is, if the support 
of~$I(i)$, seen 
as a polynomial over~$\Ae$, is restricted to the 
identity~$\unAe$.  

\thiv Remove the former initial states of~$\Ac$ that are the 
destination of no incoming transition.

\Comt 
\tha Steps~(iii) and~(iv) are necessary to insure the following property:

\emph{The standardization of a standard automaton~$\Ac$ is isomorphic to~$\Ac$}.

\thb We say `by convention' in~(iii) as we could have chosen 
different policies without loosing the above property (which is in 
the specification of \Fct{standardize}).

-- A non-proper polynomial~$I(i)$ could give rise to a spontaneous 
transition labelled with its constant term.
We prefered not to do it in order to change as few things as possible.

-- We could have decided to perform no closure  as soon as there 
exists at least one initial function which is not scalar.
We have prefered to have a choice which is more local to every intial 
state, but this is certainly disputable.


\subsection{Operations on automata}
\label{sec:ope-on-aut-A}


A small sketch is worth a long speech.

Let $\Ac=\bra{Q,A,E,\{i\},T}$ and $\Bc=\bra{R,A,F,\{j\},U}$ be two standard automata: 


\FixVCScale{.5}
\ChgEdgeLabelScale{.8}
% \ShowFrame 
% \qquad\qquad
\medskip 
\noindent 
~\hfill
\VCDraw{%
  \begin{VCPicture}{(-5,-2.5)(5,2.5)}
    \psellipse[linecolor=gray](.8,0)(3.3,2)
    \State[i]{(-4,0)}{I}
    \State[p]{(-1,1)}{P}
    \State[q]{(-1,-1)}{Q}
    \State[r]{(3,.5)}{R}
                                %
    \Initial{I}
                                %
    \FinalR{s}{I}{T_i}
    \FinalR{e}{Q}{T_q}
    \FinalL{e}{R}{T_r}
                                %
    \EdgeL{I}{P}{E_{i,p}}
    \EdgeR{I}{Q}{E_{i,q}}
    \VCPut{(1,0)}{\scalebox{3}{$\Ac$}}
  \end{VCPicture}
}
\hfill\hfill
\VCDraw{%
  \begin{VCPicture}{(-5,-2.5)(5,2.5)}
    \psellipse[linecolor=gray](.8,0)(3.3,2)
    \State[j]{(-4,0)}{I}
    \State[s]{(-1,1)}{P}
    \State[t]{(-1,-1)}{Q}
    \State[u]{(3,.5)}{R}
                                %
    \Initial{I}
                                %
    \FinalR{s}{I}{U_j}
    \FinalR{e}{Q}{U_t}
    \FinalL{e}{R}{U_u}
                                %
    \EdgeL{I}{P}{F_{j,s}}
    \EdgeR{I}{Q}{F_{j,t}}
    \VCPut{(1,0)}{\scalebox{3}{$\Bc$}}
  \end{VCPicture}
}
\hfill
~


\subsubsection{\Fct{union}}

Just the union of the two automata. It is a graph function indeed.

\subsubsection{\Fct{sum}}
\label{ssc:aut-sta-sum-A}

\Prec
\Prm{a.xml} and \Prm{b.xml} are \emph{standard} 
for the sum operation is defined only on standard automata.

\Spec 

\noindent 
$\bullet$ The standard automaton $\Ac\autplus\Bc=\bra{Q\cup R\setminus\{j\},A,G,\{i\},V}$ is defined as:
\\
\begin{minipage}{7cm}
\begin{align*}
\forall p,q\in & Q\cup R\setminus\{j\}\, ,\\
\ G_{p,q}&=
\begin{cases}
E_{p,q}&\quad {if}\quad p,q\in Q\\
F_{p,q}&\quad {if}\quad p,q\in R\\
F_{j,q}&\quad {if}\quad p=i\text{ and }q\in R\\
0_\K &\quad {otherwise}
\end{cases}\\[.5em]
%\end{align*}
%\end{minipage}
%\begin{minipage}{6cm}
%\begin{align*}
\forall p\in& Q\cup R\setminus\{j\}\, ,\\
\ V_p&=
\begin{cases}
T_i\oplus U_j &\quad {if}\quad p=i\\
T_p&\quad {if}\quad p\in Q\setminus\{i\}\\
U_p&\quad {if}\quad p\in R
\end{cases}
\end{align*}
\end{minipage}
\qquad
\VCDraw{%
  \begin{VCPicture}{(-7,-7)(5,2.5)}
    \psellipse[linecolor=gray](.8,-2.25)(4.5,4.5)
    \psellipse[linecolor=lightgray](.8,0)(3.3,2)
    \State{(-5,-2.25)}{I}
    \State[p]{(-1,1)}{P}
    \State[q]{(-1,-1)}{Q}
    \State[r]{(3,.5)}{R}
                                %
    \Initial{I}
                                %
    \FinalR{s}{I}{T_i\oplus U_j}
    \FinalR{e}{Q}{T_q}
    \FinalL{e}{R}{T_r}
                                %
    \ArcL{I}{P}{E_{i,p}}
    \ArcL{I}{Q}{}\LabelR{E_{i,q}}
    \VCPut{(0,-4.5)}{%
      \psellipse[linecolor=lightgray](.8,0)(3.3,2)
      \State[s]{(-1,1)}{S}
      \State[t]{(-1,-1)}{T}
      \State[u]{(3,.5)}{U}
                                %
      \FinalR{e}{T}{U_t}
      \FinalL{e}{U}{U_u}
                                %
      \ArcR{I}{S}{}\LabelL{F_{j,s}}
      \ArcR{I}{T}{F_{j,t}}
    }
    \VCPut{(1,-2.25)}{\scalebox{3}{$\Ac\autplus\Bc$}}
  \end{VCPicture}}

\subsubsection{\Fct{concatenate}}
\label{ssc:aut-sta-con-A}

\Prec
\Prm{a.xml} and \Prm{b.xml} are \emph{standard} 
for the concatenation operation is defined only on standard automata.

\Spec 
\noindent 
$\bullet$ The standard automaton 
$\Ac\autprod\Bc=\bra{Q\cup R\setminus\{j\},A,G,\{i\},V}$ is:
\\
\begin{minipage}[t]{6cm}
\begin{align*}
\forall p,q\in & Q\cup R\setminus\{j\}\, ,\\
\ G_{p,q}&=
\begin{cases}
E_{p,q}&\quad {if}\quad p,q\in Q\\
F_{p,q}&\quad {if}\quad p,q\in R\\
T_pF_{j,q}&\quad {if}\quad p\in Q\text{ and }q\in R\\
0_\K &\quad {otherwise}
\end{cases}
\end{align*}
\end{minipage}
\quad 
\begin{minipage}[t]{6cm}
\begin{align*}
\forall p\in& Q\cup R\setminus\{j\}\, ,\\
\\
\ V_p&=
\begin{cases}
U_p &\quad {if}\quad p\in R\\
T_p\,U_j&\quad {if}\quad p\in Q
\end{cases}
\end{align*}
\end{minipage}

\begin{center}
\VCDraw{%
  \begin{VCPicture}{(-5,-4)(14,4)}
    \psellipse[linecolor=gray](5.3,0)(8.5,3.5)
    \psellipse[linecolor=lightgray](.8,0)(3.3,2)
    \State[i]{(-4,0)}{I}
    \State[p]{(-1,1)}{P}
    \State[q]{(-1,-1)}{Q}
    \State[r]{(3,.5)}{R}
                                %
    \Initial{I}
                                %
    \FinalR{s}{I}{T_i\,U_j}
    \FinalR[.5]{n}{Q}{T_q\,U_j}
    \FinalL[.5]{n}{R}{T_r\,U_j}
                                %
    \VArcL{arcangle=50,ncurv=1}{I}{P}{E_{i,p}}
%{I}{P}{}\LabelR{E_{i,p}}
    \ArcR{I}{Q}{}\LabelL{E_{i,q}}
    \VCPut{(9,0)}{%
      \psellipse[linecolor=lightgray](.8,0)(3.3,2)
      \State[s]{(-1,1)}{S}
      \State[t]{(-1,-1)}{T}
      \State[u]{(3,.5)}{U}
                                %
      \FinalR{e}{T}{U_t}
      \FinalL{e}{U}{U_u}
    }
                                %
    \LArcL[.8]{I}{S}{T_i\,F_{j,s}}
    \LArcR[.8]{I}{T}{T_i\,F_{j,t}}
    \EdgeL[.3]{Q}{S}{T_q\,F_{j,s}}
    \EdgeR{Q}{T}{T_q\,F_{j,t}}
    \EdgeL{R}{S}{T_r\,F_{j,s}}
    \EdgeL[.8]{R}{T}{T_r\,F_{j,s}}
                                %
    \VCPut{(9,-2.25)}{\scalebox{3}{$\Ac\autprod\Bc$}}
  \end{VCPicture}}
\end{center}

\bigskip 
\subsubsection{\Fct{star}}
\label{ssc:aut-sta-sta-A}

\Prec
\Prm{a.xml} is \emph{standard} 
for the star operation is defined only on standard automata.

\Spec

\noindent 
$\bullet$ The standard automaton $\autstar{\Ac}=\bra{Q,A,E^{(\autstarsymb)},\{i\},T^{(\autstarsymb)}}$ is:
\\\begin{minipage}{6cm}
\begin{align*}
\forall p,q\in & Q,\\
\ E^{(\autstarsymb)}_{p,q}&=
\begin{cases}
T_i^*E_{i,q}&\quad {if}\quad p=i\\
T_pT_i^*E_{i,q}\oplus E_{p,q}&\quad {otherwise}
\end{cases}\\
%\end{align*}
%\end{minipage}
%\begin{minipage}{6cm}
%\begin{align*}
\forall p\in& Q,\\
\ T^{(\autstarsymb)}_p&=
\begin{cases}
T_i^*&\quad {if}\quad p=i\\
T_pT_i^*&\quad {otherwise}
\end{cases}
\end{align*}
\end{minipage}
\qquad\qquad
\VCDraw{%
  \begin{VCPicture}{(-7,-3)(5,3)}
    \psellipse[linecolor=gray](.8,0)(3.3,2.5)
    \State[i]{(-5,0)}{I}
    \State[p]{(-1,1)}{P}
    \State[q]{(-1,-1)}{Q}
    \State[r]{(3,.5)}{R}
                                %
    \Initial{I}
                                %
    \FinalR{s}{I}{T_i^*}
    \FinalL{se}{Q}{T_qT_i^*}
    \FinalL{e}{R}{T_rT_i^*}
                                %
    \ArcL{I}{P}{T_i^*\,E_{i,p}}
    \ArcR{I}{Q}{T_i^*\,E_{i,q}}
    \VArcR{arcangle=-130,ncurv=1}{R}{P}{T_r\,T_i^*\,E_{i,p}}
    \VArcL{arcangle=130,ncurv=1.1}{R}{Q}{T_r\,T_i^*\,E_{i,q}}
    \VCurveL{angleA=80,angleB=-160}{Q}{P}{T_q\,T_i^*\,E_{i,p}}
    \CLoopR[.5]{20}{Q}{T_q\,T_i^*\,E_{i,q}}
    \VCPut{(1,1)}{\scalebox{3}{$\autstar{\Ac}$}}
  \end{VCPicture}
}


\bigskip 
\subsubsection{\Fct{left-mult}}
\label{ssc:aut-sta-lft-mlt-A}

\Prec
\Prm{a.xml} is \emph{standard} for the left 
`exterior' multiplication
operation is defined only on standard automata. 

\Spec

\noindent 
$\bullet$ The standard automata $k\Ac=\bra{Q,A,E^{(k.)},\{i\},T^{(k.)}}$ is defined by:
\\
\begin{minipage}{8cm}
\begin{align*}
\forall p,q\in Q\, ,\qquad E^{(k.)}_{p,q}&=
\begin{cases}
k\,E_{p,q}&\quad {if}\quad p=i\\
E_{p,q}&\quad {otherwise}
\end{cases}\\
\forall p\in Q\, ,\qquad T^{(k.)}_p&=
\begin{cases}
k\,T_p&\quad {if}\quad p=i\\
T_p&\quad {otherwise}
\end{cases}
\end{align*}
\end{minipage}
\qquad
\VCDraw{%
  \begin{VCPicture}{(-5,-2.5)(5,2.5)}
    \psellipse[linecolor=gray](.8,0)(3.3,2)
    \State[i]{(-4,0)}{I}
    \State[p]{(-1,1)}{P}
    \State[q]{(-1,-1)}{Q}
    \State[r]{(3,.5)}{R}
                                %
    \Initial{I}
                                %
    \FinalR{s}{I}{k\,T_i}
    \FinalR{e}{Q}{T_q}
    \FinalL{e}{R}{T_r}
                                %
    \EdgeL{I}{P}{k\,E_{i,p}}
    \EdgeR{I}{Q}{k\,E_{i,q}}
    \VCPut{(1,0)}{\scalebox{3}{$k\Ac$}}
  \end{VCPicture}}


  \bigskip 
\subsubsection{\Fct{right-mult}}
\label{ssc:aut-sta-rgt-mlt-A}

\Prec
\Prm{a.xml} is \emph{standard} for the right 
`exterior' multiplication
operation is defined only on standard automata. 

\Spec
\noindent 
$\bullet$ The standard automata $\Ac k=\bra{Q,A,E,\{i\},T^{(.k)}}$ is defined by:
\\
\begin{minipage}{8cm}
\begin{equation*}
\forall p\in Q\, , \qquad T^{(.k)}_p=T_p\,k
\qquad \qquad 
\end{equation*}
\end{minipage}
\qquad
\VCDraw{%
  \begin{VCPicture}{(-5,-2.5)(5,2.5)}
    \psellipse[linecolor=gray](.8,0)(3.3,2)
    \State[i]{(-4,0)}{I}
    \State[p]{(-1,1)}{P}
    \State[q]{(-1,-1)}{Q}
    \State[r]{(3,.5)}{R}
                                %
    \Initial{I}
                                %
    \FinalR{s}{I}{T_i\,k}
    \FinalR{e}{Q}{T_q\,k}
    \FinalL{e}{R}{T_r\,k}
                                %
    \EdgeL{I}{P}{E_{i,p}}
    \EdgeR{I}{Q}{E_{i,q}}
    \VCPut{(1,0)}{\scalebox{3}{$\Ac k$}}
  \end{VCPicture}}

\bigskip 

\subsection{From automata to expressions}
\label{ssc:aut-to-exp-A}

\vcsn implements the \emph{state elimination method} for computing 
the rational expression that denotes the behaviour of an automaton.
\index{state elimination method}%
The outcome of the algorithme depends upon the order in which the 
states are `eliminated'.

In \vcsn library, this order of states is given to the fonction as a 
second parameter called `\code{chooser}'.
The \code{chooser} either runs over a list of states that is given 
explicitely, or implements a heuristics that computes at each step 
the next state to be suppressed.
Two heuristics are implemented in the library: the \emph{'na{\"\i}ve 
heuristics} which is described below, and a variant of it which takes 
into account not only the number of transitions incident to every 
given state, but also the length of the expressions that label these 
transitions it is due to Delgado and Morais and described in 
\cite{DelgMora04}.

Note that in any case and for a precise specification (in view of the derivation 
procedure in particular), one should specify the bracketting:
\begin{equation}
    p\path{\Fmsf}q\path{\Gmsf}q\path{\Hmsf}r
    \e \text{gives}\e
    p\path{(\Fmsf\cdot\Gmsf^{*})\cdot\Hmsf}r
\label{equ:eli-met}
\end{equation}
after the elimination of the state~$q$.


\subsubsection{The `na{\"\i}ve' heuristics}

\tha Make real the initial and final subliminal states~$i$ and~$t$. 
From~$i$ to every initial state~$p$, there is thus a transition with 
label~$I(p)$.
Dually, from every final state~$r$ to~$t$, there is thus a transition with 
label~$T(r)$.

\thb For every state~$p$ (outside~$i$ and~$t$) compute a two component 
index~$(l(p),k(p))$:\\
\tiret $l(p)=1$ if~$p$ is the origin of a loop, $l(p)=0$ otherwise;\\
\tiret $k(p)= [i(p)-1][o(p)-1]$ where~$i(p)$ is the in-degree of~$p$ 
and~$o(p)$ its out-degree.\\
\tiret Lexicographically order the states by their index.

\thc While there remains states,\\
\tiret choose the state~$q$ with smallest index,\\
\tiret remove it and replace the incoming and outgoing transitions 
according to~\equnm{eli-met},\\
\tiret recompute the index for those states 
that were adjacent to~$q$.

\thd Return the label of the transition from~$i$ to~$t$.


\longonly{%
\begin{ComVd}{101206}
        
\thi It is to be carefully checked whether the \code{NHChooser} and 
the \code{DMChooser} correctly implement the na{\"\i}ve and the 
Delgado--Morais heuristics respectively.


\thii I do not see why we put 
$k(p)= [i(p)-1][o(p)-1]$ rather than $k(p)= [i(p)][o(p)]$ in the 
description of the `na{\"\i}ve heuristics. 
\end{ComVd}
}%

% \thiii The implementation of the algorithm probably implies that the 
% automaton is transformed into a \code{labels-are-series} automaton, 
% whatever its native kind. 
% 

% \addtocounter{section}{1}
\section{Weighted automata and rational expressions over free monoids}
\label{sec:aut-fre-mul-spe}

\addtocounter{subsection}{1}
% \subsection{Transformations of automata}

% \addtocounter{subsubsection}{1}
% \subsubsection{\Fct{quotient}}
% \label{ssc:aut-mul-quo-A}%
% 
% As announced, a variant of Hopcroft's algorithm.
% 
% \begin{ComVd}{101205}
%     M�rite d'�tre d�crit, au moins dans les grandes lignes.
% \end{ComVd}

\subsection{Behaviour of automata}
\label{ssc:aut-mul-beh-A}%

\subsubsection{\Fct{eval}}
\label{ssc:evl-wrd-A}%

As the automaton~$\Ac$ implemented by \code{a.xml} is supposed to be 
\emph{realtime}, it is described by a representation~$\lmn$.
The coefficient of a word~$w$ in the series~$\CompAuto{\Ac}$ realised 
by~$\Ac$ is 
$\msp\lambda\matmul\mu(w)\matmul\nu\msp$
(\cf \cite[Sect.~III.3]{Saka03}).
The vector $\msp\lambda\matmul\mu(w)\msp$ is computed by induction 
on~$\ell=\lgt{w}$, the length of~$w$.
The overall complexity of the algorithm 
is~$\mathrm{O}(\ell\xmd d^{2})$ where~$d$ is the dimension of~$\Ac$.





% \addtocounter{section}{1}
\section{Automata and rational expressions on free monoids with 
weights in a field}
\label{sec:aut-fre-fld-A}

\subsection{Operations on automata}

\subsubsection{Reduction of representations over a field}


\paragraph{Automata and representation}~

Any finite automaton over~$\Ae$ with multiplicity in~$\K$ is 
equivalent to a \emph{realtime} automaton~$\Ac$ with set of 
states~$Q$:
$\msp \Ac=\aut{I,E,T}\msp$ 
where~$I$ and~$T$ are vectors in~$\K^{Q}$ and~$E$ is a square matrix 
of dimension~$Q$, whose entries are \emph{linear combination} with 
coefficients in~$\K$ \emph{of letters in}~$A$.
One can then write:
\begin{equation}
    E = \sum_{a \in A} a\mu a
    \notag
%     \label{}
\end{equation}
where every~$a\mu$ is a square matrix of dimension~$Q$ with entries 
in~$\K$.
These matices define a morphism~$\mu$ from~$\Ae$ into~$\K^{Q\x Q}$, 
and for every~$w$ in~$\Ae$ the coefficient of~$w$ in the series~$s$ 
realised by~$\Ac$ is
$\msp\bra{s,w}=I\cdot w\mu\cdot T\msp$.
The tuple~$(I,\mu,T)$ is called a \emph{representation} (of 
dimension~$Q$) of~$s$.


\paragraph{Rational series over a field}~

If~$\K$ is a \emph{field}~$\F$, 
for every $\F$-rational series~$s$, there exists an integer~$r$, 
called the \emph{rank} of~$s$ which is the minimal dimension of any 
representation of~$s$.
A representation of minimal dimension is said to be \emph{reduced}.

\begin{theorem}[Sch\"utzenberger]
    A reduced representation of a $\F$-rational series~$s$ is 
    effectively computable from any representation of~$s$.
\end{theorem}

A reduced representation of a rational series is an object comparable 
to the minimal automaton of a rational language, to the extent it is 
not unique but defined up to a basis transformation within~$\F^{^Q}$.
The theorem implies two $\F$-automata which realize~$s$ and~$t$ are 
equivalent if, and only if the reduced representation of the 
series~$s-t$ is of dimension~$0$ and this is decidable.


\paragraph{The algorithm}~

A representation~$(I,\mu,T)$ of dimension~$Q$ being given, the 
algorithm that  
underlies the theorem amounts to find a maximal prefix-closed subset~$P$ 
of~$\Ae$ such that the vectors~$I\matmul p\mu$ are independent 
(in~$\F^{Q}$).
Such set of vectors allows in turn to compute a new and equivalent 
representation, of dimension~$P$.
The substance of the theorem is that it is sufficient to perform this 
algorithm twice in a row, on the given representation and then on its 
transpose in order to get the reduced representation.

The elementary step in this algorithm is thus to determine whether a 
given vector belongs to a subspace generated by a set of given 
independent vectors and in the positive case to compute its 
coordinates in this system, that is to solve a system of linear 
equations.
In order to reach the optimal complexity, and also to be able to 
treat the case of non-commutative fields (a case which does not 
appear in \vcsnv), these systems are solved by an iterative 
method of Gaussian elimination.


\addtocounter{section}{1}
% \section{Boolean automata and rational expressions on free monoids}
% \label{sec:aut-fre-boo-spe-A}

% \subsection{Operations on automata}
% 
% \subsubsection{\Fct{minimize}}
% \label{ssc:aut-boo-min-A}%
% 
% As announced, \vcsn implements Hopcroft's algorithm.
% 
% \begin{ComVd}{101205}
%     M�rite d'�tre d�crit, au moins dans les grandes lignes.
% \end{ComVd}

% \addtocounter{subsection}{1}
% \subsection{Operations on expressions}
% 
% \subsubsection{\Fct{derived-term}}
% \label{ssc:der-ter-A}%
% 
% % L'algorithme pour construire l'automate des termes d�riv�s bool�ens
% % dans vaucanson est le suivant:
% 
% The derived term automaton is constructed (in the Boolean case) by 
% the following algorithm (r�dig� par PYA):
% 
% % \begin{ComVd}{}
% 
% 
% \thi L'expression est transform�e en \code{prat\_exp}\\
% (\code{include/vaucanson/algorithms/internal/partial\_rat\_exp.hxx}).
% Il s'agit
% d'une liste de noeuds repr�sentant donc une concat�nation de 
% sous-expressions. En fait: 
% 
%  - $\mathtt{prat}(\Ed+\Fd) = \Ed+\Fd$
%  
%  - $\mathtt{prat}(\Ed*) = \Ed*$
%  
%  - $\mathtt{prat}(\Ed.\Fd) = \mathtt{prat}(\Ed) , \Fd$
%  
% Les �tats de l'automate vont �tre des \code{prat\_exp}, la transformation de
% l'expression initiale donne l'�tat initial, l'automate est construit
% de mani�re incr�mentale. 
% 
% \thii A chaque �tat:
% 
%  - on v�rifie le terme constant de la \code{prat\_exp} et on rend l'�tat final
%    en fonction
%    
%  - pour chaque lettre de l'alphabet on d�rive la \code{prat\_exp}\\
%    (\code{include/vaucanson/algorithms/internal/partial\_rat\_exp\_derivation.hxx})
%    et on ajoute une transition en fonction.
% 
% Les �tats sont identifi�s grace � l'op�rateur \code{==} d�fini sur les
% \code{prat\_exp} dans\\
% \code{include/vaucanson/algorithms/internal/partial\_rat\_exp.hxx} .
% % \end{ComVd}

% \addtocounter{section}{1}
\section{Weighted automata over a product of two free monoids}
\label{sec:fmp-fct-spe-A}

\addtocounter{subsection}{1}
% \subsection{Transformations of transducers}

\subsection{Operations on transducers}

\addtocounter{subsubsection}{1}

\subsubsection{\Fct{composition}, \Fct{b-composition}}
\label{ssc:fmp-com-E}%

The Composition Theorem, due to Elgot and Mezei (\cf \cite{Saka03})
is one of the basic results on rational relations and finite 
transducers. 
The composition algorithm described here has been presented 
in~\cite{ClavEtAl05}.
We describe first the algorithm that realises the \Fct{b-composition} 
and then the more sophisticated one that realises the 
\Fct{composition}. 

\paragraph{Product of normalized transducers}~

We first consider two proper \emph{normalized} transducers:
\begin{equation}
    \Tc = \aut{Q,\Ae\x \Be,E,I,T}
    \ee\text{and}\ee
    \Uc = \aut{R,\Be\x \Ce,F,J,U}
    \eqvrg
    \notag
%     \label{}
\end{equation}
% $\msp \Tc = \aut{Q,\Ae\x \Be,E,I,T}\msp $ and
% $\msp \Uc = \aut{R,\Be\x \Ce,F,J,U}\msp $,
that is, the transitions of $\Tc$ are labelled in $A\x 1$ or in $1\x B$
and those of $\Uc$ are labelled in $B\x 1$ or in $1\x C$.

The proof of the Composition Theorem
% as presented in \cite{Bers79}
is equivalent to the construction of the transducer
\begin{equation*}
     \Tc\bowtie\Uc = \aut{Q\x R,A^{*}\x C^{*},G,I\x J,T\x U}
\end{equation*}
by the following rules.

\noindent
(i) If $\msp \Tran{p,(a,1),q} \in E\msp$
then for all $r\in R$
$\msp \Tran{(p,r),(a,1),(q,r)} \in G\msp$.

\noindent
(ii) If $\msp \Tran{r,(1,u),s} \in F\msp$
then for all $q\in Q$
$\msp \Tran{(q,r),(1,u),(q,s)} \in G\msp$.

\noindent
(iii) If $\msp \Tran{p,(1,x),q} \in E\msp$ \emph{and}
$\msp (r,(x,1),s) \in F\msp$
then
$\msp \Tran{(p,r),(1,1),(q,s)} \in G\msp$.

A next possible step is to eliminate the transitions with label
$(1,1)$ by means of a classical closure algorithm.
% 
\figur{tra-pro} shows an example of such product, before and after the 
elimination of spontaneous transitions.


\begin{figure}[htbp]
     \centering
\ee
\VCDraw{%[.9]
\begin{VCPicture}{(-8,-4)(4,7)}
\SmallState
%
\VCPut{(-7,0)}{%
\State{(0,0)}{A}\State{(0,3)}{B}\State{(0,-3)}{C}
\VCPut{(-2,3)}{\scalebox{3}{$\Tc_{1}$}}
               }%
%[n][s][s]
\Initial{A}\Final{A}\Final{C}
%
\ArcR{C}{A}{\IOL{1}{y}}\ArcR{A}{B}{\IOL{1}{x}}
\ArcR{B}{A}{\IOL{b}{1}}\ArcR{A}{C}{\IOL{a}{1}}
%%%%
\VCPut{(0,6)}{%
\State{(0,0)}{X}\State{(3,0)}{Y}\State{(-3,0)}{Z}
\VCPut{(-5,0)}{\scalebox{3}{$\Uc_{1}$}}         }%
\Initial[n]{X}\Final[s]{X}\Initial[n]{Z}
%
\ArcR{Z}{X}{\IOL{x}{1}}\ArcR{X}{Y}{\IOL{y}{1}}
\ArcR{Y}{X}{\IOL{1}{u}}\ArcR{X}{Z}{\IOL{1}{v}}
\VCPut{(-3,0)}{%
\State{(0,0)}{AZ}\State{(0,3)}{BZ}\State{(0,-3)}{CZ}
          }%
\VCPut{(0,0)}{%
\State{(0,0)}{AX}\State{(0,3)}{BX}\State{(0,-3)}{CX}
          }%
\VCPut{(3,0)}{%
\State{(0,0)}{AY}\State{(0,3)}{BY}\State{(0,-3)}{CY}
          }%
%
\Initial[nw]{AX}\Final[sw]{AX}\Initial[nw]{AZ}\Final[se]{CX}
%
\ChgEdgeLabelScale{.9}
\ArcR{BZ}{AZ}{\IOL{b}{1}}
\ArcR{AZ}{CZ}{\IOL{a}{1}}
\ArcR{BX}{AX}{}\LabelL{\IOL{b}{1}}
\ArcR{AX}{CX}{\IOL{a}{1}}
\ArcR{BY}{AY}{}\LabelL{\IOL{b}{1}}
\ArcR{AY}{CY}{}\LabelL{\IOL{a}{1}}
\ArcR{AY}{AX}{\IOL{1}{u}}
\ArcR{AX}{AZ}{}\LabelL{\IOL{1}{v}}
\ArcR{BY}{BX}{\IOL{1}{u}}
\ArcR{BX}{BZ}{\IOL{1}{v}}
\ArcR{CY}{CX}{}\LabelL{\IOL{1}{u}}
\ArcR{CX}{CZ}{}\LabelL{\IOL{1}{v}}
\EdgeLineDouble
\ChgEdgeLabelScale{.8}
\EdgeL[.55]{AZ}{BX}{\IOL{1}{1}}
\EdgeL[.55]{CX}{AY}{\IOL{1}{1}}
\EdgeLineSimple
%%%
\end{VCPicture}}%
\eee
\ee
\VCDraw{%[.9]
\begin{VCPicture}{(-4,-4)(4,7)}
\SmallState
%
\VCPut{(0,6)}{\scalebox{3}{$\Tc_{1}\bowtie\Uc_{1}$}}
\VCPut{(0,0)}{%
\VCPut{(-3,0)}{%
\State{(0,0)}{AZ}\State{(0,3)}{BZ}
\DimState\State{(0,-3)}{CZ}\RstState
          }%
\VCPut{(0,0)}{%
\State{(0,0)}{AX}\State{(0,-3)}{CX}
\DimState\State{(0,3)}{BX}\RstState
          }%
\VCPut{(3,0)}{%
\State{(0,-3)}{CY}
\DimState\State{(0,0)}{AY}\State{(0,3)}{BY}\RstState
          }%
         }%
%
\Initial[ne]{AX}\Final[se]{AX}
\Initial[nw]{AZ}\Final[sw]{CX}
%
\ChgEdgeLabelScale{.9}
\ArcR{BZ}{AZ}{\IOL{b}{1}}
\DimEdge
\ArcR{AZ}{CZ}{}%{\IOL{a}{1}}
\ArcR{BX}{AX}{}%{\IOL{b}{1}}
\RstEdge
\ChgEdgeLabelScale{.9}
\ArcR[.55]{AX}{CX}{\IOL{a}{1}}
\DimEdge
\ArcR{BY}{AY}{}%{\IOL{b}{1}}
\ArcR{AY}{CY}{}%{\IOL{a}{1}}
\ArcR{AY}{AX}{}%{\IOL{1}{u}}
\RstEdge
\ChgEdgeLabelScale{.9}
\ArcR{AX}{AZ}{\IOL{1}{v}}
\DimEdge
\ArcR{BY}{BX}{}%{\IOL{1}{u}}
\ArcR{BX}{BZ}{}%{\IOL{1}{v}}
\ArcR{CX}{CZ}{}%{\IOL{1}{v}}
\RstEdge
\ChgEdgeLabelScale{.9}
\ArcR{CY}{CX}{\IOL{1}{u}}
\ArcR[.55]{AZ}{AX}{\IOL{b}{1}}
\ArcR[.55]{AZ}{BZ}{\IOL{1}{v}}
\ArcR{CX}{CY}{\IOL{a}{1}}
\ArcR[.55]{CX}{AX}{\IOL{1}{u}}
%	
\end{VCPicture}}%
\ee
\caption{Composition Theorem on Boolean transducers}
\label{fig:tra-pro}
\end{figure}



\paragraph{Product of subnormalized transducers}~

This construction can easily be extended to \emph{subnormalized} 
transducers, which are such that transitions are 
labelled in
$\msp \widehat{A}\x\widehat{B} \setminus (1,1)\msp$
where $\msp\widehat{A} = A \cup \{1\}\msp$.
It amounts to replace (iii) by

\noindent
(iii') If $\msp \Tran{p,(a',x),q} \in E\msp$ with $a'\in\widehat{A}$
\emph{and}
$\msp \Tran{r,(x,u'),s} \in F\msp$ with $u'\in\widehat{C}$\\
\ee then
$\msp \Tran{(p,r),(a',u'),(q,s)} \in G\msp$.

In this form, it contains as a particular case the composition of
letter-to-letter transducers.

\paragraph{Product of subnormalized transducers and composition}~

It is known that this construction, which works perfectly well for 
Boolean transducers, does not yields a transducer for the composition 
if the multiplicities are to be taken into account.
% Let us say that two paths in $\Tc\bowtie\Uc$ are \emph{equivalent}
% if they correspond to the same pair of paths in $\Tc$ and $\Uc$.

For instance, there is one path labeled $(aa,y)$ in $\Tc_{1}$ and one
path labeled $(y,u)$ in $\Uc_{1}$; and there are \emph{two}
 paths labeled $(aa,u)$ in $\Tc_{1}\bowtie\Uc_{1}$.
Hence,  $\Tc\bowtie\Uc$ does not realize the composition of the
weighted relations realized by $\Tc$ and $\Uc$.

\paragraph{Preparation of transducers for the composition}~

In order to have a product of transducers that realises the weighted 
composition, we performa preliminary operation on both operands that 
distinguishes between transitions and the take advantage of this 
supplementary information in order to supress some transitions in the 
product.


The construction on $\Tc$ and $\Uc$ can be described
as follows:

\tha Split the states of $\Tc$ and their \emph{outgoing}
transitions in such a way they are labeled either in $(A\x 1)$ ---
black states --- or in $\widehat{A}\x B$ (or the state is final) ---
white states; the incoming transitions are duplicated on split states.
This is transducer $\Tc'$.

\thb Split the states of $\Uc$ and their \emph{incoming}
transitions in such a way they are labeled either in $(1\x C)$ ---
black states --- or in $B\x \widehat{C}$ (or the state is initial) ---
white states; the outgoing transitions are duplicated on split states.
This is transducer $\Uc'$.

\thc Apply the preceeding algorithm [steps (i), (ii) and (iii')] to
$\Tc'$ and $\Uc'$ in order to  build $\Tc'\bowtie\Uc'$.

\thd Delete the black-black states (every state in $\Tc'\bowtie\Uc'$
is a pair of states).

\thejs Trim and eliminate the transitions with label $(1,1)$ by
classical closure.

Figure~\ref{fig:com-mul} shows the construction applied to $\Tc_{1}$ and 
$\Uc_{1}$.


\begin{figure}[htbp]
      \centering
\VCDraw[.9]{%
\begin{VCPicture}{(-9,-8)(8,9)}
\SmallState
%
\VCPut{(-7,0)}{%
\State{(0,0)}{A}
\ChgStateFillColor{lightgray}
\State{(0,3)}{B}
\State{(0,-3)}{AA}
\RstStateFillColor
\State{(0,-6)}{C}
           }%
%[n][s][s]
\Initial{A}\Initial{AA}\Final{A}\Final{C}
%
\ChgLArcAngle{50}
\ChgLArcCurvature{.9}
\ChgEdgeLabelScale{.85}
\LArcR[.7]{C}{A}{\IOL{1}{y}}
\ArcR{C}{AA}{\IOL{1}{y}}
\ArcR{A}{B}{\IOL{1}{x}}
\ArcR{B}{A}{\IOL{b}{1}}
\LArcR[.7]{B}{AA}{\IOL{b}{1}}
\ArcR{AA}{C}{\IOL{a}{1}}
%%%%
\VCPut{(0,7)}{%
\State{(0,0)}{X}
\ChgStateFillColor{lightgray}
\State{(-3,0)}{Z}
\State{(3,0)}{XX}
\RstStateFillColor
\State{(6,0)}{Y}
}%
\Initial[n]{X}\Final[s]{X}\Final[s]{XX}\Initial[n]{Z}
%
\ArcR{Z}{X}{\IOL{x}{1}}
\LArcR[.3]{X}{Y}{\IOL{y}{1}}
\ArcR{XX}{Y}{\IOL{y}{1}}
\ArcR{Y}{XX}{\IOL{1}{u}}
\ArcR{X}{Z}{\IOL{1}{v}}
\LArcR[.3]{XX}{Z}{\IOL{1}{v}}
\VCPut{(-3,0)}{%
\State{(0,0)}{AZ}
\DimState
\ChgStateFillColor{lightgray}
\State{(0,3)}{BZ}\State{(0,-3)}{AAZ}
\RstStateFillColor
\State{(0,-6)}{CZ}
\RstState
           }%
\VCPut{(0,0)}{%
\DimState\State{(0,3)}{BX}\RstState
\State{(0,0)}{AX}\State{(0,-3)}{AAX}\State{(0,-6)}{CX}
           }%
\VCPut{(3,0)}{%
\DimState
\ChgStateFillColor{lightgray}
\State{(0,3)}{BXX}\State{(0,-3)}{AAXX}
\RstStateFillColor
\RstState
\State{(0,0)}{AXX}\State{(0,-6)}{CXX}
           }%
\VCPut{(6,0)}{%
\DimState
\State{(0,0)}{AY}\State{(0,3)}{BY}
\State{(0,-3)}{AAY}
\RstState
\State{(0,-6)}{CY}
           }%
%[se]
\Initial[nw]{AZ}\Initial[ne]{AX}\Final[se]{AX}
\Initial[w]{AAX}\Final{AXX}
\Final[sw]{CX}\Final[sw]{CXX}
%
\ArcR{AX}{AZ}{\IOL{1}{v}}
\LArcR[.3]{AXX}{AZ}{\IOL{1}{v}}
\ArcR[.55]{AZ}{AX}{\IOL{b}{1}}
\ArcR[.55]{AZ}{AAX}{\IOL{b}{1}}
\ArcR{AAX}{CX}{\IOL{a}{1}}
\ArcR{CY}{CXX}{\IOL{1}{u}}
\ArcR{CXX}{CY}{\IOL{a}{1}}
\LArcR{CX}{CY}{\IOL{a}{1}}
\ArcR{CX}{AXX}{}\LabelL[.7]{\IOL{1}{u}}
\ArcR[.6]{CXX}{AXX}{\IOL{1}{u}}
%
\end{VCPicture}}
      \caption{A composition that preserves multiplicity}
      \label{fig:com-mul}
\end{figure}



% \section{Weighted automata on free monoids over alphabets of pairs}
% \label{sec:alp-pai-spe}
%%%%%%%%%%%%%%%%
\endinput






\addtocounter{section}{-1}

\section{General considerations}

\begin{ComV} (091226)
    
This section has to be written.
\end{ComV}


\subsection{Ancillary, hidden and non-public functions}
\label{ssc:anc-hidd-npb-A}

\begin{ComV} (091226)
    
A classification that is meant first for the \vcsn Group.

\emph{Hidden} means: `not accessible from \tafkit but accessible from 
the library', 
\emph{non-public} means: `not accessible from the library',
\emph{ancillary} means: `accessible from \tafkit, not of interest 
by itself but as it allows to build other functions.

As we want to test as much functions as possible from \tafkit, it may 
be the case that `hidden' and even `non-public' functions may be 
called indeed from \tafkit: the distinction being more on what is 
described to the user and what is not.
\end{ComV}


\subsection{In-place and not in-place algorithms}
\label{ssc:inp-nin-A}

This distinction is not relevant for \tafkit functions as they output 
the result in the file which is given as output file, and as they 
will not write on the input file.

\begin{QstV} (091226)
    
    What happens if the output file is equal to (one of) the input 
    file(s)?
\end{QstV}


But the question is certainly to be settled for the functions in the 
library. 
In many cases, there will be two versions of the same function.

\subsection{Complexity}
\label{ssc:com-plx-A}





\section{Graph functions}
\label{sec:gra-fun-A}

\begin{enumerate}
    
\addtocounter{enumi}{-1}
\item \Fct{transpose-g}

This is a hidden (and ancillary) graph function, not accessible to the user through 
\tafkit (because it would be somewhat confusing with the true \Fct{transpose}. 
It builds the transpose of the graph, but leaves the labels untouched.

\item \Fct{accessible}, \Fct{coaccessible}, \Fct{trim}

Graph traversal. Implemented by depth-first, or breadth-first search?

Should be in-place by default in the library.

The algorithm should access \emph{entries} rather than 
\emph{transitions}.

\Fctp{coaccessible} = \Fctq{transpose-g}{\Fctq{accessible}{\Fctp{transpose-g}}}

\Fctp{trim} = \Fctq{coaccessible}{\Fctp{accessible}}

Complexity should be  proportional to the size \emph{of the result}.
This implies that tags `number of states' and `number of transitions' 
but attached to automata, and updated by every function that modify 
these parameters. 

This can be considered too heavy, in which case these functions will have 
a complexity proportional to the size \emph{of the input}.

\item \Fct{is-accessible}, \Fct{is-coaccessible}, \Fct{is-trim}

Boolean answer. Silent in the non-verbose mode.

Raise the question of having flag tags that could be used in trusted 
components.

\item \Fct{is-empty}, \Fct{is-useless}

cf. la note : sur la nature du vide. Introuvable maintenant sur le 
trac.

\end{enumerate}


\section{Automata-RatExp functions}

\subsection{Properties and transformations of automata}

\subsubsection{\Fct{standardize}}

An automaton is
said to be \emph{standard} if it has a \emph{unique initial state} which is the
destination of no transition and whose \emph{initial multiplicity} is equal to
the identity (of the multiplicity semiring or of the series semiring,
according to the current convention).

Not only every automaton is equivalent to a standard one, but a simple
procedure, called ''standardization'', transforms every 
automaton~$\Ac$ in an
equivalent standard one, and goes as follows.

\thi Add a new state~$s$ , make it initial, with initial multiplicity equal to
the identity.

\thii For every initial state~$i$ of~$\Ac$ , with initial 
function~$I(i)$ , add a
transition from~$s$ to~$i$ with label~$I(i)$, and set~$I(i)$ to~$0$
(the zero of the semiring, or of the series -- as above).

\thiii Suppress all spontaneous transition that have been created in~(ii) by a
`backward closure".

\thiv Remove the former initial states of~$\Ac$ that are the 
destination of no incoming transition.

Steps~(iii) and~(iv) are necessary to insure the following property:

''The standardization of a standard automaton~$\Ac$ is isomorphic to~$\Ac$ ''.

More informally, but more generally, they insure that the result of the
standardization is of the same "kind" as the automaton on which it is
applied (in particular, proper if~$\Ac$ is proper).

As this is informal, it is ambiguous too.
The problem comes from~(iii): when are the transitions created 
at~(ii) \emph{spontaneous}?
The proposed answer is as follows:

The initial function~$I(i)$ is either in the multiplicity semiring 
(\code{labels-are-letters} automata), or a linear combination of 
atoms (\code{labels-are-atoms} automata) or of words 
(\code{labels-are-words} automata) or a series 
(\code{labels-are-series} automata). 
In the first case, the created transitions are all spontaneous, in 
the three latter cases, the transition created between~$s$ and~$i$ is 
spontaneous if the support of~$I(i)$ is~$\unM$.



\subsubsection{\Fct{normalize}}

It is not true that every automaton is equivalent to a normalized 
one, at least if one wants to stay in the same class of automata: 
\code{labels-are-letters} automata or proper automata.
But if the behaviour of a proper automaton~$\Ac$ is proper, 
then~$\Ac$ is equivalent to a (proper) normalized automaton by a 
"normalization procedure" which plays mutatis mutandis the same role 
as the standardization and which is best described with the help of the
standardization.


The basic idea is

\Fctq{normalize}{$\Ac$} = 
\Fctq{standardize}{\Fctq{transpose-g}{\Fctq{standardize}{\Fctq{transpose-g}{$\Ac$}}}} 

but is not completely true.
Let 

$\Bc$ = 
\Fctq{standardize}{\Fctq{transpose-g}{\Fctq{standardize}{\Fctq{transpose-g}{$\Ac$}}}}. 


\thi If~$\Ac$ is proper,
let~$i$ be the (unique) initial state of~$\Bc$ and let~$\Cc$ be the automaton
obtained from~$\Bc$ by setting~$T(i)=0$ --- \ie setting to~$0$ the 
final function. Then, $\Cc$ is normalized and we write 
$\Cc$ = \Fctq{normalize}{$\Ac$}.

If~$\Ac$ is an automaton over a free monoid~$\Ae$ (anyway, if~$\Ac$ is a 
transducer (fmp), \emph{normalized} would have another meaning), it 
holds:
\begin{equation}
    \LangAuto{\Cc}=\LangAuto{\Ac}\bk\unAe 
    \e\text{if~$\Ac$ is a Boolean automaton}\e
    \CompAuto{\Cc}=\CompAuto{\Ac}\hadam\UL{\Ap}
    \e\text{if~$\Ac$ is a weighted automaton}
    \EqVrg 
    \notag
%     \label{}
\end{equation}
that is, in both cases, \Fctq{normalize}{$\Ac$} accepts or realizes 
the 'proper' part 
of the language accepted, or of the series realized by, $\Ac$.

\thii If~$\Ac$ is \emph{not proper}, it seems that the most logical 
definition would be the following.

Let~$i$ be the (unique) initial state of~$\Bc$ and let~$\Cc$ be the automaton
obtained from~$\Bc$ by adding a transition from~$i$ to~$t$, the 
(unique) initial state of~$\Bc$, with label~$T(i)$, that is, a 
spontaneous transition as soon as~$T(i)$ is a scalar.  
Then, $\Cc$ is normalized and we write 
$\Cc$ = \Fctq{normalize}{$\Ac$} and it holds:
\begin{equation}
    \LangAuto{\Cc}=\LangAuto{\Ac} 
    \e\text{if~$\Ac$ is a Boolean automaton}\e
    \CompAuto{\Cc}=\CompAuto{\Ac}
    \e\text{if~$\Ac$ is a weighted automaton}
    \EqVrg 
    \notag
%     \label{}
\end{equation}
that is, in both cases, \Fctq{normalize}{$\Ac$} accepts or realizes 
the same language or series as $\Ac$.

\begin{ComV}
The discrepancy between the two cases leads me to think that we 
should change policy.
    
    Either we should simply suppress this function from \tafkit (and 
    \vcsn).
    Or we should we should reserve it for proper Boolean automata, or 
    for \code{labels-are-letters} automata.
\end{ComV}



\subsection{Operations on automata}
\label{sec:ope-on-aut-A}


A small sketch is worth a long speech.\footnote{Attributed to Napol�on.}

Let $\Ac=\bra{Q,A,E,\{i\},T}$ and $\Bc=\bra{R,A,F,\{j\},U}$ be two standard automata: 


\FixVCScale{.5}
\ChgEdgeLabelScale{.8}
% \ShowFrame 
% \qquad\qquad
\medskip 
\noindent 
~\hfill
\VCDraw{%
  \begin{VCPicture}{(-5,-2.5)(5,2.5)}
    \psellipse[linecolor=gray](.8,0)(3.3,2)
    \State[i]{(-4,0)}{I}
    \State[p]{(-1,1)}{P}
    \State[q]{(-1,-1)}{Q}
    \State[r]{(3,.5)}{R}
                                %
    \Initial{I}
                                %
    \FinalR{s}{I}{T_i}
    \FinalR{e}{Q}{T_q}
    \FinalL{e}{R}{T_r}
                                %
    \EdgeL{I}{P}{E_{i,p}}
    \EdgeR{I}{Q}{E_{i,q}}
    \VCPut{(1,0)}{\scalebox{3}{$\Ac$}}
  \end{VCPicture}
}
\hfill\hfill
\VCDraw{%
  \begin{VCPicture}{(-5,-2.5)(5,2.5)}
    \psellipse[linecolor=gray](.8,0)(3.3,2)
    \State[j]{(-4,0)}{I}
    \State[s]{(-1,1)}{P}
    \State[t]{(-1,-1)}{Q}
    \State[u]{(3,.5)}{R}
                                %
    \Initial{I}
                                %
    \FinalR{s}{I}{U_j}
    \FinalR{e}{Q}{U_t}
    \FinalL{e}{R}{U_u}
                                %
    \EdgeL{I}{P}{F_{j,s}}
    \EdgeR{I}{Q}{F_{j,t}}
    \VCPut{(1,0)}{\scalebox{3}{$\Bc$}}
  \end{VCPicture}
}
\hfill
~


\subsubsection{\Fct{union}}

Just the union of the two automata. It is a graph function indeed.

\subsubsection{\Fct{sum}}

\emph{Precondition}:  \Prm{a.xml} and \Prm{b.xml} are \emph{standard} 
for the sum operation is defined only on standard automata.

Defined in DWM.

\bigskip 

\noindent 
$\bullet$ The standard automaton $\Ac\autplus\Bc=\bra{Q\cup R\setminus\{j\},A,G,\{i\},V}$ is defined as:
\\
\begin{minipage}{7cm}
\begin{align*}
\forall p,q\in & Q\cup R\setminus\{j\}\, ,\\
\ G_{p,q}&=
\begin{cases}
E_{p,q}&\quad {if}\quad p,q\in Q\\
F_{p,q}&\quad {if}\quad p,q\in R\\
F_{j,q}&\quad {if}\quad p=i\text{ and }q\in R\\
0_\K &\quad {otherwise}
\end{cases}\\[.5em]
%\end{align*}
%\end{minipage}
%\begin{minipage}{6cm}
%\begin{align*}
\forall p\in& Q\cup R\setminus\{j\}\, ,\\
\ V_p&=
\begin{cases}
T_i\oplus U_j &\quad {if}\quad p=i\\
T_p&\quad {if}\quad p\in Q\setminus\{i\}\\
U_p&\quad {if}\quad p\in R
\end{cases}
\end{align*}
\end{minipage}
\qquad
\VCDraw{%
  \begin{VCPicture}{(-7,-7)(5,2.5)}
    \psellipse[linecolor=gray](.8,-2.25)(4.5,4.5)
    \psellipse[linecolor=lightgray](.8,0)(3.3,2)
    \State{(-5,-2.25)}{I}
    \State[p]{(-1,1)}{P}
    \State[q]{(-1,-1)}{Q}
    \State[r]{(3,.5)}{R}
                                %
    \Initial{I}
                                %
    \FinalR{s}{I}{T_i\oplus U_j}
    \FinalR{e}{Q}{T_q}
    \FinalL{e}{R}{T_r}
                                %
    \ArcL{I}{P}{E_{i,p}}
    \ArcL{I}{Q}{}\LabelR{E_{i,q}}
    \VCPut{(0,-4.5)}{%
      \psellipse[linecolor=lightgray](.8,0)(3.3,2)
      \State[s]{(-1,1)}{S}
      \State[t]{(-1,-1)}{T}
      \State[u]{(3,.5)}{U}
                                %
      \FinalR{e}{T}{U_t}
      \FinalL{e}{U}{U_u}
                                %
      \ArcR{I}{S}{}\LabelL{F_{j,s}}
      \ArcR{I}{T}{F_{j,t}}
    }
    \VCPut{(1,-2.25)}{\scalebox{3}{$\Ac\autplus\Bc$}}
  \end{VCPicture}}

\subsubsection{\Fct{concatenate}}

\emph{Precondition}:  \Prm{a.xml} and \Prm{b.xml} are \emph{standard} 
for the concatenation operation is defined only on standard automata.

Defined in DWM.

\noindent 
$\bullet$ The standard automaton 
$\Ac\autprod\Bc=\bra{Q\cup R\setminus\{j\},A,G,\{i\},V}$ is:
\\
\begin{minipage}[t]{6cm}
\begin{align*}
\forall p,q\in & Q\cup R\setminus\{j\}\, ,\\
\ G_{p,q}&=
\begin{cases}
E_{p,q}&\quad {if}\quad p,q\in Q\\
F_{p,q}&\quad {if}\quad p,q\in R\\
T_pF_{j,q}&\quad {if}\quad p\in Q\text{ and }q\in R\\
0_\K &\quad {otherwise}
\end{cases}
\end{align*}
\end{minipage}
\quad 
\begin{minipage}[t]{6cm}
\begin{align*}
\forall p\in& Q\cup R\setminus\{j\}\, ,\\
\\
\ V_p&=
\begin{cases}
U_p &\quad {if}\quad p\in R\\
T_p\,U_j&\quad {if}\quad p\in Q
\end{cases}
\end{align*}
\end{minipage}

\begin{center}
\VCDraw{%
  \begin{VCPicture}{(-5,-4)(14,4)}
    \psellipse[linecolor=gray](5.3,0)(8.5,3.5)
    \psellipse[linecolor=lightgray](.8,0)(3.3,2)
    \State[i]{(-4,0)}{I}
    \State[p]{(-1,1)}{P}
    \State[q]{(-1,-1)}{Q}
    \State[r]{(3,.5)}{R}
                                %
    \Initial{I}
                                %
    \FinalR{s}{I}{T_i\,U_j}
    \FinalR[.5]{n}{Q}{T_q\,U_j}
    \FinalL[.5]{n}{R}{T_r\,U_j}
                                %
    \VArcL{arcangle=50,ncurv=1}{I}{P}{E_{i,p}}
%{I}{P}{}\LabelR{E_{i,p}}
    \ArcR{I}{Q}{}\LabelL{E_{i,q}}
    \VCPut{(9,0)}{%
      \psellipse[linecolor=lightgray](.8,0)(3.3,2)
      \State[s]{(-1,1)}{S}
      \State[t]{(-1,-1)}{T}
      \State[u]{(3,.5)}{U}
                                %
      \FinalR{e}{T}{U_t}
      \FinalL{e}{U}{U_u}
    }
                                %
    \LArcL[.8]{I}{S}{T_i\,F_{j,s}}
    \LArcR[.8]{I}{T}{T_i\,F_{j,t}}
    \EdgeL[.3]{Q}{S}{T_q\,F_{j,s}}
    \EdgeR{Q}{T}{T_q\,F_{j,t}}
    \EdgeL{R}{S}{T_r\,F_{j,s}}
    \EdgeL[.8]{R}{T}{T_r\,F_{j,s}}
                                %
    \VCPut{(9,-2.25)}{\scalebox{3}{$\Ac\autprod\Bc$}}
  \end{VCPicture}}
\end{center}

\bigskip 
\subsubsection{\Fct{star}}

\emph{Precondition}:  \Prm{a.xml} is \emph{standard} 
for the star operation is defined only on standard automata.

Defined in DWM.

\noindent 
$\bullet$ The standard automaton $\autstar{\Ac}=\bra{Q,A,E^{(\autstarsymb)},\{i\},T^{(\autstarsymb)}}$ is:
\\\begin{minipage}{6cm}
\begin{align*}
\forall p,q\in & Q,\\
\ E^{(\autstarsymb)}_{p,q}&=
\begin{cases}
T_i^*E_{i,q}&\quad {if}\quad p=i\\
T_pT_i^*E_{i,q}\oplus E_{p,q}&\quad {otherwise}
\end{cases}\\
%\end{align*}
%\end{minipage}
%\begin{minipage}{6cm}
%\begin{align*}
\forall p\in& Q,\\
\ T^{(\autstarsymb)}_p&=
\begin{cases}
T_i^*&\quad {if}\quad p=i\\
T_pT_i^*&\quad {otherwise}
\end{cases}
\end{align*}
\end{minipage}
\qquad\qquad
\VCDraw{%
  \begin{VCPicture}{(-7,-3)(5,3)}
    \psellipse[linecolor=gray](.8,0)(3.3,2.5)
    \State[i]{(-5,0)}{I}
    \State[p]{(-1,1)}{P}
    \State[q]{(-1,-1)}{Q}
    \State[r]{(3,.5)}{R}
                                %
    \Initial{I}
                                %
    \FinalR{s}{I}{T_i^*}
    \FinalL{se}{Q}{T_qT_i^*}
    \FinalL{e}{R}{T_rT_i^*}
                                %
    \ArcL{I}{P}{T_i^*\,E_{i,p}}
    \ArcR{I}{Q}{T_i^*\,E_{i,q}}
    \VArcR{arcangle=-130,ncurv=1}{R}{P}{T_r\,T_i^*\,E_{i,p}}
    \VArcL{arcangle=130,ncurv=1.1}{R}{Q}{T_r\,T_i^*\,E_{i,q}}
    \VCurveL{angleA=80,angleB=-160}{Q}{P}{T_q\,T_i^*\,E_{i,p}}
    \CLoopR[.5]{20}{Q}{T_q\,T_i^*\,E_{i,q}}
    \VCPut{(1,1)}{\scalebox{3}{$\autstar{\Ac}$}}
  \end{VCPicture}
}




\bigskip 
\subsubsection{\Fct{left-mult}}

\emph{Precondition}:  \Prm{a.xml} is \emph{standard} for the left 
`exterior' multiplication
operation is defined only on standard automata. 

Beware that although the multiplication is on the left the operand 
\Prm{k} is the \emph{second} argument, and thus written on the right 
of \Prm{a.xml}.



\noindent 
$\bullet$ The standard automata $k\Ac=\bra{Q,A,E^{(k.)},\{i\},T^{(k.)}}$ is defined by:
\\
\begin{minipage}{8cm}
\begin{align*}
\forall p,q\in Q\, ,\qquad E^{(k.)}_{p,q}&=
\begin{cases}
k\,E_{p,q}&\quad {if}\quad p=i\\
E_{p,q}&\quad {otherwise}
\end{cases}\\
\forall p\in Q\, ,\qquad T^{(k.)}_p&=
\begin{cases}
k\,T_p&\quad {if}\quad p=i\\
T_p&\quad {otherwise}
\end{cases}
\end{align*}
\end{minipage}
\qquad
\VCDraw{%
  \begin{VCPicture}{(-5,-2.5)(5,2.5)}
    \psellipse[linecolor=gray](.8,0)(3.3,2)
    \State[i]{(-4,0)}{I}
    \State[p]{(-1,1)}{P}
    \State[q]{(-1,-1)}{Q}
    \State[r]{(3,.5)}{R}
                                %
    \Initial{I}
                                %
    \FinalR{s}{I}{k\,T_i}
    \FinalR{e}{Q}{T_q}
    \FinalL{e}{R}{T_r}
                                %
    \EdgeL{I}{P}{k\,E_{i,p}}
    \EdgeR{I}{Q}{k\,E_{i,q}}
    \VCPut{(1,0)}{\scalebox{3}{$k\Ac$}}
  \end{VCPicture}}

\bigskip 
\subsubsection{\Fct{right-mult}}

\emph{Precondition}:  \Prm{a.xml} is \emph{standard} for the right 
`exterior' multiplication
operation is defined only on standard automata. 

\noindent 
$\bullet$ The standard automata $\Ac k=\bra{Q,A,E,\{i\},T^{(.k)}}$ is defined by:
\\
\begin{minipage}{8cm}
\begin{equation*}
\forall p\in Q\, , \qquad T^{(.k)}_p=T_p\,k
\qquad \qquad 
\end{equation*}
\end{minipage}
\qquad
\VCDraw{%
  \begin{VCPicture}{(-5,-2.5)(5,2.5)}
    \psellipse[linecolor=gray](.8,0)(3.3,2)
    \State[i]{(-4,0)}{I}
    \State[p]{(-1,1)}{P}
    \State[q]{(-1,-1)}{Q}
    \State[r]{(3,.5)}{R}
                                %
    \Initial{I}
                                %
    \FinalR{s}{I}{T_i\,k}
    \FinalR{e}{Q}{T_q\,k}
    \FinalL{e}{R}{T_r\,k}
                                %
    \EdgeL{I}{P}{E_{i,p}}
    \EdgeR{I}{Q}{E_{i,q}}
    \VCPut{(1,0)}{\scalebox{3}{$\Ac k$}}
  \end{VCPicture}}

\bigskip 


\subsection{From automata to expressions}
\label{ssc:aut-to-exp-A}


The function \Fct{aut-to-exp} implements the state elimination 
method, guided by  
the `na{\"\i}ve' heuristics.

\thi For precise specification, and in view of the derivation 
procedure, one should specify the bracketting:
\begin{equation}
    p\path{\Fmsf}q\path{\Gmsf}q\path{\Hmsf}r
    \e \text{gives}\e
    p\path{(\Fmsf\cdot\Gmsf^{*})\cdot\Hmsf}r
\label{equ:eli-met}
\end{equation}
after the elimination of the state~$q$.

\thii The `na{\"\i}ve' heuristics: 

\tha Make real the initial and final subliminal states~$i$ and~$t$. 
From~$i$ to every initial state~$p$, there is thus a transition with 
label~$I(p)$.
Dually, from every final state~$r$ to~$t$, there is thus a transition with 
label~$T(r)$.

\thb For every state~$p$ (outside~$i$ and~$t$) compute a two component 
index~$(l(p),k(p))$:\\
\tiret $l(p)=1$ if~$p$ is the origin of a loop, $l(p)=0$ otherwise;\\
\tiret $k(p)= [i(p)-1][o(p)-1]$ where~$i(p)$ is the in-degree of~$p$ 
and~$o(p)$ its out-degree.\\
\tiret Lexicographically order the states by their index.

\thc While there remains states,\\
\tiret choose the state~$q$ with smallest index,\\
\tiret remove it and replace the incoming and outgoing transitions 
according to~\equnm{eli-met},\\
\tiret recompute the index for those states 
that were adjacent to~$q$.

\thd Return the label of the transition from~$i$ to~$t$.

\thiii The implementation of the algorithm probably implies that the 
automaton is transformed into a \code{labels-are-series} automaton, 
whatever its native kind. 

\thiv there exists other heuristics. In particular, at some point, 
the `Delgado--Morais' heuristics has been encoded in \vcsn.
Could be revived, and used for demonstration of the test bench.

\subsection{From expressions to automata}
\label{ssc:exp-to-aut-A}


The function \Fct{exp-to-aut} implements the construction of the 
`position automaton' by the recursive use of the operations on 
sttandard automata as described at~\secti{ope-on-aut-A}.

\begin{ComV}
May be Alex could be more explicit (if necessary).
\end{ComV}

\section{Functions on Automata-RatExp over free monoids with numerical weights}

\subsection{Properties and transformations of automata}


\subsubsection{\Fct{quotient}}
\label{ssc:quo-A}

\subsection{Behaviour of automata}

\subsubsection{\Fct{eval}}
\label{ssc:evl-wrd-A}


\subsection{Operations on automata}

\subsubsection{\Fct{product}}
\label{ssc:pro-duc-A}

\addtocounter{section}{1}
% \section{Boolean Automata-RatExp on free monoids}
% 
% \subsection{Properties and transformations of automata}
% 
% \subsubsection{\Fct{minimize}}
% 
% \subsection{Operations on automata}
% 
% \subsubsection{\Fct{determinize}}


\section{Automata-RatExp on product of two free monoids}
\label{sec:fmp-fct-D}

\subsection{Properties and transformations of transducers}

\subsection{Operations on transducers}

\subsubsection{\Fct{composition}}

\endinput
