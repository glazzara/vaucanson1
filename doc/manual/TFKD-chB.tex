
\chapter{{FSM\,XML}, \protect\\ \e an \code{XML} format for automata}
\label{chp:fsm-xml}%

\newlength{\tagindentl}\newlength{\taglnskp}
\newlength{\tagspl}
\setlength{\tagindentl}{1.5em}
\setlength{\tagspl}{3em}
\setlength{\taglnskp}{-0.1ex}
% \pagestyle{plain}
\renewcommand\thesubsection{\arabic{subsection}}
\renewcommand\theenumii{\arabic{enumii}}
\renewcommand\theenumiii{\arabic{enumiii}}
\renewcommand\labelenumii{\theenumi.\theenumii.}
\renewcommand\labelenumiii{\theenumi.\theenumii.\theenumiii.}
\setcounter{subsection}{0}

% \endinput

\begin{Com}
    This appendix gives the description  of the 
    format \fsmxml that had been given in May 2008 under a layout 
    that is lighter than the reference card that can be found on the 
    web page of format.
    
    This description is not up to date anymore and has to be 
    updated.
\end{Com}

\fsmxml is an \xml format proposal for the description of 
weighted automata, transducers, and regular expressions.


The aim of this \xml format is to make possible, and hopefully easy,
the communication between the various programs and systems that deal
with weighted automata and transducters.  
\fsmxml is part of the \vcsn Project.  
In particular, \fsmxml is the
input/output format of \tafkit, the command line interface to the
\vcsn library.

This document gives a pseudo-formal description of the format.

All tags of the format are listed, with their children, and attributes.

\newpage 
\subsection{The root tag}
\begin{enumerate}%
\addtocounter{enumi}{-1}%

\item  \xtagf{fsmxml}

\smallskip
The unique possible root of an \fsmxml file, which can contain any 
number of automata and standalone rational expressions.

\begin{tabbing}
\xtag{fsmxml xmlns="" version=""}      \\[\taglnskp]
\tagindent\xtagf{automaton}\tagsp\= \kill
\tagindent\xtagf{automaton}\> 0 or more \occ \\[\taglnskp]
\tagindent\xtagf{regExp}   \> 0 or more \occ \\[\taglnskp]
\xftag{fsmxml}
\end{tabbing}



\end{enumerate}
\setcounter{enumitemp}{\value{enumi}}

\subsection{The value type tags}

Both \xtagf{automaton} and \xtagf{regExp} have a child in common:
the \xtagf{valueType} tag which describes the `type' of the
behaviour of the automaton or of the series denoted by the
expression.

\begin{enumerate}
\addtocounter{enumi}{\value{enumitemp}}

\item  \xtagf{valueType}
\begin{tabbing}
\xtag{valueType}      \\[\taglnskp]
\tagindent\xtagf{writingData}\tagsp\= \kill
\tagindent\xtagf{writingData} \>  \opt\\[\taglnskp]
% \tagindent\xtagf{semiring}\tagsp\= \kill
\tagindent\xtagf{semiring}\>  \req\\[\taglnskp]
\tagindent\xtagf{monoid}  \>  \req\\[\taglnskp]
\xftag{valueType}
\end{tabbing}

\item  \xtagf{semiring}

% \smallskip
\begin{tabbing}
\ptn
Pivotal att.: \xattr{type} = \xval{numerical}$|$\xval{series}\tagsp
                      \token   \req
\end{tabbing}


\begin{enumerate}

\item  \xattr{type} = \xval{numerical}

\begin{tabbing}
\xtag{semiring  type=numerical  set='' operation='' }\\[\taglnskp]
\tagindent\xtagf{writingData}\tagsp\= \kill
\tagindent\xtagf{writingData} \>  \opt\\[\taglnskp]
\xftag{semiring}
\end{tabbing}

\begin{tabbing}
\ptn Att: \ \ \=\xattr{operation} \= = \ \= \xval{classical}$|$
                        \xval{minPlus}$|$
                         \xval{maxPlus} \tagsp \=\token \req
                         \kill
\ptn
Att:\>\> \xattr{set} \' =
   \> \xval{B}$|$\xval{N}$|$\xval{Z}$|$\xval{Q}$|$\xval{R}$|$\xval{C}
   \> \token  \req \\
   \>\>\xattr{operation} \' = \>\xval{classical}$|$
                                \xval{minPlus}$|$
                                \xval{maxPlus} \tagsp \>\token\req
\end{tabbing}

\begin{tabbing}
\ptn Tag \xtagf{writingData  identitySymbol='' zeroSymbol=''} \\
\tagindent\ptn Att.: \xattr{identitySymbol} \= = ' '
                                     \tagsp \=  \strng  \req \+ \\
         \xattr{zeroSymbol}\' = ' '  \tagsp   \> \strng  \req
\end{tabbing}



\item  \xattr{type} = \xval{series}

\begin{tabbing}
\xtag{semiring  type=series}      \\[\taglnskp]
\tagindent\xtagf{writingData}\tagsp\= \kill
\tagindent\xtagf{writingData}\> \oneocc \opt\\[\taglnskp]
\tagindent\xtagf{semiring}\> \oneocc \req\\[\taglnskp]
\tagindent\xtagf{monoid} \> \oneocc \req\\[\taglnskp]
\xftag{semiring}
\end{tabbing}

\ptn Constraint: \xtagf{monoid} should not be of \xattr{type} = \xval{unit}

\end{enumerate}
\clearpage  

\item  \xtagf{monoid}

\begin{tabbing}
\ptn Pivotal att.: \xattr{type} =
            \xval{unit}$|$\xval{free}$|$\xval{product}  \tagsp   \token   \req
\end{tabbing}

\begin{enumerate}

\item  \xattr{type} = \xval{unit}

\smallskip
This type means 'no monoid' and gives the possibility of describing
     within the format graphs valued by numerical semirings only.

\begin{tabbing}
\xtagf{monoid type=unit}
\end{tabbing}

\begin{tabbing}
\ptn Constraint: 
\= not allowed in  \=\xtagf{semiring type=series}  (\cf 2.2)\\
\>\>   nor in \'\xtagf{monoid type=product}  (\cf 3.3)\\
\end{tabbing}

\item  \xattr{type} = \xval{free}

\smallskip
\begin{tabbing}
\ptn Pivotal att.:\= \ \ \ \ \xattr{genDescrip} \= = \xval{enum}$|$\xval{range}$|$\xval{set}
                            \tagsp \=\token   \req \kill 
\ptn Pivotal att.: 
\>\> \xattr{genKind} \'= \xval{simple}$|$\xval{tuple} 
\> \token   \req\\

\ptn Pivotal att.: 
\>\> \xattr{genDescrip} \'= \xval{enum}$|$\xval{range}$|$\xval{set}
                        \>\token   \req  
\end{tabbing}


This attribute \xattr{genDescrip} is put for further development.
         No alternative value is described here.

         \smallskip
\begin{enumerate}

\item  \xattr{genKind} = \xval{simple}

\smallskip
\begin{tabbing}
\xtag{monoid  type=free  genKind=simple  genDescrip='enum'  genSort=''}      \\[\taglnskp]
\tagindent\xtagf{writingData}\tagsp \=1+ \occ \=\opt \kill
\tagindent\xtagf{writingData} \> \> \opt\\[\taglnskp]
\tagindent\xtagf{monGen} \>  1+ \occ \>\req \\[\taglnskp]
\xftag{monoid}
\end{tabbing}
% (\xattr{genDescrip} = \xval{enum})

\smallskip
\begin{tabbing}
\ptn Att.: \xattr{genSort} =
    \xval{letter}$|$\xval{digit}$|$\xval{alphanum}$|$\xval{integer} 
    \tagsp \token \req \\

\ptn Tag  \xtagf{writingData  identitySymbol=''}  \\
\tagindent \xattr{identitySymbol} = ' ' \tagsp \strng \\ Tells how the
                                        identity of the monoid should
                                        be written when output (\eg
                                        in expressions)
                                    \end{tabbing}

                                    \smallskip
\item  \xattr{genKind} = \xval{tuple}

\smallskip
\begin{tabbing}
\xtag{monoid  type=free  genKind=tuple  genDim=''   genDescrip='enum'}      \\[\taglnskp]
\tagindent\tagindent
   \xtagf{genCompSort}\tagsp\=\=\xval{"genDim"}\=\occ \=\req \kill
\tagindent\xtagf{writingData}\> \> \> \>\opt\\[\taglnskp]
\tagindent\xtag{genSort}\> \> \>\oneocc \>\req \\[\taglnskp]
\tagindent\tagindent\xtagf{genCompSort}\> \>\>\xval{"genDim"} \' \occ \>\req \\[\taglnskp]
\tagindent\xftag{genSort}\\[\taglnskp]
\tagindent\xtagf{monGen} \> \> \> 1+ \'\occ  \>\req \\[\taglnskp]
\xftag{monoid}
\end{tabbing}
%  (\xattr{genDescrip} = \xval{enum})

\smallskip
\begin{tabbing}
\ptn Att.: \xattr{genDim} = ' '   \tagsp   integer strictly larger 
than 1, \req\\

\ptn Tag \xtagf{genSort}       holds the \xval{"genDim"}   
\xtagf{genCompSort} tags\\

\ptn Tag \xtagf{genCompSort value=''}  \\
\tagindent \xattr{value} = ' ' \tagsp\= has the same role as the attribute  
\xattr{genSort} in  3.2.1\\
\>       for the corresponding coordinate of the monoid generator\\
\>       and can take the same token values.
\end{tabbing}

\end{enumerate}

\smallskip
\item  \xattr{type} = \xval{product}

\begin{tabbing}
\xtag{monoid  type=product   prodDim=''}      \\[\taglnskp]
\tagindent\xtagf{writingData}\tagsp\=\xval{"prodDim"} \= \occ \= \req \kill
\tagindent\xtagf{writingData} \> \> \> \opt\\[\taglnskp]
\tagindent\xtagf{monoid}\> \> \xval{"prodDim"} \' \occ \>\req \\[\taglnskp]
\xftag{monoid}
\end{tabbing}

\smallskip
\begin{tabbing}
\ptn Att.: \xattr{prodDim} = ' '   \tagsp   integer strictly larger 
than 1, \req\\
~\\
\trt Constraint: no children  \xtagf{monoid} can be of \xattr{type} = \xval{unit}
\end{tabbing}

\end{enumerate}

 

\item \xtagf{monGen}

Describes a monoid generator for a free monoid.\\
Its form will depend on the pivotal attribute  \xattr{genKind}.\\
(The only case considered here is when  \xattr{genDescrip} = \xval{enum}.)

\begin{enumerate}

\item  \xattr{genKind} = \xval{simple}

\begin{tabbing}
\xtagf{monGen  value=''}
\end{tabbing}

\begin{tabbing}
\ptn Att.: \xattr{value} \tagsp must be consistent with \xattr{genSort} \req
\end{tabbing}

\item  \xattr{genKind} = \xval{tuple}

\begin{tabbing}
\xtag{monGen} \\[\taglnskp]
\tagindent\xtagf{monCompGen}\tagsp\xval{"genDim"} \occ \req \\[\taglnskp]
\xftag{monGen}
\end{tabbing}

\begin{tabbing}
\ptn Tag \xtagf{monCompGen  value=''}
\end{tabbing}

\trt Constraint: each \xattr{value} must be consistent with the
corresponding \xattr{genCompSort}

\end{enumerate}

\end{enumerate}
\setcounter{enumitemp}{\value{enumi}}

\newpage 
\subsection{The rational (regular) expressions}

\begin{enumerate}
\addtocounter{enumi}{\value{enumitemp}}

\item \xtagf{regExp}

\begin{tabbing}
\xtag{regExp name=""}      \\[\taglnskp]
\tagindent\xtagf{typedRegExp}\tagsp\=\oneocc \=\req\kill
\tagindent\xtagf{valueType}   \>\oneocc \>\req \\[\taglnskp]
\tagindent\xtagf{typedRegExp} \>\oneocc \>\req \\[\taglnskp]
\xftag{regExp}
\end{tabbing}

\begin{tabbing}
\ptn Att.: \xattr{name}= ' ' \tagsp  \strng   \opt
\end{tabbing}

\trt Constraint: the  \xtagf{monoid} cannot be of \xattr{type} = \xval{unit}

\trt    At this stage, one could think of a  \xtagf{writingData}  tag which would contains
    writing options for the expressions: a dot or nothing for the product,
    delimitors for the weight, etc.

\smallskip
\item \xtagf{typedRegExp}

\begin{tabbing}
\xtag{typedRegExp}      \\[\taglnskp]
\tagindent\BTRE \tagsp  plays the role of a non terminal in a grammar.\\[\taglnskp]
\xftag{typedRegExp}
\end{tabbing}

\begin{tabbing}
\BTRE =  \=\xtagf{sum}$|$\xtagf{product}$|$\xtagf{star}$|$\\
         \>\xtagf{rightExtMul}$|$\xtagf{leftExtMul}$|$\\
         \>\xtagf{zero}$|$\xtagf{one}$|$\xtagf{monElmt}
\end{tabbing}

\item \xtagf{sum}

\begin{tabbing}
\xtag{sum}\\[\taglnskp]
\tagindent\BTRE\\[\taglnskp]
\tagindent\BTRE\\[\taglnskp]
\xftag{sum}
\end{tabbing}


\item \xtagf{product}

\begin{tabbing}
\xtag{product}\\[\taglnskp]
\tagindent\BTRE\\[\taglnskp]
\tagindent\BTRE\\[\taglnskp]
\xftag{product}
\end{tabbing}


\item \xtagf{star}

\begin{tabbing}
\xtag{star}\\[\taglnskp]
\tagindent\BTRE\\[\taglnskp]
\xftag{star}
\end{tabbing}


\item \xtagf{rightExtMul} or \xtagf{leftExtMul}

\begin{tabbing}
\xtag{xxxExtMul}\\[\taglnskp]
\tagindent\xtagf{weight}\\[\taglnskp]
\tagindent\BTRE\\[\taglnskp]
\xftag{xxxExtMul}
\end{tabbing}


\item \xtagf{zero}   and   \xtagf{one}       "final" tags


\smallskip
\item \xtagf{monElmt}

\smallskip
Depends on the pivotal attribute \xattr{type}  of the
\xtagf{monoid}  in the  \xtagf{valueType}.

\begin{enumerate}

\item   \xattr{type} = \xval{free}

\begin{tabbing}
\xtag{monElmt}\\[\taglnskp]
\tagindent\xtagf{monGen}\tagsp    1+  \occ  \req \\[\taglnskp]
\xftag{monElmt}
\end{tabbing}

\item \xattr{type} = \xval{product}

\begin{tabbing}
\xtag{monElmt}\\[\taglnskp]
\tagindent\xtagf{one}$|$\xtagf{monElmt}\tagsp
                 \xval{"prodDim"} \occ\req \\[\taglnskp]
\xftag{monElmt}
\end{tabbing}

\end{enumerate}

\smallskip
\item \xtagf{weight}

\smallskip
Depends on the  pivotal attribute \xattr{type}  of the
\xtagf{semiring}  in the  \xtagf{valueType}.

\begin{enumerate}

\item  \xattr{type} = \xval{numerical}

\begin{tabbing}
\xtagf{weight value=''}
\end{tabbing}

\begin{tabbing}
\ptn Att.:  \xattr{value}= ' ' \tagsp \strng that will be interpreted
                          according to the  attribute \xattr{set}
\end{tabbing}

\trt If  \xattr{set} =\xval{Q} ,  one can think of having 2
integers values.

\smallskip
\item \xattr{type} = \xval{series}

\begin{tabbing}
\xtag{weight}\\[\taglnskp]
\tagindent\BTRE  \tagsp    of the \xtagf{valueType}  defined by  \xtagf{semiring}\\[\taglnskp]
\xftag{weight}
\end{tabbing}

\end{enumerate}

\end{enumerate}
\setcounter{enumitemp}{\value{enumi}}

\newpage 
\subsection{The automata}
\begin{enumerate}
\addtocounter{enumi}{\value{enumitemp}}

\item  \xtagf{automaton}

\begin{tabbing}
\xtag{automaton  name=''  readingDir=''}\\[\taglnskp]
\tagindent\xtagf{automatonStruct}\tagsp \= \oneocc \= \req    \kill
\tagindent\xtagf{geometricData}       \>\oneocc \>\opt \\[\taglnskp]
\tagindent \xtagf{drawingData}        \>\oneocc \> \opt\\[\taglnskp]
\tagindent\xtagf{valueType}           \>\oneocc \> \req   \\[\taglnskp]
\tagindent\xtagf{automatonStruct}     \>\oneocc \> \req    \\[\taglnskp]
\xftag{automaton}
\end{tabbing}

\begin{tabbing}
\ptn Att.: \=  \xattr{readingDir} \== ' ' \tagsp   \strng   \opt \kill
\ptn Att.:\>\>  \xattr{name} \'= ' ' \tagsp   \strng   \opt \\
         \>\>  \xattr{readingDir} \'= \xval{left}$|$\xval{right}
         \tagsp \token\\   %\opt   default = \xval{left}

\ptn Tag \xtagf{geometricData  x=''  y=''} \tagsp     gives relative
origin\\

\ptn Tag \xtagf{drawingData   drawingClass='' } \tagsp
or something more complicated
\end{tabbing}


\item  \xtagf{automStruct}

\begin{tabbing}
\xtag{automStruct}\\[\taglnskp]
\tagindent\xtagf{transitions}\tagsp \= \oneocc \= \req    \kill
\tagindent\xtagf{states}      \>\oneocc \> \req\\[\taglnskp]
\tagindent\xtagf{transitions} \>\oneocc \> \req\\[\taglnskp]
\xftag{automStruct}
\end{tabbing}


\item  \xtagf{states}

\begin{tabbing}
\xtag{states}\\[\taglnskp]
\tagindent\xtagf{state}\tagsp 0 or more  \occ \\[\taglnskp]
\xftag{states}
\end{tabbing}


\item  \xtagf{state}

\begin{tabbing}
\xtag{state  id=''  name=''  key='' }\\[\taglnskp]
\tagindent\xtagf{geometricData}\tagsp \= \oneocc \= \opt \\[\taglnskp]
\tagindent\xtagf{drawingData}   \>\oneocc \>\opt\\[\taglnskp]
\xftag{state}
\end{tabbing}

\begin{tabbing}
\ptn Att.: \=  \xattr{name} \== ' ' \tagsp  \= \strng   \opt \kill
\ptn Att.:  \>\>\xattr{id}  \' = ' '  \>\strng   \req
                           must be unique in the whole automaton. \\
            \>\> \xattr{name} \'= ' '  \> \strng   \opt   \\
            \>\>\xattr{key}  \'= ' '  \> integer  \opt    may be used
            to pass an ordering on the states. \\

\ptn Tag  \xtagf{geometricData  x=''  y=''}  \tagsp   coordinates of
the state\\

\ptn Tag \xtagf{drawingData   drawingClass='' } \tagsp or something more complicated
\end{tabbing}
\clearpage 

\item  \xtagf{transitions}

\begin{tabbing}
\xtag{transitions}\\[\taglnskp]
\tagindent\xtagf{transition} \tagsp \= 0 or more \= \occ \kill
\tagindent\xtagf{transition} \>0 or more \> \occ \\[\taglnskp]
\tagindent\xtagf{initial}  \>0 or more \> \occ \\[\taglnskp]
\tagindent\xtagf{final}    \>0 or more \> \occ \\[\taglnskp]
\xftag{transitions}
\end{tabbing}



\item  \xtagf{transition}

\begin{tabbing}
\xtag{transition  source=''  target=''}\\[\taglnskp]
\tagindent\xtagf{geometricData}\tagsp \= \oneocc \= \opt \kill
\tagindent\xtagf{geometricData} \>\oneocc \>\opt \\[\taglnskp]
\tagindent\xtagf{drawingData}   \>\oneocc \>\opt\\[\taglnskp]
\tagindent\xtagf{label}         \>\oneocc \>\req  \\[\taglnskp]
\xftag{transition}
\end{tabbing}

\begin{tabbing}
\ptn Att.: \=  \xattr{source} \== ' ' \tagsp  \= \strng   \opt \kill
\ptn Att.: \>\> \xattr{source}  \' = ' '  \> \strng   \req   must be
a valid \xattr{id}\\
           \>\> \xattr{target}  \' = ' '  \> \strng   \req   must be
           a valid \xattr{id}\\[.7ex]
\ptn Tag \xtagf{geometricData  transitionType='' labelPos='' labelDist=''
loopDir=''} \\[.5ex]
\pushtabs
\tagindent\= \ptn Pivotal att.: \= \xattr{transitionType} \==
\xval{EdgeL}$|$\xval{EdgeR}$|$\xval{ArcL}$|$\xval{ArcR} \tagsp\= \token  \req  \kill
\> \ptn Pivotal att.: \>\xattr{transitionType} \>=
\xval{EdgeL}$|$\xval{EdgeR}$|$\xval{ArcL}$|$\xval{ArcR}
\> \token  \req\\
\>\>\>\tagsp\xattr{source} must be different from \xattr{target}\\
             \>\>\xattr{transitionType} \> = \xval{Loop}  \tagsp
             \xattr{source} must be equal to  \xattr{target} \\[.5ex]

\>   \ptn Att.: \>\>\xattr{loopDir}\' =
\xval{N}$|$\xval{S}$|$\xval{E}$|$\xval{W}$|$%
\xval{NE}$|$\xval{NW}$|$\xval{SE}$|$\xval{SW}
\token \req  \\
\>\>\>\tagsp or integer between 0 and 360   (\xval{E}~0) \\
\>\>\>\tagsp but only if  \xattr{transitionType} = \xval{Loop}\\
\>\>\> \xattr{labelPos}\'= ' '   float   \opt\\
\>\>\> \xattr{labelDist}\'= ' '   float    \opt\\[.7ex]

\poptabs
\ptn Tag \xtagf{drawingData   drawingClass='' }  or something more complicated
\end{tabbing}


\item  \xtagf{initial}

\begin{tabbing}
\xtag{initial  state=''}\\[\taglnskp]
\tagindent\xtagf{geometricData}\tagsp \= \oneocc \= \opt \kill
\tagindent\xtagf{geometricData} \>\oneocc \>\opt \\[\taglnskp]
\tagindent\xtagf{drawingData}   \>\oneocc \>\opt\\[\taglnskp]
\tagindent\xtagf{label}         \>\oneocc \>\req  \\[\taglnskp]
\xftag{initial}
\end{tabbing}

\begin{tabbing}
\ptn Att.: \xattr{state}  = ' '  \tagsp \strng   \req   must be a valid \xattr{id}
\end{tabbing}

\begin{tabbing}
\ptn Tag \xtagf{geometricData  initialDir='' labelPos=''
labelDist=''} \\[.5ex]
\tagindent (\cf 19)
\end{tabbing}

\begin{tabbing}
\ptn Tag \xtagf{drawingData   drawingClass='' }  or something more complicated
\end{tabbing}

It is assumed that initial states are marked with an incoming arrow.

\smallskip\smallskip
\item  \xtagf{final}

\begin{tabbing}
\xtag{final  state=''}\\[\taglnskp]
\tagindent\xtagf{geometricData}\tagsp \= \oneocc \= \opt \kill
\tagindent\xtagf{geometricData} \>\oneocc \>\opt \\[\taglnskp]
\tagindent\xtagf{drawingData}   \>\oneocc \>\opt\\[\taglnskp]
\tagindent\xtagf{label}         \>\oneocc \>\req  \\[\taglnskp]
\xftag{final}
\end{tabbing}

\begin{tabbing}
\ptn Att.: \xattr{state}  = ' '  \tagsp \strng   \req   must be a valid \xattr{id}
\end{tabbing}

\begin{tabbing}
    \ptn Tag \xtagf{geometricData  finalMod='' finalDir='' 
    labelPos='' labelDist=''}\\[.5ex] 

\tagindent\= \ptn Pivotal att.: \= \xattr{finalMod} \==
\xval{circle}$|$\xval{arrow} \tagsp\= \token  \req  \\
\tagindent \ptn  Other att: \ \cf 19.
\end{tabbing}

\begin{tabbing}
\ptn Tag \xtagf{drawingData   drawingClass='' }  or something more complicated
\end{tabbing}

\smallskip\smallskip
\item \xtagf{label}

\begin{tabbing}
\xtag{label}\\[\taglnskp]
\tagindent\BTRE       \\[\taglnskp]
\xftag{label}
\end{tabbing}



\end{enumerate}

\renewcommand\thesubsection{\thesection.\arabic{subsection}}

\endinput
