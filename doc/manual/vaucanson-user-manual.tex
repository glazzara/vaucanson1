\documentclass[a4paper]{report}

\usepackage[american]{mybabel}
\usepackage[latin1]{inputenc}
\usepackage{a4wide}
\usepackage{myacronym}
\usepackage{amsmath}
\usepackage{amsfonts}
\usepackage{myhyperref}
\usepackage{textcomp}
\usepackage{tabularx}
\usepackage{url}
\usepackage{vaucanson-g}
\usepackage{graphicx}
\usepackage{tabularx}
\usepackage{texi}
\usepackage{misc}

% Mathematical symbols.
\renewcommand{\max}{\textrm{max}}
\renewcommand{\min}{\textrm{min}}
\newcommand{\Z}{\texorpdfstring{\ensuremath{\mathbb{Z}}}{Z}}

\newcommand{\newcal}[1]{%
  \expandafter \newcommand \csname #1c\endcsname%
  {\texorpdfstring{\ensuremath{\mathcal{#1}}}{#1}}%
}
\newcal{A}
\newcal{B}
\newcal{C}
\newcal{T}

% The same argument is output and put in the index.
\usepackage{makeidx}
\makeindex
\newcommand{\Index}[1]{\index{#1}#1}

% Bibliography.
\usepackage{natbib}

\usepackage{mylistings}
\lstset{%
  numbers=left,
  numberstyle=\tiny,
  stepnumber=5,
  numbersep=5pt,
  firstnumber=1,
  basicstyle=\small,
  frame=single,
  language=C++,
  float}

%% Whether the developer's documentation is included.
\newboolean{developer}
\ifx\withDeveloperSection\undefined
  \setboolean{developer}{false}
\else
  \setboolean{developer}{true}
\fi

% Document a figure.
\newenvironment{legend}{%
  \begin{quote}%
    }{%
  \end{quote}%
}

%% Display an interactive session.
\usepackage{alltt}
\newenvironment{shell}
{\begin{alltt}}
{\end{alltt}}

%% -------------------- %%
%% Vaucanson commands.  %%
%% -------------------- %%

%% A TAF-kit function name (i.e., the first argument for TAF-Kit
%% programs).
\newcommand{\taffn}[1]{\code{#1}}
\newcommand{\tafkit}{\textsc{TAF-Kit}\xspace}
\newcommand{\Vauc}{\textsc{Vaucanson}\xspace}

% \begin{fnsection}{NAME}
% -----------------------
% A group of taf-kit functions.
% Putting the ending hline in the end section is tempting, but fails.
\newenvironment{fnsection}[1]{%
  % \item{NAME}{ARGS}{DOC}
  % ----------------------
  % Define a taf-kit function.
  % We use \item because it has Emacs indentation support.
  \renewcommand{\item}[3]{\texttt{##1} \var{##2} & ##3\\}

  \tabularx{\textwidth}{|l|X|}%
    \hline\multicolumn{2}{|c|}{#1} \\\hline%
  }{%
  \endtabularx%
}


%% --------------------------------- %%
%% Automatic generation of content.  %%
%% --------------------------------- %%

% These two commands are recognized by extex which will extract the
% code to run to produce the generated files.  LaTeX will later
% include them.
%
% If you know a simpler means to do that, be my guest...
%
% Note that this system "caches" the result, and uses a Makefile to
% avoid useless recompilation of snippets that did not change.

% This one is not traced, so we can use it to factor definitions
% making the same calls (e.g., includegenerateddot and execdisplay).
%
% Beware that files with an embedded dots in the base name are
% improperly handled: includegraphics tries to interpret it as an extensions.
\newcommand{\includegenerated}[1]{\includegraphics[scale=.6]{#1}}

% This one is scanned by extex.pl.
\newcommand{\includegenerateddot}[1]{\includegenerated{#1}}

% \execcaption{TAG}{COMMAND}
% --------------------------
% Run the COMMAND, which is a Unix pipe, and save the result in a file
% named TAG.tex.  Input it here.  Make explicit the input/output of
% the COMMAND.
%
% If there are redirections, *THEY MUST STICK TO THE FILE*!!!
% Do not pass `echo toto > foo', but `echo toto >foo'.
\newcommand{\execcaption}[2]{\input{#1}}

% \execdisplay{TAG}{PRE-COMMAND}{POST-COMMAND}
% --------------------------------------------
% Run a taf-kit pipe ending with a display, and include the resulting
% automaton.  It will run `PRE-COMMAND display POST-COMMAND >TAG.dot',
% and include the latter.

% Typical use is:
%
% \execdisplay{a1}{vcsn-b}{a1}
%
\newcommand{\execdisplay}[3]{\includegenerated{#1}}

\title{\Vauc User's manual}
\author{The \Vauc group}

%% ---------- %%
%% Document.  %%
%% ---------- %%
\begin{document}

\maketitle

\setcounter{tocdepth}{2}
\tableofcontents

\chapter*{Introduction}
\label{sec:intro}

The \Vauc software platform is dedicated to the computation with
finite state automata.  Here, `finite state automata' is to be
understood in the broadest sense: \emph{weighted} automata on a free
monoid --- that is, automata that not only accept, or recognize,
\emph{words} but compute for every word a \emph{multiplicity} which is
taken a priori in \emph{an arbitrary semiring} --- and even weighted
automata on \emph{non free monoids}.  The latter become far too
general objects.  As for now, are implemented in \Vauc only the
(weighted) automata on (direct) products of free monoids, machines
that are often called \emph{transducers} --- that is automata that
realize (weighted) relations between words\footnote{When the relation
  is ``weighted'' the multiplicity has to be taken in a
  \emph{commutative} semiring.}.

When designing \Vauc, we had three main goals in mind: we wanted
\begin{enumerate}
\item a \emph{general purpose} software,
\item a software that allows a programming style natural to computer
  scientists who work with automata and transducers,
\item  an open and free software.
\end{enumerate}

This is the reason why we implemented so to say \emph{on top} of the
\Vauc platform a library that allows to apply a number of functions on
automata, and even to define and edit automata, without having to
bother with subtleties of \Cxx programming.  The drawback of this is
obviously that the user is given a \emph{fixed} set of functions that
apply to \emph{already typed} automata.  This library of functions
does not allow to write new algorithms on automata but permits to
combine or compose without much difficulties nor efforts a rather
large set of commands.  We call it \tafkit, standing for \emph{Typed
  Automata Function Kit}, as these commands take as input, and output,
automata whose type is fixed.  \tafkit is presented in
\autoref{sec:tafkit}.


This document presents a simple interface to \Vauc: a set of programs
tailored to be used from a traditional shell.  Since they exchange
\emph{typed} XML files, there is one program per automaton type.  Each
program supports a set of operations which depends on the type of the
automaton.

Many users of automata consider only automata whose transitions are
labeled by letters taken in an alphabet, which we call, roughly
speaking, \emph{classical} automata or \emph{Boolean} automata.  The
first program of the \tafkit, \command{vcsn-b}, allows to compute with
classical automata and is described in \autoref{sec:vcsn-b}.

\autoref{sec:vcsn-tdc} describes the program \command{vcsn-tdc} which
allows to compute with transducers, that is, automata whose
transitions are labeled by pair of words, which are elements of a
\emph{product of free monoids}, hence the name.

In \autoref{sec:vcsn-z} we consider the programs of the \tafkit that
compute with automata over a free monoid and with multiplicity, or
\emph{weight} taken in the set of integers equipped with the usual
operations of addition and multiplication, that is, the semiring $\Z$.

% ,
% or with the operations of $\min$ and addition, or the operations of
% $\max$ and addition, that is, the semirings $\Z,\min,+$ and
% $\Z,\max,+$.  Finally, the forth section describes the program
% \command{vcsn-rw-tdc} which allows to compute with transducers, viewed
% as automata on a free monoid (the input monoid) with multiplicity
% taken in the semiring of finite and even rational subsets of another
% free monoid (the output monoid).


%%% Local Variables:
%%% mode: latex
%%% ispell-local-dictionary: "american"
%%% TeX-master: "vaucanson-user-manual"
%%% End:

\chapter{Installation}

\section{Getting \Vauc}

The latest stable version of the \Vauc platform can be downloaded
from \url{http://vaucanson.lrde.epita.fr/}.  The current development
version can be retrieved from its Subversion\footnote{%
%%
  Subversion can be found at \url{http://subversion.tigris.org/}.
%%
} repository as follows:

\begin{shell}
# \kbd{svn checkout https://svn.lrde.epita.fr/svn/vaucanson/trunk vaucanson}
\end{shell}

\section{Building \Vauc}

The following commands build and install the platform (the process
takes an hour on a modern computer):
\begin{shell}
# \kbd{cd vaucanson-1.0}
\end{shell}
Then:
\begin{shell}
# \kbd{./configure}
...
# \kbd{make}
...
# \kbd{sudo make install}
...
\end{shell}

%%% Local Variables:
%%% mode: latex
%%% ispell-local-dictionary: "american"
%%% TeX-master: "vaucanson-user-manual"
%%% End:

\chapter{The \Vauc toolkit}

This chapter prevents a simple interface to \Vauc: a set of programs
tailored to be used from a traditional shell.  Since they exchange
\emph{typed} XML files, there is one program per automaton type.  Each
program support a set of operations which depends on the type of the
automaton.

Six programs are planned to be shipped in a forthcoming version:
\begin{description}
\item[vcsn-b] automata over the Boolean semiring $\mathbb{B}$;
\item[vcsn-z] automata over $(\mathbb{Z},+)$;
\item[vcsn-z-min-plus] automata over $(\mathbb{Z},min)$;
\item[vcsn-z-max-plus] automata over $(\mathbb{Z},max)$;
\item[vcsn-rw-tdc] realtime transducers;
\item[vcsn-tdc] automata over free monoid products.
\end{description}

Currently, only \command{vcsn-b}, \command{vcsn-z}, and
\command{vcsn-tdc}, are implemented.

\newpage

\section{Boolean automata}

This section focuses on the program \Index{\command{vcsn-b}}, the
TAF-Kit component dedicated to Boolean automata.

\subsection{First Contacts}

\command{vcsn-b} and its peer components of \tafkit all share the same
simple interface:

\begin{shell}
# vcsn-b \var{function} \var{automaton} \var{arguments...}
\end{shell}

\noindent
The \var{function} is the name of the operation to perform on the
\var{automaton}, specified as an XML file.  Some functions, such as
evaluation, will require additional arguments, such as the word to
evaluate.  Some other functions, such as \samp{exp-to-aut} do not have
an \var{automaton} argument.

\tafkit is made to work with Unix \emph{pipes}, that is to say, chains
of commands which feed each other.  Therefore, all the functions
produce a result on the standard output, and if an \var{automaton} is
\samp{-}, then the standard input is used.

Other than that, the interface of the \tafkit components is usual,
including options such as \option{--version} and \option{--help}:

\execcaption{vcsn-b-help}{vcsn-b --help}

The whole list of supported commands is available via
\option{--list-commands}:
\execcaption{vcsn-b-commands}{vcsn-b --list-commands}


\subsection{A first example}

\Vauc provides a set of common automata.  The function
\Index{\taffn{list-automata}} lists them all:

\execcaption{b-list}{vcsn-b list-automata}

\begin{figure}[ht] \centering
  \begin{VCPicture}{(0,-2)(6,2)}
    % states
    \State{(0,0)}{A}
    \State{(3,0)}{B}
    \State{(6,0)}{C}
    % initial--final
    \Initial{A}
    \Final{C}
    % transitions
    \EdgeL{A}{B}{a}
    \EdgeL{B}{C}{b}
    \LoopS[.5]{A}{b}
    \LoopN[.5]{A}{a}
    \LoopS[.5]{C}{b}
    \LoopN[.5]{C}{a}
  \end{VCPicture}
  \begin{legend}
    The graphical layout of this automaton was described by hand,
    using the Vaucanson-G \LaTeX{} package.  However, the following
    figures are generated by TAF-Kit, giving a very nice layout, yet
    slightly less artistic.

    The automaton is taken from \citet[Fig. I.1.1, p. 58]{sakarovitch.03.eta}.
  \end{legend}
  \caption{The automaton \index{A1@$\mathcal{A}_1$}$\mathcal{A}_1$}
  \label{fig:a1}
\end{figure}

Let's consider the Boolean automaton $\mathcal{A}_1$
(\autoref{fig:a1}).  It is part of the standard library, and
can be dumped using \Index{\taffn{dump-automaton}}:

\execcaption{a1-dump}{vcsn-b dump-automaton a1}

Usual shell indirections (\samp{|}, \samp{>}, and \samp{<}) can be
used to combine TAF-Kit commands.  For instance, this is an easy means
to bring a local copy of this file:

\execcaption{a1-xml}{vcsn-b dump-automaton a1 >a1.xml}

TAF-Kit uses XML to exchange automata, to get graphical rendering of
the automaton, you may either invoke \Index{\taffn{dot-dump}} and then
use a Dot compliant program, or use \Index{\taffn{display}} that does
both.

\execcaption{a1-dot}{vcsn-b dot-dump a1.xml >a1.dot}
\begin{center}
  \includegenerateddot{a1}
\end{center}


\subsubsection{Determinization of $\mathcal{A}_1$}
To determinize a Boolean automaton, call the
\Index{\taffn{determinize}} function:

\execcaption{a1det}{vcsn-b dump-automaton a1 | vcsn-b determinize - >a1det.xml}

To get information about an automaton, call the \Index{\taffn{info}} function:
\execcaption{a1det-info}{vcsn-b info a1det.xml}

Or use dotty to visualize it:
\execcaption{a1det.dot}{vcsn-b dot-dump a1det.xml >a1det.dot}
\begin{center}
  \includegenerateddot{a1det}
\end{center}

\subsubsection{Minimizing}

The \index{\taffn{minimize}}minimal automaton can be computed the same way:
\begin{shell}
# vcsn_b minimize a1_det.xml > a1_min.xml
\end{shell}

The commands can be composed with pipes from the shell, using
\Index{\samp{-}} to denote the standard input.
\begin{shell}
# vcsn_b determinize a1.xml | vcsn_b minimize - > a1_min.xml
\end{shell}


\subsubsection{Evaluation}

To \index{\taffn{eval}}evaluate whether a word is accepted:

\execcaption{a1.abab}{vcsn-b eval a1.xml 'abab'}
\execcaption{a1.bbba}{vcsn-b eval a1.xml 'bbba'}

\noindent
where 1 (resp. 0) means that the word is accepted (resp. not accepted)
by the automaton.

\subsection{Rational expressions and Boolean automata}

\Vauc provides functions to manipulate rational expressions associated
to Boolean automata. For instance, computing the language recognized
by a Boolean automaton can be done using \Index{\taffn{aut-to-exp}}:

\execcaption{a1.rat}{vcsn-b aut-to-exp a1.xml}
\execcaption{a1det.rat}{vcsn-b aut-to-exp a1det.xml}

\Vauc provides several algorithms that build an automaton that
recognizes a given language.  The following sequence computes the
minimal automaton of \samp{(a+b)*ab(a+b)*}.

\execcaption{l1}{vcsn-b --alphabet=ab standard "(a+b)*a.b.(a+b)*" | vcsn-b minimize - >l1.xml}
\execcaption{l1.dot}{vcsn-b dot-dump l1.xml >l1.dot}
\begin{center}
  \includegenerateddot{l1}
\end{center}

\subsection{Available functions}
%% Une definition plus rigoureuse des algorithmes devrait etre fournie
%% en annexe.
This section gives a brief definition of all functions that \Vauc
provides for manipulating Boolean automata.  All these algorithms are
invoked using \samp{vcsn-b \var{algorithm-name} [\var{arguments}]}. If
the argument is replaced by \samp{-} then the program will read an
argument from the standard input.  All algorithms dump their result to
the standard output, except the ``tests'' functions that also return
an exit status (0 if the test is successful, anything else otherwise).

\smallskip

In the following:
\begin{itemize}
\item \var{a1} and \var{a2} are two Boolean automata described
  in \Vauc XML format;
\item \var{w} is a word, for example \samp{"aabb"} if you are
  working on an alphabet that contains the letters \samp{a} and
  \samp{b};
\item \var{exp} is a rational expression denoting a language;
\item \var{n} is a nonnegative integer.
\end{itemize}


\begin{fnsection}{Input/output work with automata}
\item{define-automaton}{}{Define an automaton from scratch.}
\item{edit-automaton}{a1}{Edit an existing automaton.}
\item{info}{a1}{Print the number of states, transitions, initial and
    final states of \var{a1}.}
\item{display}{a1}{Display the automaton using Dotty.}
\item{dump}{a1}{Dump the automaton to Dot format.}  \hline
\end{fnsection}

\begin{fnsection}{Tests and evaluation on automata}
\item{are-isomorphic}{a1 a2}{Test whether \var{a1} and \var{a2} are
    isomorphic.}
\item{evaluation}{a1 w}{Test whether the word \var{w} is accepted by
    \var{a1}.}
\item{is-deterministic}{a1}{Test whether \var{a1} is deterministic.}
\item{is-empty}{a1}{Test whether \var{a1} accepts no word.}  \hline
\end{fnsection}

\begin{fnsection}{Generic algorithms for automata}
\item{accessible}{a1}{Extract the sub-automaton of accessible states
    of \var{a1}.}
\item{co-accessible}{a1}{Extract the sub-automaton of co-accessible
    states of \var{a1}.}
\item{trim}{a1}{Trim the automaton \var{a1}.}
\item{transpose}{a1}{Compute the automaton accepting the mirror
    language of the one accepted by \var{a1}.}

\item{closure}{[-bf] a1}{$\varepsilon$-removal algorithm.}
  & \option{-b} : backward closure\\
  & \option{-f} : forward closure\\
\item{concatenate}{a1 a2}{Concatenate \var{a1} to \var{a2}.}
\item{sum}{aut1 aut2}{Compute the sum of \var{a1} and \var{a2}.}
\item{normalize}{aut1}{Compute an automaton with unique initial and
    final states, with $\varepsilon$-transitions.}
\item{standardize}{aut1}{Compute an automaton with unique initial
    state without adding $\varepsilon$-transitions.}  \hline
\end{fnsection}

\begin{fnsection}{Generic algorithms for automata on letters}
\item{realtime}{[-bf] a1}{$\varepsilon$-removal algorithm and make
    every transition labeled by a letter.}
  & \option{-b} : backward closure\\
  & \option{-f} : forward closure\\
\item{product}{a1 a2}{Compute the (Cartesian) product of \var{a1} and
    \var{a2}.}
\item{power}{a1 n}{Compute the (Cartesian) product of \var{a1} by
    itself \var{n} times.}
\item{quotient}{a1}{Compute the minimal automaton in bi-simulation
    with \var{a1}.}  \hline
\end{fnsection}

\begin{fnsection}{Algorithms specific to Boolean automata}
\item{determinize}{a1}{Compute the determinized automaton of
    \var{a1}.}
\item{complement}{a1}{Compute an automaton that accepts the complement
    language of the one accepted by \var{a1}.}
\item{minimize}{[-hm] a1}{Minimize the {\em deterministic} automaton
    \var{a1}.}
  & \option{-h} : use the Hopcroft algorithm\\
  & \option{-m} : use the Moore algorithm\\
  \hline
\end{fnsection}

\begin{fnsection}{Conversion between automata and expressions}
\item{aut-to-exp}{a1}{Print a rational expression denoting the
    language accepted by \var{a1}.}
\item{expand}{exp}{Partially expand rational expressions. For
    instance, expanding \samp{a(b+ab(a+b))} will produce
    \samp{aab.(a+b)*+ab}}
\item{derived-term}{exp}{Compute the derived term automaton of
    \var{exp}.}
\item{standard}{exp}{Compute the standard (Glushkov) automaton of
    \var{exp}.}
\item{thompson-of}{exp}{Compute the Thompson automaton of \var{exp}.}
  \hline
\end{fnsection}


\newpage
\section{Transducers}

\Vauc supports two views of transducers, and therefore provides two
programs:
\begin{description}
\item[vcsn-tdc] considering a transducer as a weighted automaton of a
  product of free monoid,
\item[vcsn-rw-tdc] considering a transducer as a machine that takes a
  word as input and produce another word as (two-tape automata).
\end{description}
Both views are equivalent and \Vauc provides algorithms to pass from a
view to the other one.

\subsection{Example}

\begin{figure}[tp]
  \begin{center}
    \begin{VCPicture}{(0,-2)(6,2)}
      % states
      \State{(0,0)}{A} \State{(3,0)}{B} \State{(6,0)}{C}
      \Initial[w]{A}
      \Final[s]{A}
      % transitions
      \LoopN[.5]{A}{\IOL{0}{0}}
      \LoopN[.5]{C}{\IOL{1}{1}}
      \ArcL{A}{B}{\IOL{1}{0}}
      \ArcL{B}{A}{\IOL{1}{1}}
      \ArcL{B}{C}{\IOL{0}{0}}
      \ArcL{C}{B}{\IOL{0}{1}}
    \end{VCPicture}
    \caption{Realtime transducer $\mathcal{T}_1$ computing the
      quotient by 3 of a binary number}
    \label{fig:t1}
  \end{center}
\end{figure}

\begin{figure}[tp]
  \begin{center}
    \begin{VCPicture}{(0,-2)(3,2)}
      % states
      \State{(0,0)}{A} \State{(3,0)}{B}
      \Initial[w]{A}
      \FinalL{s}{A}{(,1)}
      \Final[e]{B}
      % transitions
      \LoopN[.5]{A}{(1,0)}
      \LoopN[.5]{B}{(1,1)}
      \LoopS[.5]{B}{(0,0)}
      \EdgeL{A}{B}{(0,1)}
    \end{VCPicture}
    \caption{Transducer $\mathcal{T}_2$ adding 1 to a binary number}
    \label{fig:t2}
  \end{center}
\end{figure}

The realtime transducer $\mathcal{T}_1$ (\autoref{fig:t1}) gives the
quotient by 3 of a binary number and the transducer $\mathcal{T}_2$
(\autoref{fig:t2}) adds 1 to a binary number.


\subsubsection{Evaluation}

\begin{shell}
# vcsn-rw-tdc evaluation quot_3_rw.xml '110'
\textit{0.1.0}
\end{shell}

\subsubsection{Domain}
The transducer $T$ only accepts binary number which are divisible by 3
as input.
\begin{shell}
# vcsn-rw-tdc domain quot_3_rw.xml > divisible_by_3.xml
\end{shell}
Now the file \file{divisible-by-3.xml} contains the description of a
Boolean automaton that accepts only the numbers divisible by 3.

\subsubsection{to-tdc}
Each transucers can be transformed to the other type of transducer
thanks to the \taffn{to-tdc} and \taffn{to-rw-tdc} functions.
\begin{shell}
# vcsn-rw-tdc to-tdc quot_3_rw.xml > quot_3.xml
# vcsn-tdc to-rw-tdc add1.xml > add1_rw.xml
\end{shell}

\subsubsection{Composing}
\begin{shell}
# vcsn-tdc compose quot_3.xml add1.xml
\end{shell}

\subsection{Available functions}
The following functions are available for both \command{vcsn-rw-tdc}
and \command{vcsn-tdc} programs.  To invoke them, run
\samp{\var{program} \var{algorithm-name} [\var{arguments}]}.

\smallskip

In the following:

\begin{itemize}
\item \var{t1} and \var{t2} are two transducers (either ``realtime''
  or not) described in \Vauc XML format;
\item \var{w} is a word, for example \samp{"aabb"} if you are working
  on an alphabet that contains the letters \samp{a} and \samp{b};
\item \var{a} is a Boolean automaton;
\item \var{t1-rw} is a realtime transducer;
\item \var{t1-fmp} is a transducer (seen as an automaton over a free
  monoid product).
\end{itemize}

\begin{fnsection}{Input/output work with transducers}
\item{define-automaton}{}{Define a transducer from scratch.}
\item{edit-automaton}{t1}{Edit an existing transducer.}
\item{info}{t1}{Print the number of states, transitions, initial and
    final states of \var{t1}.}
\item{display}{t1}{Display the transducer using DOTTY.}  \hline
\end{fnsection}

\begin{fnsection}{Tests and evaluation on tranducers}
\item{are-isomorphic}{t1 t2}{Test if the two transducers are
    isomorphic.}
\item{evaluation}{t1 w}{Compute the evaluation of \var{w} by
    \var{t1}.}
\item{is-empty}{t1}{Test if \var{t1} realizes the empty relation.}
  \hline
\end{fnsection}

\begin{fnsection}{Generic algorithm for tranducers}
\item{closure}{t1}{$\varepsilon$-removal algorithm.}

\item{compose}{t1 t2}{Compute a tranducer realizing $f_2 \circ f_1$,
    where $f_1$ (resp. $f_2$) is the function associated to \var{t1}
    (resp. \var{t2}).}

\item{domain}{t1}{Compute an automaton accepting all input accepted by
    the transducer t1.}

\item{evaluation}{t1}{Compute the evaluation of w by t1.}

\item{evaluation-aut}{t1}{Compute a Boolean automaton describing the
    words produced by the language described by a evaluated by t1.}

\item{image}{t1}{Compute an automaton describing all output produced
    by the transducer t1.}

\item{transpose}{t1}{Compute the transposed of the transducer t1.}

\item{trim}{t1}{Compute the trimmed transducer of t1.}  \hline
\end{fnsection}

\begin{fnsection}{Algorithms for transducers}
\item{sub-normalize}{t1-fmp}{Compute the sub-nomalized transducer of
    \var{t1-fmp}.}
\item{is-sub-normalize}{t1-fmp}{Test if \var{t1-fmp} is
    sub-normalized.}
\item{composition-cover}{t1-fmp}{.}%%FIXME
\item{composition-co-cover}{t1-fmp}{.}%%FIXME
\item{b-compose}{t1-fmp t2-fmp}{Compose \var{t1-fmp} and
    \var{t2-fmp}, two unweighted normalized or sub-normalized
    transducers.}
\item{to-rw-tdc}{t1-fmp}{Compute the equivalent realtime transducer
    of \var{t1-fmp}.}
\item{intersection}{a}{Transform \var{a} in a fmp transducer by
    creating, for each word, a pair containing twice this word.}
  \hline
\end{fnsection}


\begin{fnsection}{Algorithms for ``realtime'' transducers}
\item{realtime}{t1-rw}{Compute the realtime transducer of  \var{t1-rw}.}
\item{is-realtime}{t1-rw}{Test if \var{t1-rw} is realtime.}
\item{to-tdc}{t1-rw}{Compute the equivalent fmp transducer of \var{t1-rw}.}
\hline
\end{fnsection}

\newpage
\section{Weighted automata}

This part shows the use of the program \Index{\command{vcsn-z}}, but
all comments should also stand for the programs
\command{vcsn-z-min-plus} and \command{vcsn-z-max-plus}.

Again, we will toy with some of the automata provided by
\command{vcsn-z}:
\execcaption{vcsn-z-automata}{vcsn-z list-automata}

\subsection{Counting \samp{b}s}

\begin{figure}[tp] \centering
  \begin{VCPicture}{(0,-2)(3,2)}
    % states
    \State{(0,0)}{A}
    \State{(3,0)}{B}
    % initial--final
    \Initial{A} \Final{B}
    % transitions
    \EdgeL{A}{B}{b}
    \LoopS[.5]{A}{b}
    \LoopN[.5]{A}{a}
    \LoopS[.5]{B}{b}
    \LoopN[.5]{B}{a}
  \end{VCPicture}
  \begin{legend}
    Considered without weight, $\mathcal{B}_1$ accepts words with a
    \samp{b}.  With weights, it counts the number of \samp{b}s.
  \end{legend}
  \caption{The automaton \index{B1@$\mathcal{B}_1$}$\mathcal{B}_1$}
  \label{fig:b1}
\end{figure}

Let's consider $\mathcal{B}_1$ (\autoref{fig:b1}, an
$\mathbb{N}$-automaton, \textit{i.e.}  an automaton whose label's
weights are in $\mathbb{N}$.  This time the evaluation of the word
\var{w} by the automaton $\mathcal{B}_1$ will produce a number,
rather than simply accept or reject \var{w}.  For instance let's
evaluate \samp{abab} and \samp{bbab}:

\execcaption{b1.abbb}{vcsn-z dump-automaton b1 | vcsn-z eval - 'abbb'}
\execcaption{b1.abab}{vcsn-z dump-automaton b1 | vcsn-z eval - 'abab'}

\noindent
Indeed, $\mathcal{B}_1$ counts the number of \samp{b}s.

\subsubsection{Power}

Now let's consider the $\mathcal{B}_1^n$, where
\begin{displaymath}
  \mathcal{B}_1^n = \prod_{i=1}^n \mathcal{B}_1, n > 0
\end{displaymath}

\noindent
This is implemented by the \Index{\taffn{power}} function:

\execcaption{b4}{vcsn-z dump-automaton b1 | vcsn-z power - 4 >b4.xml}
\begin{shell}
# vcsn-z power b1.xml 4 > b4.xml
\end{shell}

\noindent
The file \file{b4.xml} now contains the automaton $\mathcal{B}_1^4$.
Lets see what the evaluation of the words \samp{abab} and \samp{bbab}
gives with this automaton:

\execcaption{b4.abbb}{vcsn-z eval b4.xml 'abbb'}
\execcaption{b4.abab}{vcsn-z eval b4.xml 'abab'}

This time one can notice that the automaton $\mathcal{B}_1^4$ returns
the evaluation of $\mathcal{B}_1$ at power 4.

\subsubsection{Quotient}

One drawback of doing successive products of an automaton is
that it creates a lot of new states and transitions.

\execcaption{b1.info}{vcsn-z dump-automaton b1 | vcsn-z info -}
\execcaption{b4.info}{vcsn-z info b4.xml}

One way of reducing the size of our automaton is to use the
\Index{\taffn{quotient}} algorithm.

\execcaption{b4.quot.info}{vcsn-z quotient b4.xml | vcsn-z info -}

\subsection{Available functions}

In this section you will find a brief definition of all functions for
manipulating weighted automata. The following functions are available
for both.  They are called using \command{vcsn-z},
\command{vcsn-z-max-plus}, and \command{vcsn-z-min-plus} run as
\samp{\var{program} \var{algorithm-name} [\var{arguments}]}.

\smallskip

In the following:

\begin{itemize}
\item \var{a1} and \var{a2} are two weighted automata described in
  \Vauc XML format;
\item \var{w} is a word, for example \samp{aabb} if you are working on
  an alphabet that contains the letters \samp{a} and \samp{b};
\item \var{exp} is a rational expression denoting a language;
\item \var{n} is a nonnegative integer.
\end{itemize}


\begin{fnsection}{Input/output work with weighted automata}
\item{define-automaton}{}{Define an automaton from scratch.}
\item{edit-automaton}{a1}{Edit an existing automaton.}
\item{info}{a1}{Print the number of states, transitions, initial and
    final states of \var{a1}.}
\item{display}{a1}{Display the automaton using DOTTY.}  \hline
\end{fnsection}

\begin{fnsection}{Tests and evaluation on weighted automata}
\item{are-isomorphic}{a1 a2}{Test if the two automata are isomorphic.}
\item{evaluation}{a1 w}{Compute the evaluation of \var{w} by
    \var{a1}.}
\item{is-empty}{a1}{.}  \hline
\end{fnsection}

\begin{fnsection}{Generic algorithms for automata}
\item{accessible}{a1}{Extract the sub-automaton of accessible states
    of \var{a1}.}
\item{co-accessible}{a1}{Extract the sub-automaton of co-accessible
    states of \var{a1}.}
\item{trim}{a1}{Trim the automaton \var{a1}.}
\item{transpose}{a1}{Compute the automaton accepting the mirror
    language of the one accepted by \var{a1}.}
\item{closure}{[-bf] a1}{$\varepsilon$-removal algorithm.}
  & \option{-b} : backward closure\\
  & \option{-f} : forward closure\\
\item{concatenate}{a1 a2}{Concatenate \var{a1} to \var{a2}.}
\item{sum}{a1 a2}{Compute the sum of \var{a1} and \var{a2}.}
\item{normalize}{a1}{Compute an automaton with unique initial and
    final states, with $\varepsilon$-transitions.}
\item{standardize}{a1}{Compute an automaton with unique initial state
    without adding $\varepsilon$-transitions.}  \hline
\end{fnsection}

\begin{fnsection}{Generic algorithms for automata on letters}
\item{realtime}{[-bf] a1}{$\varepsilon$-removal algorithm and make
    every transition labeled by a letter.}
  & \option{-b} : backward closure\\
  & \option{-f} : forward closure\\
\item{product}{a1 a2}{Compute the (Cartesian) product of \var{a1} and
    \var{a1}.}
\item{power}{a1 n}{Compute the (Cartesian) product of \var{a1} by
    itself \var{n} times.}
\item{quotient}{a1}{Compute the $\mathbb{Z}$-quotient of \var{a1}.}
  \hline
\end{fnsection}

\begin{fnsection}{Conversion between automata and expressions}
\item{aut-to-exp}{a1}{.}%%FIXME
\item{expand}{exp}{Partially expand rational expressions.}
\item{derived-term}{exp}{Compute the derived term automaton of
    \var{exp}.}
\item{standard}{exp}{Compute the standard (Glushkov) automaton of
    \var{exp}.}
\item{thompson-of}{exp}{Compute the Thompson automaton of \var{exp}.}
  \hline
\end{fnsection}


\section{Building your own automaton}
%%FIXME: Here we should give the usage of define_automaton function.

%%% Local Variables:
%%% mode: latex
%%% TeX-master: "vaucanson-user-manual"
%%% End:

% LocalWords:  determinize

\chapter{\Vauc as a library}

To be written.

%%% Local Variables:
%%% mode: latex
%%% TeX-master: "vaucanson-user-manual"
%%% End:

\ifthenelse{\boolean{developer}}{\chapter{Developer Guide}

The chapter is work in progress.  It is not meant for user of the
\Vauc library, but to developer and contributor who wish to include
code in \Vauc.

\section{Tools}

We use a number of tools during the development, so called
\Index{\dfn{maintainer tools}} because they are not required by the
end user.
\begin{description}
\item[Autoconf] Generates \command{configure} which probes the user
  system to configure the compilation.
\item[Doxygen] Reference documentation generator.  Available as
  \code{doxygen} in most package systems.
\item[rst2latex, rst2html] These tools are used to convert
  reStructuredText into more common formats.  Available in the
  DarwinPorts as \code{py-docutils}.
\end{description}

\section{Contributing Code}
\label{sec:contributing-code}


\subsection{Directory usage}

\index{directories}
The \Vauc package is organized as follows:

\newcommand{\dir}[2]{\path{#1} & #2\\}
\noindent
\begin{tabularx}{\textwidth}{|l|X|}
  \hline
  Directory & Usage\\
  \hline
  \dir{doc}{Documentation.}
  \dir{doc/css}{CSS style for Doxygen.}
  \dir{doc/makefiles}{Sample Makefile to reduce compilation time in
    \Vauc.}
  \dir{doc/manual}{User's (and developer's) manual.}
  \dir{doc/share}{LRDE share repository.}
  \dir{doc/xml}{XML proposal.}
  \dir{include/vaucanson}{Library start point: defines classical entry
    points such as ``boolean\_automaton.hh''.}
  \dir{include/vaucanson/algebra/concept}{Algebra concepts, ``Structure''
    part of an Element.}
  \dir{include/vaucanson/algebra/implementation}{Implementations of
    algebraic Structures.  Some specialized structures too.}
  \dir{include/vaucanson/algorithms}{Algorithms.}
  \dir{include/vaucanson/algorithms/internal}{Internal functions of
    algorithms.}
  \dir{include/vaucanson/automata/concept}{Structure of an
    automaton.}
  \dir{include/vaucanson/automata/implementation}{Its
    implementation.}
  \dir{include/vaucanson/config}{Package configuration and system
    files.}
  \dir{include/vaucanson/contexts}{Context headers.}
  \dir{include/vaucanson/design_pattern}{Element design pattern
    implementation.}
  \dir{include/vaucanson/misc}{Internal headers of the whole
    library. }
  \dir{include/vaucanson/tools}{Tools such as dumper, bencher.}
  \dir{include/vaucanson/xml}{XML implementation.}
  \hline
\end{tabularx}
\begin{tabularx}{\textwidth}{|l|X|}
  \hline
  Directory & Usage\\
  \hline
  \dir{argp}{Argp library for TAF-Kit.}
  \dir{build-aux}{Where Autotools things go.}
  \dir{data}{Misc data, like Vaucanson's XSD, Emacs files.}
  \dir{data/b}{Generated Boolean automata.}
  \dir{data/z}{Generated Automata over Z.}
  \dir{debian}{Debian packaging.}
  \dir{src/benchs}{Benches.}
  \dir{src/demos}{Demos.}
  \dir{src/tests}{Test suite.}
  \dir{taf-kit}{Typed Automata Function Kit, binaries to use \Vauc.}
  \dir{taf-kit/tests}{Test suite using the TAF-Kit.}
  \dir{vaucanswig}{Vaucanswig, a SWIG interface for \Vauc.}
  \dir{vcs}{Version Control System configuration
    (\href{http://rubyforge.org/projects/vcs}{VCS Home Page}).}
  \hline
\end{tabularx}

\subsection{Writing Makefiles}

\begin{description}
\item[Produce the output atomically] Generating an output bits by
  bits, say with a series of \samp{>> \$@}, or even with some common
  programs, can result in invalid files if the process failed at some
  point, or was interrupted.  Note that in that case the (invalid)
  output file is newer than its dependencies, therefore it will
  remain.  Instead, create an temporary output, say \samp{\$@.tmp},
  and as last step, rename it a \samp{\$@}.  Sometimes, using
  \command{move-if-change} makes more sense.

\item[Look for the economy] Do what you can to save useless
  recompilations.  Using move-if-change, especially for generated
  files, does save cycles.  But then, beware that time stamps are not
  updated, which can be troublesome if the Makefile includes
  dependency tracking, as they will not be satisfied.

\item[Always include dependencies] included for the bootstrapping
  process.  In their regular development process, the contributors
  should not have to bootstrap again, that should be only for the
  initial check-out, and some other situation where the layout of the
  project has deeply changed.  Therefore, always hook the generated to
  the changes in the generators.

\item[Hunt Makefile duplication] Prefer an included Makefile to
  copy-and-paste of bits.  That's also true for generated Makefiles:
  put the constant parts in a Makefile to be included in the generated
  Makefiles, rather than copying these bits several times.

  This makes it easier to maintain, and also improves locality: you
  can edit the included Makefile and try it in just one directory,
  instead of having to relaunch the generation of all the Makefiles.
\end{description}


\subsection{Coding Style}

Until this is written, please refer to
\href{http://www.lrde.epita.fr/~akim/compil/assignments.split/Coding-Style.html#Coding-Style}{Tiger's
  Coding Style}.

Emacs users should use the indentation style of \file{data/vaucanson.el}.

\paragraph{Be extremely conservative with header inclusions}
Do not include something that is not needed \strong{by the file
  itself}.  Do not include in a \file{*.hh} something that is required
by the \file{*.hxx} file itself.

\paragraph{Document in the \file{*.hh}}
Do not document in the implementation files, but in the declaration
files.  Unless, of course, the function is private and not
``exported''.

\paragraph{Use the same signature in the declaration and in the
  implementation}
It's a bad idea not to follow the same name on both sides, and it
confuses the reader.  Even worse is not giving names to the argument
in the declarations, giving the impression that the argument is
ignored, while using it for real in the implementation.

Keep also the same template parameter names.

Inconsistency confuses users, peer developers, and... Doxygen too.


\subsubsection{Includes}
\index{\#include}

Please, never use backward relative paths anywhere.  There are very
difficult to follow (because several such strings can designate the
same spot), they make renaming and moving virtually impossible etc.

Relative paths to sub-directories are welcome, although in many
situations they are not the best bet.

In \strong{Makefiles}, please using absolute paths starting from
\samp{\$(top\_srcdir)}.  Unfortunately, because Automake cannot grok
includes with Make macros (except... \samp{\$(top\_srcdir)}), we can't
shorten these.

For \strong{header inclusion}, stacking zillions of \samp{-I} is not
the best solution because
\begin{itemize}
\item you have to work to find what file is really included
\item you are likely to find unexpected name collisions if two
  separate directories happens to have (legitimately) two different
  files share the same name
\item etc.
\end{itemize}

So rather, stick to \emph{hierarchies} of include files, and use
qualified \samp{\#include}s.  For instance, use \samp{-I
  \$(top\_srcdir)/include -I \$(top\_srcdir)/src/tests/include} and
\samp{\#include <vaucanson/...>} falls into the first one
(\file{\$(top\_srcdir)/include} has all its content in
\file{vaucanson}), and \samp{\#include <tests/...>} falls into the
latter since \file{\$(top\_srcdir)/src/tests/include} has all its
content in \file{vaucanson}).

\subsection{Use of macros}
\index{\command{cpp}}

\ac{cpp} is evil, but code duplication is even worse.  Macros can be
useful, as in the following example:

\begin{lstlisting}[language=Vaucanson]
# define PARSER_SET_PROPERTY(prop)			\
      if (parser->canSetFeature(XMLUni::prop, true))	\
	parser->setFeature(XMLUni::prop, true);

PARSER_SET_PROPERTY(fgDOMValidation);
PARSER_SET_PROPERTY(fgDOMNamespaces);
PARSER_SET_PROPERTY(fgDOMDatatypeNormalization);
PARSER_SET_PROPERTY(fgXercesSchema);
PARSER_SET_PROPERTY(fgXercesUseCachedGrammarInParse);
PARSER_SET_PROPERTY(fgXercesCacheGrammarFromParse);

# undef PARSER_SET_PROPERTY
\end{lstlisting}

\noindent
but please, respect the following conventions.
\begin{itemize}
\item Use upper case names, unless they are part of the interface such
  as \code{for\_all\_transitions} and so forth.
\item Make them live short lives, as above: undefine them as soon as
  they are no longer needed.
\item Respect the nesting structure: if \file{foo.hh} defines a macro,
  undefine it there too, not in the included \file{foo.hxx}.
\item Indent \ac{cpp} directives.  The initial dash should always be
  in the first column, but indent the spaces (one per indentation)
  between it and the directive.  The above code snippet was included
  in an outer \code{\#if}.
\item Each header file (\file{.hh}, \file{.hxx}, \ldots) should start
  with a classic \ac{cpp} guard of the form
  \begin{lstlisting}
#ifndef FILE_HH
# define FILE_HH
  ...
#endif // !FILE_HH
  \end{lstlisting}
  GCC has some optimizations on file parsing when this scheme is seen.
\item We often rely on \texttt{grep} and tags to search things. Please
  don't clutter names with \ac{cpp} evilness.
\end{itemize}

For instance, this is bad style:

\begin{lstlisting}[language=Vaucanson]
#define VCSN_choose_semiring(Canarg, Nonarg, Typeret...)                  \
    template <class Self>                                                 \
    template <class T>                                                    \
    Typeret                                                               \
    SemiringBase<Self>::Canarg ## choose_ ## Nonarg ## starable(          \
    SELECTOR(T)) const                                                    \
    {                                                                     \
      return op_ ## Canarg ## choose_ ## Nonarg ## starable(this->self(), \
							   SELECT(T));    \
    };
    VCSN_choose_semiring(can_,non_,bool)
    VCSN_choose_semiring(,,Element<Self, T>)
    VCSN_choose_semiring(,non_,Element<Self, T>)
\end{lstlisting}

\subsection{File Names}

In \Vauc the separator is \samp{\_}, not \samp{-}.  We use the
following file extensions:
\begin{description}
\item[cc] Implementation (compilation unit)
\item[hh] Declarations and documentation
\item[hxx] Inline Implementation
\end{description}

File names should match the class they declare, with the conversion of
name conventions (i.e., from \code{MyClass} to \file{my_class.*}).

\subsection{Type Names}

Although some coding standards recommend against this practice, types
in \Vauc should end with \samp{\_t}.  One exception is traits, where
\code{ret} is commonly used.

\code{self\_t}, when defined, always refers to the current class.

\code{super\_t}, when defined, always refers to \emph{the} super
class.  When there are several, \code{super_t} is not used.  The
macros \code{INHERIT\_TYPEDEF} and \code{INHERIT\_TYPEDEF\_} rely on
this convention.

\subsection{Variable Names}
\newcommand{\textstar}{\texttt{\ensuremath{*}}}
Using long variable names clutters the code, so please, don't name
your variables and arguments like \code{automaton1} or
\code{alphabet}.  Structure members and functions should be
descriptive though.

In order to keep the variable names reasonable in size, and
understandable, there are variable name conventions: some families of
identifiers are reserved for some types of entities.  The conventions
are listed below; developers must follow it, and users are encouraged
to do it too.  In the following list, \samp{\textstar} stands for ``nothing,
or a number''.
\begin{description}
\item[al\textstar, alpha\textstar, A\textstar] alphabets
\item[a\textstar, aut\textstar] automata (\code{automaton\_t}, etc.)
\item[t\textstar, tr\textstar] transitions
\item[p\textstar, q\textstar, r\textstar, s\textstar] states (\code{hstate\_t})
\end{description}

Some variables should be consistently used to refer to some ``fixed''
values.
\begin{description}
\item[monoid\_identity] The neutral for the monoid, the empty word.
\begin{lstlisting}[language=Vaucanson]
monoid_elt_t monoid_identity = a.series().monoid().empty_;
\end{lstlisting}

\item[null\_series] The null series, the 0, the identity for the sum.
\begin{lstlisting}[language=Vaucanson]
series_set_elt_t null_series = a.series().zero_;
\end{lstlisting}

\item[semiring\_elt\_zero] The zero for the weights.
\begin{lstlisting}[language=Vaucanson]
semiring_elt_t semiring_elt_zero = a.series().semiring().wzero_;
\end{lstlisting}
\end{description}

\subsection{Commenting Code}
\label{sec:commenting-code}

Use Doxygen.  Besides the usual interface description, the Doxygen
documentation must include:
\begin{itemize}
\item references to the definitions of the algorithm, e.g., a
  reference to the ``�l�ments de la th�orie des automates'', or even
  an URL to a mailing-list archive.
\item detailed description of the assumptions, or, if you wish, pre-
  and post-conditions.
\item the name of the developer
\item use the \texttt{@pre} and \texttt{@post} tags liberally.
\end{itemize}

Don't try to outsmart your tool, even though it does not use the words
``param'' and ``arg'' as we do, stick to \emph{its} semantics (let
alone to generate correct documentation without warnings).  This is
correct:
\begin{lstlisting}[language=C++]
  /**
   * Delete memory associated with a stream upon its destruction.
   *
   * @arg \c T	Type of the pointed element.
   *
   * @param ev	IO event.
   * @param io	Related stream.
   * @param idx	Index in the internal extensible array of a pointer to delete.
   *
   * @see iomanip
   * @author Thomas Claveirole <thomas.claveirole@lrde.epita.fr>
   */
  template <class T>
  void
  pword_delete(std::ios_base::event ev, std::ios_base &io, int idx);
\end{lstlisting}
while this is not:
\begin{lstlisting}[language=C++]
  /** ...
   * @param T	Type of the pointed element.
   *
   * @arg ev	IO event.
   * @arg io	Related stream.
   * @arg idx	Index in the internal extensible array of a pointer to delete.
   * ... */
\end{lstlisting}

\subsection{Writing Algorithms}

There is a number of requirement to be met before including an
algorithms into the library:
\begin{description}
\item[Document the algorithm] See \autoref{sec:commenting-code}.
\item[Comment the code] Especially if the code is a bit tricky, or
  smart, or avoids nasty pitfalls, it \emph{must} be commented.

\item[Bind the algorithm to \tafkit]

\item[Include tests] See \autoref{sec:writing-tests} for more
  details.  Tests based on \tafkit are appreciated.  Note that tests
  require test cases: to exercise an algorithm, not any automaton will
  do, try to find relevant samples.  Again, ETA is a nice source of
  inspiration.

\item[Complete the documentation]  The pre- and post-conditions should
  also be described here.
\end{description}

When submitting a patch, make it complete (i.e., including the
aforementioned items), and provide a ChangeLog.  See
\href{http://www.lrde.epita.fr/dload/guidelines/guidelines.html}{Le
  Guide du \lrde}, section ``La maintenance de projets'' and
especially ``�crire un ChangeLog'' for more details.

Because \Vauc uses Trac, ChangeLog entries should explicit refer to
tickets (e.g., ``Fixing issue \#38: implement is\_ambiguous''), and
possible previous revisions (e.g., ``Fix a bug introduced in
[1224]'').

\subsection{Writing Tests}
\label{sec:writing-tests}


\subsection{Mailing Lists}
\label{sec:mailing-lists}

\Vauc comes with a set of mailing lists:
\begin{description}
\item[vaucanson@lrde.epita.fr] General discussions, feature request etc.
\item[vaucanson-bugs@lrde.epita.fr] To report errors in code,
  documentation, web pages, etc.
\item[vaucanson-patches@lrde.epita.fr] To submitted patches on code,
  documentation, and so forth.
\item[vaucanson-private@lrde.epita.fr] To contact privately the
  \Vauc team.
\end{description}

Please, bear in mind that there are these lists have many readers,
therefore this is a WORM medium: Write Once, Read Many.  As a
consequence:
\begin{itemize}
\item Be complete.\\
  One should not strive to understand what you are referring to, so
  always include proper references: URLs, Ticket numbers \emph{and
    summary}, etc.
\item Be concise.\\
  Write short, spell checked, understandable sentences.  Reread
  yourself, remove useless words, be proud of what you wrote.  Show
  respect to the reader.  Spare us useless messages.
\item Be structured.\\
  Quick and dirty replies with accumulated layers of replies at the
  bottom of the message is not acceptable.  The right ordering is not
  the one that is the quickest to write, but the easiest to read.
\item Be attentive.\\
  Lists are not write-only: consider the feedback that is given with
  respect.
\end{itemize}

As an example of what's not to be done, avoid answering to yourself to
point out you made a spell mistake: we can see that, and that's a
waste of time to read another message for that.  Also, there is no
hurry, it would probably be better to wait a bit to have a complete,
well thought out, message, rather than a thread of 4 messages
completing, contradicting, each other.  Finally, if you still need to
fix your message, supersede it, or even cancel it.

\section{Vaucanson I/O}
\label{sec:vaucanson-io}

January 2005

Here is some information about input and output of automata in
\href{http://www.lrde.epita.fr/vaucanson}{Vaucanson}.

\subsection{Introduction}

As usual, the structure of the data representing an automaton in a flat
file is called the file format.

There are several input and output formats for Vaucanson
automata. Obviously:
\begin{itemize}
\item {}
input formats are those that can be read from, i.e. from which an
automaton can be loaded.

\item {}
output formats are those that can be written to, i.e. to which an
automaton can be dumped.

\end{itemize}

Given these definitions, here is the meat:
\begin{itemize}
\item {}
Vaucanson supports Graphviz (dot) as an output format. Most kinds of
automata can be dumped as dot-files. Through the library this format
is simply called \texttt{dot}.

\item {}
Vaucanson supports XML as an input and output format. Most kinds of
automata can be read and written to and from XML streams, which
Vaucanson does by using the Xerces-C++ library. Through the library
this format is simply called \texttt{xml}.

\item {}
Vaucanson supports the FSM toolkit I/O format as an input and output
format. This allows for basic FSM interaction. Only certain kinds of
weighted automata can be meaningfully input and output with this
format. Through the library this format is simply called \texttt{fsm}.

\item {}
Vaucanson supports a simple informative textual format as an input
and output format. Most kinds of automata can be read and written to
and from this format. Through the library this format is simply called
\texttt{simple}.

\end{itemize}


%___________________________________________________________________________

\subsection{Dot format}

This format provides an easy way to produce a graphical representation
of an automaton.

Output using this format can be given as input to the Graphviz \texttt{dot}
command, which can in turn produce graphical representations in
Encapsulated PostScript, PNG, JPEG, and many others.

It uses Graphviz' ``directed graph'' subformat.

If you want to see what it looks like go to the \texttt{data/b}
subdirectory, build the examples and run them with the ``dot''
argument.

For Graphviz users:

Each graph generated by Vaucanson can be named with a string that also
prefixes each state name. If done so, several automata can be grouped
in a single graph by simply concatenating the Vaucanson outputs.


%___________________________________________________________________________

\subsection{XML format}

This format is intended to be an all-purpose strongly typed input and
output format for automata.

Using it requires:
\begin{itemize}
\item that the Xerces-C++ library is installed and ready to use by the
  C++ compiler that is used to compile Vaucanson.

\item configuring Vaucanson to use XML.

\item computer resources and time.
\end{itemize}

What you gain:
\begin{itemize}
\item support for the Greater and Better I/O format. See documentation
  in the \texttt{doc/xml} subdirectory for further information.
\end{itemize}

If you want to see what it looks like go to the \texttt{data/b}
subdirectory, build the examples and run them with the \texttt{xml}
argument.


%___________________________________________________________________________

\subsection{FSM format}

This format is intended to provide a basic level of compatibility with
the FSM tool kit. (FIXME: references needed)

Like FSM, support for this format in Vaucanson is limited to
deterministic automata. It probably does not work with transducers,
either.

It is not meant to be used that much apart from performance comparison
with FSM. Some code exists to simulate FSM, in
\texttt{src/demos/utilities/fsm}.

If you want to see what it looks like go to the \texttt{data/b}, build
the examples and run them with the \texttt{fsm} argument.


%___________________________________________________________________________

\subsection{Simple format}

Initially intended to be a quick and dirty debugging input and output
format, this format actually proves to be a useful, compact and
efficient textual representation of automata.

Advantages over XML:
\begin{itemize}
\item does not require additional 3rd party software,

\item simple and efficient (designed to be read and written to streams
  with very low memory footprint and minimum complexity),

\item less bytes in file,

\item not strongely typed (can be dumped from one automaton type and
  loaded to another).
\end{itemize}

Drawbacks from XML:
\begin{itemize}
\item not strongely typed (one cannot know what automaton type to
  build by only looking at the raw data).

\item currently does not (probably) support transducers.
\end{itemize}

If you want to see what it looks like go to the \texttt{data/b}, build
the examples and run them with the \texttt{simple} argument.


%___________________________________________________________________________

\subsection{Using input and output}

The library provides an infrastructure for generic I/O, which
(hopefully) will help supporting more formats in the future.

The basis for this infrastructure is the way a developer C++ using the
library will use it:

\begin{lstlisting}[language=Vaucanson]
#include <vaucanson/tools/io.hh>

/* to save an automaton */
output_stream << automaton_saver(automaton, converter, format)

/* to load an automaton */
input_stream >> automaton_loader(automaton, converter, format, merge_states)
\end{lstlisting}

Where:
\begin{description}
\item[\texttt{automaton}] is the automaton undergoing input or output.
  Note that the object must already be constructed, even to be read
  into.

\item[\texttt{converter}] is a helper class that is able to convert
  automaton transitions to character strings and possibly vice-versa.

\item[\texttt{format}] is a helper class that is able to convert the
  automaton to (and possibly from) a character string, using the
  converter as an argument.

\item[\texttt{merge\_states}] is an optional argument that should be
  omitted in most cases. For advanced users, it allows loading a
  single automaton from several different streams that share the same
  state set.
\end{description}


%___________________________________________________________________________

\subsubsection{About converters}

The \texttt{converter} argument is mandatory. There are several converter
types already available in Vaucanson. See below.

An I/O converter is a function object with one or both of the following:
\begin{itemize}
\item an operation that takes an automaton, a transition label and
  converts the transition label to a character string (std::string).
  This is called the output conversion.

\item an operation that takes an automaton, a character string and
  converts the character string to a transition label. This is called
  the input conversion.
\end{itemize}

Vaucanson already provides these converters:
\begin{description}
\item[{\texttt{vcsn::io::string\_out}, bundled with \texttt{io.hh}.}]

  Provides the output conversion only. Uses the C++ operator
  {\textless}{\textless} to create a textual representation of
  transition labels. Should work with all label types.

\item[{\texttt{vcsn::io::usual\_converter\_exp}, defined in
    \texttt{tools/usual\_io.hh}.}]

  Provides both input and output conversions. Uses the C++ operator
  {\textless}{\textless} to create a textual representation of
  transition labels, but requires also that algebra::parse can read
  back that representation into a variable of the same type. It is
  mostly used for generalized automata where transitions are labeled
  by rational expressions, hence the name.

\item[{\texttt{vcsn::io::usual\_converter\_poly<ExpType>}, defined
    in \texttt{tools/usual\_io.hh}.}]  Provides both input and
  output conversions. Converts transition labels to and from ExpType
  before (after) doing I/O. The implementation is meant to be used
  when labels are polynoms, and using the generalized (expression)
  type as ExpType.
\end{description}


%___________________________________________________________________________

\paragraph{Notes about XML and converters}

When the XML I/O format was implemented, the initial converter system
was not used. Instead a specific converter system was re-designed
specifically for this format.

(FIXME: explain why!)

(FIXME: why hasn't the generic converter for XML been ported back to
fsm and simple formats?)

Because of this, when using XML I/O the ``converter'' argument is
completely ignored by the format processor. Usually you can see
\texttt{vcsn::io::string\_output} mentioned.

(FIXME: this is terrible! it must be patched to use an empty
vcsn::io::xml\_converter\_placeholder or something like it).


%___________________________________________________________________________

\subsubsection{About formats}

The \texttt{format} argument is mandatory. It specifies an instance of the
object in charge of the actual input or output.

A format object is a function object that provides one or both the
following operations:
\begin{itemize}
\item an operation that takes an output stream, the caller
  \texttt{automaton\_saver} object, and the \texttt{converter}
  object. This is called the output operation.

\item an operation that takes an input stream and the caller
  \texttt{automaton\_loader} object. This is called the input
  operation.  Note that this operation does not uses the
  \texttt{converter} object, because it should call back the
  \texttt{automaton\_loader} object to actually perform string to
  transition label conversions.
\end{itemize}

Format objects may require arguments to be constructed, such as the
title of the automaton in the output.

Format objects for a format should be defined in a
\texttt{tools/xxx\_format.hh} file.

Vaucanson provides the following format objects:
\begin{description}
\item[{\texttt{vscn::io::dot(const std::string{\&} digraph\_title)},
    in \texttt{tools/dot\_format.hh}.}]

  Provides an output operation for the Graphviz \texttt{dot}
  subformat. The title provided when buildint the \texttt{dot} object
  in Vaucanson becomes the title of the graph in the output data and a
  prefix for state names. Therefore the title must contain only
  alphanumeric characters or the underscore (\_), and no spaces.

\item[{\texttt{vcsn::io::simple()}, in
    \texttt{tools/simple\_format.hh}.}]

  Provides both input and output operations for a simple text format.

\item[{\texttt{vcsn::xml::XML(const std::string{\&} xml\_title)}, in
    \texttt{xml/XML.hh}.}]

  Provides both input and output operations for the Vaucanson XML I/O
  format.
\end{description}

(FIXME: why not tools/xml\_format.hh with proper includes of headers
in xml/?)

(FIXME: really the FSM format should have a format object too.)


%___________________________________________________________________________

\subsection{Examples}

Create a simple dot output for an automaton a1:
\begin{lstlisting}[language=vaucanson]
std::ofstream fout("output.dot");
fout << automaton_saver(a1, vcsn::io::string_output(), vcsn::io::dot("a1"));
fout.close()
\end{lstlisting}

Output automaton a1 to XML, read it back into another automaton a2
(possibly of another type):
\begin{lstlisting}[language=Vaucanson]
std::ofstream fout("file.xml");
fout << automaton_saver(a1, NULL, vcsn::xml::XML());
fout.close()

std::ifstream fin("file.xml");
fin >> automaton_loader(a2, NULL, vcsn::xml::XML());
fin.close()
\end{lstlisting}

Do the same, but this time using the simple format. The automata are
generalized, i.e. labeled by expressions:
\begin{lstlisting}[language=Vaucanson]
std::ofstream fout("file.txt");
fout << automaton_saver(a1, vcsn::io::usual_converter_exp(), vcsn::io::simple());
fout.close()

std::ifstream fin("file.txt");
fin >> automaton_loader(a2, vcsn::io::usual_converter_exp(), vcsn::io::simple());
fin.close()
\end{lstlisting}


%___________________________________________________________________________

\subsection{Internal scenario}

What happens in Vaucanson when you write:
\begin{lstlisting}[language=Vaucanson]
fin >> automaton_loader(a1, c1, f1)
\end{lstlisting}
?

\begin{enumerate}
\item function \texttt{automaton\_loader} creates an object AL1 of
  type \texttt{automaton\_loader\_} that memorizes its arguments.

\item \texttt{automaton\_loader()} returns AL1.

\item \texttt{operator>{}>(fin, AL1)} is called.

\item \texttt{operator>{}>} says to format object f1: ``hi, please use
  fin to load something with AL1''.

\item f1 scans input stream fin. Things may happen then:
  \begin{itemize}
  \item f1 finds a state numbered N. Then it says to AL1: ``hey, make
    a new state into the output automaton, keep its handler s1 for
    yourself and remember it is associated to N''.  (callback
    \texttt{AL1.add\_state})

  \item f1 finds a transition from state numbered N to state P,
    labeled with character string S. Then it says to AL1: ``hey,
    create a transition with N, P, and S.'' (callback
    \texttt{AL1.add\_transition}).  Then:
    \begin{itemize}
    \item AL1 remembers handler for state N (s1)

    \item AL1 remembers handler for state P (s2)

    \item AL1 says to converter c1: ``hey, make me a transition label
      from S''

    \item AL1 creates transition from s1 to s2 using converted label
      into output automaton.
\end{itemize}
\end{itemize}

\item When f1 is finished, it returns control to \texttt{operator>{}>}
  and then calling code.

  Of course since everything is statically compiled using templates
  there is no performance drawback due to the intensive use of
  callbacks.

\end{enumerate}

%___________________________________________________________________________

\subsection{Convenience utilities}

For most formats the (relatively) tedious following piece of code:
\begin{lstlisting}[language=Vaucanson]
output_stream << automaton_saver(a, CONVERTER(), FORMAT(...))
\end{lstlisting} %>>
is also available as:

\begin{lstlisting}[language=Vaucanson]
FORMAT_dump(output_stream, a, ...)
\end{lstlisting}

If available, this convenience utility is defined in
\texttt{tools/XXX\_dump.hh}.

Conversely, the following piece of code:

\begin{lstlisting}[language=Vaucanson]
input_stream >> automaton_loader(a, CONVERTER(), FORMAT(...))
\end{lstlisting}

is usually also available as:

\begin{lstlisting}[language=Vaucanson]
FORMAT_load(input_stream, a, ...)
\end{lstlisting}

If available, this convenience utility is defined in
\texttt{tools/XXX\_load.hh}.

(FIXME: move fsm\_load away from fsm\_dump.hh!)

As of today (2006-03-17) the FSM format is only available using the
fsm\_load() and fsm\_dump() interface.

%%% Local Variables:
%%% mode: latex
%%% TeX-master: "vaucanson-user-manual"
%%% End:
}{}

\appendix

% Generated, do not edit by hand.
\chapter{Automaton Library}

\Vauc comes with a set of interesting automata that can be used to toy
with \tafkit (\autoref{sec:tafkit}) for instance.  In the chapter, we
present each one of these automata.


\section{Boolean Automata}
\subsection{a1}
\label{b:a1}
\execdisplay{al-b-a1}{vcsn-b dump-automaton a1 | vcsn-b}{-}

\subsection{b1}
\label{b:b1}
\execdisplay{al-b-b1}{vcsn-b dump-automaton b1 | vcsn-b}{-}

\subsection{div3base2}
\label{b:div3base2}
\execdisplay{al-b-div3base2}{vcsn-b dump-automaton div3base2 | vcsn-b}{-}

\subsection{double-3-1}
\label{b:double-3-1}
\execdisplay{al-b-double-3-1}{vcsn-b dump-automaton double-3-1 | vcsn-b}{-}

\subsection{ladybird-6}
\label{b:ladybird-6}
\execdisplay{al-b-ladybird-6}{vcsn-b dump-automaton ladybird-6 | vcsn-b}{-}

\section{\texorpdfstring{$\mathbb{Z}$}{Z}-Automata}
\subsection{b1}
\label{z:b1}
\execdisplay{al-z-b1}{vcsn-z dump-automaton b1 | vcsn-z}{-}

\subsection{c1}
\label{z:c1}
\execdisplay{al-z-c1}{vcsn-z dump-automaton c1 | vcsn-z}{-}

\section{Transducers}
\subsection{t1}
\label{tdc:t1}
\execdisplay{al-tdc-t1}{vcsn-tdc dump-automaton t1 | vcsn-tdc}{-}

\subsection{u1}
\label{tdc:u1}
\execdisplay{al-tdc-u1}{vcsn-tdc dump-automaton u1 | vcsn-tdc}{-}


\chapter{A proposal for an XML format for automata}
\label{sec:xml}

This is not a complete description of the \Vauc proposal for an XML
format for automata.  The interested reader will find such a
description at the following URL.  We just present here few examples
of files, that should give an idea on how these files are built.

\begin{lstlisting}
<automaton name="a1" xmlns="http://vaucanson.lrde.epita.fr">
   <content>
      <states>
         <state name="s0"/>
         <state name="s1"/>
         <state name="s2"/>
      </states>
      <transitions>
         <transition src="s0" dst="s0" label="a"/>
         <transition src="s0" dst="s0" label="b"/>
         <transition src="s0" dst="s1" label="a"/>
         <transition src="s1" dst="s2" label="b"/>
         <transition src="s2" dst="s2" label="a"/>
         <transition src="s2" dst="s2" label="b"/>
         <initial state="s0"/>
         <final state="s2"/>
      </transitions>
   </content>
</automaton>
\end{lstlisting}

%%% Local Variables:
%%% mode: latex
%%% TeX-master: "vaucanson-user-manual"
%%% End:


\printindex
\bibliographystyle{apalike}
\bibliography{vaucanson.bib}

\end{document}


%%% Local Variables:
%%% mode: latex
%%% TeX-master: t
%%% End:
