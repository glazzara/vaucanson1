% \documentclass{article}
% \usepackage{graphicx}
% %
% %\setlength{\textheight}{9.9in} \setlength{\textwidth}{7.41in}
% %\setlength{\topmargin}{-0.7in} \setlength{\evensidemargin}{-.031in}
% %\setlength{\oddsidemargin}{-.51in}


\setlength{\parskip}{0.1in} \setlength{\parindent}{0in}

\newcommand{\exercise}[2]{{\bf Exercise #1.} #2}
\newcommand\xmatch{\mathop{{\bf match}}}
\newcommand\xwith{\mathrel{{\bf with}}}
% \newcommand\xbar{\mathrel{|}}
\newcommand\xif{\mathop{{\bf if}}}
\newcommand\xthen{\mathrel{{\bf then}}}
\newcommand\xelse{\mathrel{{\bf else}}}
\newcommand\xlet{\mathop{{\bf let}}}
\newcommand\xrec{\mathrel{{\bf rec}}}
\newcommand\Unit{{\it Unit}}
\newcommand\Void{{\it Void}}
\newcommand\xext{\quad\mathop{\hbox{{\it ext}}}}
\newcommand\nat{{\mathbb N}}

% \newcommand\vbar{\mathrel{|}}

\newcommand\mb[1]{{\bf #1}}
\newcommand\ms[1]{{\it #1}}

% \newcommand\tab{\hskip3em}
% \newenvironment{code}{%
%   \begin{tabbing}
%     \tab\=\tab\=\tab\=\tab\=\tab\=\tab\=\tab\=\tab\=\kill}
%   {\end{tabbing}}

% \begin{document}
% \begin{center}
% \begin{Large}


% \newcommand{\VGIFig}{figures/VGI/}

\chapter{Vaucanson Graphical Interface (VGI)}
\label{chp:vgi}
Vaucanson Graphical Interface (VGI)
% \end{Large}
% \end{center}

\noindent
Available at:  \code{http://code.google.com/p/vgi/} \\
Date last updated:
July 11, 2012

% \bigskip

% \noindent
% \begin{Large}
\section{Overview}
% \end{Large}



Vaucanson Graphical Interface (VGI) is a graphical user interface
(GUI) for the Vaucanson library, a software platform dedicated to
the manipulation of weighted finite state automata.  VGI provides
the functionalities to visualize and edit finite state automata
graphically while the Vaucanson library provides computations and
algorithms operating on automata.

% \bigskip
\section{Installation}

% \noindent
% \begin{Large}
\subsection{Requirements}
% \end{Large}

\noindent
\underline{Minimum for Visualization and Graphical Editing} \\
Any operating system that supports Java (Tested on Windows XP,
Windows 7, Ubuntu, and Mac OS X) \\ Java SE 6 Runtime Environment
(JRE 6)

\noindent \underline{For Running Algorithms Provided by Vaucanson}\\
Requires TAF-Kit from the Vaucanson library, which can be compiled
from source on Unix-like operating system (Tested on Ubuntu and Mac
OS X)

\noindent \underline{For Development and Modification of Source
Code}\\ Any operating system that supports Java (Tested on Windows
XP, Windows 7, Ubuntu, and Mac OS X) \\ Java SE 6 Development Kit
(JDK 6) \\ NetBeans IDE 7.1 optional but highly recommended




% \noindent
% \begin{Large}
\subsection{Installation and Start-up}
% \end{Large}


As long as the minimum required operating system and Java runtime
environments are available, no installation is needed to use the
visualization and graphical editing functions of VGI.  To run
algorithms implemented in Vaucanson, please compile the library to
produce TAF-Kit following the instructions for Vaucanson.

To start VGI, simply double click on the {\tt vgi.jar} file in the
downloaded package.  Alternatively, one can start VGI from a
terminal window by first navigating to the directory containing {\tt
vgi.jar} and then typing the following command:
\begin{center} {\tt java -jar vgi.jar} \end{center}



% \bigskip
% 
% \noindent
% \begin{Large}
\subsection{Getting and Compiling the Source Code}
% \end{Large}


Since VGI is developed using the NetBeans IDE, it is highly
recommended that NetBeans be used to get and compile the source
code.  The complete source code is available online in the Git
repository https://code.google.com/p/vgi/ and can be downloaded
using NetBeans' built-in Git client by the following steps:
\begin{enumerate}
\item   In the menu bar of NetBeans, select {\bf Team -$>$ Git -$>$ Clone
...}
\item
For the Remote Repository step, enter {\bf
https://code.google.com/p/vgi/} for the Repository URL field, leave
the User and Password fields empty, and click on the Next button.
Note:  The repository URL starts with {\bf https} not http.
\item For the Remote Branches step, check the box beside the {\bf
master} branch and click on the Next button.
\item  For the Destination Directory step, adjust the fields as
desired or leave them as default and click on the Finish button.
\end{enumerate}

Alternatively, one can use the command-line Git client to clone the
VGI source code repository by the following command in a terminal
window: {\tt git clone  https://code.google.com/p/vgi/}

With the source code downloaded and the project opened in NetBeans,
VGI can be readily compiled and run by selecting {\bf Run -$>$ Run
Project} menu item in the menu bar.



% \bigskip
% 
% \noindent
% \begin{Large}
\section{Using VGI}

\subsection{Create a New Automaton}
% \end{Large}

\begin{enumerate}
\item   Select File -$>$ New in the menu bar of the VGI window.
\begin{center}
\includegraphics[width=3in]{\VGIFig fig1.eps}
\end{center}

\item   Select a semiring for the weight and edit the alphabet symbols
to define basic information for a weighted automaton.
Alternatively, click the Default button to quickly define a
Boolean automaton with symbols 'a' and 'b' in the alphabet
\begin{center}
\includegraphics[width=3in]{\VGIFig fig2.eps}
\end{center}

\item   Click the OK button to create a new, empty automaton with the specified settings.
\begin{center}
\includegraphics[width=3in]{\VGIFig fig3.eps}
\end{center}

\item A majority of VGI's functions are available in the context pop-up menu
after clicking the right mouse button on an object.  Try clicking the right mouse button
anywhere in the empty space in the newly created, untitled window
and the {\bf Add State} option will appear.
\begin{center}
\includegraphics[width=3in]{\VGIFig fig4.eps}
\end{center}

\item  Click the right mouse button on the newly added state
$s0$ and different options appear.  Select the {\bf Add Transition
(From)} option.
\begin{center}
\includegraphics[width=3in]{\VGIFig fig5.eps}
\end{center}

\item  Click the right mouse button on the state $s0$ again and select
the {\bf Add Transition (To)} option.
\begin{center}
\includegraphics[width=3in]{\VGIFig fig6.eps}
\end{center}

\item A loop transition is added to the state $s0$.
\begin{center}
\includegraphics[width=3in]{\VGIFig fig7.eps}
\end{center}

\item   Add another state s1 and add a transition from $s0$ to $s1$.
\begin{center}
\includegraphics[width=3in]{\VGIFig fig8.eps}
\end{center}

\item   Double click on the transition from $s0$ to $s1$ and the Weighted Regular Expression Editor appears.
\begin{center}
\includegraphics[width=3in]{\VGIFig fig9.eps}
\end{center}

\item  Click on the {\bf Alphabet Symbol} drop-down menu and select $'b'$.  Then click on the OK button.
\begin{center}
\includegraphics[width=3in]{\VGIFig fig10.eps}
\end{center}

\item 11. Click on the state $s0$ and check the {\bf Initial State}
 check box in the properties panel on the left hand side of the window to set $s0$ as an initial state.
\begin{center}
\includegraphics[width=3in]{\VGIFig fig11.eps}
\end{center}

\item  Continue editing the automaton until it appears as the screenshot below,
which contains a Boolean (classical) finite state automaton that
accepts the regular expression $a^*ba^*$.
\begin{center}
\includegraphics[width=3in]{\VGIFig fig12.eps}
\end{center}

\item  Selecting {\bf File -$>$ Save} to save the newly created automaton
somewhere so it can be used for the next part.
\begin{center}
\includegraphics[width=3in]{\VGIFig fig13.eps}
\end{center}
\end{enumerate}


% \bigskip
% 
% 
% \noindent
% \begin{Large}
\subsection{Running Algorithms Provided by Vaucanson}
% \end{Large}
% 
% \bigskip

\noindent
 Requirement:  compiled TAF-Kit
\begin{enumerate}
\item  Open the automaton that accepts the regular expression
$a^*ba^*$ and select {\bf Algorithms -$>$ Set TAF-Kit path ...} in
the menu bar of the VGI window.
\begin{center}
\includegraphics[width=3in]{\VGIFig fig14.eps}
\end{center}

\item  Working TAF-Kit scripts are usually available in the {\bf
taf-kit/tests}
subdirectory under the {\bf vaucanson-1.4} directory.
\begin{center}
\includegraphics[width=3in]{\VGIFig fig15.eps}
\end{center}

\item   Select {\bf Algorithms -$>$ Operations on all automata and rational
expressions: -$>$ aut-to-exp}.
\begin{center}
\includegraphics[width=3in]{\VGIFig fig16.eps}
\end{center}

\item The result of converting this automaton to regular expression is as expected.
\begin{center}
\includegraphics[width=3in]{\VGIFig fig17.eps}
\end{center}
\end{enumerate}


% \noindent
% \begin{Large}
\subsection{Layout}
% \end{Large}

\begin{enumerate}
\item  Here is an example of an automaton laid out haphazardly.
\begin{center}
\includegraphics[width=3in]{\VGIFig fig18.eps}
\end{center}

\item    Select {\bf  Layout -$>$ feature}
in the menu bar of the VGI window, and a more reasonable layout for the same automaton results.
\begin{center}
\includegraphics[width=3in]{\VGIFig fig19.eps}
\end{center}

\end{enumerate}

\endinput