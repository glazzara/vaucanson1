\documentclass[a4paper]{report}

\usepackage[american]{babel}
\usepackage[latin1]{inputenc}
\usepackage{a4wide}
\usepackage{amsmath}
\usepackage{amsfonts}
\usepackage{hyperref}
\usepackage{textcomp}
\usepackage{tabularx}
\usepackage{url}
\usepackage{graphicx}
\usepackage{xspace}

%% -------------------- %%
%% Tikz configuration.  %%
%% -------------------- %%

\usepackage{pgf,tikz}
\usetikzlibrary{automata}
\usetikzlibrary{arrows}
\tikzstyle{automaton}=[shorten >=1pt,
                       node distance=2cm,
                       pos=.5,
                       >=stealth',
                       every state/.style={draw,very thick,minimum size=7mm},
                       every edge/.style={draw,thick},
                       initial distance=7mm,
                       accepting distance=7mm,
                       initial text=]
\tikzstyle{loop right}=[right,out=25,in=-25,loop,pos=.5]
\tikzstyle{loop above}=[above,out=115,in=65,loop,pos=.5]
\tikzstyle{loop left}=[left,out=205,in=155,loop,pos=.5]
\tikzstyle{loop below}=[below,out=295,in=245,loop,pos=.5]
\tikzstyle{loop 60}=[out=85,in=35,loop,pos=.5]
\tikzstyle{loop 120}=[out=145,in=95,loop,pos=.5]
\tikzstyle{loop -60}=[out=-35,in=-85,loop,pos=.5]
\tikzstyle{loop -120}=[out=-95,in=-145,loop,pos=.5]
\tikzstyle{accepting}=[accepting by arrow]

\makeatletter
\tikzoption{initial angle}{\tikzaddafternodepathoption{\def\tikz@initial@angle{#1}}}
\tikzoption{accepting angle}{\tikzaddafternodepathoption{\def\tikz@accepting@angle{#1}}}
\makeatother

\newcommand{\acceptingstate}[4]{\draw[->] (#1.#2) -- node[#3] {#4} +(#2:7mm);}
\newcommand{\acceptingright}[2]{\acceptingstate{#1}{0}{above}{#2};}
\newcommand{\acceptingleft}[2]{\acceptingstate{#1}{180}{above}{#2};}
\newcommand{\acceptingabove}[2]{\acceptingstate{#1}{90}{right}{#2};}
\newcommand{\acceptingbelow}[2]{\acceptingstate{#1}{-90}{right}{#2};}

%% ---------------------- %%
%% Mathematical symbols.  %%
%% ---------------------- %%
\renewcommand{\max}{\textrm{max}}
\renewcommand{\min}{\textrm{min}}
\newcommand{\Z}{\texorpdfstring{\ensuremath{\mathbb{Z}}}{Z}}
\newcommand{\R}{\texorpdfstring{\ensuremath{\mathbb{R}}}{R}}
\newcommand{\B}{\texorpdfstring{\ensuremath{\mathbb{B}}}{B}}
\newcommand{\Ae}{\ensuremath{A^{*}}}
\newcommand{\SerSAnMon}[2]%
    {\ensuremath{#1 \langle \! \langle  #2  \rangle \! \rangle}}

\newcommand{\newcal}[1]{%
  \expandafter \newcommand \csname #1c\endcsname%
  {\texorpdfstring{\ensuremath{\mathcal{#1}}}{#1}}%
}
\newcal{A}
\newcal{B}
\newcal{C}
\newcal{D}
\newcal{T}

\newtheorem{theorem}             {Theorem}[chapter]
\newtheorem{definition} [theorem]{Definition}
\newtheorem{remark}     [theorem]{remark}

\makeatletter
\newcommand{\Cxx}{%
  \valign{\vfil\hbox{##}\vfil\cr
    {C\kern-.1em}\cr
    $\hbox{\fontsize\sf@size\z@\textbf{+\kern-0.05em+}}$\cr}%
    \xspace
}
\makeatother

\newcommand{\bslash}{\texttt{\symbol{92}}}

% A transition for a transducer.
\newcommand{\IOL}[2]{\ensuremath{(#1,#2)}}

%% ------- %%
%% Index.  %%
%% ------- %%

% The same argument is output and put in the index.
\usepackage{makeidx}
\makeindex
\newcommand{\Index}[1]{\index{#1}#1}

\usepackage{listings}
\lstset{%
  numbers=left,
  numberstyle=\tiny,
  stepnumber=5,
  numbersep=5pt,
  firstnumber=1,
  basicstyle=\small,
  frame=single,
  language=C++,
  float}

%% Display an interactive session.
\usepackage{alltt}
\newenvironment{shell}
{\begin{alltt}}
{\end{alltt}}

% @VcsnVersion@
\def\VcsnVersion{1.3.9a}

%% ----------------------- %%
%% Texinfo like commands.  %%
%% ----------------------- %%

\newcommand\kbd[1]{\textsl{\texttt{#1}}}
\newcommand\file[1]{`\texttt{#1}'}
\newcommand\command[1]{\texttt{#1}}
\newcommand\var[1]{{\ttfamily\itshape #1}}
\newcommand\code[1]{\texttt{#1}}
\newcommand\samp[1]{`\texttt{#1}'}
\newcommand\option[1]{`\texttt{#1}'}

%% -------------------- %%
%% Vaucanson commands.  %%
%% -------------------- %%

%% A TAF-kit function name (i.e., the first argument for TAF-Kit
%% programs).
\newcommand{\taffn}[1]{\code{#1}}
\newcommand{\tafkit}{\textsc{TAF-Kit}\xspace}
\newcommand{\Vauc}{\textsc{Vaucanson}\xspace}

% \begin{fnsection}{NAME}
% -----------------------
% A group of taf-kit functions.
% Putting the ending hline in the end section is tempting, but fails.
\newenvironment{fnsection}[1]{%
  % \item{NAME}{ARGS}{DOC}
  % ----------------------
  % Define a taf-kit function.
  % We use \item because it has Emacs indentation support.
  \renewcommand{\item}[3]{\texttt{##1} \var{##2} & ##3\\}

  \tabularx{\textwidth}{|l|X|}%
    \hline\multicolumn{2}{|c|}{#1} \\\hline%
  }{%
  \endtabularx%
}

\def\manualtitle{\tafkit's Documentation}

%% ---------- %%
%% Document.  %%
%% ---------- %%
\begin{document}
\vspace*{50pt}
\vskip4pt \hrule height 4pt width \hsize \vskip4pt
\begin{center}
  \Huge \manualtitle
\end{center}
\vspace*{-1.5ex}
\vskip4pt \hrule height 4pt width \hsize \vskip4pt

\hfill compiled on \today, for version \VcsnVersion

\vfill

\setcounter{tocdepth}{2}
\makeatletter
\@starttoc{toc}
\makeatother

\vfill

\chapter*{Introduction}
\label{sec:intro}

\Vauc is a free software platform dedicated to the manipulation of
finite state automata.  Here, `finite state automata' is to be
understood in the broadest sense: \Vauc supports \emph{weighted}
automata over a free monoid, and even \emph{weighted} automata on some
\emph{non free monoids} (currently only products of free
monoids---a.k.a. transducers---are supported).

\bigskip

The platform consists in a couple of components:
\begin{description}
\item[The \Vauc library] is a \Cxx library that implements objects for
  automata, rational expressions, as well as algorithms on these
  objects.  This library is generic, in the sense that it makes it
  possible to write an algorithm once and apply it to different types
  of automata.  However this genericity is achieved in a way that
  should not cause any slowdown at runtime: because the type of the
  automaton manipulated is known at compile time, compiling an
  algorithm will generate code that is almost as efficient as an
  algorithm written specifically for this type of automaton.

\item[\tafkit] is a command-line interface to the library that allows
  user to execute \Vauc's algorithms without any knowledge of \Cxx.
  Because the \Vauc library needs to know the type of automata at
  compile time, the \tafkit interface has been instantiated for a
  predefined set of common automaton types.

  \tafkit does not allow to write new algorithms nor to manipulate new
  types of automata, but it makes it possible to combine without
  efforts a large set of algorithms on common automata types.

\item[A repository of automata] that shows examples of automata of
  various types, and also contains tools to create families of
  automata.
\end{description}

Ideally this manual should document all of these components.
Presently it mostly documents \tafkit, because that is the more
accessible part of \Vauc.


\chapter{Administrativia}

\section{Getting \Vauc}

The latest version of the \Vauc platform can be downloaded from
\url{http://vaucanson.lrde.epita.fr/}.

\section{Licensing}

Vaucanson is a free software released under the GNU General Public
Licence version 2. If you are unfamiliar with this license, please
read the file \file{COPYING} (at the root of the source tree) for
details.

\section{Building \Vauc}\label{sec:building}

Detailed information is provided in the files \file{INSTALL}, which is
generic to all packages using the GNU Build System, and
\file{doc/README.pdf} which details \Vauc's specific build process.
The following installation commands will install \Vauc in
\file{/usr/local/}.

\begin{shell}
$ \kbd{cd vaucanson-\VcsnVersion}
$ \kbd{./configure}
$ \kbd{make}
$ \kbd{sudo make install}
\end{shell}

Although we discourage it, you may also use \Vauc without installing
it: you would have to use \kbd{-I
  /full-path-to/vaucanson-\VcsnVersion/include} when compiling \Cxx
programs, and move to directory
\file{vaucanson-\VcsnVersion/taf-kit/tests/} to execute \tafkit.
(This directory contains wrapper around the real \tafkit programs from
\file{taf-kit/src/} that enable them to run locally.)

\section{Getting in Touch}

Please send any question or comment to \url{vaucanson@lrde.epita.fr},
and report bugs to either our issue tracker at
\url{http://vaucanson.lrde.org/} or by mail to
\url{vaucanson-bugs@lrde.epita.fr}.

You can subscribe to these mailing lists at
\url{https://www.lrde.epita.fr/cgi-bin/mailman/listinfo/vaucanson} and
\url{https://www.lrde.epita.fr/cgi-bin/mailman/listinfo/vaucanson-bugs}
if you like, but this is not a requirement for sending e-mails.

\chapter{\tafkit}
\label{sec:tafkit}

\tafkit is a command-line interface to \Vauc.  It is a set of programs
that should be called from the shell and that can be used to chain
operations on automata.

\section{\tafkit Instances}

In the generic programming paradigm used in the \Vauc library, the
types of the automata manipulated have to be known at compile time.
\tafkit has therefore been compiled for several predefined types of
automata.  It is actually the same program that is \emph{instantiated}
for different kinds of alphabets and weights.  Here are the names of
these instances, and the kind of automata they represent:

\begin{center}
\begin{tabular}{llll}
program name & automaton type & alphabet type & weight semiring \\
\hline
\command{vcsn-char-b} & automata &
  characters & $\langle\B,\lor,\land\rangle$ \\
\command{vcsn-int-b} & automata &
  integers & $\langle\B,\lor,\land\rangle$ \\
\command{vcsn-char-z} & automata &
  characters & $\langle\Z,+,\times\rangle$\\
\command{vcsn-char-zmax} & automata &
  characters & $\langle\Z,\max,+\rangle$\\
\command{vcsn-char-zmin} & automata &
  characters & $\langle\Z,\min,+\rangle$\\
\command{vcsn-char-r} & automata &
  characters & $\langle\R,+,\times\rangle$\\
\command{vcsn-char-char-b} & automata &
  pairs of characters & $\langle\B,\lor,\land\rangle$\\
\command{vcsn-char-int-b} & automata &
  pairs of character and integer & $\langle\B,\lor,\land\rangle$\\
\command{vcsn-int-int-b} & automata &
  pairs of integers & $\langle\B,\lor,\land\rangle$\\
\command{vcsn-char-fmp-b} & transducers &
  characters & $\langle\B,\lor,\land\rangle$\\
\command{vcsn-char-fmp-z} & transducers &
  characters & $\langle\Z,+,\times\rangle$\\
\command{vcsn-int-fmp-b} & transducers &
  integers & $\langle\B,\lor,\land\rangle$\\
\command{vcsn-int-fmp-z} & transducers &
  integers & $\langle\Z,+,\times\rangle$\\
\hline
\end{tabular}
\end{center}

\medskip

Many users of automata consider only automata whose transitions are
labeled by letters taken in an alphabet, which we call, roughly
speaking, \emph{classical} automata or \emph{Boolean}
automata. \command{vcsn-char-b} is the \tafkit instance they should
use.  A variant of this program, called \command{vcsn-int-b}, handles
Boolean automata whose letters are integers.  Other variants such as
\command{vcsn-char-char-b}, \command{vcsn-char-int-b}, or
\command{vcsn-int-int-b}, support alphabets of pairs.  All of these
are called Boolean automata because each word is associated to a
Boolean \emph{weight}: either the word is accepted and its weight is
\emph{true}, or it is not and its weight is \emph{false}.

\Vauc actually supports automata with multiplicities, where words can
be associated to weights taken in any semiring.  For instance
\command{vcsn-char-z} associates each word to an integer.  The previous
table show other semirings that can be used as well.

\Vauc also supports weighted transducers.  These transducers are
actually automata over a product two free monoids.  In \Vauc we call
these FMP, for \emph{free monoid products}.  The above table lists a
few \tafkit instance for FMP.

\section{A First Contact}\label{sec:firstcontact}

\begin{figure}[tbp] \centering
  \begin{tikzpicture}[automaton]
    % states
    \node[state, initial] (A) at (0,0) {};
    \node[state] (B) at (2,0) {};
    \node[state, accepting] (C) at (4,0) {};
    % initial--final
    % transitions
    \path[->] (A) edge node[above] {$a$} (B);
    \path[->] (B) edge node[above] {$b$} (C);
    \path[->] (A) edge [loop below] node[below] {$b$} (A);
    \path[->] (A) edge [loop above] node[above] {$a$} (A);
    \path[->] (C) edge [loop below] node[below] {$b$} (C);
    \path[->] (C) edge [loop above] node[above] {$a$} (C);
  \end{tikzpicture}
  \caption{The automaton $\Ac_1$, defined over the alphabet
    $A=\{a,b\}$ recognizes any word of $A^\star$ that
    contains $ab$. \label{fig:a1}}
\end{figure}

We are about to play with automaton $\Ac_1$ pictured
on~\autoref{fig:a1}.  \tafkit comes with a set of predefined automata,
and $\Ac_1$ happens to be one of those: it is called \file{a1.xml}.
This is a Boolean automaton whose alphabet consists in two characters
$\{a,b\}$ so will shall use the \command{vcsn-char-b} instance of
\tafkit.

If you have fully installed \Vauc (see \autoref{sec:building}) you
should be able to just type any of the following commands and observe
their results.  If you only compiled \Vauc without installing it, you
should \command{cd} into the
\file{vaucanson-\VcsnVersion/taf-kit/tests/} directory and type
\samp{./vcsn-char-b} instead of \samp{vcsn-char-b} for each of the
following commands.

The following command will just make sure that \tafkit knows about
this automaton.  It will display the number of states, transitions,
initial states, and final states of $\Ac_1$.

\begin{shell}
$ \kbd{vcsn-char-b info a1.xml}
States: 3
Transitions: 6
Initial states: 1
Final states: 1
\end{shell}%$

If you have the GraphViz package installed (see \file{doc/README.pdf}
for links) you can also display that automaton with:

\begin{shell}
$ \kbd{vcsn-char-b display a1.xml}
\end{shell}%$

\noindent
The displayed automaton won't have a layout as pretty as
\autoref{fig:a1}, but it represents the same automaton nonetheless.

$\Ac_1$ is a non-deterministic automaton.  We could determinize it
with the \command{determinize} command of \tafkit.  As most commands
of \tafkit, \command{determinize} produces its output (an XML file
representing the automaton) on the standard output, so we will want to
divert it to a file using a shell redirection.

\begin{shell}
$ \kbd{vcsn-char-b determinize a1.xml > a1det.xml}
$ \kbd{vcsn-char-b info a1det.xml}
States: 4
Transitions: 8
Initial states: 1
Final states: 2
\end{shell}%$

\noindent
The determinized automaton has 4 states and 8 transitions.

Please note that \file{a1det.xml} is a file that we just created into
the current directory while \file{a1.xml} is a file that is predefined
in \Vauc's predefined automata repository.  We can call command
\command{info} on either files using the same syntax because \tafkit
will look for automata in both places.  The command \samp{vcsn-char-b
  list-automata} will list all predefined automata for this instance
of \tafkit.  See also \autoref{chap:autrepository} for a presentation
of these files.

In the pure Unix tradition, we can of course chain commands with
pipes.  For instance the above two commands could be rewritten

\begin{shell}
$ \kbd{vcsn-char-b determinize a1.xml | vcsn-char-b info -}
States: 4
Transitions: 8
Initial states: 1
Final states: 2
\end{shell}%$
\noindent
where \samp{-} stands for `\emph{read from standard input}'.


\tafkit actually supports a more efficient way of chaining commands:
the \emph{internal pipe}.  It's called \emph{internal pipe} because
the pipe logic is taken care of by \tafkit himself, but actually it is
not using a Unix pipe at all: the commands are simply serialized in
the same process, using the automata object created by the previous
one.  It is more efficient because the automaton does not have to be
converted into XML for output, and then parsed back as input of the
next command in the chain.  Here is how the above command would look
using an \emph{internal pipe}; notice how the \samp{|} symbol is
protected from its evaluation by the shell.

\begin{shell}
$ \kbd{vcsn-char-b determinize a1.xml \bslash| info -}
States: 4
Transitions: 8
Initial states: 1
Final states: 2
\end{shell}%$
\noindent

\noindent In the above command, \samp{-} does not designate the
standard input, it denotes \emph{the result of the previous command}.

\section{\tafkit's Modus Operandi}\label{sec:modusoperandi}

All \tafkit instances work identically.  They just differ on the type
of automata they handle, and may offer different algorithms because
not all algorithms work on any automata type.

Any time \tafkit is run, it breaks its command line into command names
and arguments.

\[
  \underbrace{\text{\texttt{vcsn-char-b}}}_{\text{\tafkit instance}}~
  \underbrace{\underbrace{\text{\texttt{determinize}}}_{\text{name}}~
              \underbrace{\text{\texttt{a1.xml}}}_{\text{arg.}}}_{\text{command 1}}
  \text{\texttt{\bslash|}}
  \underbrace{\underbrace{\text{\texttt{info}}}_{\text{name}}
              \underbrace{\text{\texttt{-}}}_{\text{arg.}}}_{\text{command 2}}
\]

The \emph{internal pipe}, \samp{\bslash|}, is used to separate
commands.  A command start with a name, is can be followed by several
arguments (although only one is used in the above example).
These arguments can be very different depending on the command.  The
far we have used filenames as well as \samp{-} (to designate either
the standard input or the result of the previous command).  Some
commands will also accept plain text representing for instance a word
or a rational expression.

All commands will also accept some options.  There are options to
define what the alphabet is, options to define the types to use for
input and output, even options to fine-tune how some symbols will be
printed.  We shall get back to these options in \autoref{sec:iooptions}.

\medskip

For each command, \tafkit will
\begin{enumerate}
\item parse the options
\item parse all expected arguments
      (using indications that may have been given as options)
\item execute the algorithm
\item print the result (in a format that can be controlled using
  options)
\end{enumerate}

When commands are chained internally using \samp{\bslash|} and
\samp{-}, the parsing steps and printing steps are of course omitted.


\section{Writing (Weighted) Rational Expressions}

\subsection{Rational operators}

Any word can be used as a rational expression.  Additionally the
following operators can be used to combine rational expressions.

\begin{center}
  \begin{tabular}{ll}
    \hline
    $e$\code{*} & Kleene star \\
    $e_1$$e_2$ & implicit concatenation \\
    $e_1$\code{.}$e_2$ & explicit concatenation \\
    $e_1$\code{+}$e_2$ & disjunction \\
    \code{(}$e$\code{)} & grouping \\
    \hline
  \end{tabular}
\end{center}

For instance, on the alphabet $A=\{a,b\}$, the language denoted
by the rational expression \samp{(a+b)*ab(a+b)*} contains all words
that contain \samp{ab}.

The \Vauc library always needs to know on which alphabet a rational
expression is defined in order to parse it.  This alphabet can be
indicated using option \option{--alphabet=ab} or the shorter form
\option{-aab} (see \autoref{sec:--alphabet} for more details).

For instance, here is how to create an automaton that recognizes the
same language as \samp{(a+b)*ab(a+b)*}, and make sure this automaton
is equivalent to the automaton $\Ac_1$ of \autoref{fig:a1}.

\begin{shell}
$ \kbd{vcsn-char-b exp-to-aut -aab "(a+b)*ab(a+b)*" > aut.xml}
$ \kbd{vcsn-char-b are-equivalent -v aut.xml a1.xml}
Automata are equivalent
\end{shell}%$

\noindent The \option{-v} is used to request a plain English output
from \command{are-equivalent}.  Without it, \tafkit would just set its
status code.  (See \autoref{sec:verbose} for more details.)

\textbf{Caveat:} because \Vauc builds rational expressions on top of
words, the Kleene star operator and the weights (introduced in
\autoref{sec:weight}) apply to words and not letters as it is usually
the case in other application.  For instance \samp{ab*} is the same
rational expression as \samp{(ab)*} for \Vauc, but it is different
from \samp{a.b*} or \samp{a.(b*)}.


\subsection{Empty word and null series}\label{sec:emptyword}

The default representation of the empty word (identity of the monoid)
is \samp{1} when using characters or pair alphabets.  For instance let
us try the command \command{expand}, that distributes concatenations
over disjunctions:

\begin{shell}
$ \kbd{vcsn-char-b expand -aab '(a+1)(1+b)'}
a+ab+b+1
\end{shell}%$

Of course if we use \samp{1} as one character in the alphabet, the
same symbol cannot be used for representing the empty word.  \Vauc
actually choose the first available representation of the empty word
from the following list of candidate symbols: \samp{1}, \samp{e},
\samp{\_e}, or \samp{eps}.

\begin{shell}
$ \kbd{vcsn-char-b expand -a01 '(0+e)(e+1)'}
0+01+1+e
$ \kbd{vcsn-char-b expand -a1e '(1+\_e)(\_e+e)'}
1+1e+e+_e
\end{shell}%$

For integer alphabets, the empty word is of course always \samp{e}.

Similarly the symbol used to represent the null series defaults to the
first representation from the following list that is compatible with
the alphabet: \samp{0}, \samp{z}, \samp{\_z}, or \samp{zero}.

Section~\ref{sec:writingdata} shows how you can actually specify you
own representation for these symbols


\subsection{Weights}\label{sec:weight}

Weights are written in braces as in \samp{\{3\}}.

For instance the automaton $\Cc_1$ from \autoref{sec:z:c1} corresponds
to the (weighted) rational expression
\samp{(a+b)*.b.(\{2\}a+\{2\}b)*}.

\begin{shell}
$ \kbd{vcsn-char-z aut-to-exp c1.xml}
(a+b)*.b.(\{2\} a+\{2\} b)*
\end{shell}%$

In \Vauc, and even if this does not appear yet in the \tafkit
instances, the weight semirings are not necessarily commutative: the
simple case where this will occur is when an "FMP transducer" will be
transformed into an automaton with weights in the semiring of rational
languages over the output alphabet.

For this reason, the multiplication by a weight on the left and on the
right are two distinct operations in the building of weighted rational
expressions. For instance, \samp{\{3\}(\{2\}a+b)} and
\samp{(\{2\}a+b)\{3\}} are two distinct expressions even if they
denote the same polynomial: \samp{\{6\}a + \{3\}b}, in the same way as
\samp{\{y\}(\{x\}a+b)} and \samp{(\{x\}a+b)\{y\}} are distinct
expressions, which denote distinct polynomials: \samp{\{yx\}a +
  \{y\}b} and \samp{\{xy\}a + \{y\}b} respectively.


\subsection{Trivial Identities}

Anytime a weighted rational expression is constructed inside \Vauc,
the following rewritings, called \emph{trivial identities}, are
automatically applied.

Here $E$ stands for any weighted rational expression, $w$ is any word,
and $k$ and $h$ are weights.  In these notations it should also be
obvious that $0$ and $1$ designate the identity of the monoid (empty
word) and null series, while $\{0\}$ and $\{1\}$ designate the zero
and identity of the semiring (weights).  Any subexpression of a form
listed to the left of a `$\Rightarrow$' is rewritten as indicated on
the right.

\begin{align*}
  E+0 = 0+E &\Rightarrow E
& E.1 = 1.E  &\Rightarrow  E
& \{k\}(\{h\}E) &\Rightarrow \{kh\}E
\\
  E.0 = 0.E &\Rightarrow 0
& \{1\}E = E\{1\} &\Rightarrow  E
& (E\{k\})\{h\} &\Rightarrow E\{kh\}
\\
  0^\star  &\Rightarrow 1
& 1\{k\} &\Rightarrow  \{k\}1
& (\{k\}E)\{h\} &\Rightarrow \{k\}(E\{h\})
\\
  \{k\}0 = 0\{k\} &\Rightarrow 0
& (\{k\}1).E  &\Rightarrow  \{k\}E
& w\{k\}        &\Rightarrow \{k\}w
\\
  \{0\}E = E\{0\} &\Rightarrow 0
& E.(\{k\}1)  &\Rightarrow  E\{k\}
\\
\end{align*}

These rewriting mean that it is \emph{impossible} for \Vauc to emit a
rational expression such as \samp{(\{3\}(0(ab)))*\{4\}}.  This
expression is \emph{by construction} equal to \samp{\{4\}1}.  We can
verify this using \command{identity-exp}:

\begin{shell}
$ \kbd{vcsn-char-z identity-exp --alphabet=ab "(\{3\}(0(ab)))*\{4\}"}
\{4\} 1
\end{shell}

\noindent the command \command{identity-exp} does not apply any
algorithm on the rational expression.  Its only purpose is to read and
write the rational expression using any I/O option supplied on the
command-line.  The trivial identities are rewritten while reading the
expression.

\section{Interactive Definition of Automata}

The \tafkit command \taffn{edit} provides a textual interface to
define automata interactively.  The commands takes the filename of the
automata to define or modify in argument.  If the files does not yet
exist, you should specify the alphabet of your automaton on the
command line (using \option{--alphabet=} or \option{-a} as will any
other command), and the file will be created when you exit the editor.
If the file does exist, the alphabet will be read from the file along
with the automaton itself, and the file will be overwritten upon exit.

The interface is based on a menu of choices:
\begin{shell}
$ \kbd{vcsn-char-b edit --alphabet=ab test.xml}
Automaton description:
  States: (none)
  Initial states: (none)
  Final states: (none)

  Transitions: (none)

Please choose your action:
  1. Add states.
  2. Delete a state.

  3. Add a transition.
  4. Delete a transition.

  5. Set a state to be initial.
  6. Set a state not to be initial.

  7. Set a state to be final.
  8. Set a state not to be final.

  9. Display the automaton in Dotty.

  10. Save and exit.
  11. Exit without saving.

Your choice [1-11]:
\end{shell}

\noindent
If you enter \kbd{1}, you will then be prompted for the number of
states to add, say \kbd{1} again.  The state \code{0} was created.  To
make it initial select \kbd{5}, and:

\begin{shell}
Your choice [1-10]: \kbd{5}
  For state: \kbd{0}
\end{shell}

Likewise to make it final, using choice \kbd{7}.  Finally, let's add a
transition:

\begin{shell}
Your choice [1-10]: \kbd{3}
  Add a transition from state: \kbd{0}
  To state: \kbd{0}
  Labeled by the expression: \kbd{a+b}
\end{shell}

\noindent
The automaton is generalized, that is to say, rational expressions are
valid labels.

On top of the interactive menu, the current definition of the
automaton is reported in a textual yet readable form:

\begin{shell}
Automaton description:
  States: 0
  Initial states: 0
  Final states: 0

  Transitions:
    1: From 0 to 0 labeled by (\{1\} a)+(\{1\} b)
\end{shell}

\noindent
States are numbered from 0, but transitions numbers start at 1.  Also,
note that weights are reported, although only \code{1} is valid for
Boolean automata.

Finally, hit \kbd{10} to save the resulting automaton in the file
\file{test.xml}.



\section{Command I/O options}\label{sec:iooption}

As we said in~\autoref{sec:modusoperandi}, each \tafkit command has to
read its input and write its output.  In this section we only cover
the options that may modify the Input/Output behaviors of commands.

\begin{center}
\begin{tabular}{llp{.5\textwidth}l}
  long option & short & purpose & documentation \\
  \hline
  \samp{--alphabet} & \samp{-a} & specify the alphabet of automata or rational expressions  & \S\ref{sec:--alphabet}\\
  \samp{--alphabet1} & \samp{-a} & specify the first alphabet on transducers & \S\ref{sec:--alphabet}\\
  \samp{--alphabet2} & \samp{-A} & specify the second alphabet on transducers & \S\ref{sec:--alphabet}\\
  \samp{--input} & \samp{-i} & select input format for automata and rational expressions & \S\ref{sec:formats}\\
  \samp{--output} & \samp{-o} & select output format for automata and rational expressions & \S\ref{sec:formats}\\
  \samp{--parser} & \samp{-p} & fine-tune the symbols used for input and output of rational expressions and automata & \S\ref{sec:writingdata}\\
\hline
\end{tabular}
 \end{center}

\noindent
The full list of options can be obtained with \kbd{vcsn-char-b --help}.

\subsection{Specifying alphabets}\label{sec:--alphabet}

When \tafkit reads an XML file, there is no need to specify any other
information besides the name of the file.  For instance when we read
\file{a1.xml} in~\autoref{sec:firstcontact} and determinized this
automaton, we did not have to tell \tafkit that the alphabet was
$A=\{a,b\}$.  The XML file is self-contained and already contains
this information.

Here is a situation where specifying an alphabet is mandatory:
\begin{shell}
$ \kbd{vcsn-char-b exp-to-aut aba+a}
Error: alphabet should be explicitly defined using --alphabet
\end{shell}%$

\command{exp-to-aut} is a command that takes a rational expression and
converts it into an automaton.  To be able to parse the rational
expression, \Vauc needs to know what alphabet it its using.  Here there
is no ways it can guess whether the alphabet is $A=\{a,b\}$ and
the \samp{+} is a rational operator or if it is $A=\{a,b,+\}$ and
the \samp{+} is just a letter.  Specifying the alphabet can be done
using \option{--alphabet=ab} for instance.

\begin{shell}
$ \kbd{vcsn-char-b exp-to-aut --alphabet=ab aba+a > aut.xml}
\end{shell}%$

In practice, the long \option{--alphabet=} option can be tedious
to type and we will often prefer its short equivalent \option{-a}:

\begin{shell}
$ \kbd{vcsn-char-b exp-to-aut -aab aba+a > aut.xml}
\end{shell}%$

\paragraph{Character alphabets}

For characters alphabets (as with the \samp{char} \tafkit instances
used in the above examples), the letters of the alphabets can be
arbitrary ASCII characters, and need just to be listed after
the \samp{--alphabet=} or \samp{-a} option.

When specifying characters alphabets, the characters \samp{ } (space),
\samp{"}, \samp{(}, \samp{)}, \samp{'}, \samp{=}, and \samp{\bslash},
have to be escaped with a backslash.  For instance the following
command will create an automaton that recognize numbers of the form
\samp{12,456,789}, where a comma must be used as thousand separator:
note how the comma must be escaped in the alphabet

\begin{shell}
$ \kbd{d="(0+1+2+3+4+5+6+7+8+9)"}
$ \kbd{vcsn-char-b exp-to-aut -a'0123456789\bslash,' "($d+$d$d+$d$d$d)(,$d$d$d)*" > numbers.xml}
\end{shell}%$

Some character alphabets are predefined.  These are:

\begin{center}
\begin{tabular}{ll}
\samp{letters} & The lower case letters $\{a,b,\ldots,z\}$.\\
\samp{alpha} & The upper and lower case letters $\{a,b,\ldots,z,A,B,\ldots,Z\}$.\\
\samp{digit} & All digits $\{0,1,\ldots,9\}$.\\
\end{tabular}
\end{center}

\noindent
This means that \samp{-aletters} is an abbreviation for
\samp{-aabcdefghijklmnopqrstuvwxyz}.  You can always get the above
list of predefined alphabet by typing \samp{vcsn-char-b --help}.

\paragraph{Integer alphabets}

Using integers alphabets, letters must be specified as signed integer
(they are represented by the \texttt{int} C++ type), and should be
separated by commas.  For instance the following commands will
construct an automaton that reads any sequence of coins of $1$, $2$,
$5$, $10$, $20$, or $50$ cents, as long as the values are increasing.

\begin{shell}
$ \kbd{vcsn-int-b exp-to-aut -a1,2,5,10,20,50 '1*2*5*10*20*50*' > coins.xml}
\end{shell}%$

\paragraph{Pair alphabets}

Pair alphabets should be specified using parentheses and commans to
form pairs.  For instance:

\begin{shell}
$ \kbd{vcsn-char-int-b exp-to-aut -a'(a,1)(b,2)(a,-1)' '(a,-1)(a,1)+(b,2)' > misc.xml}
\end{shell}%$

\paragraph{Transducer alphabets}

Free monoid products have two alphabets, one for each monoid.  The
instances of \tafkit that handle transducers consequently support two
options \samp{--alphabet1=} and \samp{--alphabet2=}, that can be
abbreviated respectively \samp{-a} and \samp{-A}.


\subsection{Input and Output Formats}\label{sec:formats}

\tafkit can input and output several kind of objects: automata,
rational expressions, words, weights and Boolean results.

\begin{description}
\item[words] are always read as strings given on the command line, and
  written to standard output.
\item[automata] are read from a file whose filename specified on the
  command line, and output on standard output.  \Vauc can read
  automata in two formats: FSMXML (the default), or the textual format
  of FSM.  It can also write automata in these formats, as well as in the
  \samp{dot} format that can be used for graphical output.
\item[rational expression] are by default read as strings given on the
  command line, and output as strings on standard output.
  Alternatively rational expression can be read from an FSMXML file whose
  filename is given on the command line, and output in FSMXML as well.
\item[weight results] (such as the result of the evaluation of a word on
  an automaton) are simply output as strings on the standard output.
\item[Boolean results] (such as the result of asking whether an automaton
  is empty) are returned using the status code of the
  \tafkit instance, so that these commands can be used as conditions
  in shell scripts.
\end{description}

\paragraph{Changing the format for automata and rational expressions}

The format used to input automata and rational expressions can be
controlled using the \samp{--input=} and \samp{--output=} options (or
\samp{-i}, \samp{-o} for short).  These options control the I/O
formats for both automata and rational expressions at once.  So for
instance using \samp{-ixml} will ask \tafkit to read any automaton or
rational expression using the FSMXML format.  Because rational
expression are not supported in as much formats as automata, they will
be read or written as text string if an unsupported format is
requested.

\begin{center}
\begin{tabular}{lll}
  values for \samp{-i} or \samp{-o}    & format for automata & format for rational expressions \\
  \hline
  (none)     & FSMXML              & text string \\
  \samp{xml} & FSMXML              & FSMXML \\
  \samp{fsm} & FSM                 & text string \\
  \samp{dot} (for output only) & dot   & text string \\
  \hline
\end{tabular}
\end{center}

\medskip

For instance to convert an automaton from XML to dot\footnote{dot
  files can be processed using the GraphViz package}, we would use:

\begin{shell}
$ \kbd{vcsn-char-b identity -odot aut.xml > aut.dot}
\end{shell}%$


\paragraph{Verbose Boolean results}\label{sec:verbose}

As said above, boolean results are returned using the program's status
code using the Unix convention (that is $0$ for \emph{true} and any
other value for \emph{false}). The shell makes this value available in
the \samp{\$?} variable.  The \tafkit option \samp{--verbose} or \samp{-v}
can be used to request an English interpretation of this value.

\begin{shell}
$ \kbd{vcsn-int-b is-empty coins.xml}
$ \kbd{echo $?}
1
$ \kbd{vcsn-int-b is-empty -v coins.xml}
Input is not empty
\end{shell}%$

\subsection{Specifying writing data}\label{sec:writingdata}

Section~\ref{sec:emptyword} showed how the empty word and null series
can have different representations depending on the alphabet.  \Vauc
actually allows other symbols used in rational expression to be
changed to arbitrary strings.

Here is the list of named symbols with their meaning and default
values:

\begin{center}
\begin{tabular}{lll}
symbol & meaning & default value(s) \\
\hline
\samp{OPAR} & group start & \samp{(} \\
\samp{CPAR} & group end & \samp{)} \\
\samp{PLUS} & disjunction (additive law of the series) & \samp{+} \\
\samp{TIMES} & concatenation (multiplicative law of the series) & \samp{.}\\
\samp{STAR} & Kleene star & \samp{*} \\
\samp{ONE} & empty word (identity of the monoid) & \samp{1},\samp{e},\samp{\_e},\samp{eps}\\
\samp{ZERO} & null series & \samp{0},\samp{z},\samp{\_z},\samp{zero} \\
\samp{OWEIGHT} & weight start & \samp{\{} \\
\samp{CWEIGHT} & weight end & \samp{\}} \\
\samp{SPACE} & space characters (to ignore) & \samp{ } \\
\hline
\end{tabular}
\end{center}

The \samp{--parser=} can be used to change the value of the above
tokens. Each of them must be defined as a non-empty string.  \tafkit
will check that these tokens do not collide.
For instance to use $\{(,)\}$ as alphabet, we should obviously rename
the \samp{OPAR} and \samp{CPAR} tokens.

The following command creates an automaton that recognizes the words
\samp{()}, \samp{(())}, \samp{(()())}, \samp{(()()())}, etc.


\begin{shell}
$ \kbd{vcsn-char-b -a'\(\)' --parser='OPAR=[ CPAR=]' exp-to-aut '([()]*)' >parens.xml}
\end{shell}%$

The values of these symbols, which we call the \emph{writing data},
are stored in the XML file, so there is no need to specify them again
when working from a file.

\begin{shell}
$ \kbd{vcsn-char-b aut-to-exp parens.xml}
(.(.[).(]*.).)+(.)
$ \kbd{vcsn-char-b eval parens.xml '(()())'}
1
\end{shell}%$


\paragraph{Overwriting the writing data}

When \tafkit reads an automaton or a rational expression from an XML
file (that contains writing data) or from the internal pipe, it does
not need additional information to read its input.  However the
\samp{--parser=} option can still be used to modify the way the object
will be \emph{output}.

Here is an example where we create a rational expression over the
alphabet $\{\code{(},\code{)}\}$ using \samp{[} and \samp{]} for
grouping, and store it into the file \file{p.xml}.  We can then
convert this file back into a string using either the original writing
data that were stored in the file, or overwriting these data with
different ones (here using \samp{<} and \samp{>} for grouping).

\begin{shell}
$ \kbd{vcsn-char-b -a'\(\)' --parser='OPAR=[ CPAR=]' identity-exp -oxml '([()]*)' >p.xml}
$ \kbd{vcsn-char-b identity-exp -ixml p.xml}
(.[(.)]*.)
$ \kbd{vcsn-char-b --parser='OPAR=< CPAR=>' identity-exp -ixml p.xml}
(.<(.)>*.)
\end{shell}%$


\section{An example of \Z-automaton}

This part shows some uses of the program \command{vcsn-char-z}.

\subsection{Counting \samp{b}s}

\begin{figure}[tp] \centering
  \begin{tikzpicture}[automaton]
    \node[state, initial] (A) at (0,0) {};
    \node[state, accepting] (B) at (2,0) {};
    \path[->] (A) edge node[above] {$b$} (B);
    \path[->] (A) edge[loop below] node[below] {$b$} (A);
    \path[->] (A) edge[loop above] node[above] {$a$} (A);
    \path[->] (B) edge[loop below] node[below] {$b$} (B);
    \path[->] (B) edge[loop above] node[above] {$a$} (B);
  \end{tikzpicture}
  \caption{$\Bc_1$: When defined over the Boolean semiring $\langle
    \B,\lor,\land \rangle$, $\Bc_1$ accepts words with at least one
    $b$.  When defined over the integer semiring $\langle \Z,+,\times
    \rangle$, $\Bc_1$ counts the number of $b$ in a word.\label{fig:b1}}
\end{figure}

Let's consider $\Bc_1$ for \autoref{fig:b1}: a \Z-automaton,
\textit{i.e.}  an automaton whose label's weights are in \Z.  This
time the evaluation of the word \var{w} by the automaton $\Bc_1$ will
produce a number, rather than simply accept or reject \var{w}.  For
instance let's evaluate \samp{abbb} and \samp{abab}:

\begin{shell}
$ \kbd{vcsn-char-z eval b1.xml abbb}
3
$ \kbd{vcsn-char-z eval b1.xml abab}
2
\end{shell}

\noindent
Indeed, $\Bc_1$ counts the number of \samp{b}s.

\subsubsection{Power}

Now let's consider the $\Bc_1^n$, where
\begin{displaymath}
  \Bc_1^n = \prod_{i=1}^n \Bc_1, n > 0
\end{displaymath}

\noindent
This is implemented by the \Index{\taffn{power}} function:

\begin{shell}
$ \kbd{vcsn-char-z power b1.xml 4 > b4.xml}
\end{shell}%$

\noindent
The file \file{b4.xml} now contains the automaton $\Bc_1^4$.  Let's
check that the evaluation of the words \samp{abab} and \samp{bbab} by
$\Bc_1^4$ gives the fourth power of their evaluation by $\Bc_1$:

\begin{shell}
$ \kbd{vcsn-char-z eval b4.xml abbb}
81
$ \kbd{vcsn-char-z eval b4.xml abab}
16
\end{shell}%$

\subsubsection{Quotient}

Successive products of an automaton create a lot of new states and
transitions.

\begin{shell}
$ \kbd{vcsn-char-z info b1.xml}
States: 2
Transitions: 5
Initial states: 1
Final states: 1
$ \kbd{vcsn-char-z info b4.xml}
States: 16
Transitions: 97
Initial states: 1
Final states: 1
\end{shell}%$

\noindent
One way of reducing the size of our automaton is to use the
\Index{\taffn{quotient}} algorithm.

\begin{shell}
$ \kbd{vcsn-char-z quotient b4.xml \bslash| vcsn-char-z info -}
States: 5
Transitions: 15
Initial states: 1
Final states: 1
\end{shell}%$


%%%% This is really too much detail, with no obvious usage.
%%%%
%% You can also use the \verb-ALPHABET- token an arbitrary number of
%% times, to append letters to the current alphabet. The following
%% commands:
%%
%% %% FIXME: include ({1} a) in the example
%% \begin{shell}
%% $ \kbd{vcsn-char-b --alphabet=ab exp-to-aut '((a) + b)*' >parser.xml}
%% \end{shell}
%% \begin{shell}
%% $ \kbd{vcsn-char-b --output-type=dot identity parser.xml >parser.dot}
%% \end{shell}
%%
%% \noindent
%% will give the same results as:
%% %% FIXME: include ({1} a) in the example
%% \begin{shell}
%% $ \kbd{vcsn-char-b --alphabet=a --parser="ALPHABET=b ONE=e STAR=star" exp-to-aut '(a + b)star' >parser2.xml}
%% \end{shell}
%% \begin{shell}
%% $ \kbd{vcsn-char-b --output-type=dot identity parser2.xml >parser2.dot}
%% \end{shell}
%%
%% \noindent
%% With the \verb-ALPHABET- token, you can specify two types of letters:
%%
%% \begin{itemize}
%% \item letters: one character is used to represent a letter.
%% \item words: many characters are used to represent a letter.
%% \end{itemize}
%%
%% For example, \verb-ALPHABET=letters:abc- defines the alphabet with three letters
%% $\{\verb-a-, \verb-b-, \verb-c-\}$, while \verb-ALPHABET=words:abc- defines the
%% alphabet with only one letter $\{\verb-abc-\}$. Letters are separated by commas, and
%% each \tafkit context defaults to either letters (vcsn-char-b eg) or words
%% (vcsn-int-b eg). \verb|vcsn-int-b --alphabet=0,1,2,3,4,5,6,7,8,9| and
%% \verb|vcsn-int-b --alphabet=letters:0123456789'| will give the same alphabets.
%%
%% Everywhere you want to use special characters used by the \verb|--parser|
%% option, prepending a backslash character will escape it. So, you can write:
%% \verb|ALPHABET=\,,\=| which defines the alphabet with the two letters \verb-,-
%% and \verb-=-.
%%
%% Finally whenever you may put a single character you may write a pair a words.
%% The following construction is then valid: \verb|ALPHABET=(\,,\))|, and defines
%% the alphabet with only one letter: the pair whose first component is \verb-,-
%% and the second one is \verb-)-.
%%
%% For more details, the grammar of the \verb|--parser| option is included in the
%% source code in EBNF notations.


% \section{Transducers}
% \label{sec:vcsn-char-fmp-b}
%
% While the \Vauc library supports two views of transducers, currently
% \tafkit only provides one view:
%
% \begin{description}
% \item[vcsn-char-fmp-b] considering a transducer as a weighted automaton of a
%   product of free monoid,
% \end{description}
%
% In a forthcoming release, \tafkit will provide:
%
% \begin{description}
% \item[vcsn-char-rw] considering a transducer as a machine that takes a
%   word as input and produce another word as (two-tape automata).
% \end{description}
%
% Both views are equivalent and \Vauc provides algorithms to pass from a
% view to the other one.
%
% \subsection{Example}


%% \begin{figure}[tp]
%%   \begin{center}
%
%%\begin{tikzpicture}[automaton]
%%  % states
%%  \node[state, initial left] (A) at (0,0) {};
%%  \node[state, accepting right] (B) at (3,0) {};
%%  \FinalL{s}{A}{(,1)}
%%  % transitions
%%  \path[->] (A) edge [loop above] node[above] {$(1,0)$} (A);
%%  \path[->] (B) edge [loop above] node[above] {$(1,1)$} (B);
%%  \path[->] (B) edge [loop below] node[below] {$(0,0)$} (B);
%%  \path[->] (A) edge node[above] {$(0,1)$} (B);
%%  \end{tikzpicture}
%%}
%%     \caption{Transducer $\Tc_2$ adding 1 to a binary number}
%%     \label{fig:t2}
%%   \end{center}
%% \end{figure}

% To experiment with transducers, we will use \index{T1@$\Tc_1$}$\Tc_1$,
% described in \autoref{fig:t1}, and part of the automaton library
% (\autoref{char-fmp-b:t1}).

%% \subsubsection{Evaluation}
%%
%% Currently \tafkit requires the user to specify the alphabets to use
%% using \option{--alphabet1} and \option{--alphabet2}.
%%
%% \begin{shell}
%% $ \kbd{vcsn-char-fmp-b --alphabet1=ab eval t1.xml 'bba'}
%% \end{shell}

% \subsubsection{Domain}
% The transducer $T$ only accepts binary numbers divisible by 3.
%
% \begin{shell}
% $ \kbd{vcsn-char-fmp-b identity-nodep t1.xml | vcsn-char-fmp-b --alphabet1=ab domain - >div-by-3.xml}
% \end{shell}
%
% Now the file \file{divisible-by-3.xml} contains the description of a
% Boolean automaton that accepts only the numbers divisible by 3:
%
% \begin{shell}
% $ \kbd{vcsn-char-b --output-type=dot identity div-by-3.xml >div-by-3.dot}
% \end{shell}
%
%
% %% \subsubsection{to-tdc}
% %% Each transducer can be transformed to the other type of transducer
% %% thanks to the \taffn{to-tdc} and \taffn{to-rw-tdc} functions.
% %% \begin{shell}
% %% $ \kbd{vcsn-char-rw to-tdc quot_3_rw.xml > quot_3.xml}
% %% $ \kbd{vcsn-char-fmp-b to-rw-tdc add1.xml > add1_rw.xml}
% %% \end{shell}
% %%
% %% \subsubsection{Composing}
% %% \begin{shell}
% %% $ \kbd{vcsn-char-fmp-b compose quot_3.xml add1.xml}
% %% \end{shell}
%
% \subsection{Available functions}
% The following functions are available for both \command{vcsn-char-rw}
% and \command{vcsn-char-fmp-b} programs.  To invoke them, run
% \samp{\var{program} \var{algorithm-name} [\var{arguments}]}.
%
% \begin{shell}
% $ \kbd{vcsn-char-fmp-b --list-commands}
% \end{shell}




\chapter{Automaton Repository}\label{chap:autrepository}

\Vauc comes with a set of interesting automata that can be used to toy
with \tafkit (\autoref{sec:tafkit}) for instance.  In this chapter, we
present each one of these automata.

\section{\B{} automata}

The following automata are predefined files that can be loaded with
\command{vcsn-char-b}.

\subsection{\file{a1.xml} ($\Ac_1$)}

\begin{tikzpicture}[automaton]
  \node[state, initial] (A) at (0,0) {};
  \node[state] (B) at (2,0) {};
  \node[state, accepting] (C) at (4,0) {};
  \path[->] (A) edge node[above] {$a$} (B);
  \path[->] (B) edge node[above] {$b$} (C);
  \path[->] (A) edge[loop below] node[below] {$b$} (A);
  \path[->] (A) edge[loop above] node[above] {$a$} (A);
  \path[->] (C) edge[loop below] node[below] {$b$} (C);
  \path[->] (C) edge[loop above] node[above] {$a$} (C);
\end{tikzpicture}

The automaton $\Ac_1$, defined over the alphabet $A=\{a,b\}$,
recognizes any word of $A^\star$ that contains $ab$.

\subsection{\file{b1.xml} ($\Bc_1$)}\label{sec:b:b1}

\begin{tikzpicture}[automaton]
  \node[state, initial] (A) at (0,0) {};
  \node[state, accepting] (B) at (2,0) {};
  \path[->] (A) edge node[above] {$b$} (B);
  \path[->] (A) edge[loop below] node[below] {$b$} (A);
  \path[->] (A) edge[loop above] node[above] {$a$} (A);
  \path[->] (B) edge[loop below] node[below] {$b$} (B);
  \path[->] (B) edge[loop above] node[above] {$a$} (B);
\end{tikzpicture}

The Boolean automaton $\Bc_1$ accepts words with at least one $b$.

See also \autoref{sec:z:b1} where $\Bc_1$ is defined over
$\langle \Z,+,\times\rangle$.

\subsection{\file{div3base2.xml}}

\begin{tikzpicture}[automaton]
  \node[state, initial above, accepting below]  (S0) at (0,0) {$0$};
  \node[state] (S1) at (2,0) {$1$};
  \node[state] (S2) at (4,0) {$2$};
  \path[->] (S0) edge[loop left] node[left] {$0$} (S0);
  \path[->] (S0) edge[bend left] node[above] {$1$} (S1);
  \path[->] (S1) edge[bend left] node[above] {$0$} (S2);
  \path[->] (S2) edge[loop right] node[right] {$1$} (S2);
  \path[->] (S2) edge[bend left] node[below] {$0$} (S1);
  \path[->] (S1) edge[bend left] node[below] {$1$} (S0);
\end{tikzpicture}

This deterministic automaton is defined on the alphabet
$A=\{0,1\}$ and recognizes the base-2 representations of all the
integers that are divisible by $3$.

The source code distribution of \Vauc actually comes with a
programs called \command{divkbaseb} that can generate a divisor
by any integer $k$ in any base $b$.  The program is located
in \file{data/automata/char-b/divkbaseb} in the source tree,
and take $k$ and $b$ as arguments.

In the general case the binary digits should be read from left to
right.  In the case of \file{div3base2.xml} the digits can also be
read from right to left because the automaton is symmetric.

\subsection{\file{double-3-1.xml}}

\begin{tikzpicture}[automaton]
  \node[state, initial]  (S0) at  (180:1.5cm) {$0$};
  \node[state, accepting] (S1) at (-60:1.5cm) {$1$};
  \node[state]                    (S2) at (60:1.5cm) {$2$};
  \path[->] (S0) edge[bend right=20] node[auto] {$a$} (S1);
  \path[->] (S1) edge[bend right=20] node[auto] {$a$} (S2);
  \path[->] (S2) edge[bend right=20] node[auto] {$a$} (S0);
  \path[->] (S2) edge[bend left=45] node[auto] {$b$} (S1);
  \path[->] (S1) edge[bend left=45] node[auto] {$b$} (S0);
  \path[->] (S0) edge[bend left=45] node[auto] {$b$} (S2);
\end{tikzpicture}

This deterministic automaton has three states organized as a double
ring.  It can read $a$ in one direction and $b$ in the other
direction.

The program \command{data/automata/char-b/double\_ring} can be used to
generate a ring of any number of states.  It takes that number of
states as first argument, and the list of initial states in the
remaining arguments.  The above example uses $3$ states and state $1$
as initial state, hence the name \file{double-3-1.xml}.


\subsection{\file{ladybird-6.xml}}

\begin{tikzpicture}[automaton, initial angle=45, accepting angle=-45]
  \node[state, initial, accepting]  (S0) at  (0:1.5cm) {};
  \node[state] (S1) at (-60:1.5cm) {};
  \node[state] (S2) at (-120:1.5cm) {};
  \node[state] (S3) at (-180:1.5cm) {};
  \node[state] (S4) at (-240:1.5cm) {};
  \node[state] (S5) at (-300:1.5cm) {};
  \path[->] (S0) edge[bend left] node[auto] {$a$} (S1);
  \path[->] (S1) edge[bend left] node[auto] {$a$} (S2);
  \path[->] (S2) edge[bend left] node[auto] {$a$} (S3);
  \path[->] (S3) edge[bend left] node[auto] {$a$} (S4);
  \path[->] (S4) edge[bend left] node[auto] {$a$} (S5);
  \path[->] (S5) edge[bend left] node[auto] {$a$} (S0);

  \path[->] (S1) edge[loop -60]  node[below] {$b,c$} (S1);
  \path[->] (S2) edge[loop -120] node[below] {$b,c$} (S2);
  \path[->] (S3) edge[loop left] node[left]  {$b,c$} (S3);
  \path[->] (S4) edge[loop 120]  node[above] {$b,c$} (S4);
  \path[->] (S5) edge[loop 60]   node[above] {$b,c$} (S5);

  \path[->] (S1) edge node[left] {$c$} (S0);
  \path[->] (S2) edge node[above] {$c$} (S0);
  \path[->] (S3) edge node[above] {$c$} (S0);
  \path[->] (S4) edge node[above] {$c$} (S0);
  \path[->] (S5) edge node[left] {$c$} (S0);
\end{tikzpicture}

A non-deterministic automaton of $6$ states whose determinized version
(using the powerset construction) has $2^6$ states.  We call it
ladybird simply because its drawing looks like a \emph{Coccinella}.

The \command{data/automata/char-b/ladybird} program can be used to
produce ladybirds of any size, just pass the number of states $n$ as
first argument.  We like to use ladybirds for benchmarking because
their determinization always produce the worst case with $2^n$ states.

\section{\Z{} automata}

The following automata are predefined files that can be loaded with
\command{vcsn-char-z}.

\subsection{\file{b1.xml} ($\Bc_1$)}\label{sec:z:b1}

\begin{tikzpicture}[automaton]
  \node[state, initial] (A) at (0,0) {};
  \node[state, accepting] (B) at (2,0) {};
  \path[->] (A) edge node[above] {$b$} (B);
  \path[->] (A) edge[loop below] node[below] {$b$} (A);
  \path[->] (A) edge[loop above] node[above] {$a$} (A);
  \path[->] (B) edge[loop below] node[below] {$b$} (B);
  \path[->] (B) edge[loop above] node[above] {$a$} (B);
\end{tikzpicture}

This automaton has the same structure as the one from
\autoref{sec:b:b1} but a different type.  When defined over the
$\langle \Z,+,\times\rangle$ semiring, this automaton is counting the
number of $b$ in a word.

\subsection{\file{c1.xml} ($\Cc_1$)}\label{sec:z:c1}

\begin{tikzpicture}[automaton]
  \node[state, initial] (A) at (0,0) {};
  \node[state, accepting] (B) at (2,0) {};
  \path[->] (A) edge node[above] {$1$} (B);
  \path[->] (A) edge[loop above] node[above] {$0+1$} (A);
  \path[->] (B) edge[loop above] node[above] {$\{2\}0+\{2\}1$} (B);
\end{tikzpicture}

This automaton converts a binary representation into an integer.  For
instance the word \samp{101010} is associated to the weight $42$.
Note that the alphabet is $A=\{0,1\}$ while the weights are taken in
$\Z$.  Because weights are always specified in braces, there is no
confusion possible.

\subsection{\file{d1.xml} ($\Dc_1$)}\label{sec:z:d1}

\begin{tikzpicture}[automaton]
  \node[state, initial] (A) at (0,0) {};
  \node[state, accepting] (B) at (3,0) {};
  \path[->] (A) edge node[above] {$a+\{-1\}b$} (B);
  \path[->] (A) edge[loop above] node[above] {$a+b$} (A);
  \path[->] (B) edge[loop above] node[above] {$a+b$} (B);
\end{tikzpicture}

This automaton defined on the alphabet $A=\{a,b\}$ computes the
difference between the number of $a$ and $b$ in a word.

\section{\B{} FMP}

The following transducers are predefined files that can be loaded with
\command{vcsn-char-fmp-b}.

\subsection{\file{quot3base2.xml}}

\begin{tikzpicture}[automaton]
  \node[state, initial above, accepting below] (A) at (0,0) {$0$};
  \node[state] (B) at (2,0) {$1$};
  \node[state] (C) at (4,0) {$2$};
  \path[->] (A) edge[loop left] node[left] {$\IOL{0}{0}$} (A);
  \path[->] (C) edge[loop right] node[right] {$\IOL{1}{1}$} (C);
  \path[->] (A) edge[bend left] node[above] {$\IOL{1}{0}$} (B);
  \path[->] (B) edge[bend left] node[below] {$\IOL{1}{1}$} (A);
  \path[->] (B) edge[bend left] node[above] {$\IOL{0}{0}$} (C);
  \path[->] (C) edge[bend left] node[below] {$\IOL{0}{1}$} (B);
\end{tikzpicture}

A transducer computing the quotient by $3$ of a binary number, read
from left to right.

The program \file{data/automata/char-fmp-b/quotkbaseb}, in the source
tree, can be used to construct such a divisor by $k$ in any base $b$.

\subsection{\file{t1.xml}}

\begin{tikzpicture}[automaton]
  \node[state, accepting below]                (A) at (0,0) {};
  \node[state, accepting below, initial above] (B) at (2,0) {};
  \node[state]                                 (C) at (4,0) {};
  \path[->] (A) edge[bend left] node[above] {$\IOL{1}{y}$} (B);
  \path[->] (B) edge[bend left] node[below] {$\IOL{a}{1}$} (A);
  \path[->] (B) edge[bend left] node[above] {$\IOL{1}{x}$} (C);
  \path[->] (C) edge[bend left] node[below] {$\IOL{b}{1}$} (B);
\end{tikzpicture}

\subsection{\file{u1.xml}}

\begin{tikzpicture}[automaton]
  \node[state, initial above]                  (A) at (0,0) {};
  \node[state, accepting below, initial above] (B) at (2,0) {};
  \node[state]                                 (C) at (4,0) {};
  \path[->] (A) edge[bend left] node[above] {$\IOL{x}{1}$} (B);
  \path[->] (B) edge[bend left] node[below] {$\IOL{1}{v}$} (A);
  \path[->] (B) edge[bend left] node[above] {$\IOL{y}{1}$} (C);
  \path[->] (C) edge[bend left] node[below] {$\IOL{1}{u}$} (B);
\end{tikzpicture}

\end{document}


%%% Local Variables:
%%% mode: latex
%%% ispell-local-dictionary: "american"
%%% End:

% LocalWords:  priori semiring monoids tafkit
% LocalWords:  svn vaucanson cd sudo
% LocalWords:  determinize vcsn semiring rw tdc TAF aut dmi determinizes bz xml
% LocalWords:  det detmin eval abab bbba aabb determinized Hopcroft aab fmp dom
% LocalWords:  Glushkov thompson realtime bbab abbb monoids zmin zmax
% LocalWords:  semiring ite Glushkov Rouen Kleene Schutzenberger mutatis
% LocalWords:  mutandis Kleene's Hadamard
