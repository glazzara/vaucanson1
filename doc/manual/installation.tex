\chapter{Installation}

\section{Getting \Vauc}

The latest stable version of the \Vauc platform can be downloaded
from \url{http://vaucanson.lrde.epita.fr/}.  The current development
version can be retrieved from its Subversion\footnote{%
%%
  Subversion can be found at \url{http://subversion.tigris.org/}.
%%
} repository as follows:

\begin{shell}
# \kbd{svn checkout https://svn.lrde.epita.fr/svn/vaucanson/trunk vaucanson}
\end{shell}

\section{Building \Vauc}

The following commands build and install the platform:
\begin{shell}
# \kbd{cd vaucanson-1.0}
\end{shell}
Then:
\begin{shell}
# \kbd{./configure}
...
# \kbd{make}
...
# \kbd{sudo make install}
...
\end{shell}

More detailed information is provided in the files \file{INSTALL},
which is generic to all packages using the GNU Build System, and
\file{README} which details \Vauc's specific build process.

%%% Local Variables:
%%% mode: latex
%%% ispell-local-dictionary: "american"
%%% TeX-master: "vaucanson-user-manual"
%%% End:

% LocalWords:  svn vaucanson cd sudo
