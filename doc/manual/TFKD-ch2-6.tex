

\section{Weighted automata on free monoids \protect\\
\eee over alphabets of pairs}
\label{sec:alp-pai}


An alphabet of pairs~$A$ is defined by a pair of alphabets~$B$ and~$C$
and letters in~$A$ are pairs~$(b,c)$ with~$b$ in~$B$ and~$c$ in~$C$. 
The alphabet~$A$ is thus a subset
of~$B\x C$, $(B\x C)^{*}$ is easily identified with a subset of 
$\Be\x\Ce$ and in this way
some functions apply to automata over~$\Ae$ that correspond to functions on
automata over~$\Be\x\Ce$.

The alphabets of pairs are the key to several constructions on 
automata and transducers.
One example is when letters within an expression or an automaton are 
\emph{indexed}; another one is the treatment of letter-to-letter 
transducers as automata on a free monoid.
In \tafkitv there are not many functions special to automata 
over such alphabets. 
There will be more in subsequent versions.
At this stage, what is more important is the mere existence of this 
type of automata whithin \tafkit, which already allows to demonstrate the 
usefulness of going forth and back between the class of transducers 
and the one of automata (\cf \figur{ltl-pai}). 

\renewcommand{\theenumii}{\theenumi.\arabic{enumii}}

\begin{enumerate}

\item Transformations of automata

\begin{enumerate}
\item \Fctaut{first-projection}\vrglst \Fctaut{second-projection} 
\item \Fctaut{pair-to-fmp}
% \item \Fctaut{index-to-sequentialize}, \Fctaut{index-to-cosequentialize}
% \item \Fctexp{linearize}
\end{enumerate}

\end{enumerate}


\subsection{Transformations of automata}

\subsubsection{\Fct{first-projection}, \Fct{second-projection}}
\begin{SwClCmd}
\begin{shell}
$ \kbd{vcsn first-projection a.xml > b.xml}
$
\end{shell}%
\end{SwClCmd}%
\begin{SwClTxt}
    Yields an automaton over $\Be$ (resp. $\Ce$), by keeping the first 
    (resp. second) component of every letter.
\end{SwClTxt}%
\IndexFct{first-projection}
\IndexFct{second-projection}

% \Comt
% In view of alphabets of $k$-tuples (as planned in the XML format) 
% these functions could have the following syntax:
% 
% \kbd{projection a.xml i}
% 
% where i=1 or 2.


\subsubsection{\Fct{pair-to-fmp}}

\begin{SwClCmd}
\begin{shell}
$ \kbd{vcsn pair-to-fmp a.xml > t.xml}
$
\end{shell}%
\end{SwClCmd}%
\begin{SwClTxt}
    yields a \fmpt over $\Be\x\Ce$, every letter $(b,c)$ being mapped to the 
    corresponding element of $\Be\x\Ce$.
\end{SwClTxt}%
\IndexFct{pair-to-fmp}

\Spec 
A transition 
labelled by $(a,x)(b,x)(a,y)$ becomes a transition labelled by 
$(aba,xxy)$.

% \Comt
% One of the key tools for dealing with \emph{synchronous} transducers.

% \subsubsection{\Fct{index-to-sequentialize}, \Fct{index-to-sequentialize}}
% 
% \begin{ComV}
%     \tha does not exist now. reminder.
% 
%     \thb would be useful if one wants to play with the HECCA algorithm (cf. 
% SL+JS's paper) within \tafkit.
% \end{ComV}
% 
% \subsubsection{\Fct{linearize}}
% 
% \begin{ComV}
%     \tha does not exist now. reminder.
% 
%     \thb takes advantage of the existence of the alphabet of pairs and opens the 
% possibility of a classical construction of automata from expressions 
% (so-called Berry-Sethi construction)
% \end{ComV}
% 
\endinput
