\chapter*{Introduction}
% \label{chp:intro}
\addcontentsline{toc}{chapter}{Introduction}

\vcsn is a free software platform dedicated to the manipulation of
finite state automata.  Here, `finite state automata' is to be
understood in the broadest sense: \vcsn supports \emph{weighted}
automata over a free monoid, and even \emph{weighted} automata on some
\emph{non-free monoids} (currently only automata on products of two
free monoids--- also known as \emph{transducers}---are supported).

\bigskip

The platform consists in a few components:

\begin{description}
\item[The \vcsn library] is a \Cxx library that implements objects for
  automata, rational expressions, as well as algorithms on these
  objects.  This library is generic, in the sense that it makes it
  possible to write an algorithm once and apply it to different types
  of automata.  However this genericity is achieved in a way that
  should not cause any slowdown at runtime: because the type of the
  automaton manipulated is known at compile time, compiling an
  algorithm will generate code that is almost as efficient as an
  algorithm written specifically for this type of automaton.

\item[\tafkit] is a command-line interface to the library that allows
  user to execute \vcsn's algorithms without any knowledge of \Cxx.
  Because the \vcsn library needs to know the type of automata at
  compile time, the \tafkit interface has been instantiated for a
  predefined set of common automaton types.

  \tafkit does not allow to write new algorithms nor to manipulate new
  types of automata, but it makes it possible to combine without
  efforts a large set of algorithms on common automata types.

\item[A repository of automata] that shows examples of automata of
  various types, and also contains programs, called 
  \emph{automata-factory}, which create parametrized families of
  automata.

\end{description}
It is coupled with some other modules:

\begin{description}
\item[An \xml format for automata] and expressions, called \fsmxml.
This format aims at being an interchange format for automata and thus 
at making possible, and hopefully easy, the communication between 
various programs that input or output automata.
So far,  this format is used as the normal, and default, input and 
output format for \tafkit.

\item[A graphic user interface] called \vgi, especially dedicated to
\vcsn is under development at the EE Department of the National Taiwan
University in Taipeh.  It will allow to describe automata and to visualize
the result of operation on automata in a graphical way.  All functions
defined in \tafkit will be called via the menu of \vgi.


\end{description}

Ideally, a user's manual for \vcsn should document all of these components.
We decided not to do so, not so much because it is a lot of work, but 
also as this work would not be so useful.

After several years of hard and complex developments, the evolution 
and progress of the \vcsn platform are now stuck and we have reached 
the conclusion that we have to undertake a thorough revision of the 
\vcsn library that will most probably change its interface and the one 
of the associated API.
These new developments will give rise to a new series of 
versions of \vcsn, coined \vcsn 2.x.

% On the other hand, there will be a \tafkit for these future versions of 
% \vcsn, whose functionalities will include all those of the present 
% one and whose interface will essentially be the same as the present 
% one as well.
% \tafkit \vcsnv will serve as a landmark for both functionalities and 
% performance of the first version of \vcsn.
% It will be the only documented part of \vcsnv.


However, we want to have a version of the platform that will serve as 
a landmark for both functionalities and  
performance of the first phase of \vcsn.
It will be coined \vcsnv.
Moreover, there will be a \tafkit for the future versions of \vcsn,
its  functionalities will include all those of the present one,
and its interface will essentially be the same as the \tafkit of 
\vcsnv.
\tafkitv will be the only documented part of \vcsnv.

A beta version of \vcsnv has been presented at the FSMNLP 2011 Conference, 
held in Blois, France, from July 12 to July 15 2011.
All users are encouraged to send us remarks, comments, and bug reports.
We shall make our possible to take them into account in the minor 
revisions that will be made to \vcsnv until the release of \vcsn~2.0.

\endinput 


