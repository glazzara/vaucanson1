\documentclass[a4paper]{report}

\usepackage[american]{mybabel}
\usepackage[latin1]{inputenc}
\usepackage{a4wide}
\usepackage{myacronym}
\usepackage{amsmath}
\usepackage{amsfonts}
\usepackage{myhyperref}
\usepackage{textcomp}
\usepackage{tabularx}
\usepackage{url}
\usepackage{vaucanson-g}
\usepackage{graphicx}
\usepackage{tabularx}
\usepackage{texi}
\usepackage{misc}

%% Put the TOC, the bibliography and index in the TOC.
\usepackage{tocbibind}

%% ---------------------- %%
%% Mathematical symbols.  %%
%% ---------------------- %%
\renewcommand{\max}{\textrm{max}}
\renewcommand{\min}{\textrm{min}}
\newcommand{\Z}{\texorpdfstring{\ensuremath{\mathbb{Z}}}{Z}}
\newcommand{\B}{\texorpdfstring{\ensuremath{\mathbb{B}}}{B}}
\newcommand{\Ae}{\ensuremath{A^{*}}}
\newcommand{\SerSAnMon}[2]%
    {\ensuremath{#1 \langle \! \langle  #2  \rangle \! \rangle}}

\newcommand{\newcal}[1]{%
  \expandafter \newcommand \csname #1c\endcsname%
  {\texorpdfstring{\ensuremath{\mathcal{#1}}}{#1}}%
}
\newcal{A}
\newcal{B}
\newcal{C}
\newcal{T}

\newtheorem{theorem}             {Theorem}[chapter]
\newtheorem{definition} [theorem]{Definition}
\newtheorem{remark}     [theorem]{remark}


%% ------------------------ %%
%% Index and Bibliography.  %%
%% ------------------------ %%

% The same argument is output and put in the index.
\usepackage{makeidx}
\makeindex
\newcommand{\Index}[1]{\index{#1}#1}

% Bibliography.
\usepackage{natbib}

\usepackage{listings}
\lstset{%
  numbers=left,
  numberstyle=\tiny,
  stepnumber=5,
  numbersep=5pt,
  firstnumber=1,
  basicstyle=\small,
  frame=single,
  language=C++,
  float}
\input ldf/stl.sty
\input ldf/vaucanson.sty

% Document a figure.
\newenvironment{legend}{%
  \begin{quote}%
    }{%
  \end{quote}%
}

%% Display an interactive session.
\usepackage{alltt}
\newenvironment{shell}
{\begin{alltt}}
{\end{alltt}}

\input version

%% -------------------- %%
%% Vaucanson commands.  %%
%% -------------------- %%

%% A TAF-kit function name (i.e., the first argument for TAF-Kit
%% programs).
\newcommand{\taffn}[1]{\code{#1}}
\newcommand{\tafkit}{\textsc{TAF-Kit}\xspace}
\newcommand{\Vauc}{\textsc{Vaucanson}\xspace}
\newcommand{\TFK}{\tafkit\xspace}%
\newcommand{\TFKv}{\tafkit{} \VcsnVersion\xspace}%
%% Attach weights in a type writer environment.
\newcommand{\withweighttt}[2]{\symbol{`\{}#1\symbol{`\}} #2}

% \begin{fnsection}{NAME}
% -----------------------
% A group of taf-kit functions.
% Putting the ending hline in the end section is tempting, but fails.
\newenvironment{fnsection}[1]{%
  % \item{NAME}{ARGS}{DOC}
  % ----------------------
  % Define a taf-kit function.
  % We use \item because it has Emacs indentation support.
  \renewcommand{\item}[3]{\texttt{##1} \var{##2} & ##3\\}

  \tabularx{\textwidth}{|l|X|}%
    \hline\multicolumn{2}{|c|}{#1} \\\hline%
  }{%
  \endtabularx%
}


%% --------------------------------- %%
%% Automatic generation of content.  %%
%% --------------------------------- %%

% These two commands are recognized by extex which will extract the
% code to run to produce the generated files.  LaTeX will later
% include them.
%
% If you know a simpler means to do that, be my guest...
%
% Note that this system "caches" the result, and uses a Makefile to
% avoid useless recompilation of snippets that did not change.

% This one is not traced, so we can use it to factor definitions
% making the same calls (e.g., includegenerateddot and execdisplay).
%
% Beware that files with an embedded dots in the base name are
% improperly handled: includegraphics tries to interpret it as an extensions.
\newcommand{\includegenerated}[1]{\includegraphics[scale=.6]{#1}}

% This one is scanned by extex.pl.
\newcommand{\includegenerateddot}[1]{\includegenerated{#1}}

% \execcaption{TAG}{COMMAND}
% --------------------------
% Run the COMMAND, which is a Unix pipe, and save the result in a file
% named TAG.tex.  Input it here.  Make explicit the input/output of
% the COMMAND.
%
% If there are redirections, *THEY MUST STICK TO THE FILE*!!!
% Do not pass `echo toto > foo', but `echo toto >foo'.
\newcommand{\execcaption}[2]{\input{#1}}

% \execdisplay{TAG}{PRE-COMMAND}{POST-COMMAND}
% --------------------------------------------
% Run a taf-kit pipe ending with a display, and include the resulting
% automaton.  It will run `PRE-COMMAND display POST-COMMAND >TAG.dot',
% and include the latter.

% Typical use is:
%
% \execdisplay{a1}{vcsn-b}{a1}
%
\newcommand{\execdisplay}[3]{\includegenerated{#1}}

\title{The \Vauc \TFKv Manual}
\author{The \Vauc \textsc{Group}}
\date{\VcsnDate}

%% ---------- %%
%% Document.  %%
%% ---------- %%
\begin{document}

\maketitle

\setcounter{tocdepth}{2}
\tableofcontents

\chapter*{Introduction}
\label{sec:intro}

The \Vauc software platform is dedicated to the computation with
finite state automata.  Here, `finite state automata' is to be
understood in the broadest sense: \emph{weighted} automata on a free
monoid --- that is, automata that not only accept, or recognize,
\emph{words} but compute for every word a \emph{multiplicity} which is
taken a priori in \emph{an arbitrary semiring} --- and even weighted
automata on \emph{non free monoids}.  The latter become far too
general objects.  As for now, are implemented in \Vauc only the
(weighted) automata on (direct) products of free monoids, machines
that are often called \emph{transducers} --- that is automata that
realize (weighted) relations between words\footnote{When the relation
  is ``weighted'' the multiplicity has to be taken in a
  \emph{commutative} semiring.}.

When designing \Vauc, we had three main goals in mind: we wanted
\begin{enumerate}
\item a \emph{general purpose} software,
\item a software that allows a programming style natural to computer
  scientists who work with automata and transducers,
\item  an open and free software.
\end{enumerate}

This is the reason why we implemented so to say \emph{on top} of the
\Vauc platform a library that allows to apply a number of functions on
automata, and even to define and edit automata, without having to
bother with subtleties of \Cxx programming.  The drawback of this is
obviously that the user is given a \emph{fixed} set of functions that
apply to \emph{already typed} automata.  This library of functions
does not allow to write new algorithms on automata but permits to
combine or compose without much difficulties nor efforts a rather
large set of commands.  We call it \tafkit, standing for \emph{Typed
  Automata Function Kit}, as these commands take as input, and output,
automata whose type is fixed.  \tafkit is presented in
\autoref{sec:tafkit}.


This document presents a simple interface to \Vauc: a set of programs
tailored to be used from a traditional shell.  Since they exchange
\emph{typed} XML files, there is one program per automaton type.  Each
program supports a set of operations which depends on the type of the
automaton.

Many users of automata consider only automata whose transitions are
labeled by letters taken in an alphabet, which we call, roughly
speaking, \emph{classical} automata or \emph{Boolean} automata.  The
first program of the \tafkit, \command{vcsn-b}, allows to compute with
classical automata and is described in \autoref{sec:vcsn-b}.

\autoref{sec:vcsn-tdc} describes the program \command{vcsn-tdc} which
allows to compute with transducers, that is, automata whose
transitions are labeled by pair of words, which are elements of a
\emph{product of free monoids}, hence the name.

In \autoref{sec:vcsn-z} we consider the programs of the \tafkit that
compute with automata over a free monoid and with multiplicity, or
\emph{weight} taken in the set of integers equipped with the usual
operations of addition and multiplication, that is, the semiring $\Z$.

% ,
% or with the operations of $\min$ and addition, or the operations of
% $\max$ and addition, that is, the semirings $\Z,\min,+$ and
% $\Z,\max,+$.  Finally, the forth section describes the program
% \command{vcsn-rw-tdc} which allows to compute with transducers, viewed
% as automata on a free monoid (the input monoid) with multiplicity
% taken in the semiring of finite and even rational subsets of another
% free monoid (the output monoid).


%%% Local Variables:
%%% mode: latex
%%% ispell-local-dictionary: "american"
%%% TeX-master: "vaucanson-user-manual"
%%% End:

\chapter{Installation}

\section{Getting \Vauc}

The latest stable version of the \Vauc platform can be downloaded
from \url{http://vaucanson.lrde.epita.fr/}.  The current development
version can be retrieved from its Subversion\footnote{%
%%
  Subversion can be found at \url{http://subversion.tigris.org/}.
%%
} repository as follows:

\begin{shell}
# \kbd{svn checkout https://svn.lrde.epita.fr/svn/vaucanson/trunk vaucanson}
\end{shell}

\section{Building \Vauc}

The following commands build and install the platform (the process
takes an hour on a modern computer):
\begin{shell}
# \kbd{cd vaucanson-1.0}
\end{shell}
Then:
\begin{shell}
# \kbd{./configure}
...
# \kbd{make}
...
# \kbd{sudo make install}
...
\end{shell}

%%% Local Variables:
%%% mode: latex
%%% ispell-local-dictionary: "american"
%%% TeX-master: "vaucanson-user-manual"
%%% End:

\chapter{The \Vauc toolkit}

This chapter prevents a simple interface to \Vauc: a set of programs
tailored to be used from a traditional shell.  Since they exchange
\emph{typed} XML files, there is one program per automaton type.  Each
program support a set of operations which depends on the type of the
automaton.

Six programs are planned to be shipped in a forthcoming version:
\begin{description}
\item[vcsn-b] automata over the Boolean semiring $\mathbb{B}$;
\item[vcsn-z] automata over $(\mathbb{Z},+)$;
\item[vcsn-z-min-plus] automata over $(\mathbb{Z},min)$;
\item[vcsn-z-max-plus] automata over $(\mathbb{Z},max)$;
\item[vcsn-rw-tdc] realtime transducers;
\item[vcsn-tdc] automata over free monoid products.
\end{description}

Currently, only \command{vcsn-b}, \command{vcsn-z}, and
\command{vcsn-tdc}, are implemented.

\newpage

\section{Boolean automata}

This section focuses on the program \Index{\command{vcsn-b}}, the
TAF-Kit component dedicated to Boolean automata.

\subsection{First Contacts}

\command{vcsn-b} and its peer components of \tafkit all share the same
simple interface:

\begin{shell}
# vcsn-b \var{function} \var{automaton} \var{arguments...}
\end{shell}

\noindent
The \var{function} is the name of the operation to perform on the
\var{automaton}, specified as an XML file.  Some functions, such as
evaluation, will require additional arguments, such as the word to
evaluate.  Some other functions, such as \samp{exp-to-aut} do not have
an \var{automaton} argument.

\tafkit is made to work with Unix \emph{pipes}, that is to say, chains
of commands which feed each other.  Therefore, all the functions
produce a result on the standard output, and if an \var{automaton} is
\samp{-}, then the standard input is used.

Other than that, the interface of the \tafkit components is usual,
including options such as \option{--version} and \option{--help}:

\execcaption{vcsn-b-help}{vcsn-b --help}

The whole list of supported commands is available via
\option{--list-commands}:
\execcaption{vcsn-b-commands}{vcsn-b --list-commands}


\subsection{A first example}

\Vauc provides a set of common automata.  The function
\Index{\taffn{list-automata}} lists them all:

\execcaption{b-list}{vcsn-b list-automata}

\begin{figure}[ht] \centering
  \begin{VCPicture}{(0,-2)(6,2)}
    % states
    \State{(0,0)}{A}
    \State{(3,0)}{B}
    \State{(6,0)}{C}
    % initial--final
    \Initial{A}
    \Final{C}
    % transitions
    \EdgeL{A}{B}{a}
    \EdgeL{B}{C}{b}
    \LoopS[.5]{A}{b}
    \LoopN[.5]{A}{a}
    \LoopS[.5]{C}{b}
    \LoopN[.5]{C}{a}
  \end{VCPicture}
  \begin{legend}
    The graphical layout of this automaton was described by hand,
    using the Vaucanson-G \LaTeX{} package.  However, the following
    figures are generated by TAF-Kit, giving a very nice layout, yet
    slightly less artistic.

    The automaton is taken from \citet[Fig. I.1.1, p. 58]{sakarovitch.03.eta}.
  \end{legend}
  \caption{The automaton \index{A1@$\mathcal{A}_1$}$\mathcal{A}_1$}
  \label{fig:a1}
\end{figure}

Let's consider the Boolean automaton $\mathcal{A}_1$
(\autoref{fig:a1}).  It is part of the standard library, and
can be dumped using \Index{\taffn{dump-automaton}}:

\execcaption{a1-dump}{vcsn-b dump-automaton a1}

Usual shell indirections (\samp{|}, \samp{>}, and \samp{<}) can be
used to combine TAF-Kit commands.  For instance, this is an easy means
to bring a local copy of this file:

\execcaption{a1-xml}{vcsn-b dump-automaton a1 >a1.xml}

TAF-Kit uses XML to exchange automata, to get graphical rendering of
the automaton, you may either invoke \Index{\taffn{dot-dump}} and then
use a Dot compliant program, or use \Index{\taffn{display}} that does
both.

\execcaption{a1-dot}{vcsn-b dot-dump a1.xml >a1.dot}
\begin{center}
  \includegenerateddot{a1}
\end{center}


\subsubsection{Determinization of $\mathcal{A}_1$}
To determinize a Boolean automaton, call the
\Index{\taffn{determinize}} function:

\execcaption{a1det}{vcsn-b dump-automaton a1 | vcsn-b determinize - >a1det.xml}

To get information about an automaton, call the \Index{\taffn{info}} function:
\execcaption{a1det-info}{vcsn-b info a1det.xml}

Or use dotty to visualize it:
\execcaption{a1det.dot}{vcsn-b dot-dump a1det.xml >a1det.dot}
\begin{center}
  \includegenerateddot{a1det}
\end{center}

\subsubsection{Minimizing}

The \index{\taffn{minimize}}minimal automaton can be computed the same way:
\begin{shell}
# vcsn_b minimize a1_det.xml > a1_min.xml
\end{shell}

The commands can be composed with pipes from the shell, using
\Index{\samp{-}} to denote the standard input.
\begin{shell}
# vcsn_b determinize a1.xml | vcsn_b minimize - > a1_min.xml
\end{shell}


\subsubsection{Evaluation}

To \index{\taffn{eval}}evaluate whether a word is accepted:

\execcaption{a1.abab}{vcsn-b eval a1.xml 'abab'}
\execcaption{a1.bbba}{vcsn-b eval a1.xml 'bbba'}

\noindent
where 1 (resp. 0) means that the word is accepted (resp. not accepted)
by the automaton.

\subsection{Rational expressions and Boolean automata}

\Vauc provides functions to manipulate rational expressions associated
to Boolean automata. For instance, computing the language recognized
by a Boolean automaton can be done using \Index{\taffn{aut-to-exp}}:

\execcaption{a1.rat}{vcsn-b aut-to-exp a1.xml}
\execcaption{a1det.rat}{vcsn-b aut-to-exp a1det.xml}

\Vauc provides several algorithms that build an automaton that
recognizes a given language.  The following sequence computes the
minimal automaton of \samp{(a+b)*ab(a+b)*}.

\execcaption{l1}{vcsn-b --alphabet=ab standard "(a+b)*a.b.(a+b)*" | vcsn-b minimize - >l1.xml}
\execcaption{l1.dot}{vcsn-b dot-dump l1.xml >l1.dot}
\begin{center}
  \includegenerateddot{l1}
\end{center}

\subsection{Available functions}
%% Une definition plus rigoureuse des algorithmes devrait etre fournie
%% en annexe.
This section gives a brief definition of all functions that \Vauc
provides for manipulating Boolean automata.  All these algorithms are
invoked using \samp{vcsn-b \var{algorithm-name} [\var{arguments}]}. If
the argument is replaced by \samp{-} then the program will read an
argument from the standard input.  All algorithms dump their result to
the standard output, except the ``tests'' functions that also return
an exit status (0 if the test is successful, anything else otherwise).

\smallskip

In the following:
\begin{itemize}
\item \var{a1} and \var{a2} are two Boolean automata described
  in \Vauc XML format;
\item \var{w} is a word, for example \samp{"aabb"} if you are
  working on an alphabet that contains the letters \samp{a} and
  \samp{b};
\item \var{exp} is a rational expression denoting a language;
\item \var{n} is a nonnegative integer.
\end{itemize}


\begin{fnsection}{Input/output work with automata}
\item{define-automaton}{}{Define an automaton from scratch.}
\item{edit-automaton}{a1}{Edit an existing automaton.}
\item{info}{a1}{Print the number of states, transitions, initial and
    final states of \var{a1}.}
\item{display}{a1}{Display the automaton using Dotty.}
\item{dump}{a1}{Dump the automaton to Dot format.}  \hline
\end{fnsection}

\begin{fnsection}{Tests and evaluation on automata}
\item{are-isomorphic}{a1 a2}{Test whether \var{a1} and \var{a2} are
    isomorphic.}
\item{evaluation}{a1 w}{Test whether the word \var{w} is accepted by
    \var{a1}.}
\item{is-deterministic}{a1}{Test whether \var{a1} is deterministic.}
\item{is-empty}{a1}{Test whether \var{a1} accepts no word.}  \hline
\end{fnsection}

\begin{fnsection}{Generic algorithms for automata}
\item{accessible}{a1}{Extract the sub-automaton of accessible states
    of \var{a1}.}
\item{co-accessible}{a1}{Extract the sub-automaton of co-accessible
    states of \var{a1}.}
\item{trim}{a1}{Trim the automaton \var{a1}.}
\item{transpose}{a1}{Compute the automaton accepting the mirror
    language of the one accepted by \var{a1}.}

\item{closure}{[-bf] a1}{$\varepsilon$-removal algorithm.}
  & \option{-b} : backward closure\\
  & \option{-f} : forward closure\\
\item{concatenate}{a1 a2}{Concatenate \var{a1} to \var{a2}.}
\item{sum}{aut1 aut2}{Compute the sum of \var{a1} and \var{a2}.}
\item{normalize}{aut1}{Compute an automaton with unique initial and
    final states, with $\varepsilon$-transitions.}
\item{standardize}{aut1}{Compute an automaton with unique initial
    state without adding $\varepsilon$-transitions.}  \hline
\end{fnsection}

\begin{fnsection}{Generic algorithms for automata on letters}
\item{realtime}{[-bf] a1}{$\varepsilon$-removal algorithm and make
    every transition labeled by a letter.}
  & \option{-b} : backward closure\\
  & \option{-f} : forward closure\\
\item{product}{a1 a2}{Compute the (Cartesian) product of \var{a1} and
    \var{a2}.}
\item{power}{a1 n}{Compute the (Cartesian) product of \var{a1} by
    itself \var{n} times.}
\item{quotient}{a1}{Compute the minimal automaton in bi-simulation
    with \var{a1}.}  \hline
\end{fnsection}

\begin{fnsection}{Algorithms specific to Boolean automata}
\item{determinize}{a1}{Compute the determinized automaton of
    \var{a1}.}
\item{complement}{a1}{Compute an automaton that accepts the complement
    language of the one accepted by \var{a1}.}
\item{minimize}{[-hm] a1}{Minimize the {\em deterministic} automaton
    \var{a1}.}
  & \option{-h} : use the Hopcroft algorithm\\
  & \option{-m} : use the Moore algorithm\\
  \hline
\end{fnsection}

\begin{fnsection}{Conversion between automata and expressions}
\item{aut-to-exp}{a1}{Print a rational expression denoting the
    language accepted by \var{a1}.}
\item{expand}{exp}{Partially expand rational expressions. For
    instance, expanding \samp{a(b+ab(a+b))} will produce
    \samp{aab.(a+b)*+ab}}
\item{derived-term}{exp}{Compute the derived term automaton of
    \var{exp}.}
\item{standard}{exp}{Compute the standard (Glushkov) automaton of
    \var{exp}.}
\item{thompson-of}{exp}{Compute the Thompson automaton of \var{exp}.}
  \hline
\end{fnsection}


\newpage
\section{Transducers}

\Vauc supports two views of transducers, and therefore provides two
programs:
\begin{description}
\item[vcsn-tdc] considering a transducer as a weighted automaton of a
  product of free monoid,
\item[vcsn-rw-tdc] considering a transducer as a machine that takes a
  word as input and produce another word as (two-tape automata).
\end{description}
Both views are equivalent and \Vauc provides algorithms to pass from a
view to the other one.

\subsection{Example}

\begin{figure}[tp]
  \begin{center}
    \begin{VCPicture}{(0,-2)(6,2)}
      % states
      \State{(0,0)}{A} \State{(3,0)}{B} \State{(6,0)}{C}
      \Initial[w]{A}
      \Final[s]{A}
      % transitions
      \LoopN[.5]{A}{\IOL{0}{0}}
      \LoopN[.5]{C}{\IOL{1}{1}}
      \ArcL{A}{B}{\IOL{1}{0}}
      \ArcL{B}{A}{\IOL{1}{1}}
      \ArcL{B}{C}{\IOL{0}{0}}
      \ArcL{C}{B}{\IOL{0}{1}}
    \end{VCPicture}
    \caption{Realtime transducer $\mathcal{T}_1$ computing the
      quotient by 3 of a binary number}
    \label{fig:t1}
  \end{center}
\end{figure}

\begin{figure}[tp]
  \begin{center}
    \begin{VCPicture}{(0,-2)(3,2)}
      % states
      \State{(0,0)}{A} \State{(3,0)}{B}
      \Initial[w]{A}
      \FinalL{s}{A}{(,1)}
      \Final[e]{B}
      % transitions
      \LoopN[.5]{A}{(1,0)}
      \LoopN[.5]{B}{(1,1)}
      \LoopS[.5]{B}{(0,0)}
      \EdgeL{A}{B}{(0,1)}
    \end{VCPicture}
    \caption{Transducer $\mathcal{T}_2$ adding 1 to a binary number}
    \label{fig:t2}
  \end{center}
\end{figure}

The realtime transducer $\mathcal{T}_1$ (\autoref{fig:t1}) gives the
quotient by 3 of a binary number and the transducer $\mathcal{T}_2$
(\autoref{fig:t2}) adds 1 to a binary number.


\subsubsection{Evaluation}

\begin{shell}
# vcsn-rw-tdc evaluation quot_3_rw.xml '110'
\textit{0.1.0}
\end{shell}

\subsubsection{Domain}
The transducer $T$ only accepts binary number which are divisible by 3
as input.
\begin{shell}
# vcsn-rw-tdc domain quot_3_rw.xml > divisible_by_3.xml
\end{shell}
Now the file \file{divisible-by-3.xml} contains the description of a
Boolean automaton that accepts only the numbers divisible by 3.

\subsubsection{to-tdc}
Each transucers can be transformed to the other type of transducer
thanks to the \taffn{to-tdc} and \taffn{to-rw-tdc} functions.
\begin{shell}
# vcsn-rw-tdc to-tdc quot_3_rw.xml > quot_3.xml
# vcsn-tdc to-rw-tdc add1.xml > add1_rw.xml
\end{shell}

\subsubsection{Composing}
\begin{shell}
# vcsn-tdc compose quot_3.xml add1.xml
\end{shell}

\subsection{Available functions}
The following functions are available for both \command{vcsn-rw-tdc}
and \command{vcsn-tdc} programs.  To invoke them, run
\samp{\var{program} \var{algorithm-name} [\var{arguments}]}.

\smallskip

In the following:

\begin{itemize}
\item \var{t1} and \var{t2} are two transducers (either ``realtime''
  or not) described in \Vauc XML format;
\item \var{w} is a word, for example \samp{"aabb"} if you are working
  on an alphabet that contains the letters \samp{a} and \samp{b};
\item \var{a} is a Boolean automaton;
\item \var{t1-rw} is a realtime transducer;
\item \var{t1-fmp} is a transducer (seen as an automaton over a free
  monoid product).
\end{itemize}

\begin{fnsection}{Input/output work with transducers}
\item{define-automaton}{}{Define a transducer from scratch.}
\item{edit-automaton}{t1}{Edit an existing transducer.}
\item{info}{t1}{Print the number of states, transitions, initial and
    final states of \var{t1}.}
\item{display}{t1}{Display the transducer using DOTTY.}  \hline
\end{fnsection}

\begin{fnsection}{Tests and evaluation on tranducers}
\item{are-isomorphic}{t1 t2}{Test if the two transducers are
    isomorphic.}
\item{evaluation}{t1 w}{Compute the evaluation of \var{w} by
    \var{t1}.}
\item{is-empty}{t1}{Test if \var{t1} realizes the empty relation.}
  \hline
\end{fnsection}

\begin{fnsection}{Generic algorithm for tranducers}
\item{closure}{t1}{$\varepsilon$-removal algorithm.}

\item{compose}{t1 t2}{Compute a tranducer realizing $f_2 \circ f_1$,
    where $f_1$ (resp. $f_2$) is the function associated to \var{t1}
    (resp. \var{t2}).}

\item{domain}{t1}{Compute an automaton accepting all input accepted by
    the transducer t1.}

\item{evaluation}{t1}{Compute the evaluation of w by t1.}

\item{evaluation-aut}{t1}{Compute a Boolean automaton describing the
    words produced by the language described by a evaluated by t1.}

\item{image}{t1}{Compute an automaton describing all output produced
    by the transducer t1.}

\item{transpose}{t1}{Compute the transposed of the transducer t1.}

\item{trim}{t1}{Compute the trimmed transducer of t1.}  \hline
\end{fnsection}

\begin{fnsection}{Algorithms for transducers}
\item{sub-normalize}{t1-fmp}{Compute the sub-nomalized transducer of
    \var{t1-fmp}.}
\item{is-sub-normalize}{t1-fmp}{Test if \var{t1-fmp} is
    sub-normalized.}
\item{composition-cover}{t1-fmp}{.}%%FIXME
\item{composition-co-cover}{t1-fmp}{.}%%FIXME
\item{b-compose}{t1-fmp t2-fmp}{Compose \var{t1-fmp} and
    \var{t2-fmp}, two unweighted normalized or sub-normalized
    transducers.}
\item{to-rw-tdc}{t1-fmp}{Compute the equivalent realtime transducer
    of \var{t1-fmp}.}
\item{intersection}{a}{Transform \var{a} in a fmp transducer by
    creating, for each word, a pair containing twice this word.}
  \hline
\end{fnsection}


\begin{fnsection}{Algorithms for ``realtime'' transducers}
\item{realtime}{t1-rw}{Compute the realtime transducer of  \var{t1-rw}.}
\item{is-realtime}{t1-rw}{Test if \var{t1-rw} is realtime.}
\item{to-tdc}{t1-rw}{Compute the equivalent fmp transducer of \var{t1-rw}.}
\hline
\end{fnsection}

\newpage
\section{Weighted automata}

This part shows the use of the program \Index{\command{vcsn-z}}, but
all comments should also stand for the programs
\command{vcsn-z-min-plus} and \command{vcsn-z-max-plus}.

Again, we will toy with some of the automata provided by
\command{vcsn-z}:
\execcaption{vcsn-z-automata}{vcsn-z list-automata}

\subsection{Counting \samp{b}s}

\begin{figure}[tp] \centering
  \begin{VCPicture}{(0,-2)(3,2)}
    % states
    \State{(0,0)}{A}
    \State{(3,0)}{B}
    % initial--final
    \Initial{A} \Final{B}
    % transitions
    \EdgeL{A}{B}{b}
    \LoopS[.5]{A}{b}
    \LoopN[.5]{A}{a}
    \LoopS[.5]{B}{b}
    \LoopN[.5]{B}{a}
  \end{VCPicture}
  \begin{legend}
    Considered without weight, $\mathcal{B}_1$ accepts words with a
    \samp{b}.  With weights, it counts the number of \samp{b}s.
  \end{legend}
  \caption{The automaton \index{B1@$\mathcal{B}_1$}$\mathcal{B}_1$}
  \label{fig:b1}
\end{figure}

Let's consider $\mathcal{B}_1$ (\autoref{fig:b1}, an
$\mathbb{N}$-automaton, \textit{i.e.}  an automaton whose label's
weights are in $\mathbb{N}$.  This time the evaluation of the word
\var{w} by the automaton $\mathcal{B}_1$ will produce a number,
rather than simply accept or reject \var{w}.  For instance let's
evaluate \samp{abab} and \samp{bbab}:

\execcaption{b1.abbb}{vcsn-z dump-automaton b1 | vcsn-z eval - 'abbb'}
\execcaption{b1.abab}{vcsn-z dump-automaton b1 | vcsn-z eval - 'abab'}

\noindent
Indeed, $\mathcal{B}_1$ counts the number of \samp{b}s.

\subsubsection{Power}

Now let's consider the $\mathcal{B}_1^n$, where
\begin{displaymath}
  \mathcal{B}_1^n = \prod_{i=1}^n \mathcal{B}_1, n > 0
\end{displaymath}

\noindent
This is implemented by the \Index{\taffn{power}} function:

\execcaption{b4}{vcsn-z dump-automaton b1 | vcsn-z power - 4 >b4.xml}
\begin{shell}
# vcsn-z power b1.xml 4 > b4.xml
\end{shell}

\noindent
The file \file{b4.xml} now contains the automaton $\mathcal{B}_1^4$.
Lets see what the evaluation of the words \samp{abab} and \samp{bbab}
gives with this automaton:

\execcaption{b4.abbb}{vcsn-z eval b4.xml 'abbb'}
\execcaption{b4.abab}{vcsn-z eval b4.xml 'abab'}

This time one can notice that the automaton $\mathcal{B}_1^4$ returns
the evaluation of $\mathcal{B}_1$ at power 4.

\subsubsection{Quotient}

One drawback of doing successive products of an automaton is
that it creates a lot of new states and transitions.

\execcaption{b1.info}{vcsn-z dump-automaton b1 | vcsn-z info -}
\execcaption{b4.info}{vcsn-z info b4.xml}

One way of reducing the size of our automaton is to use the
\Index{\taffn{quotient}} algorithm.

\execcaption{b4.quot.info}{vcsn-z quotient b4.xml | vcsn-z info -}

\subsection{Available functions}

In this section you will find a brief definition of all functions for
manipulating weighted automata. The following functions are available
for both.  They are called using \command{vcsn-z},
\command{vcsn-z-max-plus}, and \command{vcsn-z-min-plus} run as
\samp{\var{program} \var{algorithm-name} [\var{arguments}]}.

\smallskip

In the following:

\begin{itemize}
\item \var{a1} and \var{a2} are two weighted automata described in
  \Vauc XML format;
\item \var{w} is a word, for example \samp{aabb} if you are working on
  an alphabet that contains the letters \samp{a} and \samp{b};
\item \var{exp} is a rational expression denoting a language;
\item \var{n} is a nonnegative integer.
\end{itemize}


\begin{fnsection}{Input/output work with weighted automata}
\item{define-automaton}{}{Define an automaton from scratch.}
\item{edit-automaton}{a1}{Edit an existing automaton.}
\item{info}{a1}{Print the number of states, transitions, initial and
    final states of \var{a1}.}
\item{display}{a1}{Display the automaton using DOTTY.}  \hline
\end{fnsection}

\begin{fnsection}{Tests and evaluation on weighted automata}
\item{are-isomorphic}{a1 a2}{Test if the two automata are isomorphic.}
\item{evaluation}{a1 w}{Compute the evaluation of \var{w} by
    \var{a1}.}
\item{is-empty}{a1}{.}  \hline
\end{fnsection}

\begin{fnsection}{Generic algorithms for automata}
\item{accessible}{a1}{Extract the sub-automaton of accessible states
    of \var{a1}.}
\item{co-accessible}{a1}{Extract the sub-automaton of co-accessible
    states of \var{a1}.}
\item{trim}{a1}{Trim the automaton \var{a1}.}
\item{transpose}{a1}{Compute the automaton accepting the mirror
    language of the one accepted by \var{a1}.}
\item{closure}{[-bf] a1}{$\varepsilon$-removal algorithm.}
  & \option{-b} : backward closure\\
  & \option{-f} : forward closure\\
\item{concatenate}{a1 a2}{Concatenate \var{a1} to \var{a2}.}
\item{sum}{a1 a2}{Compute the sum of \var{a1} and \var{a2}.}
\item{normalize}{a1}{Compute an automaton with unique initial and
    final states, with $\varepsilon$-transitions.}
\item{standardize}{a1}{Compute an automaton with unique initial state
    without adding $\varepsilon$-transitions.}  \hline
\end{fnsection}

\begin{fnsection}{Generic algorithms for automata on letters}
\item{realtime}{[-bf] a1}{$\varepsilon$-removal algorithm and make
    every transition labeled by a letter.}
  & \option{-b} : backward closure\\
  & \option{-f} : forward closure\\
\item{product}{a1 a2}{Compute the (Cartesian) product of \var{a1} and
    \var{a1}.}
\item{power}{a1 n}{Compute the (Cartesian) product of \var{a1} by
    itself \var{n} times.}
\item{quotient}{a1}{Compute the $\mathbb{Z}$-quotient of \var{a1}.}
  \hline
\end{fnsection}

\begin{fnsection}{Conversion between automata and expressions}
\item{aut-to-exp}{a1}{.}%%FIXME
\item{expand}{exp}{Partially expand rational expressions.}
\item{derived-term}{exp}{Compute the derived term automaton of
    \var{exp}.}
\item{standard}{exp}{Compute the standard (Glushkov) automaton of
    \var{exp}.}
\item{thompson-of}{exp}{Compute the Thompson automaton of \var{exp}.}
  \hline
\end{fnsection}


\section{Building your own automaton}
%%FIXME: Here we should give the usage of define_automaton function.

%%% Local Variables:
%%% mode: latex
%%% TeX-master: "vaucanson-user-manual"
%%% End:

% LocalWords:  determinize

\chapter{Vaucanswig}
\label{sec:swig}

\section{Introduction to Vaucanswig}

Vaucanswig is a set of SWIG definitions which allow to use \Vauc in a
high-level, dynamic, language such as Python, Perl, PHP or Ruby.

\subsection{Introduction}

\Vauc is a \Cxx library that uses static genericity.

SWIG is an interface generator for C and \Cxx libraries, that allow
their use from a variety of languages: CHICKEN, C\#, Scheme, Java,
O'Caml, Perl, Pike, PHP, Python, Ruby, Lisp and TCL.

Unfortunately, running SWIG directly on the \Vauc library does not
work: most of \Vauc features are expressed using \Cxx meta-code,
which means that basically there is no real code in \Vauc for SWIG
to work on.

Vaucanswig comes in between SWIG and \Vauc: it describes to SWIG
some explicit \Vauc types and algorithms implementations so that
SWIG can generate the inter-language interface.

\subsection{Usage}

For  any  SWIG-supported   language,  using  Vaucanswig  requires  the
following steps:

\begin{enumerate}
\item generation of the language interface from SWIG input sources
  (\code{.i} files) provided by Vaucanswig,

\item compilation of the interface into extensions to the language
  library (e.g.  dynamically loadable shared package module for
  Python).

\item loading the extension into the target language.
\end{enumerate}

Vaucanswig provides no material nor tools to achieve these two steps,
except for the Python language target (see below).  Refer to the SWIG
documentation for information about generating language extensions
from SWIG input files for other languages.

\subsection{What is provided?}

\subsubsection{Glossary}

In the  next sections, the  name "category" will  refer to the  set of
features related to a particular algebraic configuration in \Vauc.

The following categories are predefined in Vaucanswig:

\begin{tabular}{*{6}{|c}|}
\hline
Category            & Semiring values  & Monoid values  & Series          &Series values  & Expression values\\
\hline
\code{usual}        & \code{bool}      &  \code{string} &   B<<A*>>       & \code{polynom}&   \code{exp}\\
\code{numerical}    & \code{int}       &  \code{string} &   Z<<A*>>       & \code{polynom}&   \code{exp}\\
\code{tropical\_min} & \code{int}       &  \code{string} &   Z(min,+)<<A*>>& \code{polynom}&   \code{exp}\\
\code{tropical\_max} & \code{int}       &  \code{string} &   Z(max,+)<<A*>>& \code{polynom}&   \code{exp}\\
\hline
\end{tabular}

These are the standard contexts defined in \Vauc. They are defined
in Vaucanswig in the file \code{expand.sh}.

\subsubsection{What is in a category?}

For   a  given   category  *D*,   Vaucanswig  defines   the  following
**modules**:

\begin{description}
\item[\code{vaucanswig\_D\_context}] Algebra and algebraic context.
\item[\code{vaucanswig\_D\_automaton}] Automata types (standard and
  generalized).
\item[\code{vaucanswig\_D\_alg\_...}]  Algorithm wrappers.
\item[\code{vaucanswig\_D\_algorithms}] General wrapper for all
  algorithms.
\end{description}

Each of these modules becomes an extension package/module/namespace in
the target language.

\subsubsection{Algebra}

For  a   given  category  *D*,   the  module  \code{vaucanswig\_D\_context}
contains the following **classes**:

\code{D\_alphabet\_t}:
     Alphabet element with constructor from a string of generator
     letters::

        (constructor): string -> D\_alphabet\_t

\code{D\_monoid\_t}:
     Monoid structural element with the following members:

     - standard \Vauc constructors and operators,

     - method to construct a word element from a simple string::

        make: string -> D\_monoid\_elt\_t

     - method to generate the identity value::

        identity: -> D\_monoid\_elt\_t

\code{D\_monoid\_elt\_t}:
     Word  (monoid element) with  standard \Vauc  constructors and
     operators.


\code{D\_semiring\_t}:
      Semiring structural element with the following members:

      - standard \Vauc constructors and operators,

      - method to construct a weight element from a number::

         make: int -> D\_semiring\_elt\_t

      - methods to generate the identity and zero values::

         identity: -> D\_semiring\_elt\_t
         zero: -> D\_semiring\_elt\_t

\code{D\_semiring\_elt\_t}:
     Weight  (semiring element)  with standard  \Vauc constructors
     and operators.

\code{D\_series\_set\_t}:
      Series structural element with the following members:

      - standard \Vauc constructors and operators,

      - methods to construct a series element from a number or
        string::

          make: int -> D\_series\_set\_elt\_t
          make: string -> D\_series\_set\_elt\_t

      - methods to  generate the identity and zero  values as polynoms
        or expressions::

          identity: -> D\_series\_set\_elt\_t
          zero: -> D\_series\_set\_elt\_t
          exp\_identity: -> D\_exp\_t
          exp\_zero: -> D\_exp\_t

\code{D\_series\_set\_elt\_t}, \code{D\_exp\_t}:
      Polynom  and  expressions  (series  elements  with  polynom  and
      expression implementations) with standard \Vauc constructors
      and operators.

\code{D\_automata\_set\_t}:
      Structural  element  for  automata. Include  standard  \Vauc
      constructors.

\code{D\_context}:
      Convenience  class   with  utility  methods.   It  provides  the
      following members:

      - constructors::

         (constructor): D\_automata\_set\_t -> D\_context
         (copy constructor): D\_context -> D\_context

      - accessors for structural elements::

         automata\_set: -> D\_automata\_set\_t
         series: -> D\_series\_set\_t
         monoid: -> D\_monoid\_t
         semiring: -> D\_semiring\_t
         alphabet: -> D\_alphabet\_t

      - shortcut constructors for elements::

         semiring\_elt: int -> D\_semiring\_elt\_t
         word: string -> D\_monoid\_elt\_t
         series: int -> D\_series\_set\_elt\_t
         series: word -> D\_series\_set\_elt\_t
         series: D\_exp\_t -> D\_series\_set\_elt\_t
         exp: D\_series\_set\_elt\_t -> D\_exp\_t
         exp: string -> D\_expt\_t

In  addition  to these  classes,  the module  \code{vaucanswig\_D\_context}
contains the following **function**::

    make\_context: D\_alphabet\_t -> D\_context

Algebra usage
\code{}\code{}\code{}`

All classes are equipped with a \code{describe} method for
textual representation of values. Example use (Python):

\begin{lstlisting}[language=Python]
>>> from vaucanswig_usual_context import *
>>> c = make_context(usual_alphabet_t("abc"))

>>> c.exp("a+b+c").describe()
'usual_exp_t@0x81a2e60 = ((a+b)+c)'

>>> (c.exp("a")*c.exp("a+b+c")).star().describe()
'usual_exp_t@0x81a20f8 = (a.((a+b)+c))*'

>>> from vaucanswig_tropical_min_context import *
>>> c = make_context(tropical_min_alphabet_t("abc"))

>>> c.series().identity().describe()
'tropical_min_serie_t@0x81ad8b8 = 0'
>>> c.series().zero().describe()
'tropical_min_serie_t@0x81a6de8 = +oo'
\end{lstlisting}

\subsubsection{Automata}

For  a  given  category  *D*,  the  module  \code{vaucanswig\_D\_automaton}
contains the following **classes**:

\code{D\_auto\_t}:
  The standard automaton type for this category.

\code{gen\_D\_auto\_t}:
  The  generalized (with  expression labels)  automaton type  for this
  category.

These class provides the following constructors::

  (constructor): D\_context -> D\_auto\_t
  (constructor): D\_context -> gen\_D\_auto\_t
  (copy constructor): D\_auto\_t -> D\_auto\_t
  (copy constructor): gen\_D\_auto\_t -> gen\_D\_auto\_t
  (constructor): D\_auto\_t -> gen\_D\_auto\_t

For   convenience  purposes,  a   \code{gen\_D\_auto\_t}  instance   can  be
constructed from a \code{D\_auto\_t}  (generalization). The opposite is not
possible, of course.

In addition to the standard \Vauc methods, these classes have been
augmented with the following operators:

\code{describe()}:
   Give a short description for the object.

\code{save(filename)}:
   Save data to a file.

\code{load(filename)}:
   Load data from a file. The automaton must be already defined (empty)
   and its structural element must be compatible with the file data.

\code{dot\_run(tmpf, cmd)}:
   Dump the automaton to file named \code{tmpf}, then run command \code{cmd}
   on file \code{tmpf}. The file is in dot format compatible with Graphviz\_.

.. \_Graphviz: http://www.research.att.com/sw/tools/graphviz/

Example use:

\begin{lstlisting}[language=Python]
>>> from vaucanswig_usual_automaton import *
>>> a = usual_auto_t(c)
>>> a.add_state()
0
>>> a.add_state()
1
>>> a.add_state()
2
>>> a.del_state(1)
>>> for i in a.states():
...   print i
...
0
2
>>> a.dot_run("tmp", "dot_view")

>>> a.save("foo")
>>> a2 = usual_auto_t(c)
>>> a2.load("foo");
>>> a2.states().size()
2
\end{lstlisting}


\subsubsection{Algorithms}

As a  general rule of thumb,  if some algorithm \code{foo}  is defined in
the source file \code{vaucanson/algorithms/bar.hh} then:

\begin{itemize}
\item the module \code{vaucanswig\_D\_alg\_bar} contains a function
  \code{foo},
\item the module \code{vaucanswig\_D\_algorithms} contains
  \code{D.foo}.
\end{itemize}


\subsection{Adding new algorithms}
The Vaucanswig  generator automatically build  Vaucanswig modules from
definitions found in the \Vauc source files.

You  can  add   a  new  algorithm  to  vaucanswig   simply  by  adding
declarations of the form::

\begin{lstlisting}[language=C++]
// INTERFACE: ....
\end{lstlisting}

to the \Vauc headers.


\subsubsection{Example}

Let's consider the \Vauc header \code{foo.hh} in
\code{include/vaucanson/algorithms}, which contains the following
code::

\begin{lstlisting}[language=C++]
// INTERFACE: Exp foo1(const Exp& other) { return vcsn::foo1(other); }
template<typename S, typename T>
Element<S, T> foo1(const Element<S, T>& exp);

// INTERFACE: Exp foo1(const Exp& other1, const Exp& other2) { return vcsn::foo2(other1, other2); }
template<typename S, typename T>
Element<S, T> foo1(const Element<S, T>& exp);
\end{lstlisting}

Then,  after  running  \code{expand.sh}  (the Vaucanswig  generator)  for
category *D*, the module \code{vaucanswig\_D\_alg\_foo} becomes available::

\begin{lstlisting}[language=C++]
   foo1: D_exp_t -> D_exp_t
   foo2: (D_exp_t, D_exp_t) -> D_exp_t
\end{lstlisting}

In   addition,  the   special  algorithm   class  \code{D},   defined  in
\code{vaucanswig\_D\_algorithms}, also contains 'foo1' and 'foo2'.

\subsubsection{Limitations}

When writing  \code{// INTERFACE:} comments, the following  notes must be
taken into consideration:

\begin{itemize}
\item The comment must stand on a single line.  Indeed,
  \code{expand.sh} does not currently support multi-line interface
  declarations.

\item The following special macro names are available:

  \begin{description}
  \item[\code{Exp}] The expression type for the category.
  \item[\code{Serie}] The polynom/serie type for the category.
  \item[\code{Automaton}, \code{GenAutomaton}] The automaton types for
    the category.
  \item[\code{HList}] A list of state or transition handlers
    (integers). This type is \code{std::list<int>} in \Cxx and a
    standard sequence of numbers in the target language.
  \end{description}

\item When accessing automata, a special behavior stands. Instead of
  writing:

  \begin{lstlisting}[language=C++]
    // INTERFACE: void foo(Automaton& a) { return vcsn::foo(a); }
    // INTERFACE: void foo(GenAutomaton& a) { return vcsn::foo(a); }
  \end{lstlisting}

  one should write instead:

  \begin{lstlisting}[language=C++]
    // INTERFACE: void foo(Automaton& a) { return vcsn::foo(*a); }
    // INTERFACE: void foo(GenAutomaton& a) { return vcsn::foo(*a); }
  \end{lstlisting}

  Indeed,  \code{Automaton}   and  \code{GenAutomaton}  do   not  expand  to
  \Vauc automata types, but to  a wrapper type. The real automaton
  can be reached by means of operator*().

\end{itemize}

\subsection{Python support}

For convenience purposes, Python interfaces for Vaucanswig are
included in the distribution.  They are automatically compiled and
installed with \Vauc if enabled.  To enable these modules, run the
\code{configure} script like this:

\begin{lstlisting}
configure --enable-vaucanswig
\end{lstlisting}

\subsection{Licence}

Vaucanswig  is part  of \Vauc,  and is  distributed under  the GNU
General Public Licence. See the file \file{COPYING} for details.

\subsection{Contact}

For  any  comments, requests  or  suggestions,  please  write mail  to
\code{vaucanson@lrde.epita.fr}.


\section{Building language interfaces with Vaucanswig}
\label{sec:swig:build}

This section describes how to use Vaucanswig to produce interfaces
with other languages.

\subsection{Background}

Vaucanswig is a set of \href{http://www.swig.org}{SWIG} wrapper
definitions for the \Vauc library.

SWIG takes Vaucanswig as input, and generates code to link between any
supported scripting language and \Cxx. In that sense, Vaucanswig is
already "meta", because it ultimately supports several scripting
languages. But still, even Vaucanswig itself is automatically
generated, and this "meta-build" process is described in
\autoref{sec:swig:meta}.

The document you are reading explains how to \emph{use} Vaucanswig once it
has been generated.


\subsection{General idea}

Once Vaucanswig has been generated, it is composed of input files to
SWIG.

To use \Vauc in a target scripting language, two steps are necessary:

\begin{enumerate}
\item Produce \Cxx sources for the interface (running SWIG).\\
  This step only requires Vaucanswig sources and a decent version of
  SWIG.
\item Compile these sources.\\
  This step requires the \Vauc library and the extension libraries
  for the selected target language.
\end{enumerate}

\subsection{SWIG modules (\texttt{MODULES})}

Vaucanswig defines a number of SWIG modules.

The list of SWIG modules, hereinafter named \texttt{MODULES}, contains:

\begin{tabular}{|l|p{.6\linewidth}|}
  \hline
  Name of module      & Description
  \\
  \hline
  \code{core}	     & the core of vaucanswig.
  \\
   \code{K\_context}    & for each \var{K}, the definition of the
                         algebraic context \var{K} (\var{K} can be
                         \code{usual}, \code{numerical},
                         \code{tropical} and so on)
  \\
   \code{K\_automaton}  & definition of the Automaton and Expression
   		       types in context \var{K}
  \\
   \code{K\_alg\_A}	     & for each algorithm *A*, the definition of the
                         specific instance of *A* in context
                         \var{K}. (*A* can be "complete", "standard",
                         "product" and so on)
  \\
   \code{K\_algorithms} & a convienient wrapper for context \var{K} with
                         "shortcuts" to all the algorithms instanciated
                         for \var{K}.
  \\
  \hline
\end{tabular}

Note that the name of SWIG modules are closely related to the namespace
where the corresponding features can be found in the target scripting language.

Then, for each module *M*, two items are available:

\begin{tabular}{|l|p{.6\linewidth}|}
  \hline
  Item                     & Description\\
  \hline
   \file{src/vaucanswig\_M.i} &  the dedicated SWIG source file\\

   \file{src/M.deps} &		 (optional, may not exist) a file containing a list
   			 of modules that *M* is dependent upon. If the file is
  			 empty, two cases apply:
                           \begin{itemize}
                           \item *M* is "core" - no dependency
                           \item *M* is not "core" - it depends on "core".
                           \end{itemize}
                           \\
  \hline
\end{tabular}

The first item is the most important. The second is only useful to create
automated build processes which require dependency rules.


\subsection{\Cxx sources specific to the target scripting language (T.S.L.)}

Each TSL needs a different set of wrapper for the Vaucanswig modules.

For any given TSL, source files for the \code{MODULES} can be created by
SWIG by running the following pseudo-algorithm::

\begin{lstlisting}
  $ for M in ${MODULES}; do
      ${SWIG} -noruntime -c++ -${TSL} \
         -I${VAUCANSWIGDIR}/src \
	 -I${VAUCANSWIGDIR}/meta \
	 -I${VAUCANSON_INCLUDES}  \
 	 ${VAUCANSWIGDIR}/src/vaucanswig_${M}.i
    done
\end{lstlisting}%$

Where:

\begin{itemize}
\item \code{\$\{TSL\}} is the SWIG option pertaining to the language
  (python, java ...)
\item \code{\$\{VAUCANSWIGDIR\}} is the root directory of Vaucanswig.
\item \code{\$\{SWIG\}} is the path to the SWIG binary.
\item \code{\$\{VAUCANSON\_INCLUDES\}} is the base directory of the
  \Vauc library.
\end{itemize}

\subsection{Compilation of the binaries for the target scripting language}

The previous step creates a bunch of \Cxx source files of the form::

   \file{vaucanswig\_\$\{M\}\_wrap.cxx}

They should be compiled with the \Cxx compiler supported by the TSL.

The \Cxx compilation should use the following flags:

\begin{itemize}
\item \samp{-DINTERNAL\_CHECKS -DSTRICT -DEXCEPTION\_TRAPS}\\
  Use for more secure code in \Vauc.

\item \samp{-I\$\{VAUCANSON\_INCLUDES\}}\\
  Specify the location of the \Vauc library headers.

\item \samp{-I\$\{VAUCANSWIGDIR\}/src -I\$\{VAUCANSWIGDIR\}/meta}\\
  Needed by Vaucanswig.
\end{itemize}

In addition, any "compatibility" flags required by \Vauc for this
particular \Cxx compiler should be used as well.


\subsection{Automake support for Python as a TSL}

According to the previous section, a \file{Makefile.am} file is
generated in the subdirectory \file{python/}.

It contains four main parts:
\begin{description}
\item[A header]~\\
\begin{lstlisting}[language=Make]
##
## Set INCLUDES for compilation of C++ code.
##

# FIXME: the python path is hardcoded, this is NOT good.
INCLUDES = -I/usr/include/python2.2 \
           -I$(srcdir)/../src -I$(srcdir)/../meta \
  	   -I$(top_srcdir)/include -I$(top_builddir)/include

##
## Set AM_... flags.
##

# According to spec.
AM_CPPFLAGS = -DINTERNAL_CHECKS -DSTRICT -DEXCEPTION_TRAPS
# We want lots of debugging information in the wrapper code.
AM_CXXFLAGS = $(CXXFLAGS_DEBUG)
# For Libtool, to generate dynamically loadable modules.
AM_LDFLAGS = -module -avoid-version
\end{lstlisting}%$

\item[The list of binary targets]
(the shared objects - DLL)

\begin{lstlisting}[language=Make]
# for each MODULE:
pyexec_LTLIBRARIES += libvs_$(MODULE).la
\end{lstlisting}%$

\item[The list of Python source files]~\\

\begin{lstlisting}[language=Make]
# for each MODULE:
python_PYTHON += vaucanswig_$(MODULE).py
\end{lstlisting}%$

\item[Build specifications for binary targets]~\\

\begin{lstlisting}[language=Make]
# for each MODULE:
libvs_$(MODULE)_la_SOURCES = vaucanswig_$(MODULE)_wrap.cxx

# If the module is "core":
#    # This should be the only dependency against static, non-template
#    # Vaucanswig code. And make it a dependency to the SWIG runtime.
#    libvs_core_la_LIBADD = ../meta/libvv.la -lswigpy

# Else:
#    If src/$(MODULE).deps is empty:
#	      libvs_$(MODULE)_la_LIBADD = libvs_core.la
#    Else:
#        for each DEPENDENCY in src/$(MODULE).deps do:
#	        libvs_$(MODULE)_la_LIBADD += libvs_$(DEPENDENCY).la
\end{lstlisting}%$
\end{description}

Additionnaly, the following (not important) parts are generated for
convenience purposes:

\begin{itemize}
\item Rules to rerun SWIG in case something changes in Vaucanswig::
\begin{lstlisting}[language=Make]
vaucanswig_*_wrap.cxx vaucanswig_*.py: ../src/vaucanswig_*.i
	$(SWIG) -noruntime -c++ -python -I... \
	               -o vaucanswig_*_wrap.cxx \
		       ../src/vaucanswig_*.i
\end{lstlisting}%$

\item Installation and uninstallation hooks.
\end{itemize}


\subsection{Automake support for the TSL-independent code}

In order to make things comply to the spirit of the Autotools, a
convenience \file{Makefile.am} is generated in the \file{src/}
directory.

It contains a definition of EXTRA\_DIST with all the SWIG module
source files, of the form: \file{vaucanswig\_\$(MODULE).i}.


\section{Generating and extending Vaucanswig sources}
\label{sec:swig:meta}

The \autoref{sec:swig:build} describes how to use Vaucanswig to create
a wrapper for \Vauc in a scripting language.  (read it first)

This document instead describes how Vaucanswig itself is generated,
currently using the infamous \file{expand.sh} script.

\subsection{The list of Vaucanswig modules}

Once generated, Vaucanswig is a set of SWIG modules. This list of
modules is algorithmically generated. The overall process to build the
list of module names is as follows:

\begin{enumerate}
\item put \code{core} in the \code{MODULES} list.

\item create an auxiliary list \code{ALGS} of algorithm families.

  (detailed below, gives \code{alg\_sum}, \code{alg\_complete}, ...)

\item create an auxiliary list \code{KINDS} of algebra contexts

  (contains \code{boolean}, \code{z}, \code{z\_max\_plus}, ...)

\item extend \code{ALGS} with "\code{context}", "\code{algorithms}"
  and "\code{automaton}".

\item make the cross product of \code{KINDS} and \code{ALGS} putting a
  "\_" between the two parts of each generated name.

\item add the results of this cross product to the \code{MODULES}
  list.
\end{enumerate}

\subsection{The list of algorithm families (\code{ALGS} in step 2 above)}


In Vaucanswig, an "algorithm family" is the set of algorithms declared
in a single \Vauc header file. Most families declare only one
algorithm, but usually with several forms (using overloading). In
Vaucanswig, each algorithm family is related to a SWIG source file:
\code{src/vaucanswig\_alg\_NAME.i} where \code{NAME} is the name of the
algorithm family.

Each family source file contains the following items:

\begin{itemize}
\item a link to its \Cxx header.

\item the definition of a bunch of SWIG macros which are able to
  instanciate the algorithm *declarations* for the type set given as
  parameters.

\item the definition of a bunch of SWIG macros which are able to
  instanciate algorithm *wrappers* for the set of types given as
  parameters.
\end{itemize}

To create the list of algorithm families and associated SWIG sources,
the geneeration script proceeds as follows:

\begin{enumerate}
\item Find all files in the \Vauc includes that declare algorithms
  using the "\code{// INTERFACE:}" construct.

\item For each such include file, proceed as follows:

  \begin{enumerate}
  \item Prepend the base name of the file with "\code{alg\_}" to make a
    "family name".

  \item Create \code{src/vaucanswig\_(family\_name).i} containing the
    relevant SWIG code

  \item Put the generated family name (with prefix) in the \code{ALGS}
    list.
  \end{enumerate}
\end{enumerate}

\subsection{The cross-product of contexts and generic code (step 5 above)}

This is where you find all the magic. :)

This is the step where *real* code (i.e. non-template) is produced.

The goal of this step is to build the list of SWIG modules names *and*
the source file for each SWIG module. The basic idea is simple. It
relies on the following two facts:

\begin{enumerate}
\item each algorithm family defined above defines macros that take
  types as parameters and produce non-template declarations and
  definitions.

\item each algebra context defines a set of types, that fit as
  parameters in the macros for algorithm families.
\end{enumerate}

Now the rest is quite simple. Since we have two lists \code{KINDS}
(contexts) and \code{ALGS} (algorithm families), proceed as follows::

\begin{lstlisting}
  for each K of KINDS, do:
    for each A of ALGS, do:

      # Step 5.1
      instanciate macros...
      ... from src/vaucanswig_alg_${A}.i
      ... using ${K}
      ... into src/vaucanswig_${K}_${A}.i

      # Step 5.2
      add "${K}_${A}" to the MODULES list.

    # the following step is not fundamental, but required for later
    # compilation:

    # Step 5.3 (still in the K loop)
    add "${K}_context" to src/${K}_automaton.deps

    for each algorithm family F, do:

       # Step 5.4
       add "${K}_automaton" to src/${K}_${F}.deps

       # Step 5.5
       add "${K}_${F}" to src/${K}_algorithms.deps
\end{lstlisting}

The result of steps 5.3, 5.4 and 5.5 above can later be used to create
dynamic link dependencies between object code for modules (see
\code{build-process.txt}). It creates the following dependency graph::

\begin{lstlisting}
  core -> K1_context -> K1_automaton -> K1_F1 -> K1_algorithms
                                     -> K1_F2 ->
				     -> K1_F3 ->

       -> K2_context -> K2_automaton -> K2_F1 -> K2_algorithms
                                     -> K2_F2 ->
				     -> K2_F3 ->
\end{lstlisting}

  (and so on)

\subsection{The transparency property}

At every level, a property can be recognized. If an algorithm
\code{foo()} is declared (\Cxx) in \file{bar.h}, then:
\begin{itemize}
\item \code{bar} is the "algorithm family" of \code{foo()}

\item for each selected context \var{K}, exactly one SWIG module exists
  and is called called \samp{K\_bar}.

\item the goal is that at the end of the compilation, in the target
  scripting language you can write::
  \begin{lstlisting}[language=sh]
K_bar.foo()
# (or equivalent)
  \end{lstlisting}
\end{itemize}


\subsection{What is \emph{not} automatic}

Some work is required from the part of the developer:
\begin{itemize}
\item keeping \samp{// INTERFACE:} tags in \Vauc headers.

\item deciding a list of contexts to instanciate in Vaucanswig.

\item running the generator for Vaucanswig generic code whenever
  the \Vauc library is updated.

\item distributing the generated generic sources and building rules
  afterwards.
\end{itemize}

\subsection{Things not easy to change *yet*}

In this section, \var{K} stands for any algebra context.

The set of \var{K} -dependent types available in wrapper code in the
\samp{// INTERFACE:} tags is not yet easily configurable, because it
involves a huge piece of hand-written dedicated code.

For the moment, the following types are available for each
context \code{K}:

\begin{tabular}{|l|p{.6\linewidth}|}
  \hline
   Name of type   & Description\\
  \hline
     Automaton	& the automaton type labeled by series \\
     GenAutomaton& the corresponding type labeled by expressions \\
     Series	& the type of series in K \\
     Exp	& the type of expressions in K \\
     HList	& a type for lists of unsigned integer
                  (to be used as automaton handlers where required)\\
  \hline
\end{tabular}

Adding more of these is not difficult, but very tedious. It involves
adding a new argument in various argument list in various SWIG macros
in the code. These will be documented later.

But still, it remains \strong{very difficult} to bind in Vaucanswig
any algorithm that operates on more than one algebra context at the
same time. "Very difficult" here means that some major work is
required to change Vaucanswig to support this case.



%%% Local Variables:
%%% mode: latex
%%% ispell-local-dictionary: "american"
%%% TeX-master: "vaucanson-user-manual"
%%% End:

\chapter{\Vauc as a library}

To be written.

%%% Local Variables:
%%% mode: latex
%%% TeX-master: "vaucanson-user-manual"
%%% End:

\chapter{Developer Guide}

The chapter is work in progress.  It is not meant for user of the
\Vauc library, but to developer and contributor who wish to include
code in \Vauc.

\section{Tools}

We use a number of tools during the development, so called
\Index{\dfn{maintainer tools}} because they are not required by the
end user.
\begin{description}
\item[Autoconf] Generates \command{configure} which probes the user
  system to configure the compilation.
\item[Doxygen] Reference documentation generator.  Available as
  \code{doxygen} in most package systems.
\item[rst2latex, rst2html] These tools are used to convert
  reStructuredText into more common formats.  Available in the
  DarwinPorts as \code{py-docutils}.
\end{description}

\section{Contributing Code}
\label{sec:contributing-code}


\subsection{Directory usage}

\index{directories}
The \Vauc package is organized as follows:

\newcommand{\dir}[2]{\path{#1} & #2\\}
\noindent
\begin{tabularx}{\textwidth}{|l|X|}
  \hline
  Directory & Usage\\
  \hline
  \dir{doc}{Documentation.}
  \dir{doc/css}{CSS style for Doxygen.}
  \dir{doc/makefiles}{Sample Makefile to reduce compilation time in
    \Vauc.}
  \dir{doc/manual}{User's (and developer's) manual.}
  \dir{doc/share}{LRDE share repository.}
  \dir{doc/xml}{XML proposal.}
  \dir{include/vaucanson}{Library start point: defines classical entry
    points such as ``boolean\_automaton.hh''.}
  \dir{include/vaucanson/algebra/concept}{Algebra concepts, ``Structure''
    part of an Element.}
  \dir{include/vaucanson/algebra/implementation}{Implementations of
    algebraic Structures.  Some specialized structures too.}
  \dir{include/vaucanson/algorithms}{Algorithms.}
  \dir{include/vaucanson/algorithms/internal}{Internal functions of
    algorithms.}
  \dir{include/vaucanson/automata/concept}{Structure of an
    automaton.}
  \dir{include/vaucanson/automata/implementation}{Its
    implementation.}
  \dir{include/vaucanson/config}{Package configuration and system
    files.}
  \dir{include/vaucanson/contexts}{Context headers.}
  \dir{include/vaucanson/design_pattern}{Element design pattern
    implementation.}
  \dir{include/vaucanson/misc}{Internal headers of the whole
    library. }
  \dir{include/vaucanson/tools}{Tools such as dumper, bencher.}
  \dir{include/vaucanson/xml}{XML implementation.}
  \hline
\end{tabularx}
\begin{tabularx}{\textwidth}{|l|X|}
  \hline
  Directory & Usage\\
  \hline
  \dir{argp}{Argp library for TAF-Kit.}
  \dir{build-aux}{Where Autotools things go.}
  \dir{data}{Misc data, like Vaucanson's XSD, Emacs files.}
  \dir{data/b}{Generated Boolean automata.}
  \dir{data/z}{Generated Automata over Z.}
  \dir{debian}{Debian packaging.}
  \dir{src/benchs}{Benches.}
  \dir{src/demos}{Demos.}
  \dir{src/tests}{Test suite.}
  \dir{taf-kit}{Typed Automata Function Kit, binaries to use \Vauc.}
  \dir{taf-kit/tests}{Test suite using the TAF-Kit.}
  \dir{vaucanswig}{Vaucanswig, a SWIG interface for \Vauc.}
  \dir{vcs}{Version Control System configuration
    (\href{http://rubyforge.org/projects/vcs}{VCS Home Page}).}
  \hline
\end{tabularx}

\subsection{Writing Makefiles}

\begin{description}
\item[Produce the output atomically] Generating an output bits by
  bits, say with a series of \samp{>> \$@}, or even with some common
  programs, can result in invalid files if the process failed at some
  point, or was interrupted.  Note that in that case the (invalid)
  output file is newer than its dependencies, therefore it will
  remain.  Instead, create an temporary output, say \samp{\$@.tmp},
  and as last step, rename it a \samp{\$@}.  Sometimes, using
  \command{move-if-change} makes more sense.

\item[Look for the economy] Do what you can to save useless
  recompilations.  Using move-if-change, especially for generated
  files, does save cycles.  But then, beware that time stamps are not
  updated, which can be troublesome if the Makefile includes
  dependency tracking, as they will not be satisfied.

\item[Always include dependencies] included for the bootstrapping
  process.  In their regular development process, the contributors
  should not have to bootstrap again, that should be only for the
  initial check-out, and some other situation where the layout of the
  project has deeply changed.  Therefore, always hook the generated to
  the changes in the generators.

\item[Hunt Makefile duplication] Prefer an included Makefile to
  copy-and-paste of bits.  That's also true for generated Makefiles:
  put the constant parts in a Makefile to be included in the generated
  Makefiles, rather than copying these bits several times.

  This makes it easier to maintain, and also improves locality: you
  can edit the included Makefile and try it in just one directory,
  instead of having to relaunch the generation of all the Makefiles.
\end{description}


\subsection{Coding Style}

Until this is written, please refer to
\href{http://www.lrde.epita.fr/~akim/compil/assignments.split/Coding-Style.html#Coding-Style}{Tiger's
  Coding Style}.

Emacs users should use the indentation style of \file{data/vaucanson.el}.

\paragraph{Be extremely conservative with header inclusions}
Do not include something that is not needed \strong{by the file
  itself}.  Do not include in a \file{*.hh} something that is required
by the \file{*.hxx} file itself.

\paragraph{Document in the \file{*.hh}}
Do not document in the implementation files, but in the declaration
files.  Unless, of course, the function is private and not
``exported''.

\paragraph{Use the same signature in the declaration and in the
  implementation}
It's a bad idea not to follow the same name on both sides, and it
confuses the reader.  Even worse is not giving names to the argument
in the declarations, giving the impression that the argument is
ignored, while using it for real in the implementation.

Keep also the same template parameter names.

Inconsistency confuses users, peer developers, and... Doxygen too.


\subsubsection{Includes}
\index{\#include}

Please, never use backward relative paths anywhere.  There are very
difficult to follow (because several such strings can designate the
same spot), they make renaming and moving virtually impossible etc.

Relative paths to sub-directories are welcome, although in many
situations they are not the best bet.

In \strong{Makefiles}, please using absolute paths starting from
\samp{\$(top\_srcdir)}.  Unfortunately, because Automake cannot grok
includes with Make macros (except... \samp{\$(top\_srcdir)}), we can't
shorten these.

For \strong{header inclusion}, stacking zillions of \samp{-I} is not
the best solution because
\begin{itemize}
\item you have to work to find what file is really included
\item you are likely to find unexpected name collisions if two
  separate directories happens to have (legitimately) two different
  files share the same name
\item etc.
\end{itemize}

So rather, stick to \emph{hierarchies} of include files, and use
qualified \samp{\#include}s.  For instance, use \samp{-I
  \$(top\_srcdir)/include -I \$(top\_srcdir)/src/tests/include} and
\samp{\#include <vaucanson/...>} falls into the first one
(\file{\$(top\_srcdir)/include} has all its content in
\file{vaucanson}), and \samp{\#include <tests/...>} falls into the
latter since \file{\$(top\_srcdir)/src/tests/include} has all its
content in \file{vaucanson}).

\subsection{Use of macros}
\index{\command{cpp}}

\ac{cpp} is evil, but code duplication is even worse.  Macros can be
useful, as in the following example:

\begin{lstlisting}[language=Vaucanson]
# define PARSER_SET_PROPERTY(prop)			\
      if (parser->canSetFeature(XMLUni::prop, true))	\
	parser->setFeature(XMLUni::prop, true);

PARSER_SET_PROPERTY(fgDOMValidation);
PARSER_SET_PROPERTY(fgDOMNamespaces);
PARSER_SET_PROPERTY(fgDOMDatatypeNormalization);
PARSER_SET_PROPERTY(fgXercesSchema);
PARSER_SET_PROPERTY(fgXercesUseCachedGrammarInParse);
PARSER_SET_PROPERTY(fgXercesCacheGrammarFromParse);

# undef PARSER_SET_PROPERTY
\end{lstlisting}

\noindent
but please, respect the following conventions.
\begin{itemize}
\item Use upper case names, unless they are part of the interface such
  as \code{for\_all\_transitions} and so forth.
\item Make them live short lives, as above: undefine them as soon as
  they are no longer needed.
\item Respect the nesting structure: if \file{foo.hh} defines a macro,
  undefine it there too, not in the included \file{foo.hxx}.
\item Indent \ac{cpp} directives.  The initial dash should always be
  in the first column, but indent the spaces (one per indentation)
  between it and the directive.  The above code snippet was included
  in an outer \code{\#if}.
\item Each header file (\file{.hh}, \file{.hxx}, \ldots) should start
  with a classic \ac{cpp} guard of the form
  \begin{lstlisting}
#ifndef FILE_HH
# define FILE_HH
  ...
#endif // !FILE_HH
  \end{lstlisting}
  GCC has some optimizations on file parsing when this scheme is seen.
\item We often rely on \texttt{grep} and tags to search things. Please
  don't clutter names with \ac{cpp} evilness.
\end{itemize}

For instance, this is bad style:

\begin{lstlisting}[language=Vaucanson]
#define VCSN_choose_semiring(Canarg, Nonarg, Typeret...)                  \
    template <class Self>                                                 \
    template <class T>                                                    \
    Typeret                                                               \
    SemiringBase<Self>::Canarg ## choose_ ## Nonarg ## starable(          \
    SELECTOR(T)) const                                                    \
    {                                                                     \
      return op_ ## Canarg ## choose_ ## Nonarg ## starable(this->self(), \
							   SELECT(T));    \
    };
    VCSN_choose_semiring(can_,non_,bool)
    VCSN_choose_semiring(,,Element<Self, T>)
    VCSN_choose_semiring(,non_,Element<Self, T>)
\end{lstlisting}

\subsection{File Names}

In \Vauc the separator is \samp{\_}, not \samp{-}.  We use the
following file extensions:
\begin{description}
\item[cc] Implementation (compilation unit)
\item[hh] Declarations and documentation
\item[hxx] Inline Implementation
\end{description}

File names should match the class they declare, with the conversion of
name conventions (i.e., from \code{MyClass} to \file{my_class.*}).

\subsection{Type Names}

Although some coding standards recommend against this practice, types
in \Vauc should end with \samp{\_t}.  One exception is traits, where
\code{ret} is commonly used.

\code{self\_t}, when defined, always refers to the current class.

\code{super\_t}, when defined, always refers to \emph{the} super
class.  When there are several, \code{super_t} is not used.  The
macros \code{INHERIT\_TYPEDEF} and \code{INHERIT\_TYPEDEF\_} rely on
this convention.

\subsection{Variable Names}
\newcommand{\textstar}{\texttt{\ensuremath{*}}}
Using long variable names clutters the code, so please, don't name
your variables and arguments like \code{automaton1} or
\code{alphabet}.  Structure members and functions should be
descriptive though.

In order to keep the variable names reasonable in size, and
understandable, there are variable name conventions: some families of
identifiers are reserved for some types of entities.  The conventions
are listed below; developers must follow it, and users are encouraged
to do it too.  In the following list, \samp{\textstar} stands for ``nothing,
or a number''.
\begin{description}
\item[al\textstar, alpha\textstar, A\textstar] alphabets
\item[a\textstar, aut\textstar] automata (\code{automaton\_t}, etc.)
\item[t\textstar, tr\textstar] transitions
\item[p\textstar, q\textstar, r\textstar, s\textstar] states (\code{hstate\_t})
\end{description}

Some variables should be consistently used to refer to some ``fixed''
values.
\begin{description}
\item[monoid\_identity] The neutral for the monoid, the empty word.
\begin{lstlisting}[language=Vaucanson]
monoid_elt_t monoid_identity = a.series().monoid().empty_;
\end{lstlisting}

\item[null\_series] The null series, the 0, the identity for the sum.
\begin{lstlisting}[language=Vaucanson]
series_set_elt_t null_series = a.series().zero_;
\end{lstlisting}

\item[semiring\_elt\_zero] The zero for the weights.
\begin{lstlisting}[language=Vaucanson]
semiring_elt_t semiring_elt_zero = a.series().semiring().wzero_;
\end{lstlisting}
\end{description}

\subsection{Commenting Code}
\label{sec:commenting-code}

Use Doxygen.  Besides the usual interface description, the Doxygen
documentation must include:
\begin{itemize}
\item references to the definitions of the algorithm, e.g., a
  reference to the ``�l�ments de la th�orie des automates'', or even
  an URL to a mailing-list archive.
\item detailed description of the assumptions, or, if you wish, pre-
  and post-conditions.
\item the name of the developer
\item use the \texttt{@pre} and \texttt{@post} tags liberally.
\end{itemize}

Don't try to outsmart your tool, even though it does not use the words
``param'' and ``arg'' as we do, stick to \emph{its} semantics (let
alone to generate correct documentation without warnings).  This is
correct:
\begin{lstlisting}[language=C++]
  /**
   * Delete memory associated with a stream upon its destruction.
   *
   * @arg \c T	Type of the pointed element.
   *
   * @param ev	IO event.
   * @param io	Related stream.
   * @param idx	Index in the internal extensible array of a pointer to delete.
   *
   * @see iomanip
   * @author Thomas Claveirole <thomas.claveirole@lrde.epita.fr>
   */
  template <class T>
  void
  pword_delete(std::ios_base::event ev, std::ios_base &io, int idx);
\end{lstlisting}
while this is not:
\begin{lstlisting}[language=C++]
  /** ...
   * @param T	Type of the pointed element.
   *
   * @arg ev	IO event.
   * @arg io	Related stream.
   * @arg idx	Index in the internal extensible array of a pointer to delete.
   * ... */
\end{lstlisting}

\subsection{Writing Algorithms}

There is a number of requirement to be met before including an
algorithms into the library:
\begin{description}
\item[Document the algorithm] See \autoref{sec:commenting-code}.
\item[Comment the code] Especially if the code is a bit tricky, or
  smart, or avoids nasty pitfalls, it \emph{must} be commented.

\item[Bind the algorithm to \tafkit]

\item[Include tests] See \autoref{sec:writing-tests} for more
  details.  Tests based on \tafkit are appreciated.  Note that tests
  require test cases: to exercise an algorithm, not any automaton will
  do, try to find relevant samples.  Again, ETA is a nice source of
  inspiration.

\item[Complete the documentation]  The pre- and post-conditions should
  also be described here.
\end{description}

When submitting a patch, make it complete (i.e., including the
aforementioned items), and provide a ChangeLog.  See
\href{http://www.lrde.epita.fr/dload/guidelines/guidelines.html}{Le
  Guide du \lrde}, section ``La maintenance de projets'' and
especially ``�crire un ChangeLog'' for more details.

Because \Vauc uses Trac, ChangeLog entries should explicit refer to
tickets (e.g., ``Fixing issue \#38: implement is\_ambiguous''), and
possible previous revisions (e.g., ``Fix a bug introduced in
[1224]'').

\subsection{Writing Tests}
\label{sec:writing-tests}


\subsection{Mailing Lists}
\label{sec:mailing-lists}

\Vauc comes with a set of mailing lists:
\begin{description}
\item[vaucanson@lrde.epita.fr] General discussions, feature request etc.
\item[vaucanson-bugs@lrde.epita.fr] To report errors in code,
  documentation, web pages, etc.
\item[vaucanson-patches@lrde.epita.fr] To submitted patches on code,
  documentation, and so forth.
\item[vaucanson-private@lrde.epita.fr] To contact privately the
  \Vauc team.
\end{description}

Please, bear in mind that there are these lists have many readers,
therefore this is a WORM medium: Write Once, Read Many.  As a
consequence:
\begin{itemize}
\item Be complete.\\
  One should not strive to understand what you are referring to, so
  always include proper references: URLs, Ticket numbers \emph{and
    summary}, etc.
\item Be concise.\\
  Write short, spell checked, understandable sentences.  Reread
  yourself, remove useless words, be proud of what you wrote.  Show
  respect to the reader.  Spare us useless messages.
\item Be structured.\\
  Quick and dirty replies with accumulated layers of replies at the
  bottom of the message is not acceptable.  The right ordering is not
  the one that is the quickest to write, but the easiest to read.
\item Be attentive.\\
  Lists are not write-only: consider the feedback that is given with
  respect.
\end{itemize}

As an example of what's not to be done, avoid answering to yourself to
point out you made a spell mistake: we can see that, and that's a
waste of time to read another message for that.  Also, there is no
hurry, it would probably be better to wait a bit to have a complete,
well thought out, message, rather than a thread of 4 messages
completing, contradicting, each other.  Finally, if you still need to
fix your message, supersede it, or even cancel it.

\section{Vaucanson I/O}
\label{sec:vaucanson-io}

January 2005

Here is some information about input and output of automata in
\href{http://www.lrde.epita.fr/vaucanson}{Vaucanson}.

\subsection{Introduction}

As usual, the structure of the data representing an automaton in a flat
file is called the file format.

There are several input and output formats for Vaucanson
automata. Obviously:
\begin{itemize}
\item {}
input formats are those that can be read from, i.e. from which an
automaton can be loaded.

\item {}
output formats are those that can be written to, i.e. to which an
automaton can be dumped.

\end{itemize}

Given these definitions, here is the meat:
\begin{itemize}
\item {}
Vaucanson supports Graphviz (dot) as an output format. Most kinds of
automata can be dumped as dot-files. Through the library this format
is simply called \texttt{dot}.

\item {}
Vaucanson supports XML as an input and output format. Most kinds of
automata can be read and written to and from XML streams, which
Vaucanson does by using the Xerces-C++ library. Through the library
this format is simply called \texttt{xml}.

\item {}
Vaucanson supports the FSM toolkit I/O format as an input and output
format. This allows for basic FSM interaction. Only certain kinds of
weighted automata can be meaningfully input and output with this
format. Through the library this format is simply called \texttt{fsm}.

\item {}
Vaucanson supports a simple informative textual format as an input
and output format. Most kinds of automata can be read and written to
and from this format. Through the library this format is simply called
\texttt{simple}.

\end{itemize}


%___________________________________________________________________________

\subsection{Dot format}

This format provides an easy way to produce a graphical representation
of an automaton.

Output using this format can be given as input to the Graphviz \texttt{dot}
command, which can in turn produce graphical representations in
Encapsulated PostScript, PNG, JPEG, and many others.

It uses Graphviz' ``directed graph'' subformat.

If you want to see what it looks like go to the \texttt{data/b}
subdirectory, build the examples and run them with the ``dot''
argument.

For Graphviz users:

Each graph generated by Vaucanson can be named with a string that also
prefixes each state name. If done so, several automata can be grouped
in a single graph by simply concatenating the Vaucanson outputs.


%___________________________________________________________________________

\subsection{XML format}

This format is intended to be an all-purpose strongly typed input and
output format for automata.

Using it requires:
\begin{itemize}
\item that the Xerces-C++ library is installed and ready to use by the
  C++ compiler that is used to compile Vaucanson.

\item configuring Vaucanson to use XML.

\item computer resources and time.
\end{itemize}

What you gain:
\begin{itemize}
\item support for the Greater and Better I/O format. See documentation
  in the \texttt{doc/xml} subdirectory for further information.
\end{itemize}

If you want to see what it looks like go to the \texttt{data/b}
subdirectory, build the examples and run them with the \texttt{xml}
argument.


%___________________________________________________________________________

\subsection{FSM format}

This format is intended to provide a basic level of compatibility with
the FSM tool kit. (FIXME: references needed)

Like FSM, support for this format in Vaucanson is limited to
deterministic automata. It probably does not work with transducers,
either.

It is not meant to be used that much apart from performance comparison
with FSM. Some code exists to simulate FSM, in
\texttt{src/demos/utilities/fsm}.

If you want to see what it looks like go to the \texttt{data/b}, build
the examples and run them with the \texttt{fsm} argument.


%___________________________________________________________________________

\subsection{Simple format}

Initially intended to be a quick and dirty debugging input and output
format, this format actually proves to be a useful, compact and
efficient textual representation of automata.

Advantages over XML:
\begin{itemize}
\item does not require additional 3rd party software,

\item simple and efficient (designed to be read and written to streams
  with very low memory footprint and minimum complexity),

\item less bytes in file,

\item not strongely typed (can be dumped from one automaton type and
  loaded to another).
\end{itemize}

Drawbacks from XML:
\begin{itemize}
\item not strongely typed (one cannot know what automaton type to
  build by only looking at the raw data).

\item currently does not (probably) support transducers.
\end{itemize}

If you want to see what it looks like go to the \texttt{data/b}, build
the examples and run them with the \texttt{simple} argument.


%___________________________________________________________________________

\subsection{Using input and output}

The library provides an infrastructure for generic I/O, which
(hopefully) will help supporting more formats in the future.

The basis for this infrastructure is the way a developer C++ using the
library will use it:

\begin{lstlisting}[language=Vaucanson]
#include <vaucanson/tools/io.hh>

/* to save an automaton */
output_stream << automaton_saver(automaton, converter, format)

/* to load an automaton */
input_stream >> automaton_loader(automaton, converter, format, merge_states)
\end{lstlisting}

Where:
\begin{description}
\item[\texttt{automaton}] is the automaton undergoing input or output.
  Note that the object must already be constructed, even to be read
  into.

\item[\texttt{converter}] is a helper class that is able to convert
  automaton transitions to character strings and possibly vice-versa.

\item[\texttt{format}] is a helper class that is able to convert the
  automaton to (and possibly from) a character string, using the
  converter as an argument.

\item[\texttt{merge\_states}] is an optional argument that should be
  omitted in most cases. For advanced users, it allows loading a
  single automaton from several different streams that share the same
  state set.
\end{description}


%___________________________________________________________________________

\subsubsection{About converters}

The \texttt{converter} argument is mandatory. There are several converter
types already available in Vaucanson. See below.

An I/O converter is a function object with one or both of the following:
\begin{itemize}
\item an operation that takes an automaton, a transition label and
  converts the transition label to a character string (std::string).
  This is called the output conversion.

\item an operation that takes an automaton, a character string and
  converts the character string to a transition label. This is called
  the input conversion.
\end{itemize}

Vaucanson already provides these converters:
\begin{description}
\item[{\texttt{vcsn::io::string\_out}, bundled with \texttt{io.hh}.}]

  Provides the output conversion only. Uses the C++ operator
  {\textless}{\textless} to create a textual representation of
  transition labels. Should work with all label types.

\item[{\texttt{vcsn::io::usual\_converter\_exp}, defined in
    \texttt{tools/usual\_io.hh}.}]

  Provides both input and output conversions. Uses the C++ operator
  {\textless}{\textless} to create a textual representation of
  transition labels, but requires also that algebra::parse can read
  back that representation into a variable of the same type. It is
  mostly used for generalized automata where transitions are labeled
  by rational expressions, hence the name.

\item[{\texttt{vcsn::io::usual\_converter\_poly<ExpType>}, defined
    in \texttt{tools/usual\_io.hh}.}]  Provides both input and
  output conversions. Converts transition labels to and from ExpType
  before (after) doing I/O. The implementation is meant to be used
  when labels are polynoms, and using the generalized (expression)
  type as ExpType.
\end{description}


%___________________________________________________________________________

\paragraph{Notes about XML and converters}

When the XML I/O format was implemented, the initial converter system
was not used. Instead a specific converter system was re-designed
specifically for this format.

(FIXME: explain why!)

(FIXME: why hasn't the generic converter for XML been ported back to
fsm and simple formats?)

Because of this, when using XML I/O the ``converter'' argument is
completely ignored by the format processor. Usually you can see
\texttt{vcsn::io::string\_output} mentioned.

(FIXME: this is terrible! it must be patched to use an empty
vcsn::io::xml\_converter\_placeholder or something like it).


%___________________________________________________________________________

\subsubsection{About formats}

The \texttt{format} argument is mandatory. It specifies an instance of the
object in charge of the actual input or output.

A format object is a function object that provides one or both the
following operations:
\begin{itemize}
\item an operation that takes an output stream, the caller
  \texttt{automaton\_saver} object, and the \texttt{converter}
  object. This is called the output operation.

\item an operation that takes an input stream and the caller
  \texttt{automaton\_loader} object. This is called the input
  operation.  Note that this operation does not uses the
  \texttt{converter} object, because it should call back the
  \texttt{automaton\_loader} object to actually perform string to
  transition label conversions.
\end{itemize}

Format objects may require arguments to be constructed, such as the
title of the automaton in the output.

Format objects for a format should be defined in a
\texttt{tools/xxx\_format.hh} file.

Vaucanson provides the following format objects:
\begin{description}
\item[{\texttt{vscn::io::dot(const std::string{\&} digraph\_title)},
    in \texttt{tools/dot\_format.hh}.}]

  Provides an output operation for the Graphviz \texttt{dot}
  subformat. The title provided when buildint the \texttt{dot} object
  in Vaucanson becomes the title of the graph in the output data and a
  prefix for state names. Therefore the title must contain only
  alphanumeric characters or the underscore (\_), and no spaces.

\item[{\texttt{vcsn::io::simple()}, in
    \texttt{tools/simple\_format.hh}.}]

  Provides both input and output operations for a simple text format.

\item[{\texttt{vcsn::xml::XML(const std::string{\&} xml\_title)}, in
    \texttt{xml/XML.hh}.}]

  Provides both input and output operations for the Vaucanson XML I/O
  format.
\end{description}

(FIXME: why not tools/xml\_format.hh with proper includes of headers
in xml/?)

(FIXME: really the FSM format should have a format object too.)


%___________________________________________________________________________

\subsection{Examples}

Create a simple dot output for an automaton a1:
\begin{lstlisting}[language=vaucanson]
std::ofstream fout("output.dot");
fout << automaton_saver(a1, vcsn::io::string_output(), vcsn::io::dot("a1"));
fout.close()
\end{lstlisting}

Output automaton a1 to XML, read it back into another automaton a2
(possibly of another type):
\begin{lstlisting}[language=Vaucanson]
std::ofstream fout("file.xml");
fout << automaton_saver(a1, NULL, vcsn::xml::XML());
fout.close()

std::ifstream fin("file.xml");
fin >> automaton_loader(a2, NULL, vcsn::xml::XML());
fin.close()
\end{lstlisting}

Do the same, but this time using the simple format. The automata are
generalized, i.e. labeled by expressions:
\begin{lstlisting}[language=Vaucanson]
std::ofstream fout("file.txt");
fout << automaton_saver(a1, vcsn::io::usual_converter_exp(), vcsn::io::simple());
fout.close()

std::ifstream fin("file.txt");
fin >> automaton_loader(a2, vcsn::io::usual_converter_exp(), vcsn::io::simple());
fin.close()
\end{lstlisting}


%___________________________________________________________________________

\subsection{Internal scenario}

What happens in Vaucanson when you write:
\begin{lstlisting}[language=Vaucanson]
fin >> automaton_loader(a1, c1, f1)
\end{lstlisting}
?

\begin{enumerate}
\item function \texttt{automaton\_loader} creates an object AL1 of
  type \texttt{automaton\_loader\_} that memorizes its arguments.

\item \texttt{automaton\_loader()} returns AL1.

\item \texttt{operator>{}>(fin, AL1)} is called.

\item \texttt{operator>{}>} says to format object f1: ``hi, please use
  fin to load something with AL1''.

\item f1 scans input stream fin. Things may happen then:
  \begin{itemize}
  \item f1 finds a state numbered N. Then it says to AL1: ``hey, make
    a new state into the output automaton, keep its handler s1 for
    yourself and remember it is associated to N''.  (callback
    \texttt{AL1.add\_state})

  \item f1 finds a transition from state numbered N to state P,
    labeled with character string S. Then it says to AL1: ``hey,
    create a transition with N, P, and S.'' (callback
    \texttt{AL1.add\_transition}).  Then:
    \begin{itemize}
    \item AL1 remembers handler for state N (s1)

    \item AL1 remembers handler for state P (s2)

    \item AL1 says to converter c1: ``hey, make me a transition label
      from S''

    \item AL1 creates transition from s1 to s2 using converted label
      into output automaton.
\end{itemize}
\end{itemize}

\item When f1 is finished, it returns control to \texttt{operator>{}>}
  and then calling code.

  Of course since everything is statically compiled using templates
  there is no performance drawback due to the intensive use of
  callbacks.

\end{enumerate}

%___________________________________________________________________________

\subsection{Convenience utilities}

For most formats the (relatively) tedious following piece of code:
\begin{lstlisting}[language=Vaucanson]
output_stream << automaton_saver(a, CONVERTER(), FORMAT(...))
\end{lstlisting} %>>
is also available as:

\begin{lstlisting}[language=Vaucanson]
FORMAT_dump(output_stream, a, ...)
\end{lstlisting}

If available, this convenience utility is defined in
\texttt{tools/XXX\_dump.hh}.

Conversely, the following piece of code:

\begin{lstlisting}[language=Vaucanson]
input_stream >> automaton_loader(a, CONVERTER(), FORMAT(...))
\end{lstlisting}

is usually also available as:

\begin{lstlisting}[language=Vaucanson]
FORMAT_load(input_stream, a, ...)
\end{lstlisting}

If available, this convenience utility is defined in
\texttt{tools/XXX\_load.hh}.

(FIXME: move fsm\_load away from fsm\_dump.hh!)

As of today (2006-03-17) the FSM format is only available using the
fsm\_load() and fsm\_dump() interface.

%%% Local Variables:
%%% mode: latex
%%% TeX-master: "vaucanson-user-manual"
%%% End:


\appendix

% Generated, do not edit by hand.
\chapter{Automaton Library}

\Vauc comes with a set of interesting automata that can be used to toy
with \tafkit (\autoref{sec:tafkit}) for instance.  In the chapter, we
present each one of these automata.


\section{Boolean Automata}
\subsection{a1}
\label{b:a1}
\execdisplay{al-b-a1}{vcsn-b dump-automaton a1 | vcsn-b}{-}

\subsection{b1}
\label{b:b1}
\execdisplay{al-b-b1}{vcsn-b dump-automaton b1 | vcsn-b}{-}

\subsection{div3base2}
\label{b:div3base2}
\execdisplay{al-b-div3base2}{vcsn-b dump-automaton div3base2 | vcsn-b}{-}

\subsection{double-3-1}
\label{b:double-3-1}
\execdisplay{al-b-double-3-1}{vcsn-b dump-automaton double-3-1 | vcsn-b}{-}

\subsection{ladybird-6}
\label{b:ladybird-6}
\execdisplay{al-b-ladybird-6}{vcsn-b dump-automaton ladybird-6 | vcsn-b}{-}

\section{\texorpdfstring{$\mathbb{Z}$}{Z}-Automata}
\subsection{b1}
\label{z:b1}
\execdisplay{al-z-b1}{vcsn-z dump-automaton b1 | vcsn-z}{-}

\subsection{c1}
\label{z:c1}
\execdisplay{al-z-c1}{vcsn-z dump-automaton c1 | vcsn-z}{-}

\section{Transducers}
\subsection{t1}
\label{tdc:t1}
\execdisplay{al-tdc-t1}{vcsn-tdc dump-automaton t1 | vcsn-tdc}{-}

\subsection{u1}
\label{tdc:u1}
\execdisplay{al-tdc-u1}{vcsn-tdc dump-automaton u1 | vcsn-tdc}{-}


\chapter{Bits of Automaton Theory}
\label{sec:theory}

\section{On standard and normalized automata}

\subsection{Standard automata}

\subsubsection{Definition}

\begin{definition}[\Index{Standard Automaton}]
  An automaton (any kind, automata over any monoid with any
  multiplicity) is said to be \dfn{standard} if it has a unique
  initial state which is the destination of no transition and whose
  'initial multiplicity' is equal to the identity (of the multiplicity
  semiring or of the series semiring, according to the current
  convention).
\end{definition}

\begin{remark}
  These terminology and definition are to be found in ETA and are not
  (yet) universally known or accepted.
\end{remark}

\subsubsection{Standardization}

Not only every automaton is equivalent to a standard one, but a simple
procedure, called 'standardization', transforms every automaton $A$ in
an equivalent standard one, and goes as follows.

\begin{enumerate}
\item Add a new state $s$, make it initial, with initial multiplicity
  equal to the identity.

\item For every initial state i of $A$, with initial multiplicity
  $I(i)$, add a transition from $s$ to i with label I(i), and set I(i)
  to 0 (the zero of the semiring, or of the series -- as above).

\item\label{ite:suppress} Suppress all epsilon-transition from the
  created transitions by a backward closure.

\item\label{ite:take} Take the accessible part of the result.
\end{enumerate}

\begin{remark}
  Steps \autoref{ite:suppress} and \autoref{ite:take} are necessary to
  insure the following property:

  The standardization of a standard automaton $A$ is isomorphic to $A$ .

  More informally, but more generally, they insure that the result of
  the standardization is of the same "kind" as the automaton on which
  it is applied (in particular, without epsilon-transition if $A$ is
  without epsilon-transition).
\end{remark}



\subsubsection{Standard automaton of an expression}

A classical algorithm --- often credited to Glushkov --- transforms a
rational (ie regular) expression (of litteral length n ) into a
standard automaton (with n+1 states). This automaton is known in the
litterature as the 'Glushkov automaton' or as the 'position automaton'
of the expression.

\begin{remark}
  It is folklore that the epsilon-transition removal --- via a
  "backward closure" --- applied to the 'Thompson automaton' of a
  rational expression produces the standard automaton of the
  expression.
\end{remark}

\begin{remark}
  These definitions, constructions and properties are fairly classical
  for classical automata. Their generalization to automata with
  multiplicity is more recent (mostly written by the "Rouen school"
  around the year 2000).
\end{remark}

\subsection{Normalized automaton}

\subsubsection{Definition}


An automaton (any kind, automata over any monoid with any
multiplicity) is said to be \index{normalized}\dfn{normalized} if

\begin{enumerate}
\item it has a unique initial state
  \begin{itemize}
  \item which is the destination of no transition,
  \item whose 'initial multiplicity' is equal to the identity (of the
    multiplicity semiring or of the series semiring, according to the
    current convention),
  \item and whose 'final multiplicity' is equal to the zero (with the
    same convention);
  \end{itemize}

\item and, symmetrically, it has a unique final state
  \begin{itemize}
  \item which is the source of no transition,
  \item whose 'final multiplicity' is equal to the identity,
  \item and whose 'initial multiplicity' is equal to the zero.
  \end{itemize}
\end{enumerate}

\begin{remark}
  The terminology is rather unfortunate, for there are already so many
  different "normalized" things. The notion however, is rather
  classsical, under this name, at least for classical Boolean
  automata, because of one classical proof of Kleene theorem. For the
  same reason, it is a proposition credited to Schutzenberger that
  every weighted automaton $A$ is equivalent to a normalized one,
  provided the empty word is not in the support of the series realized
  by $A$, although the word normalized is not used there. The
  terminology is even more unfortunate since "normalized transducer"
  has usually an other meaning, and corresponds to transducers whose
  transitions have label of the form either (a,1) or (1,b) .
\end{remark}

\subsubsection{Normalization}

It is not true that every automaton is equivalent to a normalized one.
This holds only for automata whose accepted language does not contain
the empty word (for classical automata) or whose realized series gives
a zero coefficient to the empty word (for weighted automata). There
exists however a "normalization procedure" which plays mutatis
mutandis the same role as the standardization and which is best
described with the help of the standardization.

Let $A$ be an automaton.

Let $B$ = standardize(transpose(standardize(transpose(A))))

Let i be the (unique) initial state of $B$ and let C be the automaton
obtained from $B$ by setting T(i)=0 --- ie setting to 0 the terminal
function. Then, C is normalized, we write C = normalize(A) and it
holds:

for classical automata: The language accepted by normalize(A) is equal
to the language accepted by $A$ minus the empty word (if it is accepted
by A):

\begin{displaymath}
  L(normalize(A))  =  L(A) \ 1_{X^*}
\end{displaymath}

\noindent
(where  X  is the alphabet.)

for weighted automata: the series realized by normalize(A) is equel to
the one realized by $A$, but for the coefficient of $1_{X^*}$ which is 0:

%% FIXME: Can't use \circledot here.  What is the package for it?
\begin{displaymath}
  |normalize(A)|  =  |A| \odot  char(X^+)
\end{displaymath}

\noindent
(where  $char(X^+)$  is the characteric series of  $X^+$.

that is, in both cases, normalize(A) accepts or realizes the 'proper'
part of the language accepted, or of the series realized by, $A$ .

\subsection{Operations on automata}

These families of automata have been considered in order to establish
one direction of Kleene's theorem, the one that amounts to show that
languages accepted (or series realized) by finite automata are closed
under rational operations: sum, product and star.

\subsubsection{The sum}

The sum is never a problem: the union of two automata is an automaton
whose behaviour is the sum of the behaviours of these automata.

\begin{remark}
  If we consider automata with unique initial and/or final state, it
  would be a bad idea to realize the sum by merging the initial and/or
  final states of the two automata in order to recover automata of the
  same kind --- unless these initial states have no incoming
  transitions and/or these final states have no outgoing transitions,
  that is if we consider standard automata, transpose of standard
  automata, or normalized automata.
\end{remark}
\subsubsection{The concatenation}

The product (of accepted language or of realized series) is carried
out by the "concatenation" of automata --- since we keep the word
"product" for the cartesian product of automata which realizes the
intersection of languages or the hadamard product of series. For
classical automata, the concatenation of $A$ and $B$ can be described as
follows: add an epsilon-transition from every final state of $A$ to
every initial state of $B$, and suppress the epsilon-transition (if
necessary, and by any closure algorithm). For weighted automata, the
'same' algorithm is more easily described by using a standardization
step: compute A' the 'co-standardized' automaton of $A$, compute B' the
standardized automaton of $B$, add an epsilon-transition (with label
identity) from the unique final state of A' to the unique initial
state of B' and suppress this new transition (if necessary and by any
closure algorithm).

\begin{remark}
  Along the same line as above, if $A$ has a unique final state $t$ and
  B a unique initial state $j$, it would be a bad idea to realize the
  concatenation of $A$ and $B$ by merging t and $j$ ---unless t has no
  outgoing transition, that is if $A$ is 'co-standard' or normalized.
\end{remark}


\subsubsection{The star}

The "star" of an automaton $A$, realizing the star of the accepted
language or of the realized series, is even more subtle.

If $A$ is normalized, it is easily carried out by the merging of the
initial and final states of $A$ . Since the series accepted by a
normalized automaton is proper, its star is always defined, this is
the advantage of the construction. On the other hand, star(A) is not
normalized anymore, and if this operation is used inside an algorithm
that buils an automaton from an expression, it yields an explosion of
the number of states.

If $A$ is standard, with initial state $i$ and initial multiplicity c
(usually a scalar), the star of |A| is defined if, and only if, the
$c^{*}$ is defined \cite[Prop. III.2.6]{sakarovitch.03.eta} --- if $A$ has
no epsilon-transition.  In this case, star(A) is defined as follows:

\begin{enumerate}
\item replace the initial multiplicity by $c^*$;

\item for every final state $t$ of $A$, add a new transition from $t$
  to $i$ with label $T(t) \times 1_{X^*}$;

\item Suppress the epsilon-transition via backward closure.
\end{enumerate}
\begin{remark}
  If $A$ is not standard, it would be a bad idea to use the above
  construction, even letting asside the multiplicity --- although it
  may have occured to knowledgeable people.
\end{remark}

\subsection{Conclusion}

Normalized and standard automata have been introduced in relation with
the proof of Kleene's theorem. If one does not want to introduce
epsilon-transition, the notion of normalized automata yields certainly
the most straightforward argument. The advantage of standard automata
is that they not only can be used for the same proof, but they also
yields an efficient algorithm, both for the size of the result and for
the computational complexity, to transform a rational expression into
an automaton.

It took me some times to get to this conclusion. If I were to rewrite
a new edition of \cite{sakarovitch.03.eta}, I would not mention
normalized automata besides exercices and historical notes. All the
theory would be presented with standard automata only.

%%% Local Variables:
%%% mode: latex
%%% ispell-local-dictionary: "american"
%%% TeX-master: "vaucanson-user-manual"
%%% End:

\chapter{A proposal for an XML format for automata}
\label{sec:xml}

This is not a complete description of the \Vauc proposal for an XML
format for automata.  The interested reader will find such a
description at the following URL.  We just present here few examples
of files, that should give an idea on how these files are built.

\begin{lstlisting}
<automaton name="a1" xmlns="http://vaucanson.lrde.epita.fr">
   <content>
      <states>
         <state name="s0"/>
         <state name="s1"/>
         <state name="s2"/>
      </states>
      <transitions>
         <transition src="s0" dst="s0" label="a"/>
         <transition src="s0" dst="s0" label="b"/>
         <transition src="s0" dst="s1" label="a"/>
         <transition src="s1" dst="s2" label="b"/>
         <transition src="s2" dst="s2" label="a"/>
         <transition src="s2" dst="s2" label="b"/>
         <initial state="s0"/>
         <final state="s2"/>
      </transitions>
   </content>
</automaton>
\end{lstlisting}

%%% Local Variables:
%%% mode: latex
%%% TeX-master: "vaucanson-user-manual"
%%% End:

\chapter{Algorithms specifications}
\label{chap:specification}

\section{Vocabulary}

In \Vauc, we use a precise vocabulary to speak about automaton. As is
it specific to our project and some expressions may be not widely used
or approved by the automata community, we choose to define them here.

\begin{description}
\item[$\mathbb{B}$] Boole's semiring.

\item[Boolean automaton] is a ``classical'' automaton. Precisely, it
  is a automaton over a free monoid which transitions are labeled by
  letters of an alphabet with multiplicity in $\mathbb{B}$.

\item[automaton with multiplicity in $\mathbb{B}$] is an automaton
  over any kind of monoid (in \Vauc we have free monoid and product
  of free monoids) with its multiplicity in $\mathbb{B}$.

\item[realtime automaton] is an automaton over a monoid which
  transitions are labelled by letters only (not words).

\item[FMP-transducer] is a transducer over a free monoid product.

\item[RW-transducer] is a transducer over a series $\mathbb{K'}<< \mathbb{K}<<A^*>> >>$.
\end{description}

\section{Algorithms applicability in \Vauc}

%% FIXME: Precise which algorithms takes only realtime, deterministic,
%% automata as input.

%% FIXME: Precise what we can expect from resulting automaton.


\subsection{Algorithms on graph}

\begin{description}
\item[accessible] (accessible.hh)
\item[accessible\_states] (accessible.hh)
\item[coaccessible] (accessible.hh)
\item[coaccessible\_states] (accessible.hh)
\item[trim] (trim.hh)
\item[useful\_states] (trim.hh)
\item[sub\_automaton] (sub\_automaton.hh)
\item[is\_void] %% FIXME: is it called is_empty in vaucanson?
\end{description}

\subsection{Algorithms on labeled graphs}
%% Algorithmes sur les graphes �tiquet�s (sans interpreter les �tiquettes)}

\begin{description}
\item[are\_isomorphic] (isomorph.hh)
\item[aut\_to\_exp] (aut\_to\_exp.hh)
\item[sum] (sum.hh)
\item[thompson\_of] (thompson.hh)
\item[is\_normalized] (normalized.hh)
\item[normalize] (normalized.hh)
\item[union\_of\_normalized] (normalized.hh)
\item[concatenate\_of\_normalized] (normalized.hh)
\item[star\_of\_normalized] (normalized.hh)
\item[standard\_of] (standard\_of.hh)
\item[standardize] (standard.hh)
\item[is\_standard] (standard.hh)
\item[union\_of\_standard] (standard.hh)
\item[concat\_of\_standard] (standard.hh)
\item[star\_of\_standard] (standard.hh)
\end{description}

\subsection{Algorithms on labeled graphs (epsilon-transitions are distinguish)}
%%== Algorithmes sur les graphes �tiquet�s ( trans. spontanees distinguees) ==

\begin{description}
\item[generalized]
\item[closure] (closure.hh)
\item[backward\_closure] (closure.hh)
\item[forward\_closure] (closure.hh)
\item[concatenate] (concatenate.hh)
\item[cut\_up] (cut\_up.hh)
\end{description}

\subsection{Algorithms on graphs labeled on $\mathbb{K}<<A^*>>$}
%%== Algorithmes sur les graphes �tiquet�s sur K << A* >> ==

\begin{description}
\item[is\_realtime] (realtime\_decl.hh)
\item[backward\_realtime] (backward\_realtime.hh)
\item[forward\_realtime] (forward\_realtime.hh)
\item[realtime] (realtime.hh)
\end{description}

\subsection{Algorithms on graphs labeled on series of letter with
  multiplicities ($\sum{(a, \mathbb{K}_{a*}a)}$)}
%%== Algorithmes sur les graphes �tiquet�s sur une s�rie de lettres � coefficients (sum(a, k_a*a)) ==

\begin{description}
\item[product] (product.hh)
\item[eval] (eval.hh)
\item[evaluation] (evaluation.hh)
\item[is\_ambiguous]
\item[is\_deterministic] (determinize.hh)
\item[is\_sequential]
\item[quotient] (minimization\_hopcroft.hh)
\item[derived\_term\_automaton] (derived\_term\_automaton.hh)
\item[broken\_derived\_term\_automaton] (derived\_term\_automaton.hh)
\item[complete] (complete.hh)
\item[is\_complete] (complete.hh)
\item[transpose] (transpose.hh)
\end{description}

\subsection{Algorithms on Boolean automata}
%%== Algorithmes sur les automates bool�ens ==

\begin{description}
\item[determinize] (determinize.hh)
\item[brzozowski] (brzozowski.hh)
\item[berry\_sethi] (berry\_sethi.hh)
\item[canonical] (aci\_canonical.hh)
\item[complement] (complement.hh)
\item[minimization\_moore] (minimization\_moore.hh)
\item[co\_minimization\_moore] (minimization\_moore.hh)
\item[minimization\_hopcroft] (minimization\_hopcroft.hh)
\item[search] (search.hh)
\end{description}

\subsection{Algorithms on automata with multiplicities in $\mathbb{K}<<A^*>>$}
%%== Algorithmes sur les automates � multiplicit� dans K<<A*>> (rw-transducers) ==
%% Les algorithmes V fonctionnent evidemment dans cette classe

\begin{description}
\item[domain]
\item[image]
\item[extension] (extension.hh)
\item[inverse]
\end{description}

\subsection{Algorithms on realtime transducers}
%%== Algorithmes sur les transducteurs realtime ==

\begin{description}
\item[evaluation] (evaluation.hh)
\item[realtime\_composition] (realtime\_composition.hh)
\item[realtime\_to\_fmp] (realtime\_to\_fmp.hh)
\end{description}

\subsection{Algorithms on realtime RW-transducers}
%%== Algorithmes sur les RW-transducteurs lettre a letter ==

\begin{description}
\item[letter\_to\_letter\_composition] (letter\_to\_letter\_composition.hh)
\end{description}

\subsection{Algorithms on FMP-transducers}
%%== Algorithmes sur les FMP ==

\begin{description}
\item[domain] (projections\_fmp.hh)
\item[image] (projections\_fmp.hh)
\item[extension] (extension.hh)
\item[identity] (projections\_fmp.hh)
\item[evaluation\_fmp] (evaluation\_fmp.hh)
\item[inverse]
\item[insplitting] (outsplitting.hh)
\item[outsplitting] (outsplitting.hh)
\item[sub\_normalize] (sub\_normalize.hh)
\item[normalized\_composition] (normalized\_composition.hh)
\item[fmp\_to\_realtime] (fmp\_to\_realtime.hh)
\end{description}

\subsection{Algorithms on Boolean FMP-transducers}
%%== Algorithmes sur les FMP booleens ==

\begin{description}
\item[b\_composition] (normalized\_composition.hh)
\end{description}

\subsection{Algorithms on regular expressions over $\mathbb{K}<<A^*>>$}
%%== Algorithmes sur les expressions sur K<<A*>> ==

\begin{description}
\item[flatten] (krat\_exp\_flatten.hh)
\item[expand] (krat\_exp\_expand.hh)
\end{description}




%%% Local Variables:
%%% mode: latex
%%% ispell-local-dictionary: "american"
%%% TeX-master: "vaucanson-user-manual"
%%% End:


\printindex

\bibliographystyle{apalike}
\bibliography{vaucanson.bib}

\end{document}


%%% Local Variables:
%%% mode: latex
%%% ispell-local-dictionary: "american"
%%% TeX-master: t
%%% End:
