\thispagestyle{empty}
\vspace*{20pt}
\vskip4pt \hrule height 4pt width \hsize \vskip4pt
\begin{center}
  \Huge 
  \vcsnv\\
%   Specifications
  \tafkit Documentation
\end{center}
\vspace*{-1.5ex}
\vskip4pt \hrule height 4pt width \hsize \vskip4pt
\vspace*{20pt}
\vfill

% \longonly{%
    \begin{center}
        \textsc{about this document}
    \end{center}

{\itshape 
The \vcsn platform is an on-going project of a free
software platform dedicated to the manipulation of finite state
automata, which started about ten years ago already. 
It is 
conducted at LTCI, Telecom ParisTech, IGM, University Paris-Est, and 
LRDE, EPITA, in Paris.
The last version of the platform, coined \vcsnv, is meant to be the 
last one of a first phase of this project.

This document describes a part of this version only, the \tafkit, and 
will serve as a user's manual for it.
\tafkit will be the only 
documented part of \vcsnv.

A second phase of the project, officially starting March~1st, 2011, 
is now engaged with the same partners together with the team of Prof. 
Hsu-Chun Yen at the EE Department of the National Taiwan University 
in Taipeh.
It will give rise to versions \vcsn 2.x.y, hopefully as soon as 
possible, and fully documented.


\begin{flushright}
%     J.~S.\\
    July 2012    
\end{flushright}
}%
% }%
% {\itshape 
% This document is based on a first draft of a \vcsn User's Manual 
% written by Alexandre Duret-Lutz in April~2009 for the 
% version~1.2.95a.
% 
% It is first meant to describe as precisely as possible the 
% specifications of \tafkit within \vcsnv. 
% It is a working document and should help to finalizing \tafkit. 
% The discrepancies, in names and functionalities --- there should 
% not be many --- will raise final discussions and decisions.
% When \vcsnv will be released, the same document, with the adequate 
% corrections, will serve as a rather complete user's manual for 
% \tafkit
% which will be the only 
% documented part of that version. 
% 
% 
% The corresponding version of the \vcsn platform --- \vcsnv --- is meant 
% to be the last one of a first phase of this project.
% Officially starting January 2011, a second phase will be engaged, 
% with different interface specification.
% It will give rise to versions \vcsn 2.x.y.
% 
% 
% \begin{flushright}
%     J.~S.\\
%     July 2011    
% \end{flushright}
% }%
% 
\vfill
    
\begin{center}
        \textsc{authoring}
    \end{center}

	\noindent 
The authors of the \vcsn platform are jointly represented  
 as the \vcsn \textsc{Group}.
% 
%  \bigskip
The permanents of this group are:

\medskip
	\noindent 
Alexandre Duret-Lutz, LRDE, EPITA, Paris 
\PushLine 
{\tt Alexandre.Duret-Lutz@lrde.epita.fr}

	\noindent 
Sylvain Lombardy, IGM, Universit\'e Paris Est Marne-la-Vall\'ee,
\PushLine 
 {\tt lombardy@univ-mlv.fr}
 
	\noindent 
Jacques Sakarovitch, LTCI, CNRS / Telecom-ParisTech, Paris 
\PushLine 
 {\tt sakarovitch@enst.fr}
 
 \bigskip
	\noindent 
The authors of the \vgi Graphical User Interface are represented by

\medskip
\noindent
Hsu-Chun Yen, EE Dept., National Taiwan University, Taipeh
\PushLine 
{\tt yen@cc.ee.ntu.edu.tw}


\vspace*{20pt}

\begin{center}
        \textsc{acknowledgements}
    \end{center}

\noindent
Since March~1st, 2011, the work on the \vcsn platform is supported by 
the Agence Nationale de la Recherche with the project 
ANR-10-INTB-0203.

\smallskip 
\noindent
The work on the Graphical User Interface \vgi is  supported by the  
NSC-100-2923-E-002-001-MY3  project since January~1st, 2011. 


\newpage
% \tableofcontents
\begin{center}
    {\Large \textbf{Table of contents}}
\end{center}

\makeatletter
\@starttoc{toc}
\makeatother



%%%%%%%%%%%%%%%%%%%%
\endinput 


