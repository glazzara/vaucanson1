\documentclass{article}

\usepackage{listings}
\usepackage{url}
\usepackage{ulem}
\usepackage{amsmath}
\usepackage{amsfonts}
\usepackage{palatino}
\usepackage{a4wide}
\usepackage{xspace}
\usepackage{listings}

% macros.tex
%
% $Id$

\newenvironment{code}[1][]%
{\begin{center}
     \rule{1cm}{0.02cm} \textit{#1} \rule{1cm}{0.02cm} 
  \end{center}
  \verbatim }
{\endverbatim \begin{center} \rule{2cm}{0.02cm} \end{center}}



% TODO: �crire les macros pour les exemples de code.
% TODO: v�rifier que l'HTML est ok.
% TODO: �tablir un code de lecture (remarque de lecture, exemple, ...)

\begin{document}

\title{Vaucanson -- a tutorial}

\maketitle

\tableofcontents

\section{Overview}

\subsection{Presentation}

Vaucanson is a C++ library for finite state machine manipulation. 

Vaucanson is:

\begin{itemize}
  
\item \textbf{generic}: a general algorithm is written once and is
  instantiated for the good parameters at use ;
  
\item \textbf{algorithm oriented}: the system is meant to provide
  primitive services to write algorithms ;

\item \textbf{meta}: the C++ is enriched to obtain a flexible framework.

\end{itemize}

\subsection{Distribution, installation, basic usage}

The tarball can be found at \url{http://www.lrde.epita.fr/twiki/VaucansonLib}.

The installation is the classical:

\begin{verbatim}
./configure
make
make install (as root)
\end{verbatim}

If you are not root on your system, you can install Vaucanson by typing:

\begin{verbatim}
./configure --with-prefix=$HOME/include/ 
make
make install
\end{verbatim} % $

The only difference is that you must specify a -I \$(HOME)/include in
your compilation flags.

\subsection{Directory tree}

The Vaucanson library is organized as follows: 

\begin{verbatim}
`-- vaucanson
    |-- algebra
    |   |-- concept
    |   `-- concrete
    |       |-- alphabets
    |       |-- free_monoid
    |       |-- letter
    |       |-- semiring
    |       `-- series
    |           `-- rat
    |-- algorithms
    |-- automata
    |   |-- concept
    |   `-- concrete
    |-- config
    |-- fundamental
    |-- internal
    |-- misc
    `-- tools
\end{verbatim}

\begin{itemize}
\item \textbf{fundamental}: the core of the library, its goal is to
  enriched C++. (see section \ref{sec:fundamental}) ;

\item \textbf{config}: the internal configuration system ;
\item \textbf{internal}: the internal C++ headers ;
\item \textbf{algebra}: the algebra module ;
\item \textbf{automata}: the automata module ;
\item \textbf{algorithms}: the algorithms set ;
\item \textbf{misc}: some tools to interact with external tools ;
\item \textbf{tools}: some useful tools for daily work.
\end{itemize}

\subsection{A simple example}

% FIXME: choose a good example.

\subsection{Plan}

The goal of this document is to present the philosophy of the library
and to demonstrate some features of the system in a practical manner.

\section{Preliminaries}

%FIXME: vcsn namespace, template C++ references, articles, STL ... etc etc

\section{Fundamental: enriched c++}

The fundamental module provides some sugar to build the system
genericity. It can be seen as the core of the system and, then,
non-expert users do not have to understand it. Yet, a so-called
'Element' design pattern is used in every line of Vaucanson to enable
orthogonal specialization and genericity: one must be aware of it
before coding with the Vaucanson library.

\subsection{Element/Set design pattern}

As mentionned in (FIXME), Vaucanson is designed in an algebraic
manner. To reinforce this view and to provide a simple way to separate
implementation and theoritical behaviour, we use a unified way to
structure all the object used in the library: the Element pattern. The
'Basics' section is essential whereas the other can be skipped for a
first reading.

\subsubsection{Basics}

First, in Vaucanson, everything is an instance of
\textbf{vcsn::Element$<$S, T$>$} where the 'S' parameter is
interpreted as the set of the element and 'T' as its implementation.
For example, Element$<$Series, polynom$>$ is an element of the series
implemented with the polynom class. What is important is that every
Element$<$S, T$>$ provides the same interface. You can access the
implementation by the 'value()' method and the set by the 'set()' one.
As a consequence, objects are build incrementally from their set. 

% FIXME: Write valid Vaucanson code please !
% FIXME: add all the code into the src/demos.
\begin{verbatim}
Alphabet a;
FreeMonoid f(a);
Semiring s;
Series s(f, s);
Element<Series, polynom> s;
Element<FreeMonoid, std::string> m(f);
Element<Semiring, int>   w;
s.set();          // the series set.
s.set().monoid(); // the monoid.
s.value();        // the polynom.
m.set().alphabet() == s.set().monoid().alphabet();
s[m] = w;
\end{verbatim}

Second, the particular behaviour of an implementation 'T' viewed as an
element of a set 'S' is defined by specialization of the class
\textbf{vcsn::MetaElement$<$S, T$>$}. For instance, the
MetaElement$<$Series, polynom$>$ provides a 'is\_stareable()' method
that returns true if we can take the star of the current serie. The
MetaElement class is not a user class, it is just a way to define the
interpretation of the implementation as a particular concept.

% FIXME:Add example.

% FIXME: consequence => separation in the library between concrete/concept.

\subsubsection{A short example}

\subsubsection{Inheritance between Element}

\subsubsection{Orthogonal specialization}

\subsubsection{Dynamic/Static properties}


\section{Dealing with algebraic structures}

Even if classical context are provided (like letter acceptor),
Vaucanson enables the user to precisely define the algebraic context
to work with. 

The next sections show how it can be done for the main algebraic
components: alphabet, free monoid, semiring, serie and rational
expression.

\subsection{Alphabet}

An alphabet is collection of letters. 

% FIXME: add a glossary
In Vaucanson, we have the set of all alphabet over a particular type
of letter denoted by the AlphabetSetBase abstract class. A final class
SetAlphabet implements this class in a trivial manner: it is an empty
class for static typing purpose. 

% We also have a DecoratedAlphabets class which represents the set of alphabet 
% decorated with particular symbol like '?' or '~(.)' \ldots 
% A creuser \ldots

An element of the alphabet set (so an alphabet) can be implemented in a lot of
different ways. For example, we can use std::set or std::list for
dynamic alphabets but there are also static alphabets.

\paragraph{Example 1 (dynamical alphabet)}

\paragraph{Type Creation} 
First we have to precise wich kind of letter we wish manipulate inside our 
futur alphabet. You can use the builtins types of C/C++ (like char, int, ...)
or use more specific types proposed in Vaucanson. To do that, one trivial
typedef be enough :

\begin{verbatim}
typedef char Letter;
\end{verbatim}

in order to have an alphabet of char, or we can type : 

\begin{verbatim}
typedef static_ranged<char, static_char_interval<'a','z'> >  Letter;
\end{verbatim}

in order to have a more specific kind of symbols inside your alphabet. 
If you choose this last option, symbols can only be letters between 'a'
and 'z', nothing more.

Then we have to create the type of set of alphabets inside which we will 
manipulate one of them. We can do this very naturally :

\begin{verbatim}
typedef AlphabetSet<Letter>  Alphabets;
\end{verbatim}

Lastly, we can create the type of the alphabet we will effectively use,
with the Element pattern :

\begin{verbatim}
typedef Element<Alphabets, std::set<Letter> >  Alphabet;
\end{verbatim}

This previous should be understand like : ``an element of the set of all
alphabets, implemented by a set of Letter''.
In fact this code is already present inside Vaucanson, more exactly inside
a file named ``predefs.hh'' :

\begin{verbatim}
==============================================================================

 namespace small_alpha_letter {
      
      typedef static_ranged<char, static_char_interval<'a','z'> >  Letter;
      typedef AlphabetSet<Letter>				   Alphabets;
      typedef Element<Alphabets, std::set<Letter> >		   Alphabet;

    } // small_alpha_letter

    namespace char_letter {

      typedef AlphabetSet<char>				   Alphabets;
      typedef Element<Alphabets, std::set<char> >		   Alphabet;

    } // char_letter

    namespace int_letter {

      typedef AlphabetSet<int>					   Alphabets;
      typedef Element<Alphabets, std::set<int> >		   Alphabet;

    } // int_letter

==============================================================================
\end{verbatim}

You have only to choose the right namespace for your work, or create a new.
Next a short example of alphabet manipulation :

\begin{verbatim}
==============================================================================

  using namespace vcsn;
  using namespace algebra;
  using namespace small_alpha_letter;
  
  using std::cout;
  using std::endl;
  
  Alphabet A;
  
  Letter a('a');
  
  A.insert(a);
  A.insert(Letter('b'));
  
  cout << "Size of alphabet : "
       << A.size() << endl;                // return 2
  
  
  cout << "Is \'a\' inside alphabet ? (0 or 1) : "
       << A.contains('a') << endl;         //return true

  cout << "Is \'1\' inside alphabet ? (0 or 1) : "
       << A.contains('1') << endl;         //return false
  
  
  cout << "element of alphabet are : ";
  for (Alphabet::iterator i = A.begin(); i != A.end(); i++)
    cout << *i << " ";
  cout << endl;
  
  cout << "random sequence of 10 symbols of alphabet : ";
  for (unsigned i = 0; i < 10; i++)
    cout << A.choose() << " ";
  cout << endl;
       
==============================================================================
\end{verbatim}


\subsection{Free monoid}

Now we use an algebraic structure called a free moinoid. It is the set of
words we can represent with an alphabet (a collection of symbols) and
a specific operation. 

\subsection{Semiring}

\subsection{Series}

\subsection{Rational expression}

\section{Automaton}

\subsection{Simple usage}

\section{Algorithms}

\section{Your 'grep'}

\subsection{From a regular expression to an automaton}

\subsection{Speed}

\subsection{Multiplicity in action}

\section{Play with multiplicity}

\section{Advanced use}

\subsection{Extending Vaucanson}

\end{document}